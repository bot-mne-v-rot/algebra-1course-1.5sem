\section{Комплексификация вещественного линейного пространства}
Пусть $V$ -- Линейное пространство над $\R$. По нему мы построим $V_{\C}$ -- линейное пространство над $\C$.
Как множество, оно будет опреляться следующим образом: $V_{\C} = V \times V$. 

Теперь введем на этом множестве необходимые операции: 
\begin{itemize}
    \item Сложение: $(v, w) + (v', w') = (v + v', w + w')$
    \item Умножение на комплексное число: $(\alpha + i \beta)(v, w) = (\alpha v - \beta w, \alpha w + \beta v)$ 
\end{itemize}
Несложно проверить, что $V_{\C}$ с этими операциями -- линейное пространство над $\C$.

\begin{theorem-non}(Базис $V_{\C}$)
    Пусть $e_1, \dots, e_n$ -- базис $V$. Тогда $(e_1, 0), \dots, (e_n, 0)$ --- базис $V_{\C}$
\end{theorem-non}
\begin{proof} \quad 

    Для начала заметим, что $(\alpha + i \beta) \cdot (v, 0) = (\alpha v, \beta v)$. В частности: 
    \begin{gather}
        \sum\limits_{j=0}^{n} (\alpha_j + i \beta_j)(e_j, 0) = \left( \sum \alpha_j e_j, \sum \beta_j e_j \right)
    \end{gather}
    
    Таким образом, $\{ \Lin(e_j, 0), j = 1, \dots, n \} = V_{\C}$. Линейная независимость векторов тоже видна из равенства $(2)$. 
    Если сумма $(2)$ оказалась равна 0, то $\alpha_j = \beta_j = 0, j = 0, \dots, n$
\end{proof}

\follow $\dim_\C V_{\C} = \dim_\R V$

\begin{conj}
    $V_{\C}$ называется комплексификацией $V$
\end{conj}

Заметим, что любой вектор оттуда можно записать как $(v, w) = (v, 0) + (0, w) = (v, 0) + i(w, 0)$.
Так что захотелось отождествить $(v, 0)$ с $v$. И тогда пара $(v, w) = v + iw$.

Важно отметить на будещее, что из разнообразных базисов в $\R$ мы будем получать базисы комплексификаций $V$, 
но при этом, у комплексификаций есть и свои отдельные базисы, где не все вектора будут вещественными. 