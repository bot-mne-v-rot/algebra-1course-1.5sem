\section{Изменение координат вектора под действием линейного отображения}
\begin{theorem}
    Пусть $E$ --- базис $V$, $F$ --- базис $W$, $\varphi \in \Hom(V, W)$. \\
    Тогда $[\varphi(v)]_F = [\varphi]_{E, F} \cdot [v]_E$
\end{theorem}
Получается следующая коммутативная диаграмма:
\begin{center}
    \shorthandoff{"}
    \begin{tikzcd}
        v \arrow[dd, maps to] & V \arrow[rr, "\varphi"] \arrow[dd, "\varepsilon_E"] &  & W \arrow[dd, "\varepsilon_F"]                  & w \arrow[dd, maps to] \\
                            &                                    &  &                               &                       \\
        {[v]_E}               & K^n \arrow[rr]                     &  & K^m                           & {[w]_F}               \\
                            & {[v]_E} \arrow[rr, maps to]        &  & {[\varphi]_{E,F} \cdot [v]_E} &                      
    \end{tikzcd}
    \shorthandon{"}
\end{center}
\begin{proof}
    Т.к. $E, F$ --- вектора-строки из базисных векторов, получаем: 
    $v = E \cdot [v]_E$,  $\varphi(v) = F \cdot [\varphi(v)]_F$.

    С другой стороны, $\varphi(v) = \varphi(E) \cdot [v]_E$, где
    $\varphi(E) = (\varphi(e_1), \dots, \varphi(e_n))$ -- вектор-строка
    из образов базисных векторов. Более того, 
    $\varphi(E) = F \cdot [\varphi]_{E, F}$, т.к. 
    $\varphi(e_i) = F \cdot [\varphi(e_i)]_F$.

    Собираем вместе:
    \begin{align*}
        \varphi(v) &= F \cdot [\varphi(v)]_F \\
        &= (F \cdot [\varphi]_{E, F}) \cdot [v]_E \\
        &= F \cdot ([\varphi]_{E, F} \cdot [v]_E)
    \end{align*}

    Получаем, что линейная комбинация векторов $F$ с коэф.
    $[\varphi(v)]_F$ равна ЛК векторов $F$ с коэф.
    $[\varphi]_{E, F} \cdot [v]_E$. Т.к. $F$ --- базис $W$, делаем вывод,
    что $[\varphi(v)]_F = [\varphi]_{E, F} \cdot [v]_E$.
\end{proof}

\notice Проще всего понять, что происходит, можно на матрицах.

Пусть 
$E = (e_1, \dots, e_n)$,
$[v]_E = \begin{pmatrix}
\alpha_1 \\ 
\vdots \\ 
\alpha_n
\end{pmatrix}$, 
$[\varphi(v)]_F = \begin{pmatrix}
\beta_1 \\ 
\vdots \\ 
\beta_m
\end{pmatrix}$,
$[\varphi(e_i)]_F = \begin{pmatrix}
\gamma_{1i} \\ 
\vdots \\ 
\gamma_{mi}
\end{pmatrix}$. Тогда
$$
\begin{pmatrix}
    \beta_1 \\ 
    {
        \tikz[overlay]
        \fill[green!20] (-0.1,-0.2) rectangle (0.5,0.4);
    }
    \beta_2 \\ 
    \vdots \\ 
    \beta_m
\end{pmatrix}
=
\begin{pmatrix}
    \gamma_{11} & \gamma_{12} & \dots & \gamma_{1n} \\
    {
        \tikz[overlay]
        \fill[blue!20] (0,-0.2) rectangle (3.5,0.4);
    }
    \gamma_{21} & \gamma_{22} & \dots & \gamma_{2n} \\
    \vdots & \vdots & \ddots & \vdots \\
    \gamma_{m1} & \gamma_{m2} & \dots & \gamma_{mn} \\
\end{pmatrix}
\cdot
\begin{pmatrix}
    {
        \tikz[overlay]
        \fill[red!20] (-0.1,0.3) rectangle (0.5,-1.9);
    }
    \alpha_1 \\ 
    \alpha_2 \\ 
    \vdots \\ 
    \alpha_n
\end{pmatrix}
$$

Получается, что: 
\colorbox{green!20}{$\beta_i$}$= \sum \limits_{k=1}^{n}$ 
\colorbox{blue!20}{$\gamma_{ik}$}
$\cdot$
\colorbox{red!20}{$\alpha_k$}

И это можно переписать вот так:
$$
\underbrace{
\begin{pmatrix}
    \beta_1 \\
    \vdots \\ 
    \beta_m
\end{pmatrix}
}_{=[\varphi(v)]_F}
=
\alpha_1
\underbrace{
\begin{pmatrix}
    \gamma_{11} \\ 
    \vdots \\ 
    \gamma_{m1}
\end{pmatrix}
}_{=[\varphi(e_1)]_F}
+
\alpha_2
\underbrace{
\begin{pmatrix}
    \gamma_{12} \\ 
    \vdots \\ 
    \gamma_{m2}
\end{pmatrix}
}_{=[\varphi(e_2)]_F}
+
\dots
+
\alpha_n
\underbrace{
\begin{pmatrix}
    \gamma_{1n} \\ 
    \vdots \\ 
    \gamma_{mn}
\end{pmatrix}
}_{=[\varphi(e_n)]_F}
$$
И действительно, 
$\varphi(v) = \alpha_1 \varphi(e_1) + \dots + \alpha_n \varphi(e_n)$.

\notice Пусть $E, E'$ --- базисы $V$. Тогда $M_{E \to E'} =
[\E_V]_{E', E}$, где $\E_V \in \Hom(V, V)$ --- тождественное отображение на $V$.
Действительно, в матрице перехода между базисами 
$M_{E \to E'}$ мы раскладываем новый базис $E'$ по старому базису $E$,
что по сути то же самое, что в матрице линейного отображения, только
никакого отображения мы к векторам не применяем. Или другими словами,
мы применяем к базисным векторам тождественное отображение.
