\section{Центр и коммутант группы. Критерий абелевости факторгруппы}
\begin{conj}
    Центр группы -- множесто элементов данной группы, которые коммутируют со всеми ее элементами:
    \[ Z(G) = \{ g \in G \, | \, \forall h \in G : gh = hg \} \]
\end{conj}

\begin{theorem-non}
    Центр группы является ее нормальной подгруппой.
\end{theorem-non}
\begin{proof} \quad \\
    \begin{itemize}
        \item Замкнутость остносительно операции: 
        \begin{gather*}
            g_1, g_2 \in Z(G), h \in G \\
            h(g_1g_2) = g_1hg_2 = (g_1g_2)h \Rightarrow g_1g_2 \in Z(G)
        \end{gather*}
        \item Очевидно, что нейтральный элемент лежит в $Z(G)$
        \item Существование обратного:
        \begin{gather*}
            g \in Z(G), h \in G \\
            g^{-1}h = (h^{-1}g)^{-1} = (gh^{-1})^{-1} = hg^{-1} \Rightarrow g^{-1} \in Z(G)
        \end{gather*}
        \item Нормальность:
        \begin{gather*}
            g \in Z(G), h \in G \\
            hgh^{-1} = hh^{-1}g = g \Rightarrow hgh^{-1} \in G
        \end{gather*}
    \end{itemize}
\end{proof}

\begin{conj}
    Коммутатором для элементов $g, h \in G$ называется эллемент $[g, h] = ghg^{-1}h^{-1}$.
\end{conj}
\begin{notice}
    \begin{enumerate}
        \item Коммутатором двух элементов является нейтральный элемент, если и только если они коммутируют.
        \begin{proof} \quad \\
            $"\Rightarrow": ghg^{-1}h^{-1} = e \Rightarrow ghg^{-1} = h \Rightarrow gh = hg$ \\
            $"\Leftarrow": gh = hg \Rightarrow ghg^{-1} = h \Rightarrow ghg^{-1}h^{-1} = e$
        \end{proof}
        \item Обратный к коммутатору тоже является коммутатором, причем $[g, h]^{-1} = [h, g]$.
        \begin{proof}
            $[g, h]^{-1} = (ghg^{-1}h^{-1})^{-1} = hgh^{-1}g^{-1} = [h, g]$
        \end{proof}
    \end{enumerate}
    
\end{notice}


\begin{conj}
    Коммутант -- подгруппа, порожденная всеми коммутаторами группы. Обозначается как $[G, G]$.
\end{conj}

\begin{theorem-non}
    Коммутант является нормальной подгруппой.
\end{theorem-non}
\begin{proof}
    Это подгруппа по определнию, значит, осталось проверить нормальность.
    Сперва проверим, что сопряжение коммутатора дает коммутатор. \\
    Введем следующее обозначение для сопряжения: $aga^{-1} = g^a$.
    Легко видеть, что отображение $g \mapsto g^a$ является автоморфизмом.
    Тогда
    \begin{gather*}
        [g, h]^a = g^ah^a(g^{-1})^a(h^{-1})^a = g^ah^a(g^a)^{-1}(h^a)^{-1} = [g^a, h^a]
    \end{gather*}
    Отсюда уже легко доказать нормальность, ведь любой элемент $[G, G]$ раскладывается в произведение коммутаторов:
    \[ ([g_1, h_1][g_2, h_2]\dots[g_m, h_m])^a = [g_1^a, h_1^a][g_2^a, h_2^a]\dots[g_m^a, h_m^a] \in [G, G] \]
\end{proof}

\begin{theorem-non}
    Пусть $H \lhd G$. 
    Тогда следующие 2 утверждения эквивалентны: 
    \begin{enumerate}
        \item $G / H$ абелева группа
        \item $H \supset [G, G]$
    \end{enumerate}
\end{theorem-non}
\begin{proof} \quad \\
    \begin{gather*}
        \begin{split}
            \text{$G / H$ абелева} 
            &\Leftrightarrow \forall g_1, g_1 \in G \quad g_1Hg_2H = g_2Hg_1H \\
            &\Leftrightarrow (g_1H)(g_2H)(g_1H)^{-1}(g_2H)^{-1} = eH  \\
            &\Leftrightarrow (g_1g_2g_1^{-1}g_2^{-1})H = eH (\text{произведение классов -- класс произведения})\\
            &\Leftrightarrow [g_1, g_2] \in H \\
            &\Leftrightarrow [G, G] \subset H
        \end{split}
    \end{gather*}
\end{proof}