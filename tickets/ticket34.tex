\section{Двойственный базис. Двойственное отображение}
\begin{conj}
    Пусть $V$ -- линейное пространство над $K$.
    Двойственным к $V$ пространством называют пространство линейных отображений из $V$ в $K$.
    (Еще встречаются названия ``дуальное'' или ``сопряженное'' пространство).
    \begin{gather*}
        V^* := \Hom(V, K)
    \end{gather*}
    Элементы $V^*$ называют линейными функционалами на $V$.
\end{conj}

\underline{Пример:}
Пусть $V = C[0, 1]$ -- пространство непрерывных функций на отрезке $[0, 1]$. 
Примером линейного функционала $\in V^*$ будет $\varphi$:
\begin{gather*}
    \varphi : f \longmapsto f(0)
\end{gather*}
\begin{theorem-non}
    Пусть $\dim V = n < +\infty$. Тогда $\dim V^* = n$
\end{theorem-non}
\begin{proof} \quad 

    Доказывать собственно нечего, так как у нас был следующий общий факт: 

    $V, W$ --- конечномерные. 
    \begin{gather*}
        \Hom(V, W) \cong M(m, n, K), \text{ где } m = \dim V, n = \dim W \\
        \Longrightarrow \dim \Hom(V, W) = mn
    \end{gather*}

    Ну а в нашем случае второе пространство имеет размерность 1, значит 
    мы просто получаем размерность первого пространства.
\end{proof}
Если у нас зафиксирован некий базис в первом пространстве, то можно каноническим образом построить 
так называемый ``двойственный'' базис во втором пространстве.
\begin{conj}
    Пусть $e_1, \dots, e_n$ --- базис $V$
    \begin{gather*}
        e^i : V \longrightarrow K \\
        e_j \longmapsto \delta_{ij}
    \end{gather*}
    
\end{conj}

\begin{theorem-non}
    $(e^1, \dots, e^n)$ -- базис $V^*$
\end{theorem-non}
\begin{proof} \quad
    
    Так как мы уже знаем размерность $V^*$, осталось проверить линейную независимость. 

    Пусть некая линейная комбинация $\alpha_1 e^1 + \dots + \alpha_n e^n = 0$.
    Подставим в нее $e_j$ и получим 
    \begin{gather*}
        (\alpha_1 e^1 + \dots + \alpha_n e^n) (e_j) = 0
    \end{gather*}
    То есть получим, что все $\alpha_j = 0$. 
    Выходит они линейно независимы и их количество равно размерности. Значит это базис.
\end{proof}

\begin{conj}
    $(e^1, \dots, e^n)$ -- двойственный базис к $e_1, \dots, e_n$
\end{conj}