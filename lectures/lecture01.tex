\section{Лекция номер 1}

\subsection{Что-то про СЛУ}

$A: V \rightarrow W$

Было доказано: $\dim{\ker{A}} + \dim{\mathrm{Im}{V}} = \dim{V}$

Как это применить к изучению СЛУ?

Представим, что есть система из $m$ уравнений, $n$ неизвестных и она однородна: 

$AX = 0$, $A\in M(m, n, k)$

Пусть $U = \{ X | AX = 0 \} \subset K^n$ - пространство решений (подпространство пронстранства всех столбцов высоты $n$).

Требуется понять, чему равна размерность пространства решений. 

Для этого нужно интерпретировать $U$ как ядро некоторого линейного отображения. Это нетрудно, потому что $U$ состоит из тех столбцов, из которых получается 0 в результате некоторой операции. Остается операцию (умножение на матрицу $A$) интерпретировать как некоторое линейное отображение. 

Умножение на матрицу $A$ сопоставляет столбцу высоты $n$ столбец высоты $m$: 

$K^n \stackrel{\varphi}{\rightarrow} K^m$

$c \mapsto  Ac$

$\varphi \in \Hom(K^n, K^m)$

Остается применить равенство с размерностями.

Ядро умножения на матрицу $A$ – $U$.

$\dim{U} + \dim{\mathrm{Im}{\varphi}} = n$

Осталось понять, что представляет из себя размерность образа.

Рассмотрим:

$A \left(
  \begin{matrix}
  c_1 \\ 
  \vdots \\
  c_n
\end{matrix}
\right) = c_1A_1 + c_2A_2 + \cdots + c_nA_n$

$A = (A_1A_2\cdots A_n)$

Следовательно, образ $\varphi$ – множество всевозможных линейных комбинаций столбцов: 

$\Rightarrow \mathrm{{Im}{\varphi}} = \Lin(A_1, \cdots, A_n)$

$\Rightarrow \dim{\mathrm{{Im}{\varphi}}} =\rk A$

\follow 
$\dim{U} = n - \rk A$

% \chapter{Теория групп} !!!

\subsection{Группы и подгруппы, новые примеры}
\begin{itemize}

\item Аддитивные группы колец:

\begin{itemize}
\item $\mathbb{Z}$, $\mathbb{Q}$, $\mathbb{R}$, $\mathbb{C}$, $A[X]$, $K(X)$, $M(n, A)$
\item Кольца вычетов (группы по сложению) $\mathbb{Z}/m\mathbb{Z} = \mathbb{Z}/(m) = \mathbb{Z}/\stackrel{m}{\equiv}\ = \{ [0], [1], \cdots, [m-1] \}$
\end{itemize}

Введем сложение и умножение на классах вычетов: 
\begin{gather*}
  [a] + [b] = [a + b] \\
  [a]\cdot [b] = [a \cdot b]
\end{gather*}

Это определение нуждается в проверке корректности.
\begin{gather*}
\begin{cases}
    [a] = [a']  \\
    [b] = [b']
\end{cases}
\stackrel{?}{\Rightarrow}
\begin{cases}
  [a' + b'] = [a + b]  \\
  [a' \cdot b'] = [a \cdot b]
\end{cases}
\end{gather*}

\begin{gather*}
\begin{cases}
  a \stackrel{m}{\equiv} a' \\
  b \stackrel{m}{\equiv} b'
\end{cases}
\stackrel{?}{\Rightarrow}
\begin{cases}
  a + b \equiv a' + b' \\ 
  a \cdot b \equiv a' \cdot b'
\end{cases}
\end{gather*}

Из курса дискретной математики известно, что замена представителей классов на других представителей тех же классов не влияет на результат операций.

\begin{theorem-non}
  $(\mathbb{Z}/m \mathbb{Z}, +, \cdot)$ --- коммутативное ассоциативное кольцо с единицей.

  \begin{proof}
    Рутинная проверка, тривиально.
  \end{proof}
\end{theorem-non}

\notice
Можно обобщить построение кольца классов вычетов до любого коммутативного кольца. 

$R$ --- коммутативное кольцо, $f\in R$, $\stackrel{f}{\equiv}$ --- отношение сравнимости по модулю элемента $f$.

$R/(f) = R/\stackrel{f}{\equiv}$ --- факторкольцо. 

В частности можно самостоятельно обдумать такое: 

\begin{itemize}
  \item $\mathbb{R}[X]/(X^2 + 1)$ --- легко отождествляется с полем комплексных чисел. 
  \item $R$ --- коммутативное кольцо, $I\subset R$ --- идеал в нем. Тогда $a \stackrel{I}{\equiv} b \Leftrightarrow a - b \in I$
\end{itemize} 

\item Группы обратимых элементов кольца

Пусть $A$ --- ассоциативное кольцо с единицей. 

Тогда мы имеем дело с еще одной группой. Элементы $A$ не образуют группу по умножению, но если ограничиться обратимыми элементами ($A^*$), то они образуют группу по умножению. 

Будем занудами, проверим замкнутость: 
\begin{gather*}
  ab = 1 \\
  a'b' = 1 = b'a' \\
  \Rightarrow (aa')(b'b) = 1  
\end{gather*}

Теперь имеем такие примеры: 

$\mathbb(Z)^*$, $\mathbb{R}^*$, $\mathbb{Q}^*$, $\mathbb{C}^*$, $M(n, K)^* = GL(n, K)$ --- общая линейная группа степени $n$ над полем $K$.

В $GL(n, K)$ есть важная подгруппа (специальная линейная группа), состоит из матриц, у которых определитель равен единице. 
\begin{gather*}
  SL(n, k) = \{ A \in GL(n, K) | \det{A} = 1 \} \\
  SL(n, k) < GL(n, k)
\end{gather*}

Теперь в кольце классов вычетов можно рассмотреть группу обратимых элементов: 

\begin{gather*}
  (\mathbb{Z}/m \mathbb{Z})^* = \{ [a] | (a, m) = 1 \} \\
  |(\mathbb{Z}/m \mathbb{Z})^*| = \varphi(m)
\end{gather*}

$\varphi(m)$--- функция Эйлера, сопоставляет число количество натуральных, не превышающих его и взаимно простых с ним

\item 
$S_n = S(\{ 1, \cdots, n \})$

$A_n = \{ \sigma \in S_n | \sign{\sigma} = 1 \}$ --- группа четных перестановок

$A_n < S_n$, так как произведение четных перестановок четно, обратная к четной перестановке четна и тождественная перестановка четна. 

\item Группы симметрий геометрических фигур

$\Phi$ --- фигура

$\Sigma(\Phi)$ --- самосовмещения этой фигуры

$D_n$ --- группы симметрий правильного $n$-угольника

$R_k$ --- Повороты на углы кратные $n$: $k \cdot \frac{2\pi}{n}$

$S_k$ --- Оси симметрии: $k=0,1,\cdots, n-1$ (на каждую вершину одна ось)
% В планах нарисовать шестиугольник чтобы показать симметрии, но сейчас нет времени
$|D_n| = 2n$

Показано, что их хотя бы $2n$, но на самом деле их не больше. В это можно просто поверить, можно установить самостоятельно.

В $D_n$ есть циклическая абелева подгруппа порядка $n$, состоящая из поворотов.

\begin{tikzpicture}
  \draw[gray, thick] (-1,-1.539) -- (-1.618,0.363);
  \draw[gray, thick] (-1.618,0.363) -- (0,1.539);
  \draw[gray, thick] (0,1.539) -- (1.618,0.363);
  \draw[gray, thick] (1.618,0.363) -- (1,-1.539);
  \draw[gray, thick] (1,-1.539) -- (-1,-1.539);

  \draw[thick, densely dotted] (0, -1.839) -- (0, 1.839);

  \filldraw[black] (0,0) circle (2pt) node[anchor=west] {Ось симметрии};
  
\end{tikzpicture}

\item $\mathbb{T} < \mathbb{C}^*$

$\mathbb{T} = \{ z\ :\ |z| = 1 \}$

Подгруппа, так как произведение комплексных чисел с модулем $1$ будет иметь модуль $1$, все обратные тоже имеют модуль $1$.

$\mu_n$ (группа (по умножению) корней из единицы степени $n$) < $\mathbb{T} < \mathbb{C}^*$ 

Ранее отмечалось, что $\mu_n$ -- циклическая. Напомним, что $\zeta_n \cdot \zeta_1 = \zeta_{n+1}$

$\mu_n =\ \langle \zeta_1 \rangle\ =\ \langle\zeta_{n-1}\rangle$
\end{itemize} 

\subsection{Подгруппа, порожденная подмножеством}
Что такое циклическая подгруппа?

$G$ -- группа, $g\in G$, тогда

$\langle g \rangle\ :=\ \{\ g^n\ |\ n\in\ \mathbb{Z}\ \}$

\begin{theorem-non}
Это подгруппа.

\begin{proof}
Замкнутость: 

$g^m\cdot g^n = g^{m+n}$

Обратный: 

$(g^m)^{-1} = g^{-m}$

Нейтральный: 

$e = g^0$

\end{proof}
\end{theorem-non}

\begin{conj} Циклическая группа

$G$ --- циклическая, если $\exists\ g:\ G = \langle g \rangle$
\end{conj}

\begin{conj} Порядок элемента

  $\ord{g} = \min{ \{ n\in \mathbb{N} : g^n = e \} }$

  $\ord{g} = +\infty \Longleftrightarrow \nexists n\in \mathbb{N} : g^n = e$
\end{conj}

\begin{theorem-non}

  \begin{itemize}
  \item $|\langle g \rangle| = \ord{g}$
  \item $\ord{g} = n < +\infty \Longrightarrow \langle g \rangle = \{ g^0, g^1, \cdots, g^{n-1} \}$
  \end{itemize}

\begin{proof}
  \begin{itemize}
    \item $\ord{g} = +\infty$

    $i \neq j \Longrightarrow g^i \neq g^j$ (так как $g^{i-j} \neq e$, иначе противоречие с порядком)

    $\Longrightarrow |\langle g \rangle| = +\infty$
    
    \item $\ord{g} = n < +\infty$
    
    Начнем со второго утверждения. 
    
    $m = nq + r$, $0 \leq r \leq r - 1$
    
    $g^m = \equalto{(g^n)}{e}^q \cdot g^r = g^r$

    То есть любая степень $g$ это степень $g$ в нужном диапазоне ($[0; n-1]$)

    Таким образом, $|\langle g \rangle| \leq n$

    Осталось увидеть, что $0 \leq i < j \leq n - 1 \Longrightarrow g^i\neq g^j $

    Если $g^i = g^j$, то $g^{j-i} = e$ --- противоречие с определением порядка, так как $1 \leq j - i \leq n - 1$
  \end{itemize}
\end{proof}
\end{theorem-non}

\notice Циклическая группа единственна (с точностью до изоморфизма).

$\ord{g} = +\infty \Longrightarrow \langle g \rangle \cong \mathbb{Z}$

$\ord{g} = n \Longrightarrow \langle g \rangle \cong \mathbb{Z}/n \mathbb{Z}$

Строгое доказательство будет изложено в дальнейшем.

\begin{conj}
  Подгруппа, порожденная подмножеством.

  $\langle M \rangle = \{ g_1 \cdot \ldots \cdot g_s\ |\ s \geq 0,\ g_1, \ldots, g_s \in M \cup M^{-1} \}$

  $M^{-1} = \{ g^{-1}\ |\ g\in M \}$
  
\end{conj}

\notice $\langle \{g\} \rangle \ =\ \langle g \rangle$
\begin{theorem-non}
  $\langle M \rangle$ --- подгруппа
  \begin{proof}
    Нейтральный:

    $e\in \langle M \rangle$ хотя бы $s = 0$

    Замкнутость: 

    $(g_1 \cdots g_s)(g'_1 \cdots g'_t) = g_1 \cdots g_s g'_1 \cdots g'_t \in \langle M \rangle$

    Обратный:
    $(g_1 \cdots g_s)^{-1} = g_s^{-1} \cdots g_1^{-1} \in \langle M \rangle$  
  \end{proof}
  
\end{theorem-non}
\notice $H_i < G, i\in I \Longrightarrow \bigcap\limits_{i\in I} H_i < G$

Это тривиально: произведение любых двух элементов, лежащих в пересечении, лежит в пересечении. Обратный к любому тоже лежит в каждой из подгрупп, значит, лежит в пересечении. Единица Тоже.

\begin{theorem-non}
  $\langle M \rangle = \bigcap\limits_{\substack{H < G, \\ M \subset H}} H $

\begin{proof}
  $\mbox{\Large$\subset$}$
  \begin{gather*}
    \begin{cases}
      H < G \\
      M \subset H
    \end{cases}
    \stackrel{?}{\Longrightarrow} \langle M \rangle \subset H \\
    M \subset H \Longrightarrow M^{-1} \subset H \\
    \Longrightarrow g_1 \cdots g_s \in H\ (g_i\in M \cap M^{-1}) \\
    \Longrightarrow \langle M \rangle \subset H
  \end{gather*}

  $\mbox{\Large$\supset$}$
  \begin{gather*}
    \begin{cases}
      \langle M \rangle < G \\
      M \subset \langle M \rangle
    \end{cases}
    \Longrightarrow \langle M \rangle \supset \bigcap\limits_{\substack{H < G, \\ M \subset H}} H
  \end{gather*}
  
\end{proof}
\end{theorem-non}

\notice Обозначение

$\langle \{ g_1, \cdots, g_t \} \rangle = \langle g_1, \cdots, g_t \rangle$

\notice Группы перестановок и симметрии порождаются двумя элементами

$S_n = \langle (1 2 \cdots n), (12) \rangle$

$D_n = \langle R_1, R_0 \rangle $

\subsection{Гомоморфизмы}

\begin{conj}
  Отображение $G \stackrel{\varphi}{\longrightarrow} G'$ $G, G'$ --- группы

  называется гомоморфизмом, если 

  $\forall g_1, g_2 \in G$ $\varphi(g_1 \cdot g_2) = \varphi(g_1) \cdot \varphi(g_2)$
\end{conj}

Примеры:
\begin{itemize}
  \item $\mathbb{R} \longrightarrow \mathbb{C}$
  
  $\alpha \mapsto \cos{\alpha} + i\sin{\alpha}$

  Согласитесь, что если перемножить два числа с аргументами $\alpha$ и $\beta$, то получится число с аргументом $\alpha + \beta$.

  $(\cos{a} + i\sin{a})(\cos{b} + i\sin{b}) = \cos{a}\cdot\cos{b} - \sin{a}\cdot\sin{b} + i(\sin{a}\cdot\cos{b} + \sin{b}\cdot\cos{a}) = \cos(a + b) + i\sin(a + b)$

  \item $G$ --- абелева группа, $m\in \mathbb{Z}$
  
  $G \longrightarrow G$

  $g \mapsto g^m$

  $(g_1 \cdot g_2)^m = g_1^m \cdot g_2^m$
  
  Абелевость нужна, чтобы переставить элементы местами в 
  
  $(g_1 \cdot g_2) \cdot (g_1 \cdot g_2) \cdot \cdots (g_1 \cdot g_2)$
  % в планах красиво подчеркнуть, что тут m скобок

  \item  $S_n \longrightarrow \mathbb{Z}^*$

  $\sigma \mapsto \sign{\sigma}$

  Знак произведения перестановок равен произведению знаков.
\end{itemize}

\begin{conj}
  $\mathrm{Im}{\varphi} = \{ \varphi(g)\ |\ g\in G \}$

  $\Ker{\varphi} = \{ g \in G\ |\ \varphi(g) = e \}$
\end{conj}

\begin{theorem-non}
  Образ нейтрального элемента это нейтральный элемент:

  \begin{proof}
    $\varphi(e_G)\varphi(e_G) = e_{G'}$

    $\varphi(e_G) = e_{G'}$
  \end{proof}
\end{theorem-non}

\begin{theorem-non}
  $\mathrm{Im}{\varphi} < G$, $\Ker{\varphi} < G$
  \begin{proof}
    Наличие обратного:

    $\varphi(\equalto{g \cdot g^{-1}}{\varphi(g)\cdot \varphi(g^{-1})}) = \varphi(e_G) = e_{G'}$
    
    $\Longrightarrow \varphi(g)^{-1} = \varphi(g^{-1}) \in \mathrm{Im}{G}$

    Наличие нейтрального:

    $e_{G'} = \varphi(e_G) \in \mathrm{Im}{\varphi}$

    Замкнутость очевидна из свойств гомоморфизма. 

    Таким образом, $\mathrm{Im}{\varphi} < G'$
  \end{proof}
\end{theorem-non}
