\section{Лекция номер 9}

\subsection{Про p-примарные подпространства}
\begin{theorem}
    Пусть $\A \in \End V$ 

    $\mu_{\A} = p^{n_1}_1 \cdot \dots \cdot p^{n_s}_s$, \quad $p_i$ --- различные унитарные неприводимые многочлены.

    Тогда
    \[ V = \bigotimes_{i = 1}^{s} W_{p_i} \]

    Напомним, что $W_{p_i} = \{ v \in V \, | \, \mu_{\A, v} = p^t_i \, t \in \N \}$

    \begin{proof}
        $ $ \newline
        \quad Для доказательства применим несколько раз предыдущую лемму:
        \[ V = \bigotimes_{i = 1}^{s} V_i, \quad V_i \text{--- $\A$---инвариантно } \]
        
        Откуда получим, что $\mu_{\A |_{V_i}} | p^{n_i}_i$

        Это ещё не всё, мы лишь знаем, что каждый $v \in V_i$ является $p_i$---примарным, т.е. $V_i \subset W_{p_i}$
        
        \quad Теперь проверим, что $W_{p_i} \subset V_i$ 

        Возьмём $w \in W_{p_s}$ и разложим в сумму векторов из $V = \bigotimes_{i = 1}^{s} W_{p_i}$:
        \begin{gather*}
            w = v_1 + v_2 + \dots + v_s, \quad v_i \in V_i \\
            \underbrace{w - v_s}_{\text{$ \in W_{p_s} $}}  = v_1 + v_2 + \dots + v_{s - 1} \\
        \end{gather*}

        Получили, что $\mu_{\A, w - v_s} = p^N_s$ для некоторого $N$

        С другой же стороны, $\mu_{\A, v_i} = p^{n_j}_j$

        И тогда $(\mu_{\A, v_1} \cdot \dots \cdot \mu_{\A, v_{s - 1}})(\A)(v_j) = 0$ для $j = 0, \dots, s - 1$

        $\Longrightarrow (\mu_{\A, v_1} \cdot \dots \cdot \mu_{\A, v_{s - 1}})(\A)(\underbrace{v_1 + \dots + v_{s - 1}}_{\text{$ w - v_s $}} ) = 0$

        Получается, что $\underbrace{\mu_{\A, v_1} \cdot \dots \cdot \mu_{\A, v_{s - 1}}}_{\text{$ p^{n_1}_1 \cdot \dots \cdot p^{n_{s - 1}}_{s - 1} $}} \vdots p^{n_s}_s$ (?! противоречие, т.к. каждый $p^{n_i}_i$ неприводим)

        $\Longrightarrow n_s = 0 \Longrightarrow w - v_s = 0$

        Получается, что $W_{p_s} = V_s$
    \end{proof}
\end{theorem}

\begin{theorem}
    Пусть $V$ это $p$ --- примарное пространство, тогда $\exists v_1, \dots, v_t \in V$, т.ч.

    \[ V = \bigotimes_{i = 1}^{t} L_{v_i} \]

    \begin{proof}
        Оставлена лектором без доказательства.
    \end{proof}
\end{theorem}

Что это всё будет означать на матричном языке?

\vspace*{3mm}

Обозначим $[\A |_{L_{v_i}}]_{E_i}$ за $C(\mu_{\A, v_i})$

\[ E_i = (v_i, \A v_i, \dots, \A^{d - 1} v_i), \quad d = \deg \mu_{\A, v_i} \]

Напомним, как выглядит эта матрица:
\[
    \left(\begin{array}{ccccc}
        0 & 0 & \dots & 0 & -\alpha_0 \\ 
        1 & 0 & \dots & 0 & -\alpha_1 \\ 
        0 & 1 & \dots & 0 & -\alpha_2 \\ 
        \vdots & \vdots & \vdots & \vdots & \vdots \\ 
        0 & 0 & 0 & 1 & -\alpha_{d - 1}
    \end{array}\right)    
\]

Так вот, предыдущая теорема говорит нам о том, что матрица такого оператора будет блочно-диагональной:
\[
  [ \A ]_{E} = \left(\begin{array}{ccc}
  C(\mu_{\A, v_1}) &  & 0 \\ 
   & \ddots &  \\ 
  0 &  & C(\mu_{\A, v_t})
  \end{array}\right)  
\]

Где $E = E_1\dots E_t$

\begin{follow}
    Пусть $\chi_{\A} = \pm p^{m_1}_1 \cdot \dots \cdot p^{m_s}_s$, где $p_i$ --- различные неприводимые

    Тогда 
        \[ \mu_{\A} = p^{n_1}_1 \cdot \dots \cdot p^{n_s}_s, \quad 1 \leqslant n_i \leqslant m_i \]
    
    \begin{proof}
    $ $ \newline
        Осталось на упражнение.
    \end{proof}
\end{follow}

\vspace*{5mm}

Отдельно рассмотрим случай, когда неприводимые множители линейны, т.е. когда характеристический многочлен раскладывается на линейные множители.
Изучение этого не является экзотикой, поскольку в $\C$ это любой многочлен, а в $\R$ таких много.

\subsection{Жорданова нормальная форма}
%таймкод 12:23 (https://www.youtube.com/watch?v=lhO-ihwtCu0&list=PL-XfgOapHEQcpZjEQnZPZ7zGBBStGCMxx&index=4&t=1706s)

\subsection{Билинейные формы и евклидовы пространства.}

\begin{conj}(Билинейная форма)
Пусть $V$~--- ЛП $K$

Билинейная форма на $V$ это отображение $\B: V \times V \to K$, линейное по двум аргументам:
\begin{itemize}
    \item $\B(\alpha_1 v_1 + \alpha_2 v_2, w) = \alpha_1 \B(v_1, w) + \alpha_2\B(v_2, w)$
    \item Для второго аргумента аналогично.
\end{itemize}
\end{conj}

\notice Функционал~--- линейное отображение в поле $K$.

\underline{Примеры:}
\begin{enumerate}
    \item Стандартное скалярное произведение
    $V = K^n$

    $\B(
       \left(\begin{array}{c}
       \alpha_1 \\ 
       \vdots \\ 
       \alpha_n
       \end{array}\right),
       \left(\begin{array}{c}
       \beta_1 \\ 
       \vdots \\
       \beta_n
       \end{array}\right)
    ) = \alpha_1 \beta_1 + \dots + \alpha_n \beta_n$

    \item $V = C[0, 1]$~--- непрерывные функции на отрезке $[0, 1]$ \\
    $\B(f, g) = \int_{0}^{1} fg$ (интеграл линеен)

    \item Определитель:

    $V = K^2$ \\
    $\B(
        \left(\begin{array}{c}
        \alpha_1 \\ 
        \alpha_2
        \end{array}\right),
        \left(\begin{array}{c}
        \beta_1 \\ 
        \beta_2
        \end{array}\right)
    ) = \alpha_1 \beta_2 - \alpha_2 \beta_1$
\end{enumerate}


$V$~--- конечномерное, $E = (e_1, \dots, e_n)$~--- базис.
$\B(\alpha_1 e_1 + \dots + \alpha_n e_n, \beta_1 e_1 + \dots + \beta_n e_n) = 
\sum_{i=1}^{n} \sum_{j = 1}^{n} \alpha_i \beta_j \B(e_i, e_j)$

\begin{conj}(Матрица Грама)
$[\B]_E = \left(\begin{array}{ccc}
\B(e_1, e_1) & \dots & \B(e_1, e_n) \\ 
\dots & \dots & \dots \\ 
\B(e_n, e_1) & \dots & \B(e_n, e_n)
\end{array}\right)$

$[\B]_E$~---Матрица Грама (далее МГ) $\B$ в $E$
\end{conj}

Пусть $X = \left(\begin{array}{c}
\alpha_1 \\ 
\vdots \\ 
\alpha_n
\end{array}\right)$,
а $Y = \left(\begin{array}{c}
\beta_1 \\ 
\vdots \\ 
\beta_n
\end{array}\right)$

Тогда $\B(EX, EY) = X^T \cdot B \cdot Y$ ($B$ --- МГ $\B$ в базисе $E$)

Легко видеть, что
$BY = \left(\begin{array}{c}
    \B(e_1, EY) \\ 
    \dots \\ 
    \B(e_n, EY)
\end{array}\right)$

%TODO("Затехать примеры матриц Грама")

\begin{theorem}(Матрица Грама при замене базиса)
    Пусть $E' = EC, \quad C \in GL(n, K)$

    $X = CX'$

    $\B(ECX', ECY')= \B(EX, EY) = X^TBY = X$
\end{theorem}

\begin{conj}(Ранг билинейной формы)
    Пусть $\B$~--- билинейная форма на конечномерном $V$.
    Рангом билинейной формы $\rk{\B}$ является $\rk{[\B]_E}$ для произвольного базиса $E$.
\end{conj}

\notice Это инвариант $\B$
Действительно, $\rk{C^T \cdot B \cdot C} = \rk{B}$ для $C^T, C \in GL(n, K), \, B$~--- МГ $\B$

\begin{conj}(Симметрическая билинейная форма)
    Если $\forall v, w \in V: \, \B(v, w) = \B(w, v)$, то 
    $\B$~--- симмитрическая билинейная форма.
\end{conj}

\begin{conj}(Кососимметрическая билинейная форма)
    Если $\forall v, w \in V: \, \B(v, w) = -\B(w, v)$, то 
    $\B$~--- кососимметрическая билинейная форма.
\end{conj}

\begin{conj}(Квадратичная форма)
    Пусть $V$~--- ЛП / $K, \, char K \neq 2$
    Отображение $q: V \to K$ называется квадратичной формой,
    
    если существует симмитричая билинейная форма $\B$ на $V$, т.ч.:
    $$
        \forall v \in V: q(v) = \B(v, v)
    $$
\end{conj}

Пусть $E = (e_1, \dots, e_n)$~--- базис.

Можно заметить, что $q(x_1e_1, \dots, x_ne_n) = X^T B X$~--- однородный многочлен степени $2$

$B = \left(\begin{array}{cc}
b_11 & b_12 \\ 
b_12 & b_22
\end{array}\right)$

Тогда $X^TBX = b_{11}x^2_1 + 2b_{12}x_1x_2 + b_{22}x^2_2$

Билинейную форму можно восстановить по квадратичной (помним, что $char K \neq 2$):
$$
    \B(v, w) = \frac{1}{2}(
        \underbrace{q(v + w)}_{\text{$\B(v + w, v + w)$}} - q(v) - q(w)
        )
$$
$\B$ называют поляризацией $q$