\section{Поле разложения многочлена. Примеры. Существование}
\begin{conj}
    Пусть $K$ -- поле, $f \in K[x]$ -- некоторый многочлен, $L/K$ -- расширение. Тогда $L$ называется \textbf{полем разложения} $f$, если:
    \begin{enumerate}
        \item $f$ там раскладывается на линейные множетели. $f = a_0 \cdot (X - x_1) \dots (X - x_n)$ в $L[X]$.
        \item $L$ получается из $K$ присоединением всех этих корней. $L = K(x_1, \dots, x_n)$. То есть $L$ является наименьшим полем с таким свойством. 
    \end{enumerate}
\end{conj}
Примеры:
\begin{enumerate}
    \item $\Q\left(\sqrt 2\right)$~--- поле разложения $X^2-2$ над $\Q$
    \item $\Q\left(\sqrt[3]{2}\right)$~--- не поле разложения $X^3-2$ над $\Q$ (содержит только один корень)
    \item $\Q\left(\sqrt[3]{2}, \omega\right)$ (где $\omega$~--- какой-то из комплексных корней $X^3-2$)~--- поле разложения $X^3-2$ над $\Q$
    \item $\C$~--- не поле разложения $X^3-2$ над $\Q$ (все корни содержит, но не минимально).
\end{enumerate}

\begin{theorem}
    Пусть $K$ -- поле, $f$ -- его многочлен. Тогда у $f$ существует поле разложения над $K$. 
\end{theorem}
\begin{proof}
    Индукция по степени $f$. 

    \quad \underline{База $f = 1$:} Тогда у него есть единственный корень и он лежит в $K$, то есть $K$ -- само поле разложения. 

    \quad \underline{Переход:} Пусть $f_0$ -- любой неприводимый делитель $f$.
    Возьмем $L := K[X] / (f_0)$, тогда, как мы выясняли ранее, $L=K(x_1)$, где $x_1$~--- некоторый корень $f_0$ и, соответственно, $f$.
    Тогда по теореме Безу в $L$ многочлен $f$ раскладывается как $f=(X-x_1)g$, где $\deg g = \deg f - 1$.
    Тогда по предположению индукции существует поле разложения $g$ в $L$, равное $M=L(x_2, x_3, \ldots, x_n)$, где $x_2,x_3,\ldots, x_n$~--- оставшиеся корни $f$.
    Но тогда $M=L(x_2, x_3, \ldots, x_n) = (K(x_1))(x_2, x_3, \ldots, x_n) = K(x_1, x_2, \ldots, x_n) \Rightarrow$ $M$ является полем разложения $f$ над $K$.
\end{proof}
\notice На самом деле поле разложения единственно с точностью до изоморфизма, но доказывать мы это не будем.
