\section{Полуторалинейные формы и унитарные пространства}
Сейчас мы начинаем обиширно работать с пространствами над полем комплексных чисел, так что 
давайте перенесем ряд понятий, которые мы вводили в вещественном случае на комплексный. 

Унитарное пространство -- комплексный аналог понятия евклидова пространства. Также может быть названо 
Эрмитовым пространством. Пусть $V$ -- линейное пространство над $\C$. В качестве скалярного произведения выступает так 
называемая полуторалинейная форма. 

\begin{conj}
    Полутролинейная форма на $V$ -- это отображение: 
    \begin{gather*}
        \B: V \times V \longrightarrow \C
    \end{gather*}
    Такое, что выполняются следующие свойства: 
    \begin{enumerate}
        \item $\B$ линейно по 1 аргументу: $\B(\alpha_1 v_1 + \alpha_2 v_2, w) = \alpha_2 \B(v_2, w)$
        \item $\B(v, \alpha_1 w_1 + \alpha_2 w_2) = \bar{\alpha_1} \B(v, w_1) + \bar{\alpha_2}\B(v, w_2)$
    \end{enumerate}
    \underline{Примеры:}
    \begin{enumerate}
        \item Стандартное скалярное произведение в $V = \C^n$:
        
        \begin{gather*}
            \B\left(\left(\begin{array}{c}
            \alpha_1 \\ 
            \vdots \\ 
            \alpha_n
            \end{array}\right), 
            \left(\begin{array}{c}
            \beta_1 \\ 
            \vdots \\ 
            \beta_n
            \end{array}\right)\right) 
            = \alpha_1 \bar{\beta_1} + \dots + \alpha_n \bar{\beta_n}
        \end{gather*}

        Преимущество относительно обычного скалярного произведения заключается в том, что если мы возьмем любой такой столбец и
        скалярно умножим на себя, то получим неотрицательно вещественное число, а если мы будем так делать без комплексного 
        сопряжения, то может получиться все что угодно.

        \item $V = C_{\C} [0, 1]$ -- непрерывные комплексно-значные функции на отрезке $[0, 1]$.
        \begin{gather*}
            \B(f, g) = \int\limits_0^1 f \cdot \bar{g}
        \end{gather*}
    \end{enumerate}    
\end{conj}

Заметим, что, как и обычно, можно определить матрицу Грама. 
Пусть $E$ -- базис $V$. $e_1, \dots, e_n$ -- вектора $E$. Тогда есть матрица Грама $[\B]_E = (\B(e_i, e_j))$. 
Если у нас есть произвольные вектора $v = E \cdot b$ и $w = E \cdot c$. Тогда: 
\begin{gather*}
    \B(v, w) = b^T \cdot [\B]_E \cdot \bar{c}
\end{gather*}
\underline{Fan fact}:
Если базис $E' = EC$, то $[\B]_{E'} = C^T \cdot [\B]_{E} \cdot \bar{C}$

Введем понятие эрмитовой формы, которая является аналогом симметрической билинейной формы в комплексном случае. 
\begin{conj}
    Эрмитовой формой на $V$ называется полутролинейная форма $\B$, такая что:  
    \begin{gather*}
        \forall u, v: \B(v, u) = \overline{\B(u, v)}
    \end{gather*}
\end{conj}
Для таких форм справедлива теорема Лагранжа. Доказательство абсолютно аналогично вещественному случаю. 
\begin{theorem}
    Пусть $V$ конечномерно, $\B$ --  эрмитова форма на $V$. Тогда $\exists$ базис $E$, такой, что матрица Грама $[\B]_E$ диагональна.
\end{theorem}

\notice $\B$ эрмитова $\Longleftrightarrow [\B]^*_E  = [\B]_E$, где оператор ``$*$'' -- это: 
\begin{conj}
    Пусть $A \in M(m, n, \C)$. Тогда $A^* := \overline{(A^T)}$ --- матрица, сопряженная к $A$.
\end{conj}
В частности, если $[\B]_E$ диагональна, то она вещественна.

Диагонализировать форму можно разумеется по разному, но, опять таки, для эрмитовой формы справедлив закон инерции:
Число положительных и число отрицательных чисел в диагональной матрице $[\B]_E$ -- инварианты $\B$.

\begin{conj}
    Скалярным произведением на $V$ называется положительно определённая эрмитова форма на V, то есть такая, что
    \begin{enumerate}
        \item $\forall v \in V: \B(v, v) \geqslant 0$
        \item $\B(v, v) = 0 \Longleftrightarrow v = 0$
    \end{enumerate}
\end{conj}

Иначе говоря, число положительных чисел на диагонали должно совпадать с размерностью пространства. 

\begin{conj}
    Унитарным пространством называется линейное пространство над $\C$ с фиксированным скалярным произведением.
\end{conj}

\begin{conj}
    Базис $e_1, \dots, e_n$ унитарного пространства V называется ортонормированным,
    если $\forall i, j: \, (e_i, e_j) = \delta_{ij}$. Иными словами, если матрица Грама этого базиса представляет собой единичную матрицу: 
    $\Gamma_E = E_n$. 
\end{conj}

Пусть $E' = EC$, где $C \in GL(n, \C)$, мы знаем, что $\Gamma_{E'} = C^T \cdot \Gamma_E \cdot \bar{C}$. 
\begin{align*}
    E' \text{ ортонормированный } &\Longleftrightarrow \Gamma_{E'} = E_n \\
    &\Longleftrightarrow C^T \cdot \bar{C} = E_n \\
    &\Longleftrightarrow C^* \cdot C = E_n
\end{align*}

\begin{conj}
    Матрица $C \in GL(n, \C)$ называется унитарной, если $C^* \cdot C = E_n$
\end{conj}

\begin{conj}
    Множество унитарных матриц: $U(n) := \{ C \in GL(n, \C) \, | \, C$ --- унитарна\}
\end{conj}

\begin{theorem}
    $U(n) < GL(n, \C)$ -- унитарная группа степени $n$.
\end{theorem}

\begin{theorem} (Неравенство КБШ). Как и обычно, $\norm{v} = \sqrt{(v, v)}$. 
    \begin{gather*}
        \forall u,v \in V: \abs{(u, v)}^2 \leqslant \norm{u} \cdot \norm{v}
    \end{gather*}
\end{theorem}
Пруфов теорем не будет, автор принял Линал. 