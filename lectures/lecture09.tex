\section{Лекция номер 9}

\subsection*{Про p-примарные подпространства}
Первая половина лекции не готова. Вторая половина про Билинейные формы готова.
% от 0:00 до 28:26

\subsection*{Билинейные формы и евклидовы пространства.}

\begin{conj}(Билинейная форма)
Пусть $V$~--- ЛП $K$

Билинейная форма на $V$ это отображение $\B: V \times V \to K$, линейное по двум аргументам:
\begin{itemize}
    \item $\B(\alpha_1 v_1 + \alpha_2 v_2, w) = \alpha_1 \B(v_1, w) + \alpha_2\B(v_2, w)$
    \item Для второго аргумента аналогично.
\end{itemize}
\end{conj}

\notice Функционал~--- линейное отображение в поле $K$.

\underline{Примеры:}
\begin{enumerate}
    \item Стандартное скалярное произведение
    $V = K^n$

    $\B(
       \left(\begin{array}{c}
       \alpha_1 \\ 
       \vdots \\ 
       \alpha_n
       \end{array}\right),
       \left(\begin{array}{c}
       \beta_1 \\ 
       \vdots \\
       \beta_n
       \end{array}\right)
    ) = \alpha_1 \beta_1 + \dots + \alpha_n \beta_n$

    \item $V = C[0, 1]$~--- непрерывные функции на отрезке $[0, 1]$ \\
    $\B(f, g) = \int_{0}^{1} fg$ (интеграл линеен)

    \item Определитель:

    $V = K^2$ \\
    $\B(
        \left(\begin{array}{c}
        \alpha_1 \\ 
        \alpha_2
        \end{array}\right),
        \left(\begin{array}{c}
        \beta_1 \\ 
        \beta_2
        \end{array}\right)
    ) = \alpha_1 \beta_2 - \alpha_2 \beta_1$
\end{enumerate}


$V$~--- конечномерное, $E = (e_1, \dots, e_n)$~--- базис.
$\B(\alpha_1 e_1 + \dots + \alpha_n e_n, \beta_1 e_1 + \dots + \beta_n e_n) = 
\sum_{i=1}^{n} \sum_{j = 1}^{n} \alpha_i \beta_j \B(e_i, e_j)$

\begin{conj}(Матрица Грама)
$[\B]_E = \left(\begin{array}{ccc}
\B(e_1, e_1) & \dots & \B(e_1, e_n) \\ 
\dots & \dots & \dots \\ 
\B(e_n, e_1) & \dots & \B(e_n, e_n)
\end{array}\right)$

$[\B]_E$~---Матрица Грама (далее МГ) $\B$ в $E$
\end{conj}

Пусть $X = \left(\begin{array}{c}
\alpha_1 \\ 
\vdots \\ 
\alpha_n
\end{array}\right)$,
а $Y = \left(\begin{array}{c}
\beta_1 \\ 
\vdots \\ 
\beta_n
\end{array}\right)$

Тогда $\B(EX, EY) = X^T \cdot B \cdot Y$ ($B$ --- МГ $\B$ в базисе $E$)

Легко видеть, что
$BY = \left(\begin{array}{c}
    \B(e_1, EY) \\ 
    \dots \\ 
    \B(e_n, EY)
\end{array}\right)$

%TODO("Затехать примеры матриц Грама")

\begin{theorem}(Матрица Грама при замене базиса)
    Пусть $E' = EC, \quad C \in GL(n, K)$

    $X = CX'$

    $\B(ECX', ECY')= \B(EX, EY) = X^TBY = X$
\end{theorem}

\begin{conj}(Ранг билинейной формы)
    Пусть $\B$~--- билинейная форма на конечномерном $V$.
    Рангом билинейной формы $\rk{\B}$ является $\rk{[\B]_E}$ для произвольного базиса $E$.
\end{conj}

\notice Это инвариант $\B$
Действительно, $\rk{C^T \cdot B \cdot C} = \rk{B}$ для $C^T, C \in GL(n, K), \, B$~--- МГ $\B$

\begin{conj}(Симметрическая билинейная форма)
    Если $\forall v, w \in V: \, \B(v, w) = \B(w, v)$, то 
    $\B$~--- симмитрическая билинейная форма.
\end{conj}

\begin{conj}(Кососимметрическая билинейная форма)
    Если $\forall v, w \in V: \, \B(v, w) = -\B(w, v)$, то 
    $\B$~--- кососимметрическая билинейная форма.
\end{conj}

\begin{conj}(Квадратичная форма)
    Пусть $V$~--- ЛП / $K, \, char K \neq 2$
    Отображение $q: V \to K$ называется квадратичной формой,
    
    если существует симмитричая билинейная форма $\B$ на $V$, т.ч.:
    $$
        \forall v \in V: q(v) = \B(v, v)
    $$
\end{conj}

Пусть $E = (e_1, \dots, e_n)$~--- базис.

Можно заметить, что $q(x_1e_1, \dots, x_ne_n) = X^T B X$~--- однородный многочлен степени $2$

$B = \left(\begin{array}{cc}
b_11 & b_12 \\ 
b_12 & b_22
\end{array}\right)$

Тогда $X^TBX = b_{11}x^2_1 + 2b_{12}x_1x_2 + b_{22}x^2_2$

Билинейную форму можно восстановить по квадратичной (помним, что $char K \neq 2$):
$$
    \B(v, w) = \frac{1}{2}(
        \underbrace{q(v + w)}_{\text{$\B(v + w, v + w)$}} - q(v) - q(w)
        )
$$
$\B$ называют поляризацией $q$