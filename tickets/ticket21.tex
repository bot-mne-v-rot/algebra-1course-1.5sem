\section{Билинейные формы. Изменение матрицы билинейной формы при линейном преобразовании}
\begin{conj}
    Билинейная форма на $V$, где $V$ это линейное пространство над полем $K$, это отображение $\B: V \times V \to K$, линейное по двум аргументам:
    \begin{itemize}
        \item $\B(\alpha_1 v_1 + \alpha_2 v_2, w) = \alpha_1 \B(v_1, w) + \alpha_2\B(v_2, w)$
        \item $\B(v, \alpha_1 w_1 + \alpha_2 w_2) = \alpha_1 \B(v, w_1) + \alpha_2\B(v, w_2)$
    \end{itemize}
    \end{conj}
    
    \vspace*{5mm}
    То есть мы фиксируем какой-то из аргументов и получаем линейный функционал на $V$.
    Линейный функционал -- линейное отображение в поле $K$.
    
    \underline{Примеры:}
    \begin{enumerate}
        \item Стандартное скалярное произведение на $V = K^n$:
    
        \[ \B \left(
           \left(\begin{array}{c}
           \alpha_1 \\ 
           \vdots \\ 
           \alpha_n
           \end{array}\right),
           \left(\begin{array}{c}
           \beta_1 \\ 
           \vdots \\
           \beta_n
           \end{array}\right) 
        \right) = \alpha_1 \beta_1 + \dots + \alpha_n \beta_n \]
    
        \item Пусть $V = C[0, 1]$ -- непрерывные функции на отрезке $[0, 1]$. 
        
        Тогда $\B(f, g) = \int_{0}^{1} fg$ -- билинейная форма, так как интеграл линеен.
    
        \item Пусть $V = K^2$. \\
        Тогда следующее отображение является билейной формой: \[ \B \left(
            \left(\begin{array}{c}
            \alpha_1 \\ 
            \alpha_2
            \end{array}\right),
            \left(\begin{array}{c}
            \beta_1 \\ 
            \beta_2
            \end{array}\right)
        \right) = \alpha_1 \beta_2 - \alpha_2 \beta_1 \]
        По смыслу это определитель составленной из этих столбцов матрицы.
    \end{enumerate}
    
    \vspace*{5mm}
    
    Пусть $V$ -- конечномерное, $E = (e_1, \dots, e_n)$ -- его базис.
    Тогда билинейную форму, заданную на произвольных векторах, можно выразить через базисные: 
    \[ \B(\alpha_1 e_1 + \dots + \alpha_n e_n, \beta_1 e_1 + \dots + \beta_n e_n) = 
    \sum_{i=1}^{n} \sum_{j = 1}^{n} \alpha_i \beta_j \B(e_i, e_j) \]
    
    Отсюда логичным оборазом вытекает следующее определение.
    
    \begin{conj} Матрица Грама для билиненой формы $\B$ в базисе $E$ это матрица вида
    \[ [\B]_E = \left(\begin{array}{ccc}
    \B(e_1, e_1) & \dots & \B(e_1, e_n) \\ 
    \dots & \ddots & \dots \\ 
    \B(e_n, e_1) & \dots & \B(e_n, e_n)
    \end{array}\right) \] 
    \end{conj}
    
    \vspace*{5mm}
    
    Поймем, как с помощью матрицы Грама быстро посчитать $\B(\alpha_1 e_1 + \dots + \alpha_n e_n, \beta_1 e_1 + \dots + \beta_n e_n)$. 
    Введем обозначение для столбцов координат: $X = \begin{pmatrix}
        \alpha_1 \\
        \vdots \\
        \alpha_n
    \end{pmatrix}, Y = \begin{pmatrix}
        \beta_1 \\
        \vdots \\
        \beta_n
    \end{pmatrix}$. 
    Тогда верна следующая формула: \begin{gather*}
        B(EX, EY) = X^TBY \\
    \end{gather*}
    Она напрямую следует из матричных соображений: \begin{gather*}
        X^TB = (\alpha_1 B_{11} + \dots + \alpha_n B_{n1}, \dots, \alpha_1 B_{1n} + \dots + \alpha_n B_{nn}) \\
        X^TBY = \alpha_1 B_{11}\beta_1 + \dots + \alpha_n B_{n1}\beta_1 + \dots + \alpha_1 B_{1n}\beta_n + \dots + \alpha_n B_{nn}\beta_n 
    \end{gather*}
    
    \vspace*{5mm}
    
    Чему равна матрица Грама в наших примерах?
    Если в первом примере взять стандратный базис, то матрица Грама очевидно будет единичной.
    Во втором примере у нас пространство бесконечномерное, поэтому никакой матрицы Грама нет. 
    Если в третьем примере взять базис $E = \left(\begin{pmatrix}
        1 \\ 0
    \end{pmatrix}, \begin{pmatrix}
        0 \\ 1
    \end{pmatrix} \right)$, то матрица Грама будет равна $[\B]_E = \begin{pmatrix}
        0 & 1 \\
        -1 & 0
    \end{pmatrix}$.
    
    \vspace*{5mm}
    
    Давайте посмотрим на то, как будет меняться матрица Грама при замене базиса.
    
    \begin{theorem}(Матрица Грама при замене базиса)
        Пусть $E' = EC$, где $C \in GL(n, K)$.
        Тогда \[ [\B]_{E'} = C^T[\B]_EC \]
    \end{theorem}
    \begin{proof}
        Знаем, что $E' = EC$ и $X = CX'$, где $X$ -- координаты вектора.
        С одной стороны: \[ \B(EX, EY) = X^T[\B]_EY = (X')^TC^T[\B]_ECY' \]
        \quad С другой стороны: \[ \B(EX, EY) = \B(ECX', ECY') = B(E'X', E'Y') = (X')^T[\B]_{E'}Y' \]
        \quad Получаем, что для любых векторов $X'$ и $Y'$ выполняется равенство $(X')^TC^T[\B]_ECY' = (X')^T[\B]_{E'}Y'$.
        Отсюда уже следует, что $[\B]_{E'} = C^T[\B]_EC$.
        Почему?
        Возьмем за $X'$ $i$-тый базисный вектор, а за $Y'$ -- $j$-тый.
        Тогда умножение на $(X')^T$ и $Y'$ будет соответствовать вырезанию из матрицы элемента на позиции $(j, i)$.
        Выбирая различные $i$ и $j$ мы добьемся того, чтобы матрицы совпали.
    \end{proof}
    
    \vspace*{5mm}
    
    \begin{conj}
        Рангом билинейной формы на конечномерном пространстве $V$ является $\rk{[\B]_E}$ для произвольного базиса $E$.
        Обозначается как $\rk{\B}$.
    \end{conj}
    
    \notice Это инвариант $\B$, так как при умножении на обратимую ранг не меняется.
    