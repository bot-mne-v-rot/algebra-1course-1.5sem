\section{Связь между алгебраической и геометрической кратностью собственного значения}
\begin{follow}
    Геометричекая кратность собственного значения не превосходит его алгебраической кратности.
\end{follow}
\begin{proof}
    В качестве $W$ возьмем $\Ker(\A - \lambda\mathcal{E}_V)$.
    Очевидно, что собственное подпространство является инвариантным (мы просто умножаем каждый вектор на скаляр, значит остаемся в том же подпространстве).
    Получаем, что $\chi_{\A|_W} | \, \chi_{\A}$

    \quad Заметим, что матрица $[\A|_W]_E$ в любом базисе $E$ имеет очень простой вид:
    \begin{gather*}
        \begin{pmatrix}
            \lambda & 0 & \dots & 0 \\
            0 & \lambda & \dots & 0 \\
            \dots & \dots & \dots & \dots \\
            0 & 0 & \dots & \lambda
        \end{pmatrix}
    \end{gather*}
    \quad Отсюда очевидным образом следует, что $\chi_{A|_W} = (\lambda - x)^{g_\lambda}$, так как при подсчете определителя вклад внесет только произведение элементов на главной диагонали, а их $g_\lambda$ штук.
    Получаем, что \[ (\lambda - x)^{g_\lambda} \; | \; \chi_{\A} \Rightarrow \text{имеем как минимум $g_\lambda$ корней $\lambda$} \Rightarrow g_\lambda \leqslant a_\lambda \]
\end{proof}