\section{Конечность расширения, полученного присоединением конечного числа алгебраических элементов}
\begin{conj}
    Пусть $L_1 / K$, $L_2 / K$. $\psi : L_1 \to L_2$ --- \textbf{изоморфизм над $K$}, если это изоморфизм полей и $\forall c \in K : \psi(c) = c$. 
\end{conj}

\begin{theorem}
    Пусть $K$ --- поле, $f \in K[x]$ неприводим. Тогда для $K' = K[x] / \langle f \rangle$ справедливо, что $K' = K([x])$, $\Irr_K [x] = f$ и $[K' : K] = \deg f$.
\end{theorem}
\begin{proof}
    Очевидно (нет), $K' / K$ --- расширение. Пусть $d = \deg f$. Тогда \\ $K' = \Lin_K(1, [x], \dots, [x]^{d-1})$. А значит, $K([x]) = K'$.

    Проверим, что $\Irr_K [x] = f$. $f([x]) = f + \langle f \rangle =
    0 + \langle f \rangle = [0]$. Возьмём $g \in K[x]$, т.ч. $g \neq 0$ и $\deg g < \deg f$. Тогда $g([x]) = g + \langle f \rangle \neq [0]$, т.к. $g \nmid f$. Значит, $\Irr_K [x] = f$.

    Понятно (нет), что $(1, [x], \dots, [x]^{d-1})$ --- базис $K'$.
    Тогда $[K' : K] = d$.
\end{proof}

Таким образом, мы классифицировали простые расширения $K$. Это такие $K[x] / \langle f \rangle$, где $f \in K[x]$ неприводим, со стандартным вложением в них поля $K$.

\textbf{Примеры:}
\begin{itemize}
    \item $\Q(\sqrt{5}) = \{ a + b \sqrt{5} \mid a, b \in \Q \}$
    \item $\Q(\sqrt[3]{5}) = \{ a + b \sqrt{5} + c(\sqrt{5})^2 \mid a, b, c \in \Q \}$
    \item и тому подобные...
\end{itemize} 

\follow Конечнопорождённое расширение является алгебраическим тогда и только тогда, когда все присоединяемые элементы алгебраические.
\begin{proof}
    В одну сторону это понятно. Если расширение алгебраическое, тогда все присоединенные элементы будут алгебраическими просто по определению.

    В другую сторону утверждение более содержательное. Сформулируем и докажем даже более сильное утверждение: пусть $a_1, \dots, a_n$ --- алг. над $K$, тогда $K(a_1, \dots, a_n)/K$ --- конечное расширение. Из конечности расширения следует его алгебраичность.

    Докажем индукцией по $n$.
    \begin{itemize}
        \item База $n = 1$. Только что видели.
        \item Переход $n - 1 \to n$.
        
        Запишем присоединение по одному из прошлых замечаний:
        $$ K(a_1, \dots, a_n) = K(a_1)(a_2)\dots(a_n) $$

        Если $a_n$ алгебраично над $K$, то оно уж тем более алгебраично над более широким полем $K(a_1, \dots, a_{n-1})$ --- мы всегда можем взять тот же самый многочлен. Тогда \\$K(a_1, \dots, a_{n-1})(a_n) / K(a_1, \dots, a_{n-1})$ --- конечное расширение. По ИП $K(a_1, \dots, a_{n-1}) / K$ конечно. Применяя мультипликативность степени, получаем, что $K(a_1, \dots, a_n)$ тоже конечно.
    \end{itemize}
\end{proof}

\follow Пусть $a, b$ алгебраичны над $K$. Тогда $a \pm b$, $ab$ тоже алгебраичны над $K$.
\begin{proof}
    Как мы только что поняли, $K(a, b) / K$ --- конечное расширение. А значит, алгебраичное. А значит, любой элемент в $K(a, b)$ алгебраичен над $K$.
\end{proof}