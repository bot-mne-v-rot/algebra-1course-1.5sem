\section{Порядки элементов и циклические подгруппы}
Что такое циклическая подгруппа?

$G$ -- группа, $g\in G$, тогда

$\langle g \rangle\ :=\ \{\ g^n\ |\ n\in\ \mathbb{Z}\ \}$

\begin{theorem-non}
Это подгруппа.

\begin{proof}
Замкнутость: 

$g^m\cdot g^n = g^{m+n}$

Обратный: 

$(g^m)^{-1} = g^{-m}$

Нейтральный: 

$e = g^0$

\end{proof}
\end{theorem-non}

\begin{conj} Циклическая группа

$G$ --- циклическая, если $\exists\ g:\ G = \langle g \rangle$
\end{conj}

\begin{conj} Порядок элемента

  $\ord{g} = \min{ \{ n\in \mathbb{N} : g^n = e \} }$

  $\ord{g} = +\infty \Longleftrightarrow \nexists n\in \mathbb{N} : g^n = e$
\end{conj}

\begin{theorem-non} \quad

  \begin{itemize}
  \item $|\langle g \rangle| = \ord{g}$
  \item $\ord{g} = n < +\infty \Longrightarrow \langle g \rangle = \{ g^0, g^1, \cdots, g^{n-1} \}$
  \end{itemize}

\begin{proof} \quad

  \begin{itemize}
    \item $\ord{g} = +\infty$

    $i \neq j \Longrightarrow g^i \neq g^j$ (так как $g^{i-j} \neq e$, иначе противоречие с порядком)

    $\Longrightarrow |\langle g \rangle| = +\infty$
    
    \item $\ord{g} = n < +\infty$
    
    Начнем со второго утверждения. 
    
    $m = nq + r$, $0 \leq r \leq r - 1$
    
    $g^m = \equalto{(g^n)}{e}^q \cdot g^r = g^r$

    То есть любая степень $g$ это степень $g$ в нужном диапазоне ($[0; n-1]$)

    Таким образом, $|\langle g \rangle| \leq n$

    Осталось увидеть, что $0 \leq i < j \leq n - 1 \Longrightarrow g^i\neq g^j $

    Если $g^i = g^j$, то $g^{j-i} = e$ --- противоречие с определением порядка, так как $1 \leq j - i \leq n - 1$
  \end{itemize}
\end{proof}
\end{theorem-non}

\notice Циклическая группа единственна (с точностью до изоморфизма).

$\ord{g} = +\infty \Longrightarrow \langle g \rangle \cong \mathbb{Z}$

$\ord{g} = n \Longrightarrow \langle g \rangle \cong \mathbb{Z}/n \mathbb{Z}$

Строгое доказательство будет изложено в дальнейшем.
