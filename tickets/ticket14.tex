\section{Подгруппы циклических групп}
\begin{enumerate}
    \item Подгруппы $\Z$ -- это всевозможные $l \Z$, где $l \in \N_0$ (то есть натуральные числа или 0). 
    \begin{proof}
        Было
    \end{proof}
    \item Пусть $n$ -- натуральное число. Тогда все подгруппы $\Z / n\Z$ -- это циклические подгруппы $\cycle{\bar d}$, 
    где $d$ пробегает натуральные делители $n$. 
    
    При этом $(\Z/ n\Z)/ \cycle{\bar d} \cong \Z/d\Z$. 
    
    Например, перечислим все подгруппы $\Z/10\Z$: 
    \begin{align*}
        \cycle{\bar 1} &= \Z/ 10 \Z \\
        \cycle{\bar 2} &= \{\bar 2, \bar 4, \bar 6, \bar 8, \bar 0\} \\
        \cycle{\bar 5} &= \{\bar 5, \bar 0\} \\
        \cycle{\bar{10}} &= \cycle{\bar 0} = \{\bar 0\} \\
    \end{align*}
    \begin{proof}
        Воспользуемся теоремой о соответствии. $\Z  \overset{\pi_{n\Z}}{\longrightarrow} \Z/ n\Z$. По теореме о соответствии, 
        подгруппами в $\Z/ n\Z$ будут образы подгрупп в $\Z$, которые содержали $n\Z$. Таким образом, $\pi_{n\Z}(d\Z)$, где 
        $d\Z \supset n\Z$. А когда такое бывает? Тогда, когда все числа, кратные $n$, будут кратны $d$. То есть это условие 
        равносильно тому, что $d \mid n$.

        $\pi_{n\Z}(d\Z) = \{dm + n\Z\mid m \in \Z\} = \{m (d + n\Z)\mid m \in Z\} = \{d + n\Z\} = \cycle{\bar d}$

        Итак, мы описали все подгруппы $\Z/ n\Z$. Это будут циклические подгруппы порожденные разными натуральными 
        делителями $n$ и нулем. 

        Ну и финальная точка в доказательстве -- применение теоремы о факторгруппе факторгруппы: $(\Z/n\Z)/\cycle{\bar d} = (\Z/n\Z)/(d\Z/ n\Z) \cong \Z/d\Z$
    \end{proof}
\end{enumerate}

\notice На самом деле, доказав последний факт, мы в каком-то смысле внесли ясность в предыдущую часть, так как в ее доказательстве 
мы замяли вопрос о том, не может ли быть такого, что $\cycle{\bar a} = \cycle{\bar b}$. Но сейчас мы поняли, что у фактогрупп разные порядки.
\begin{gather*}
    \ord{\Z/d\Z} = d \Longrightarrow (\Z/n\Z)/\cycle{\bar d} = d \Longrightarrow (\Z/n\Z : \cycle{\bar d}) = d \\
    \text{(так как мы знаем, что порядок факторгруппы равен индексу)} \\ 
    \text{а также } \ord{\Z/n\Z : \cycle{\bar d}} = \underbrace{\ord{\Z/n\Z}}_n / \ord{\cycle{\bar d}} \Longrightarrow \ord{\cycle{\bar d}} = \frac{n}{d} \\
    \cycle{\bar d_1} \neq \cycle{\bar d_2}, \text{если } d_1, d_2 \text{ -- различные делители } n
\end{gather*}

Теперь давайте поговорим о произведении двух подгрупп. 
Напомним, что если у нас есть группа $G$ и есть любые подмножества $M, N$ в ней, то можно рассматривать 
множество $MN = \{ mn \mid m \in M, n \in N\}$. Чаще всего данную конструкцию применяют, когда $M,N$ -- подгруппы $G$. 
При это не лишним будет отметить, что даже если они и правда подгруппы, то из этого не следует, что $MN < G$. 

\example \begin{gather*}
    G = S_3, H = \{(12)\} = \{e, (12)\}, K = \{(13)\} = \{e, (13)\} \\
    HK = \{e, (12), (13), (132)\} \nless S_3 \; (\text{т. Лагранжа})
\end{gather*}

\begin{lemma}
    Пусть $H \lhd G, K < G$. Тогда $HK = KH$ -- подгруппа в $G$
\end{lemma}

\begin{proof}
    Сначала докажем включение $HK \subset KH$. Возьмем $h \in H, k \in K$. Тогда $hk = k\underbrace{k^{-1}hk}_{\in H} \in KH$. 
    Аналогично можно получить, что $HK \supset KH$. Таким образом, равенство мы доказали. Осталось проверить три свойства подгруппы. 
    Непустота гарантированна, т. к. как минимум $ee$ точно попал в множество. Замкнутость относительно обратного тоже есть т. к. 
    $(HK)^{-1} = K^{-1} H^{-1} = KH = HK$. Наконец, замкнутость относительно умножения тоже имеется, так как $(HK)(HK) = H(KH)K = H(HK)K
    = (HH)(KK) = HK$. 
\end{proof}