\section{Лекция номер 6}

\subsection{Конечные и конечнопорождённые абелевы группы}

\begin{conj} $ $\\
    Пусть $A, B$ -- группы.\\
    Тогда $A \oplus B := A \times B$ -- \textbf{прямая сумма} 
    групп $A$ и $B$ в аддитивной записи.
\end{conj}

\begin{lemma}
    Пусть $m, n \in \N$, $\gcd(m, n) = 1$.
    Тогда $\Z/(mn)\Z \cong (\Z/m\Z) \oplus (\Z/n\Z)$.
\end{lemma}
\begin{proof}
    Два доказательства:
    \begin{itemize}
        \item[I.] Рассмотрим отображение:
        \begin{flalign*}
            \varphi \colon \Z &\to (\Z/m\Z) \oplus (\Z/n\Z) &&\\
            a &\mapsto ([a]_m, [a]_n) &&
        \end{flalign*} 
        Очевидно, $\varphi$ -- гомоморфизм.
        $$\Ker \varphi = \{ a \mid a \equiv 0 \mod{m},
        a \equiv 0 \mod{n} \} = \{ a \mid a \equiv 0 
        \mod{\operatorname{lcm}(m, n)}\} = mn \Z$$

        По теореме о гомоморфизме:
        \begin{align*}
            &\Z/\Ker \varphi \cong \Imm \varphi \Leftrightarrow
            \Z/mn\Z \cong \Imm \varphi \Rightarrow 
            \abs{\Imm \varphi} = mn; \\
            &(\Z/m\Z) \oplus (\Z/n\Z) \subset \Imm \varphi,
            \abs{(\Z/m\Z) \oplus (\Z/n\Z)} = mn \Rightarrow \\
            &\Rightarrow \Imm \varphi = (\Z/m\Z) \oplus (\Z/n\Z)
            \cong \Z / mn \Z
        \end{align*}

        \item[II.]
        \begin{align*}
            & \ord([1]_m, [1]_n) = \min\{ l \in \N \mid 
            ([l]_m, [l]_n) = ([0]_m, [0]_n)\} = mn \Rightarrow \\
            & \Rightarrow (\Z/m\Z) \oplus (\Z/n\Z) = 
            \cycle{([1]_m, [1]_n)} \Rightarrow \\
            & \Rightarrow \text{циклическая группа порядка } mn 
            \cong \Z/mn\Z
        \end{align*} 
    \end{itemize}
\end{proof}

\begin{conj}
    $\Z/p^t\Z$ -- это $p$-примарная циклическая группа,
    где $p \in \mathbb{P}, t \in \N$.
\end{conj}

\begin{theorem-non}
    Пусть $G$ -- конечная абелева группа. Тогда $G$ изоморфна
    прямой сумме примарных циклических групп. Порядки этих
    групп определены однозначно с точностью до перестановки.
\end{theorem-non}
\begin{proof}
    \textit{Доказательства не будет -- автор принял линал.}
\end{proof}
\begin{example}
    Опишем все абелевы группы порядка 200:
    \begin{eqnarray*}
        200 &=& 5^2 \cdot 2^3 = \\
            &=& 5 \cdot 5 \cdot 2^3 = \\
            &=& 5^2 \cdot 2^2 \cdot 2 = \\
            &=& 5 \cdot 5 \cdot 2^2 \cdot 2 = \\
            &=& 5^2 \cdot 2 \cdot 2 \cdot 2 = \\
            &=& 5 \cdot 5 \cdot 2 \cdot 2 \cdot 2
    \end{eqnarray*}
    Этим разложениям соответствуют группы:
    \begin{gather*}
        (\Z/5^2\Z) \oplus (\Z/2^3\Z) \\
        (\Z/5\Z) \oplus (\Z/5\Z) \oplus (\Z/2^3\Z) \\
        (\Z/5^2\Z) \oplus (\Z/2^2\Z) \oplus (\Z/2\Z) \\
        (\Z/5\Z) \oplus (\Z/5\Z) \oplus (\Z/2^2\Z) \oplus (\Z/2\Z) \\
        \vdots \\
        (\Z/5\Z)^2 \oplus (\Z/2\Z)^3 \cong (\Z/10\Z)^2 \oplus (\Z/2\Z)
    \end{gather*}
\end{example}

\follow Пусть $K$ -- поле, $G < K^*$, $\abs{G} < \infty$.
Тогда $G$ -- циклическая.

\begin{proof} $ $
    \textit{Далее аддитивная запись}
     
    $$G \cong \bigoplus_{i = 1}^m \Z/p_i^{l_i}\Z$$

    Предположим, что среди $p_i$ есть совпадающие. НУО, $p_1 = p_2$.

    В $(\Z/p_1^{l_1}\Z) \oplus (\Z/p_2^{l_2}\Z)$ рассмотрим 
    $g = ([ap_1^{l_1-1}]_{p_1^{l_1}}, [bp_1^{l_2 - 1}]_{p_1^{l_2}})$,
    где $0 \leqslant a, b \leqslant p_1 - 1$. \\
    $p_1 g = ([0]_{p_1^{l_1}}, [0]_{p_1^{l_2}})$, значит $\ord g = p_1
    \text{ или } 1$ (1 в случае $a = b = 0$). \\
    Это означает, что $\{ g \in G \mid p_1 g = 0\} \geqslant p_1^2$
    (мы перебираем все возможные $a$ и $b$). \\
    При этом $\underbrace{\{ g \in G \mid p_1 g = 0 \}}_
    {\text{группа кольца } \Rightarrow \text{ адд. запись}}
    \subset \underbrace{\{ x \in K^* \mid x^{p_1} = 1\}}_
    {\text{группа обратимых } \Rightarrow \text{ мульт. запись}}$

    Но у $x^{p_1} - 1$ есть не более $p_1$ корней -- противоречие.

    Таким образом, все $p_i$ различны $\Rightarrow$ все $p_i$
    взаимно просты $\xRightarrow[]{\text{Л.}}$
    $$\xRightarrow[]{\text{Л.}}
    G \cong \bigoplus_{i = 1}^m \Z/p_i^{l_i}\Z \cong
    \Z/(\prod_{i = 1}^m p_i^{l_i})\Z$$
    А это значит, что $G$ -- циклическая.

\end{proof}

\begin{conj}
    \textbf{Конечнопорождённая группа} --- группа с 
    конечным числом образующих.
\end{conj}
\begin{conj} $ $\\
    Пусть $G$ -- абелева группа.\\
    $G_{tor} = \{ g \in G \mid \ord g < \infty \}$
    --- \textbf{подгруппа кручений}.\\
    \textit{tortion -- кручение}
\end{conj}
\notice Абелевость $G$ здесь важна. В неабелевой группе
определённое таким образом множество не обязательно является
подгруппой $G$.

Легко видеть: $(G/G_{tor})_{tor} = \{ 0 \}$. 
(\textit{вот мне не легко, но кому легко, тот молодец})

\underline{\textbf{Факты:}}
\begin{enumerate}
    \item $G_{tor} = 0 \Rightarrow G \cong \Z^r$, 
    где $r$ -- инвариант.
    \item В общем случае, $G \cong (G / G_{tor}) \oplus G_{tor}$
    \item $G_{tor}$ -- конечная абелева группа
\end{enumerate}
\notice В частности, конечно порождённая абелева группа изоморфна
конечной прямой сумме циклических.

\subsection{Линейные отображения}

\begin{theorem}[Свойство универсальности базиса]
    Пусть $V, W$ -- ЛП/$K$; $e_1, \dots, e_n$ --- базис $V$;
    $w_1, \dots, w_n \in W$. Тогда 
    $\exists! \varphi \in \Hom(V, W) : \varphi(e_i) = w_i, i = 1..n$.
\end{theorem}
\begin{proof}
    Пусть $$\varphi(\alpha_1 e_1 + \dots + \alpha_n e_n) =
    \alpha_1 w_1 + \dots + \alpha_n w_n $$
    Очевидно, $\varphi$ линейно, однозначно задано для всех векторов 
    из $V$ (т.к. любой вектор однозначно раскладывается по базису)
    и $\varphi(e_i) = w_i$.

    Таким образом мы доказали существование. Докажем единственность.
    Пусть $ \widetilde{\varphi}(e_i) = w_i = \varphi(e_i)$. Тогда
    в силу линейности:
    $$\widetilde{\varphi}(\alpha_1 e_1 + \dots + \alpha_n e_n) = 
    \alpha_1 \widetilde{\varphi}(e_1) + \dots + \alpha_n \widetilde{\varphi}(e_n) =
    \alpha_1 w_1 + \dots + \alpha_n w_n = \varphi(\alpha_1 e_1 + \dots + \alpha_n e_n)$$
    А значит, $\widetilde{\varphi} = \varphi$.
\end{proof}
\notice В этом предложении конечномерность $W$ не требуется.

\begin{conj}
    Пусть $\varphi \in \Hom(V, W)$, $E = (e_1, \dots, e_n)$ -- базис $V$,
    $F = (f_1, \dots, f_m)$ -- базис $W$. \textbf{Матрицей $\varphi$ 
    относительно базисов $E$ и $F$} называют
    $$ [ \varphi ]_{E, F} = \begin{pmatrix}
        [ \varphi(e_1)]_F & [ \varphi(e_2)]_F & \dots & [ \varphi(e_n) ]_F
    \end{pmatrix} \in M(m, n, K)
    $$
\end{conj}

\begin{conj}
    Введём структуру линейного пространства над $\Hom(V, W)$. \\
    Пусть $\varphi, \psi \in \Hom(V, W)$, $\alpha, \beta \in K$.
    Тогда 
    \begin{gather*}
        (\alpha \varphi + \beta \psi)(v) := \alpha \varphi(v) +
        \beta \psi(v) \\
        \alpha \varphi + \beta \psi \in \Hom(V, W)
    \end{gather*}  
    Аксиомы ЛП тривиально выполняются.
\end{conj}

\follow из свойства универсальности базиса следующее отображение
является изоморфизмом ЛП
\begin{align*}
    \varepsilon \colon \Hom(V, W) &\to M(m, n, K)\\
    \varphi &\mapsto [ \varphi ]_{E, F}
\end{align*}

\follow Пусть $\dim V = n$, $\dim W = m$. Тогда $\dim \Hom(V, W) = mn$.
\begin{proof}
    $\Hom(V, W) \cong M(m, n, K)$, а $\dim M(m, n, K) = mn$.
\end{proof}

\begin{theorem}
    Пусть $E$ --- базис $V$, $F$ --- базис $W$, $\varphi \in \Hom(V, W)$. \\
    Тогда $[\varphi(v)]_F = [\varphi]_{E, F} \cdot [v]_E$
\end{theorem}
Получается следующая коммутативная диаграмма:
\begin{center}
    \shorthandoff{"}
    \begin{tikzcd}
        v \arrow[dd, maps to] & V \arrow[rr, "\varphi"] \arrow[dd, "\varepsilon_E"] &  & W \arrow[dd, "\varepsilon_F"]                  & w \arrow[dd, maps to] \\
                            &                                    &  &                               &                       \\
        {[v]_E}               & K^n \arrow[rr]                     &  & K^m                           & {[w]_F}               \\
                            & {[v]_E} \arrow[rr, maps to]        &  & {[\varphi]_{E,F} \cdot [v]_E} &                      
    \end{tikzcd}
    \shorthandon{"}
\end{center}
\begin{proof}
    Т.к. $E, F$ --- вектора-строки из базисных векторов, получаем: 
    $v = E \cdot [v]_E$,  $\varphi(v) = F \cdot [\varphi(v)]_F$.

    С другой стороны, $\varphi(v) = \varphi(E) \cdot [v]_E$, где
    $\varphi(E) = (\varphi(e_1), \dots, \varphi(e_n))$ -- вектор-строка
    из образов базисных векторов. Более того, 
    $\varphi(E) = F \cdot [\varphi]_{E, F}$, т.к. 
    $\varphi(e_i) = F \cdot [\varphi(e_i)]_F$.

    Собираем вместе:
    \begin{align*}
        \varphi(v) &= F \cdot [\varphi(v)]_F \\
        &= (F \cdot [\varphi]_{E, F}) \cdot [v]_E \\
        &= F \cdot ([\varphi]_{E, F} \cdot [v]_E)
    \end{align*}

    Получаем, что линейная комбинация векторов $F$ с коэф.
    $[\varphi(v)]_F$ равна ЛК векторов $F$ с коэф.
    $[\varphi]_{E, F} \cdot [v]_E$. Т.к. $F$ --- базис $W$, делаем вывод,
    что $[\varphi(v)]_F = [\varphi]_{E, F} \cdot [v]_E$.
\end{proof}

\notice Проще всего понять, что происходит, можно на матрицах.

Пусть 
$E = (e_1, \dots, e_n)$,
$[v]_E = \begin{pmatrix}
\alpha_1 \\ 
\vdots \\ 
\alpha_n
\end{pmatrix}$, 
$[\varphi(v)]_F = \begin{pmatrix}
\beta_1 \\ 
\vdots \\ 
\beta_m
\end{pmatrix}$,
$[\varphi(e_i)]_F = \begin{pmatrix}
\gamma_{1i} \\ 
\vdots \\ 
\gamma_{mi}
\end{pmatrix}$. Тогда
$$
\begin{pmatrix}
    \beta_1 \\ 
    {
        \tikz[overlay]
        \fill[green!20] (-0.1,-0.2) rectangle (0.5,0.4);
    }
    \beta_2 \\ 
    \vdots \\ 
    \beta_m
\end{pmatrix}
=
\begin{pmatrix}
    \gamma_{11} & \gamma_{12} & \dots & \gamma_{1n} \\
    {
        \tikz[overlay]
        \fill[blue!20] (0,-0.2) rectangle (3.5,0.4);
    }
    \gamma_{21} & \gamma_{22} & \dots & \gamma_{2n} \\
    \vdots & \vdots & \ddots & \vdots \\
    \gamma_{m1} & \gamma_{m2} & \dots & \gamma_{mn} \\
\end{pmatrix}
\cdot
\begin{pmatrix}
    {
        \tikz[overlay]
        \fill[red!20] (-0.1,0.3) rectangle (0.5,-1.9);
    }
    \alpha_1 \\ 
    \alpha_2 \\ 
    \vdots \\ 
    \alpha_n
\end{pmatrix}
$$

Получается, что: 
\colorbox{green!20}{$\beta_i$}$= \sum \limits_{k=1}^{n}$ 
\colorbox{blue!20}{$\gamma_{ik}$}
$\cdot$
\colorbox{red!20}{$\alpha_k$}

И это можно переписать вот так:
$$
\underbrace{
\begin{pmatrix}
    \beta_1 \\
    \vdots \\ 
    \beta_m
\end{pmatrix}
}_{=[\varphi(v)]_F}
=
\alpha_1
\underbrace{
\begin{pmatrix}
    \gamma_{11} \\ 
    \vdots \\ 
    \gamma_{m1}
\end{pmatrix}
}_{=[\varphi(e_1)]_F}
+
\alpha_2
\underbrace{
\begin{pmatrix}
    \gamma_{12} \\ 
    \vdots \\ 
    \gamma_{m2}
\end{pmatrix}
}_{=[\varphi(e_2)]_F}
+
\dots
+
\alpha_n
\underbrace{
\begin{pmatrix}
    \gamma_{1n} \\ 
    \vdots \\ 
    \gamma_{mn}
\end{pmatrix}
}_{=[\varphi(e_n)]_F}
$$
И действительно, 
$\varphi(v) = \alpha_1 \varphi(e_1) + \dots + \alpha_n \varphi(e_n)$.

\notice Пусть $E, E'$ --- базисы $V$. Тогда $M_{E \to E'} =
[\E_V]_{E', E}$, где $\E_V \in \Hom(V, V)$ --- тождественное отображение на $V$.
Действительно, в матрице перехода между базисами 
$M_{E \to E'}$ мы раскладываем новый базис $E'$ по старому базису $E$,
что по сути то же самое, что в матрице линейного отображения, только
никакого отображения мы к векторам не применяем. Или другими словами,
мы применяем к базисным векторам тождественное отображение.

\begin{theorem}
    Пусть $U, V, W$ --- конечномерные ЛП/$K$; $E, F, G$ --- их базисы;
    $\varphi \in \Hom(U, V)$, $\psi \in \Hom(V, W)$. Тогда
    $[\psi \circ \varphi]_{E,G} = [\psi]_{F,G} \cdot [\varphi]_{E,F}$.
\end{theorem}
\begin{proof}
    По определению:
    \begin{align*}
        \varphi(E) &= F \cdot [\varphi]_{E, F} \\
        \psi(F) &= G \cdot [\psi]_{F, G} \\
        \psi(\varphi(E)) &= G \cdot [\psi \circ \varphi]_{E, G} 
    \end{align*}
    Применим $\psi$ к первому равенству.
    Важно, что $[\varphi]_{E, F}$ это матрица-скаляр, поэтому при применении к ней $\psi$ ничего не происходит:
    \[ \psi(\varphi(E)) = \psi(F) \cdot [\varphi]_{E, F} = G \cdot [\psi]_{F, G} \cdot [\varphi]_{E, F}  \]
    Получилось то же самое, что и в третьем равенстве:
    \[ \psi(\varphi(E)) = G \cdot [\psi \circ \varphi]_{E, G} = G \cdot [\psi]_{F, G} \cdot [\varphi]_{E, F}  \]
    $G$ --- базис $\Longrightarrow$ 
    $[\psi \circ \varphi]_{E,G} = [\psi]_{F,G} \cdot [\varphi]_{E,F}$
\end{proof}

\follow Пусть $E, E'$ -- базисы $V$; $F, F'$ -- базисы $W$;
$C = M_{E \to E'}$, $D = M_{F \to F'}$, $\varphi \in \Hom(V, W)$.
Тогда 
$$ [\varphi]_{E', F'} = D^{-1} \cdot [\varphi]_{E, F} \cdot C $$
\begin{proof}
    Заметим, что:
    $$ \varphi = \E_W \circ \varphi \circ \E_V $$

    Тогда по предыдущему предложению:
    $$ [\varphi]_{E', F'} = 
    \stackbelow{[\E_W]_{F, F'}}{M_{F' \to F}} 
    \cdot [\varphi]_{E, F}
    \cdot \stackbelow{[\E_V]_{E', E}}{M_{E \to E'}}$$
    Как мы знаем, $M_{F' \to F} = \left(M_{F \to F'}\right)^{-1}$
\end{proof}

\begin{conj} $ $\\
    \textbf{Ранг} линейного отображения $\rk \varphi
    = \dim \Imm \varphi$. \\
    \textbf{Дефект} линейного отображения $= \dim \Ker \varphi$.
\end{conj}

\begin{theorem}
    Пусть $V, W$ --- конечномерные ЛП/$K$, $\varphi \in \Hom(V, W)$.
    Тогда в $V,W$ существуют базисы $E, F$ такие, что
    $[\varphi]_{E, F}$ --- окаймлённая единичная.
\end{theorem}
\begin{proof} Два доказательства:
    \begin{itemize}
        \item[I.] Пусть $E_0, F_0$ --- какие-то базисы $V,W$.
        У $[\varphi]_{E_0, F_0}$ есть $PDQ$-разложение, т.е.
        $[\varphi]_{E_0, F_0} = PDQ$, где $P, Q$ --- обратимые,
        $D$ --- окаймлённая единичная.  

        Нужно построить $E, F$ так, что $M_{E_0 \to E} = Q^{-1}$,
        $M_{F_0 \to F} = P$. Пусть 
        $E := E_0 \cdot Q^{-1}$, $F := F_0 \cdot P$. 
        Мы доказывали, что базис, домноженный на обратимую матрицу
        --- тоже какой-то базис.
        
        Тогда:
        $$ [\varphi]_{E, F} = M_{F_0 \to F}^{-1} \cdot [\varphi]_{E_0, F_0}
        \cdot M_{E_0 \to E} = P^{-1} PDQ Q^{-1} = D $$

        \item[II.] Пусть $r = \rk \varphi = \dim \Imm \varphi$, 
        $n = \dim V$, $m = \dim W$.
        Тогда $\dim \Ker \varphi = n - r$. 
        
        Пусть $e_1', \dots, e_{n-r}'$ --- базис $\Ker \varphi$,
        $e_1, \dots, e_r$ --- его дополнение до базиса $V$.
        Пусть $f_1 = \varphi(e_1), \dots, f_r = \varphi(e_r)$.

        Проверим, что $f_1, \dots, f_r$ --- ЛНС.
        Пусть:
        $$\stackbelow{\alpha_1 f_1 + \dots + \alpha_r f_r}
        {\varphi(\alpha_1 e_1 + \dots + \alpha_r e_r)} = 0$$
        Значит, $\alpha_1 e_1 + \dots + \alpha_r e_r \in \Ker \varphi$.
        А там есть свой базис, поэтому:
        $$ \alpha_1 e_1 + \dots + \alpha_r e_r =
        \alpha_1' e_1' + \dots + \alpha_{n-r}' e_{n-r}'$$
        Но $e_1', \dots, e_{n-r}', e_1, \dots, e_r$ --- базис $V$,
        поэтому $\alpha_1 = \dots = \alpha_r = 0$.

        Раз $f_1, \dots, f_r$ --- ЛНС, пусть $f_{r+1}, \dots, f_m$ ---
        дополнение её до базиса $W$.

        Пусть $E = (e_1, \dots, e_r, e_1', \dots, e_{n-r}')$,
        $F = (f_1, \dots, f_m)$. 
        Получаем:
        \begin{align*}
            [\varphi]_{E,F} &= 
            (\stackbelow{[\varphi(e_1)]_F}{[f_1]_F} 
            \quad \dots \quad 
            \stackbelow{[\varphi(e_r)]_F}{[f_r]_F} \quad
            \stackbelow{[\varphi(e_1')]_F}{[0]_F} \quad \dots \quad 
            \stackbelow{[\varphi(e_{n-r})']_F}{[0]_F}) = \\
            &= 
            \stackrel{\begin{matrix}
                & & & r & & &
            \end{matrix}}
            {\begin{pmatrix}
                1 & 0 & \dots & 0 & 0 & \dots & 0 \\
                0 & 1 & \dots & 0 & 0 & \dots & 0 \\
                \vdots & \vdots & \ddots & \vdots & \vdots & \ddots & \vdots \\
                0 & 0 & \dots & 1 & 0 & \dots & 0 \\
                0 & 0 & \dots & 0 & 0 & \dots & 0 \\
                \vdots & \vdots & \ddots & \vdots & \vdots & \ddots & \vdots \\
                0 & 0 & \dots & 0 & 0 & \dots & 0 \\
            \end{pmatrix}}
            \begin{matrix}
                \\ \\ r \\ \\ \\
            \end{matrix} \\
            &=
            \begin{pmatrix}
                E_r & 0 \\
                0 & 0
            \end{pmatrix}
        \end{align*} 
        
    \end{itemize}
\end{proof}

\begin{theorem}
    Пусть $V, W$ --- конечномерные ЛП/$K$; $E, F$ --- базисы $V, W$;
    $\varphi \in \Hom(V, W)$. Тогда $\rk [\varphi]_{E, F} = \rk \varphi$.
\end{theorem}
\begin{proof}
    Два доказательства:
    \begin{itemize}
        \item[I.] 
        \begin{align*}
            \Imm \varphi &\cong \Imm (\varepsilon_F \circ \varphi) \\
            &= \Imm (([\varphi]_{E,F} \times) \circ \varepsilon_E) & 
            (``[\varphi]_{E,F} \times'' \text{ --- оператор умножения на $[\varphi]_{E,F}$ слева}) \\
            &= \Imm ([\varphi]_{E,F} \times) \\
            &\equiv \text{лин. обол. столбцов } [\varphi]_{E,F} \\
            \Longrightarrow & \dim \Imm \varphi = \rk [\varphi]_{E,F}
        \end{align*}
        \item[II.] $\rk [\varphi]_{E,F}$ не зависит от $E, F$, т.к.
        при замене базисов происходит умножение на обратимые матрицы.

        Из второго доказательства предыдущего предложения: 
        $\rk [\varphi]_{E,F} = \rk D = r$, но $r = \rk \varphi$.
    \end{itemize}
\end{proof}
