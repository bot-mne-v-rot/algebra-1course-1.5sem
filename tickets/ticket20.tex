\section{Действие группы на множестве. Определение и примеры}
\begin{conj}
    Говорят, что задано действие группы $G$ на множестве $M$, если
    \begin{enumerate}
        \item задано отображение $(g, m) \mapsto gm$ из  $G \times M$ в $M$, обладающее двумя свойствами
        \begin{enumerate}
            \item $\forall g_1, g_2 \in G \;\; \forall m \in M : (g_1g_2)m = g_1(g_2m)$
            \item $\forall m \in M \;\; em = m$
        \end{enumerate}
        \item задан гомоморфизм $\varphi: G \to S(M)$, где $S(M)$ - группа биекций $M$ на себя
    \end{enumerate}
\end{conj}
Получим из одного определения другое.

\underline{Определение 1 $\to$ Определение 2:} Определим наш гомоморфизм $\varphi$ так:
\begin{gather*}
    \varphi: G \to S(M) \\
    g \mapsto (m \mapsto gm)
\end{gather*} 
Отображение $m \mapsto gm$ являестя биекцией, так как у него есть обратное отображение $m \mapsto g^{-1}m$.

Изучим отображение $\varphi(g_1g_2)$. Это элемент $S(M)$, давайте применим его к какому-то $m$.
\begin{gather*}
    \varphi(g_1g_2)(m) \underbrace{=}_{\text{по опр. гом-ма}} (g_1g_2)m \underbrace{=}_{\text{св-во опр. 1}} g_1(g_2m) = \varphi(g_1)(g_2m) = \varphi(g_1)(\varphi(g_2)(m)) \\
    \Rightarrow \varphi(g_1g_2) = \varphi(g_1) \circ \varphi(g_2)
\end{gather*}
Значит, гомоморфизм построен корректно.

\underline{Определение 2 $\to$ Определение 1:} Определим наше отображение из $G \times M$ в $M$ так:
\begin{gather*}
    G \times M \to M \\
    (g, m) \mapsto \varphi(g)(m)
\end{gather*}
Проверим, что сохранились свойства:
\begin{enumerate}
    \item \begin{gather*}
        (g_1g_2)m \underbrace{=}_{\text{по опр. отображения}} \varphi(g_1g_2)(m) \underbrace{=}_{\text{св-во гом-ма}} (\varphi(g_1) \circ \varphi(g_2))(m) = g_1(g_2m)
    \end{gather*}
    \item \[ em = \varphi(e)(m) = id_m(m) = m \]
\end{enumerate}

\underline{Примеры:}
\begin{enumerate}
    \item $G = \R, M = \mathbb{C}$ \\ $gm := m(\cos g + i\sin g)$ -- поворот на комплексной плоскости на угол $g$
    \item $G = GL_n(K)$ -- обратимые матрицы порядка $n$, $M = K^n$ -- столбцы \\ $gm := g * m$ - умножение матрицы на столбец
    \item $G$ -- любая группа, $M = G$ \\ $gm := g * m$, где $*$ -- групповая операция \\ Это действие $G$ на себе левыми сдвигами.
    \item $G$ -- любая группа, $M = G$ \\ $gm := m^g = gmg^{-1}$ \\ Это действие $G$ на себе сопряжениями.
\end{enumerate}