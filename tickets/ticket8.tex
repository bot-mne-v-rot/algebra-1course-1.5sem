\section{Факторгруппа}
\begin{conj} \quad

    Пусть $H \lhd G$. Введём на $G/H$ структуру группы:
    \begin{flalign*}
        ``*'' \colon G/H \times G/H &\to G/H &&\\
        (A, B) &\mapsto AB &&
    \end{flalign*}
    где $AB = \{ ab \mid a \in A, b \in B \}$.

    Такая группа называется \textbf{факторгруппой} группы $G$ по
    нормальной подгруппе $H$.
\end{conj}
\begin{proof}
    Убедимся в корректности определения.

    Отметим, что т.к. $H \lhd G$, то 
    $\forall g \in G \;\; gH = Hg$

    Почему $AB \in G/H$? Пусть $A = gH$, $B = g'H$.\\
    Тогда $AB = (gH)(g'H) = g(Hg')H = g(g'H)H = (gg')(HH) =
    gg'H \in G/H$.

    Проверим аксиомы группы:
    \begin{itemize}
        \item Ассоциативность:
        
        Пусть $A, B, C \in G/H$. Тогда
        $(AB)C = A(BC) \Leftrightarrow \\ \Leftrightarrow
        \{ (ab)c \mid a \in A, b \in B, c \in C \} =
        \{ a(bc) \mid a \in A, b \in B, c \in C \}$.\\
        Это верно, т.к. операция на элементах $G$ была ассоциативна,
        т.е. $(ab)c = a(bc)$.

        \item Наличие нейтрального:
        
        $eH$ -- нейтральный в $G/H$, так как:\\
        $gH \cdot eH = gHH = gH$, \\
        $eH \cdot gH = e(Hg)H = e(gH)H = (eg)(HH) = gH$.

        \item Наличие обратного:
        
        $g^{-1} H$ -- обратный к $gH$, так как:\\
        $gH \cdot g^{-1}H = g(Hg^{-1})H = g(g^{-1}H)H =
        (gg^{-1})(HH) = eH$, \\
        $g^{-1}H \cdot gH = g^{-1}(Hg)H = g^{-1}(gH)H =
        (g^{-1}g)(HH) = eH$.

    \end{itemize}
\end{proof}

\begin{example}
    \begin{enumerate}
        \item $\Z / m\Z$ -- совпадает с тем, что мы ранее определяли для
        колец.
        \item $G/G = \{eG\}$ -- тривиальная группа. Все элементы попадают
        в один класс.
        \item $G/\{e\} \cong G$.
    \end{enumerate}
\end{example}

\notice Если $H \lhd G$, то $gH = g'H$ $\Longleftrightarrow$ 
$g = g'h$, где $h \in H$ $\Longleftrightarrow$ 
$g = \widetilde{h}g'$, где $\widetilde{h} \in H$.\\
Элементы $g$ и $g'$ называются \textbf{сравнивыми} по нормальной
подгруппе $H$.

\notice $ghg^{-1}$ называется \textbf{сопряжённым} к $h$ при помощи $g$.
Тогда определение нормальной подгруппы можно переформулировать так, 
что нормальная подгруппа -- это та, что помимо всех своих элементов
содержит все сопряжённые к ним.