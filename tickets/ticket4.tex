\section{Левые и правые смежные классы. Индекс подгруппы}
\begin{conj}
    Пусть $G$ -- группа, $H$ -- подгруппа $G$, т.е. $H < G$;
    $g_1, g_2 \in G$. Тогда $g_2 \sim g_1$, если $g_2 = g_1 h$ для
    некоторого $h \in H$.
\end{conj}

\begin{theorem-non}
    ``$\sim$'' --- отношение эквивалентности.
\end{theorem-non}
\begin{proof} $ $

    \begin{itemize}
        \item Рефлексивность:
        
        $g = ge$, но $H < G$, значит $e \in H$.

        \item Симметричность:
        
        Если $g_2 = g_1 h$ для $h \in H$, то $g_1 = g_2 h^{-1}$. \\
        $H < G \Rightarrow h^{-1} \in H$.

        \item Транзитивность:
        
        Если $g_2 = g_1 h$, $g_3 = g_2 h'$, $h, h' \in H$, \\
        то $g_3 = g_2 h' = (g_1 h) h' = g_1 (hh')$. \\
        $H < G \Rightarrow hh' \in H$.

    \end{itemize}
\end{proof}

\begin{conj} $ $\\
    $G$ -- группа; $M$, $N$ -- множества; $g \in G$. Тогда \\
    $gM := \{ gm \mid m \in M \}$. \\
    $Ng := \{ ng \mid n \in N \}$. \\
    $MN := \{ mn \mid m \in M, n \in N \}$. \\
    $M^{-1} := \{ m^{-1} \mid m \in M \}$.
\end{conj}

\begin{conj} $ $\\
    $G/\sim \,\, =: G/H$ -- \textbf{множество левых 
    смежных классов} $G$ по $H$. \\
    $[g] = \{ gh \mid h \in H \} =: gH$ -- 
    \textbf{левый смежный класс элемента}. \\
    $Hg := \{ hg \mid h \in H \}$ -- 
    \textbf{правый смежный класс элемента}. \\
    $H \backslash G$ -- \textbf{множество правых 
    смежных классов} $G$ по $H$. \\
\end{conj}

\notice В некоторых учебниках по алгебре $gH$ называют
правым смежным классом.

\notice Если $h \in H$, то $hH = eH = H$.

\notice Правые и левые смежные классы не обязательно равны. \\
\begin{example}
    $S_3 = \{e, (1 \; 2), (1 \; 3), (2 \; 3), 
    (1 \; 2 \; 3), (1 \; 3 \; 2)\}$. \\
    $H = \langle (1 \; 2) \rangle = \{e, (1 \; 2)$, \\
    $eH = (1 \; 2)H = H = H(1 \; 2) = He$, \\
    $(1 \; 3) H = \{ (1 \; 3), (1 \; 2 \; 3) \} = (1 \; 2 \; 3) H$, \\
    $(2 \; 3) H = \{ (2 \; 3), (1 \; 3 \; 2) \} = (1 \; 3 \; 2) H$; \\
    при этом $H (1 \; 3) = \{ (1 \; 3), (1 \; 3 \; 2) \}$.
\end{example}

\begin{conj}
    $H < G$. \textbf{Индексом $H$ в $G$} называют
    $(G : H) := \abs{G/H}$.
\end{conj}
\notice $(G : H)$ -- это некоторая мощность, которая в том числе 
может быть бесконечной.

\begin{theorem-non}
    $\abs{G/H} = \abs{H\backslash G}$
\end{theorem-non}
\begin{proof} $ $

    Рассмотрим отображение:
    \begin{flalign*}
        \varphi\colon G/H &\to H \backslash G &&\\
        A &\mapsto A^{-1} = \{ a^{-1} \mid a \in A\} &&
    \end{flalign*}
    Убедимся, что отображение задано корректно.
    $A = gH$ для нек. $g \in G$. \\ $A^{-1} = (gH)^{-1}
    = \{ (gh)^{-1} \mid h \in H \} 
    = \{ h^{-1}g^{-1} \mid h \in H \}
    = H^{-1}g^{-1} = Hg^{-1} \in H \backslash G$.

    Рассмотрим другое отображение:
    \begin{flalign*}
        \psi\colon H \backslash G &\to G/H &&\\
        A &\mapsto A^{-1} = \{ a^{-1} \mid a \in A\} &&
    \end{flalign*}
    Аналогично, оно задано корректно. При этом нетрудно
    убедиться, что 
    $\psi \circ \varphi = \operatorname{id}_{G/H}$ и
    $\varphi \circ \psi = \operatorname{id}_{H \backslash G}$.
    Таким образом, $\varphi$ и $\psi$ -- взаимно обратные биекции,
    значит, множества $G/H$ и $H \backslash G$ равномощны.

\end{proof}

\notice Если группа была конечной, то индекс подгруппы конечен.
С другой стороны, если группа была бесконечной, то индекс не
обязательно бесконечный. Например, $(\Z : m\Z) = m$.

