\section{Свойства внешнего прямого произведения}
Пусть есть $G, G'$ -- группы. Построим новую группу 
$G \times G'. \; (g_1, g_1')(g_2, g_2') = (g_1g_2, g_1'g_2')$. Легко видеть, 
что $(G \times G', \; \cdot \; )$ -- группа. 

\begin{theorem-non} 
    Рассмотрим отображения: 
    \begin{align*}
        i_1: G \longrightarrow G \times G' &\qquad g \longmapsto (g, e) \\
        i_2: G' \longrightarrow G \times G' &\qquad g' \longmapsto (e, g') \\
        \pi_1: G \times G' \longrightarrow G &\qquad (g, g') \longmapsto g \\
        \pi_2: G \times G' \longrightarrow G' &\qquad (g, g') \longmapsto g'
    \end{align*}
\end{theorem-non}
\textbf{\textit{Fun facts: }}
\begin{enumerate}
    \item $i_1, i_2$ (вложения) -- мономорфизмы групп
    
    $\pi_1, \pi_2$ (проекции) -- эндоморфизмы групп
    \item $\Imm{i_1} = \Ker{\pi_2} = G \times \{e\}$
    
    $\Imm{i_2} = \Ker{\pi_1} = \{e\} \times G$
    \item $\pi_1 \circ i_1 = id_G, \quad \pi_2 \circ i_2 = id_{G'}$
    
    $\pi_1 \circ i_2 = e: G' \longrightarrow G, \; g' \longmapsto e$ (единичный гомоморфизм)

    $\pi_2 \circ i_1 = e: G \longrightarrow G', \; g \longmapsto e$ (единичный гомоморфизм)
\end{enumerate}

Стоит также заметить, что $i_1, i_2$ индуцируют 
\begin{gather*}
    G \longrightarrow G \times \{e\} \text{ -- изоморфизм} \\
    G' \longrightarrow \{e\} \times G' \text{ -- изоморфизм}
\end{gather*}

Бывает ситуации, когда у нас изначально есть некая группа, у нее две подгруппы и в какой-то момент становится 
ясно, что группа устроена как прямое произведение этих самых подгрупп. В качестве примера такой группы можно взять 
комплексные числа. Группа комплексных чисел по умножению изоморфна группе пар, где первый элемент пары -- 
модуль комплексного числа, а второй -- тригонометрическая часть. 
