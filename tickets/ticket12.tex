\section{Подстановка оператора в многочлен. Аннуляторы оператора и вектора}
\begin{conj} (Подстановка оператора в многочлен)

    Пусть $ f \in K[x], f = a_m x^m + \dots + a_1x + a_0$.
    Тогда подстановка оператора $\A$ в многочлен $f$ это выражение вида
    \[ f(\A) = a_m \A^m + \dots + a_1\A + a_0 \mathcal{E}_V \in \End V \quad (\A^i \text{--- композиция \,}  i \text{\, операторов} ) \]
\end{conj}

\begin{theorem} Свойства подстановки оператора в многочлен.

    Пусть $f, g \in K[x]$. Тогда:
    \begin{enumerate}
        \item $ (fg)(\A) = f(\A) g(\A)$.
        \item $ f(\A) g(\A) = g(\A) f(\A)$.
        \item $ \Ker f(\A) , \Imm f(\A)$  ---  $\A $-инвариантные подпространства.
    \end{enumerate}
    \begin{proof} \quad
    
    \begin{enumerate}
        \item Очевидно из перемножения многочленов.
        \item Следует из первого через $fg = gf$ (ведь многочлены всегда коммутируют).
        \item Сперва про ядро. Пусть $v \in \Ker f(\A)$ и надо доказать, что $\A v \in \Ker f(a)$.
        Введем $g(x) = x$. Тогда: \begin{gather*}
            \A v = g(\A) (v) \\
            \Rightarrow f(\A) (\A v) = f(\A) g(\A) (v) = g(\A) \cdot \underbrace{f(\A) (v)}_{\text{$ 0 $}} = 0 \Longrightarrow \A v \in \Ker f(\A)
        \end{gather*}
        
        Теперь про образ. Пусть $v \in \Imm f(\A)$ и надо доказать, что $\A v \in \Imm f(\A)$.
        Введем $g(x) = x$. Тогда: \begin{gather*}
            v \in \Imm f(\A) \Longleftrightarrow v = f(\A) (w) \\
            \Rightarrow \A v = g(\A) f(\A) (w) = f(\A) g(\A) (w) \in \Imm f(\A)
        \end{gather*}
    \end{enumerate}
    \end{proof}
\end{theorem}

Вспомним, что кольцо многочленов является евклидовым кольцом, что в свою очередь является областью главных идеалов (т.е. любой идеал евклидова кольца порождён одним элементом). 
Этот факт уже доказывался в первом семестре, но Жуков решил повторить доказательство. 

\begin{lemma}
    В евклидовом кольце любой идеал главный.

    \begin{proof}
    Пусть $R$ -- евклидово кольцо, $I \subset R$ его идеал (т.е. это подгруппа, а также $\forall b \in R$ выполняется $bI \subset I$) и $\nu$ -- евклидова норма.
    
    \quad Случай $I = 0$ тривиален, так как тогда $I = (0)$.

    \quad Пусть теперь $I \neq 0$. 
    Зафиксируем элемент $a \in I$, у которого евклидова норма минимальна (она принимает целые неотрицательные значения, поэтому такой точно найдется).
    Докажем, что $I = (a)$. Пусть $ b \in I$. Поделим на $a$ с остатком:
    \begin{gather*}
        b = aq + r, \quad \text{ где } \nu(r) < \nu(a) \text{ или } r = 0
    \end{gather*}
    \quad Первый случай невозможен, так как $ \nu(a) $ минимальна. 
    Значит, остаток равен нулю, тогда $b = aq$, т.е. $b \in (a)$.
    \end{proof}
\end{lemma}

\vspace*{3mm}

Этот факт мы сейчас будем использовать.

\begin{conj} 
    Будем называть $f \in K[x]$ аннулятором (аннулирующим многочленом) $\A$, если $f(\A) = 0$.
\end{conj}

\begin{theorem}
    Множество аннуляторов оператора -- главный идеал в $K[x]$.
\end{theorem}
\begin{proof}
    Очевидно, что свойства группы выполняются, поэтому осталось только доказать, что это идеал.
    Пусть $ f \in I$. Тогда:
    \[ \forall g \in K[x]: \, (gf)(\A) = g(\A) \underbrace{f(\A)}_{0} = 0 \Longrightarrow gf \in I \]
    \quad Этот идеал главный, так как $K[x]$ евклидово кольцо.
\end{proof}

Раз идеал главный, логично было бы как-то выделить его образующий.

\vspace*{3mm}

\begin{conj}
    Пусть $\A \in \End V$, $ I = \{ f \,  |  \, f$ --- аннулятор $\A \} = (\mu_{\A})$.
    Тогда $\mu_{\A}$ --- минимальный многочлен оператора $\A$.
\end{conj}

\vspace*{3mm}

\notice Порождающий элемент главного идеала определён с точностью до ассоциированности, поэтому минимальный многочлен не единственнен.
Но зато это дает нам право считать, что он унитарный (старший коэффициент 1).


\notice А вдруг это нулевой идеал? О каком унитарном многочлене вообще тогда речь? 
Такого быть не может! Дело в том, что операторы образуют конечномерное пространство: $\dim \End V = n^2$, где $n = \dim V$.
Тогда \begin{gather*}
    \{ \mathcal{E}, \A, \A^2, \dots, \A^{n^2} \} - \text{ ЛЗС (т.к. мы перечислили $n^2 + 1$ элемент нашего пространства)} \\
    \Longrightarrow \, \exists \alpha_0, \alpha_1, \dots, \alpha_{n^2} \text{ не все 0 } : \alpha_0 \mathcal{E} + \alpha_1 \A \dots + \alpha_{n^2} \A^{n^2} = 0 \\
    \Longrightarrow f = \alpha_0 + \alpha_1 x + \dots + \alpha_{n^2} x \text{--- ненулевой аннулятор $\A$}
\end{gather*}

\begin{conj}
    Пусть $v \in V, \A \in \End V$.
    Говорят, что $f \in K[x]$ -- аннулятор вектора $v$ (по отношению к оператору $\A$), если $f(\A) (v) = 0$.
\end{conj}