\section{Лекция номер 11}
Квадратная матрица $A \in M(n, K)$ называется ортогональной, если она обратна к своей транспонированной: $AA^T = E_n$. 
Давайте поймем, какие критерии данное определение налагает на $A$. Пусть $A = (a_{ij})$, тогда: 
\begin{gather*}
    AA^T [i, j] = \sum\limits_{k}^n a_{i k} \cdot a_{jk}
\end{gather*}
Отсюда видим, что строки нашей матрицы $A$ ортогональны, а также нормированны.
\begin{lemma}
    $A$ ортогональна $\Longrightarrow A^T$ ортогональна
    \begin{gather*}
        A^T A^{TT} = \underbrace{A^T}_{= A^{-1}} A = E_n
    \end{gather*}
\end{lemma}
\begin{conj}
    Введем обозначение для множества ортогональных матриц: 
    \begin{gather*}
        O(n, K) = \{ A \in M(n, K) \mid A \text{ ортогональна } \}
    \end{gather*}
\end{conj}
\begin{theorem-non}
    Множество ортогональных -- подгруппа в множестве обратимых матриц.
\end{theorem-non}
\begin{proof} \quad 

    \begin{itemize}
        \item Проверим замкнутость относительно умножения: 
            \begin{gather*}
                A, B \in O(n, K) \\
                (AB)(AB)^T = A\underbrace{B B^T}_{E_n} A^T = A A^T = E_n 
            \end{gather*}
        \item Проверим замкнутость относительно взятия обратного: 
            \begin{gather*}
                A \in O(n, K) \\
                A^{-1} (A^{-1})^T = A^T A^{TT} = E_n
            \end{gather*}
    \end{itemize}
    --- ортогональная группа степени $n$ над полем $K$
\end{proof}

\begin{theorem-non}
    $V$ евклидово протранство, $E$ -- ортонормированный базис, $E'$ -- какой-либо базис $V$,
    $C$ -- матрица перехода от $E$ к $E'$. Тогда $E'$ ортонормированный $\Longleftrightarrow C \in O(n, K)$ 
\end{theorem-non}
\begin{proof}
    Пусть $\Gamma_E$ -- матрица Грама скалярного произведения в базисе $E$. Тогда
    \begin{gather*}
        \Gamma_{E'} = C^T \underbrace{\Gamma_E}_{E_n} C \\ 
        \Gamma_{E'} = E_n \Longleftrightarrow C^T C = E_n \Longleftrightarrow C \in O(n, K)
    \end{gather*} 
\end{proof}

Теперь поговорим о понятии ортогонального дополнения к подпространству. 
\begin{conj}
    $U < V$ -- линейное подпространство. Ортогональное дополнение к $U$ -- это: 
    \begin{gather*}
        U^\perp = \{ v \in V \mid \forall u \in U : u \perp v \}
    \end{gather*}
\end{conj}

\begin{theorem-non} \quad

    Зафиксируем, что $\dim V < +\infty$

    \begin{enumerate}
        \item $U^\perp$ -- линейное подпространство $V$
        \begin{proof}
            $(v_1 + v_2, u) = (v_1, u) + (v_2, u), \qquad (\alpha v, u) = \alpha (v, u)$
        \end{proof}
        \item $V = U \oplus U^\perp$
        \begin{proof}
            Пусть $f_1, \dots, f_m$ -- какой-либо базис $U$, а $f_{m+1}, \dots, f_n$ -- его дополнение до базиса $V$. 
            Существует $e_1, \dots, e_n$ -- ортонормированный базис $V$, такой, что $\forall l: \Lin{(e_1, \dots, e_l)} = \Lin{(f_1, \dots, f_l)}$. 
            В частности, $\Lin{(e_1, \dots, e_m)} = U$. 
            
            Поймем, из чего состоит ортогональное дополнение к $U$. 
            \begin{align*}
                v &= \alpha_1 e_1 + \dots + \alpha_n e_n \in U^\perp \\
                &\Longleftrightarrow v \perp e_i, i=1, \dots, m \\
                &\Longleftrightarrow \alpha_i = 0, i=1, \dots, m \; (\text{так как } (v, e_i) = \alpha_i) \\
                &\Longleftrightarrow v \in \Lin{(e_{m+1}, \dots, e_n)}
            \end{align*}
            Таким образом, $U^\perp = \Lin{(e_{m+1}, \dots, e_n)} \Longrightarrow V = U \oplus U^\perp$
        \end{proof}
        \item $U_1 \subset U_2 \Longrightarrow U_1^\perp \supset U_2^\perp$
        \begin{proof}
            Тривиально.
        \end{proof}
        \item $(U^\perp)^\perp = U$
        \begin{proof} \quad 

            \begin{itemize} 
                \item[``$\supset$''] Очевидно, так как любой вектор из $U$ ортогонален ко всем векторам из ортогонального дополнения по определению. 
                \item[``$\subset$''] Из второго свойства знаем, что $\dim{U^\perp} = \dim{V} - \dim{U}$. Применим это свойство еще раз и получим, что $\dim{(U^\perp)^\perp} = \dim{V} - \dim{U^\perp} = \dim{U}$
                
                Раз размерности равны, и у нас есть одно включение, то получаем равенство.
            \end{itemize}
        \end{proof}
        \item $(U_1 + U_2)^\perp = U_1^\perp \cap U_2^\perp$
        \begin{proof}
            $v \in (U_1 + U_2)^\perp \Longleftrightarrow v \in U_1^\perp$ и $v \in U_2^\perp$
        \end{proof}
        \item $(U_1 \cap U_2)^\perp = U_1^\perp + U_2^\perp$
        \begin{proof}
            Выводится из пятого и четвертого пунктов: 
            \begin{align*}
                (U_1 \cap U_2)^\perp &= \left( (U_1^\perp)^\perp \cap (U_2^\perp)^\perp \right)^\perp \\
                &= \left( (U_1^\perp + U_2^\perp)^\perp \right)^\perp \\
                &= U_1^\perp + U_2^\perp  
            \end{align*}
        \end{proof}
        \item $V^\perp = 0, \; 0^\perp = V$
        \begin{proof}
            Очевидно благодаря равенству размерностей: 
            \begin{gather*}
                \dim{V^\perp} = \dim{V} - \dim{V} = 0 \\
                0^\perp = (V^\perp)^\perp = V
            \end{gather*}
        \end{proof}
    \end{enumerate}
\end{theorem-non}
\begin{conj}
    Пусть у нас есть некоторое $U$ -- подпространство $V$. 
    Если мы раскладываем вектор $v$ в сумму двух векторов, первый из которых лежит в $U$, а второй -- в $U^\perp$, то они имеют специальные 
    названия: 
    \begin{gather*}
        u = u_1 + u_2, u_1 \in U, u_2 \in U^\perp
    \end{gather*}
    $u_1$ -- это ортогональная проекция $u$ на $u_1$. \\
    $u_2$ -- это ортогональное дополнение $u$ по отношению к $u_1$.
\end{conj}

Отметим, что длина ортогонального дополнения, это и есть в точности расстояние от точки до подпространства. 
Напомним, что в евклидовом пространстве у нас есть метрика $\rho(u, v) = \norm{u - v}$. Введем расстояние от точки до множества:
    
Пусть $M$ --- метрическое пространство, $x \in M, N \subset M$, тогда: 
\begin{gather*}
    \rho(x, N) = \inf\limits_{\mu \in N} \rho(x, \mu)
\end{gather*}

\begin{theorem-non}
    Пусть $V$ -- конечномерное евклидово пространство. $U \subset V$ -- подпространство. 
    $v \in V$, сразу разложим его в сумму ортогональной проекции и ортогонального дополнения: $v = u_1 + u_2, u_1 \in U, u_2 \in U^{\perp}$. Тогда: 
    \begin{gather*}
         \rho(v, U) = \norm{u_2}, \quad \rho(v, U) = \rho(v, u_1)
    \end{gather*}
\end{theorem-non}
\begin{proof} \quad

        Зафиксируем произвольное $z \in U$ и посчитаем расстояние от $v$ до $u_1 + z$. 
        \begin{align*}
            \rho(v, u_1 + z)^2 &= \rho(u_1 + u_2, u_1 + z)^2 \\
            &= \norm{z - u_2}^2 \\
            &= (z - u_2, z - u_2) \\
            (\text{так как } z \perp u_2) &= {\underbrace{\norm{z}}_{\geqslant 0}}^2 + \norm{u_2}^2 \\
            &\geqslant \norm{u_2}^2
        \end{align*}
        То есть $\rho(v, u_1 + z)^2$ достигает минимума при $\norm{z} = 0$ и в таком случае будет равна $\norm{u_2}^2$

        Таким образом: 
        \begin{gather*}
            \inf\limits_z \rho(v, u_1 + z) = \norm{u_2}
        \end{gather*}
        А этот инфимум и является расстоянием $\rho(v, U)$, то есть $\rho(v, u_1) = \norm{u_2}$
    \end{proof}

\subsection{Унитарные пространства}
Сейчас мы начинаем обширно работать с пространствами над полем комплексных чисел, так что 
давайте перенесем ряд понятий, которые мы вводили в вещественном случае на комплексный. 

Унитарное пространство -- комплексный аналог понятия евклидова пространства. Также может быть названо 
Эрмитовым пространством. Пусть $V$ -- линейное пространство над $\C$. В качестве скалярного произведения выступает так 
называемая полуторалинейная форма. 

\begin{conj}
    Полутролинейная форма на $V$ -- это отображение: 
    \begin{gather*}
        \B: V \times V \longrightarrow \C
    \end{gather*}
    Такое, что выполняются следующие свойства: 
    \begin{enumerate}
        \item $\B$ линейно по 1 аргументу: $\B(\alpha_1 v_1 + \alpha_2 v_2, w) = \alpha_1 \B(v_1, w) + \alpha_2 \B(v_2, w)$
        \item $\B$ полулинейно по 2 аргументу: $\B(v, \alpha_1 w_1 + \alpha_2 w_2) = \bar{\alpha_1} \B(v, w_1) + \bar{\alpha_2}\B(v, w_2)$
    \end{enumerate}
    \underline{Примеры:}
    \begin{enumerate}
        \item Стандартное скалярное произведение в $V = \C^n$:
        
        \begin{gather*}
            \B\left(\left(\begin{array}{c}
            \alpha_1 \\ 
            \vdots \\ 
            \alpha_n
            \end{array}\right), 
            \left(\begin{array}{c}
            \beta_1 \\ 
            \vdots \\ 
            \beta_n
            \end{array}\right)\right) 
            = \alpha_1 \bar{\beta_1} + \dots + \alpha_n \bar{\beta_n}
        \end{gather*}

        Преимущество относительно обычного скалярного произведения заключается в том, что если мы возьмем любой такой столбец и
        скалярно умножим на себя, то получим неотрицательное вещественное число, а если мы будем так делать без комплексного 
        сопряжения, то может получиться все что угодно.

        \item $V = C_{\C} [0, 1]$ -- непрерывные комплексно-значные функции на отрезке $[0, 1]$.
        \begin{gather*}
            \B(f, g) = \int\limits_0^1 f \cdot \bar{g}
        \end{gather*}
    \end{enumerate}    
\end{conj}

Заметим, что, как и обычно, можно определить матрицу Грама. 
Пусть $E$ -- базис $V$. $e_1, \dots, e_n$ -- вектора $E$. Тогда есть матрица Грама $[\B]_E = (\B(e_i, e_j))$. 
Если у нас есть произвольные векторы $v = E \cdot b$ и $w = E \cdot c$. Тогда: 
\begin{gather*}
    \B(v, w) = b^T \cdot [\B]_E \cdot \bar{c}
\end{gather*}
\underline{Fan fact}:
Если базис $E' = EC$, то $[\B]_{E'} = C^T \cdot [\B]_{E} \cdot \bar{C}$

Введем понятие эрмитовой формы, которая является аналогом симметрической билинейной формы в комплексном случае. 
\begin{conj}
    Эрмитовой формой на $V$ называется полутролинейная форма $\B$, такая что:  
    \begin{gather*}
        \forall u, v: \B(v, u) = \overline{\B(u, v)}
    \end{gather*}
\end{conj}
Для таких форм справедлива теорема Лагранжа. Доказательство абсолютно аналогично вещественному случаю. 
\begin{theorem}
    Пусть $V$ конечномерно, $\B$ --  эрмитова форма на $V$. Тогда $\exists$ базис $E$, такой, что матрица Грама $[\B]_E$ диагональна.
\end{theorem}

\notice $\B$ эрмитова $\Longleftrightarrow [\B]^*_E  = [\B]_E$, где оператор ``$*$'' -- это: 
\begin{conj}
    Пусть $A \in M(m, n, \C)$. Тогда $A^* := \overline{(A^T)}$ --- матрица, сопряженная к $A$.
\end{conj}
В частности, если $[\B]_E$ диагональна, то она вещественна.

Диагонализировать форму можно разумеется по разному, но, опять таки, для эрмитовой формы справедлив закон инерции:
Число положительных и число отрицательных чисел в диагональной матрице $[\B]_E$ -- инварианты $\B$.

\begin{conj}
    Скалярным произведением на $V$ называется положительно определённая эрмитова форма на V, то есть такая, что
    \begin{enumerate}
        \item $\forall v \in V: \B(v, v) \geqslant 0$
        \item $\B(v, v) = 0 \Longleftrightarrow v = 0$
    \end{enumerate}
\end{conj}

Иначе говоря, число положительных чисел на диагонали должно совпадать с размерностью пространства. 

\begin{conj}
    Унитарным пространством называется линейное пространство над $\C$ с фиксированным скалярным произведением.
\end{conj}

\begin{conj}
    Базис $e_1, \dots, e_n$ унитарного пространства V называется ортонормированным,
    если $\forall i, j: \, (e_i, e_j) = \delta_{ij}$. Иными словами, если матрица Грама этого базиса представляет собой единичную матрицу: 
    $\Gamma_E = E_n$. 
\end{conj}

Пусть $E' = EC$, где $C \in GL(n, \C)$, мы знаем, что $\Gamma_{E'} = C^T \cdot \Gamma_E \cdot \bar{C}$. 
\begin{align*}
    E' \text{ ортонормированный } &\Longleftrightarrow \Gamma_{E'} = E_n \\
    &\Longleftrightarrow C^T \cdot \bar{C} = E_n \\
    &\Longleftrightarrow C^* \cdot C = E_n
\end{align*}

\begin{conj}
    Матрица $C \in GL(n, \C)$ называется унитарной, если $C^* \cdot C = E_n$
\end{conj}

\begin{conj}
    Множество унитарных матриц: $U(n) := \{ C \in GL(n, \C) \, | \, C$ --- унитарна\}
\end{conj}

\begin{theorem}
    $U(n) < GL(n, \C)$ -- унитарная группа степени $n$.
\end{theorem}

\begin{theorem} (Неравенство КБШ). Как и обычно, $\norm{v} = \sqrt{(v, v)}$. 
    \begin{gather*}
        \forall u,v \in V: \abs{(u, v)}^2 \leqslant \norm{u} \cdot \norm{v}
    \end{gather*}
\end{theorem}
Пруфов теорем не будет, автор принял Линал. 

\subsection{Комплексификация}

Пусть $V$ -- Линейное пространство над $\R$. По нему мы построим $V_{\C}$ -- линейное пространство над $\C$.
Как множество, оно будет опреляться следующим образом: $V_{\C} = V \times V$. 

Теперь введем на этом множестве необходимые операции: 
\begin{itemize}
    \item Сложение: $(v, w) + (v', w') = (v + v', w + w')$
    \item Умножение на комплексное число: $(\alpha + i \beta)(v, w) = (\alpha v - \beta w, \alpha w + \beta v)$ 
\end{itemize}
Несложно проверить, что $V_{\C}$ с этими операциями -- линейное пространство над $\C$.

\begin{theorem-non}(Базис $V_{\C}$)
    Пусть $e_1, \dots, e_n$ -- базис $V$. Тогда $(e_1, 0), \dots, (e_n, 0)$ --- базис $V_{\C}$
\end{theorem-non}
\begin{proof} \quad 

    Для начала заметим, что $(\alpha + i \beta) \cdot (v, 0) = (\alpha v, \beta v)$. В частности: 
    \begin{gather}
        \sum\limits_{j=0}^{n} (\alpha_j + i \beta_j)(e_j, 0) = \left( \sum \alpha_j e_j, \sum \beta_j e_j \right)
    \end{gather}
    
    Таким образом, $\{ \Lin(e_j, 0), j = 1, \dots, n \} = V_{\C}$. Линейная независимость векторов тоже видна из равенства $(2)$. 
    Если сумма $(2)$ оказалась равна 0, то $\alpha_j = \beta_j = 0, j = 0, \dots, n$
\end{proof}

\follow $\dim_\C V_{\C} = \dim_\R V$

\begin{conj}
    $V_{\C}$ называется комплексификацией $V$
\end{conj}

Заметим, что любой вектор оттуда можно записать как $(v, w) = (v, 0) + (0, w) = (v, 0) + i(w, 0)$.
Так что захотелось отождествить $(v, 0)$ с $v$. И тогда пара $(v, w) = v + iw$.

Важно отметить на будещее, что из разнообразных базисов в $\R$ мы будем получать базисы комплексификаций $V$, 
но при этом, у комплексификаций есть и свои отдельные базисы, где не все вектора будут вещественными. 

\begin{theorem-non}
Пусть $V$ евклидово пространство, $V_{\C}$ --- его комплексификация

$(v + iw, v' + iw') = (v, v') + (w, w') + i((w, v') - (v, w'))$ -- скалярное произведение на $V_{\C}$, продолжающее 
скалярное произведение на $V$.
\end{theorem-non}
\begin{proof} \quad

    \begin{itemize}
        \item Линейность по первому аргументу тривиальна
        \item $(u, u') = \overline{(u', u)}$ (мнимая часть поменяет знак)
        \item $(v + iw, v + iw) = (v, v) + (w, w) + 0 \geqslant 0$. И равняется $0$ только если мы брали скалярный квадрат нулевого вектора, то есть при $v = w = 0$
    \end{itemize}
\end{proof}
\notice Пусть $E$ -- базис $V$. Тогда 
$(e_i, e_j)_V = (e_i, e_j)_{V_{\C}}$

Таким образом, если взять какой то базис, то Матрица Грама будет той же самой, когда мы перейдем в унитарное пространство. А также
если базис был ортонормированным, то и получающийся из него, комплексный базис будет ортонормированным. 

\subsection{Двойственное пространство}
\begin{conj}
    Пусть $V$ -- линейное пространство над $K$.
    Двойственным к $V$ пространством называют пространство линейных отображений из $V$ в $K$.
    (Еще встречаются названия ``дуальное'' или ``сопряженное'' пространство).
    \begin{gather*}
        V^* := \Hom(V, K)
    \end{gather*}
    Элементы $V^*$ называют линейными функционалами на $V$.
\end{conj}

\underline{Пример:}
Пусть $V = C[0, 1]$ -- пространство непрерывных функций на отрезке $[0, 1]$. 
Примером линейного функционала $\in V^*$ будет $\varphi$:
\begin{gather*}
    \varphi : f \longmapsto f(0)
\end{gather*}
\begin{theorem-non}
    Пусть $\dim V = n < +\infty$. Тогда $\dim V^* = n$
\end{theorem-non}
\begin{proof} \quad 

    Доказывать собственно нечего, так как у нас был следующий общий факт: 

    $V, W$ --- конечномерные. 
    \begin{gather*}
        \Hom(V, W) \cong M(m, n, K), \text{ где } m = \dim V, n = \dim W \\
        \Longrightarrow \dim \Hom(V, W) = mn
    \end{gather*}

    Ну а в нашем случае второе пространство имеет размерность 1, значит 
    мы просто получаем размерность первого пространства.
\end{proof}
Если у нас зафиксирован некий базис в первом пространстве, то можно каноническим образом построить 
так называемый ``двойственный'' базис во втором пространстве.
\begin{conj}
    Пусть $e_1, \dots, e_n$ --- базис $V$
    \begin{gather*}
        e^i : V \longrightarrow K \\
        e_j \longmapsto \delta_{ij}
    \end{gather*}
    
\end{conj}

\begin{theorem-non}
    $(e^1, \dots, e^n)$ -- базис $V^*$
\end{theorem-non}
\begin{proof} \quad
    
    Так как мы уже знаем размерность $V^*$, осталось проверить линейную независимость. 

    Пусть некая линейная комбинация $\alpha_1 e^1 + \dots + \alpha_n e^n = 0$.
    Подставим в нее $e_j$ и получим 
    \begin{gather*}
        (\alpha_1 e^1 + \dots + \alpha_n e^n) (e_j) = 0
    \end{gather*}
    То есть получим, что все $\alpha_j = 0$. 
    Выходит они линейно независимы и их количество равно размерности. Значит это базис.
\end{proof}

\begin{conj}
    $(e^1, \dots, e^n)$ -- двойственный базис к $e_1, \dots, e_n$
\end{conj}