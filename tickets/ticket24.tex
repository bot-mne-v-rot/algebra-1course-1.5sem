\section{Положительно определённые симметрические билинейные (квадратичные) формы. Классификация над $\C$}
\begin{theorem-non}
    Пусть $V$ -- конечномерное линейное пространство над $\C$, $\B$ -- симметрическая билинейная форма. Тогда существует 
    базис $E$ пространства $V$, такой, что: 
    \begin{gather*}
        [\B]_E = \begin{pmatrix*}
            E_r & 0 \\
            0 & 0
        \end{pmatrix*} 
    \end{gather*} 
    Для некоторого $r$ -- инварианта $\B$. То есть такое условие может выполняться для нескольких различных базисов, но $r$ всегда будет получться одно и то же. 
\end{theorem-non}
\begin{proof}
    Так как диагонализировать мы уже умеем, существует $E_0 = (e_1, \dots, e_n)$, такой, что: 
    \begin{gather*}
        [\B]_{E_0} = \diag{\lambda_1, \dots, \lambda_n}
    \end{gather*}
    Мы можем перенумеровать лямбды так, что $\lambda_1, \dots, \lambda_r \neq 0$, а $\lambda_{r+1} = \dots = \lambda_n = 0$. 

    Для любого $j$ существует $\alpha_j \in \C : \alpha_j^2 = \lambda_j$. 
    Построим $E = (\alpha^{-1}_1 e_1, \dots, \alpha^{-1}_r e_r, e_{r+1}, \dots, e_n)$. 
    Тогда: 
    \begin{gather*}
        [\B]_E = \diag{\underbrace{\alpha_1^{-2} \lambda_1, \dots, \alpha_r^{-2} \lambda_r}_{1, \dots, 1}, \underbrace{\lambda_{r+1}, \dots, \lambda_n}_{0, \dots, 0}}
    \end{gather*}
    То есть мы получили матрицу, нужного нам вида. Осталось понять, что $r$ -- инвариант. Для этого хватает вспомнить, что: 
    \begin{gather*}
        r = \rk{\begin{pmatrix*}
            E_r & 0 \\
            0 & 0
        \end{pmatrix*}} = \rk{\B}
    \end{gather*}
    Так что $r$ -- очевидным образом инвариант. 
\end{proof}