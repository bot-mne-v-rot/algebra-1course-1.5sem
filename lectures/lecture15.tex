\section{Лекция номер 15}
\subsection{Поля разложения}

\begin{conj}
    Пусть $K$ -- поле, $f \in K[x]$ -- некоторый многочлен, $L/K$ -- расширение. Тогда $L$ называется \textbf{полем разложения} $f$, если:
    \begin{enumerate}
        \item $f$ там раскладывается на линейные множетели. $f = a_0 \cdot (X - x_1) \dots (X - x_n)$ в $L[X]$.
        \item $L$ получается из $K$ присоединением всех этих корней. $L = K(x_1, \dots, x_n)$. То есть $L$ является наименьшим полем с таким свойством. 
    \end{enumerate}
\end{conj}
Примеры:
\begin{enumerate}
    \item $\Q\left(\sqrt 2\right)$~--- поле разложения $X^2-2$ над $\Q$
    \item $\Q\left(\sqrt[3]{2}\right)$~--- не поле разложения $X^3-2$ над $\Q$ (содержит только один корень)
    \item $\Q\left(\sqrt[3]{2}, \omega\right)$ (где $\omega$~--- какой-то из комплексных корней $X^3-2$)~--- поле разложения $X^3-2$ над $\Q$
    \item $\C$~--- не поле разложения $X^3-2$ над $\Q$ (все корни содержит, но не минимально).
\end{enumerate}

\begin{theorem}
    Пусть $K$ -- поле, $f$ -- его многочлен. Тогда у $f$ существует поле разложения над $K$. 
\end{theorem}
\begin{proof}
    Индукция по степени $f$. 

    \quad \underline{База $f = 1$:} Тогда у него есть единственный корень и он лежит в $K$, то есть $K$ -- само поле разложения. 

    \quad \underline{Переход:} Пусть $f_0$ -- любой неприводимый делитель $f$.
    Возьмем $L := K[X] / (f_0)$, тогда, как мы выясняли ранее, $L=K(x_1)$, где $x_1$~--- некоторый корень $f_0$ и, соответственно, $f$.
    Тогда по теореме Безу в $L$ многочлен $f$ раскладывается как $f=(X-x_1)g$, где $\deg g = \deg f - 1$.
    Тогда по предположению индукции существует поле разложения $g$ в $L$, равное $M=L(x_2, x_3, \ldots, x_n)$, где $x_2,x_3,\ldots, x_n$~--- оставшиеся корни $f$.
    Но тогда $M=L(x_2, x_3, \ldots, x_n) = (K(x_1))(x_2, x_3, \ldots, x_n) = K(x_1, x_2, \ldots, x_n) \Rightarrow$ $M$ является полем разложения $f$ над $K$.
\end{proof}
\notice На самом деле поле разложения единственно с точностью до изоморфизма, но доказывать мы это не будем.

\subsection{Классификация простых полей}

\begin{conj}
    Поле называется \textbf{простым}, если оно не содержит собственных (т.е. отличных от себя самого) подполей.
\end{conj}

\begin{theorem-non}
    \begin{enumerate}
        \item Любое простое поле изоморфно либо $\Q$, либо $\mathbb{F}_p$
        \item Любое поле содержит единственное простое подполе.
    \end{enumerate}
\end{theorem-non}
\begin{proof}
    Рассмотрим произвольное поле $K$ и гомоморфизм $\varphi$, действующий
    \begin{align*}
        & \qquad \Z \longrightarrow K \\
        & n \mapsto \underbrace{1+1+\cdots+1}_{n \text{ раз}}\quad (\forall n\in \N) \\
        & 0 \mapsto 0 \\
        & -n \mapsto -(\underbrace{1+1+\cdots+1}_{n \text{ раз}})\quad (\forall n\in \N)
    \end{align*}

    Если $\text{char } K = p$, то $\Ker \varphi = p\Z$, отсюда по теореме о гомоморфизме для групп
    \begin{gather*}
        \Z / p\Z \cong \Imm \varphi
    \end{gather*}

    (Теорема дает нам только изоморфизм этих множеств как групп по сложению, однако тривиально проверяется, что $\varphi$ сохраняет умножение, поэтому это еще и изоморфизм колец с единицей).
    При этом мы знаем, что $\Z / p\Z$ является полем, так что $\Imm \varphi$ тоже будет полем, т.е. $\Imm \varphi$~--- подполе в $K$.
    Поэтому если мы (как в пункте $1$) хотим чтобы $K$ было простым, то необходимо $K=\Imm \varphi$
    Если мы (как в пункте $2$) рассматриваем произвольное $K$, то мы нашли его простое подполе.\medskip

    Если теперь $\text{char} K = 0$, то это значит $\Ker \varphi = 0$, что позволяет нам рассмотреть отображение $\varphi'$:
    \begin{gather*}
        \Q \longrightarrow K \\
        \frac mn \mapsto \frac{\varphi(m)}{\varphi(n)}
    \end{gather*}

    У него тоже нулевое ядро, поэтому по теореме о гомоморфизме
    \begin{gather*}
        \Q \cong \Imm \varphi'
    \end{gather*}
    При этом $\Q$ является полем $\Rightarrow$ $\Imm \varphi'$ будет подполем в $K$.
    Поэтому если (см. пункт $1$) $K$ простое, то $K=\Imm \varphi'$.
    А если (см. пункт $2$) $K$ произвольное, то мы нашли искомое простое подполе.\medskip

    Осталось проверить, что простое подполе единственно.
    Ну действительно, пусть $F_1, F_2$~--- простые подполя $K$.
    Пересечение полей тоже будет полем, поэтому $F_1 \cap F_2$ будет подполем в $F_1$ и $F_2$.
    Но все их подполя, по определению, собственные, поэтому $F_1 = F_1 \cap F_2 = F_2$.
\end{proof}

Таким образом, у любого поля существует простое подполе, полностью определяемое характеристикой исходного.
Отсюда вытекает

\follow Пусть $K$~--- конечное поле, $\text{char} K = p$. Тогда $|K|=p^n$ для некоторого $n\in\N$.
\begin{proof}
    Из доказательства предыдущей теоремы видно, что $\mathbb{F}_p$ (или изоморфное ему)~--- подполе в $K$.
    Но тогда $K$ можно рассматривать как линейное пространство над $\mathbb{F}_p$.
    Но это и значит, что $|K|=|\mathbb{F}_p|^n=p^n$, где $n = [K:\mathbb{F}_p]$.
\end{proof}

\subsection{Эндоморфизм возведения в степень $p$}
\begin{lemma}
    Пусть $K$~--- поле характеристики $p$.
    Тогда отображение $\varphi: a \mapsto a^p$ является эндоморфизмом (и называется \underline{эндоморфизмом Фробениуса}).
\end{lemma}
\begin{proof}
    Проверяем все необходимые свойства.
    \begin{enumerate}
        \item $\varphi(0)=0, \varphi(1)=1$~--- выполнены
        \item $\varphi(ab) = (ab)^p = a^p \cdot b^p = \varphi(a)\varphi(b)$~--- проверили, что умножение сохраняется
        \item $\varphi(a+b)=\varphi(a)+\varphi(b)$.
        Это неочевидное, но тоже верное свойство: $\varphi(a+b)=(a+b)^p = \sum_{k=0}^p \binom{p}{k} a^k b^{p-k}$.
        Если вспомнить, что $\binom{p}{k} = \frac{p!}{k!(p-k)!}$ и что $p$ простое, то можно заметить, что $p \mid \binom{p}{k}$ при $k\notin \{0, p\}$.
        А поскольку характеристика поля равна $p$, то получается, что $\binom{p}{k}=0$ при всех $k$ кроме $0, p$.
        Так что в сумме остаются только два слагаемых и получается $\varphi(a+b)=(a+b)^p = a^p + b^p = \varphi(a) + \varphi(b)$.
    \end{enumerate}
\end{proof}
\follow Если $K$ конечно, то такое $\varphi$ является автоморфизмом $K$.
\begin{proof}
    Очевидно (например т.к. поле является областью целостности), что $\Ker \varphi = 0$, поэтому $\varphi$ инъективен.
    А т.к. $K$ конечно, то это дает и сюръективность $\Rightarrow$ $\varphi$ биективен, что и означает, что это автоморфизм.
\end{proof}
\notice Когда из контекста ясно, о каком поле идет речь, автоморфизм Фробениуса на нем можно обозначать $Fr$.

\subsection{Существование поля из $p^n$ элементов}
Мы поняли, что размер поля имеет вид $p^n$, где $p\in\mathbb{P}$, $n\in \N$ (см. следствие из пункта про простые поля).
На самом деле, верно и обратное
\begin{theorem-non}
    Пусть $p$ простое, $n$~--- натуральное.
    Тогда существует поле из $p^n$ элементов.
\end{theorem-non}
\begin{proof}
    Для удобства введем $q := p^n$
    Пусть $L$~--- поле разложения многочлена $X^q-X$ над $\mathbb{F}_p$.
    Заметим, что у $X^q-X$ нет кратных корней.
    Действительно, в первом семестре мы проверяли что кратный корень должен быть также корнем формальной производной многочлена.
    Но производная равна $qX^{q-1}-1$.
    Поскольку характеристика поля равна $p$ а $p\mid q$, то первое слагаемое зануляется и остается константная $-1$ $\Rightarrow$ у производной корней нет.

    С другой стороны $X^q-X$ раскладывается на линейные множители в $L$ (ведь это поле разложения).
    Значит, можем рассмотреть множество $M$ корней многочлена $X^q-X$ в $L$ и для него будет верно $|M|=q$.
    Однако $M$ будет являться подполем!
    Проверим это.
    \begin{enumerate}
        \item $0,1\in M$~--- очевидно
        \item Замкнутость по умножению: пусть $a, b\in M$, т.е. $a,b$~--- корни $X^q-X$.
        Соответственно, $a^q=a, b^q=b$.
        Тогда $(ab)^q=a^q b^q = ab \Rightarrow$ $ab$ тоже корень $X^q-X$ $\Rightarrow$ $ab\in M$
        \item Замкнутость по сложению: пусть $a, b\in M$, опять же, $a^q=a, b^q=a$.
        Тогда $(a+b)^q = a^q + b^q = a + b$ $\Rightarrow$ $a+b \in M$ (равенство $(a+b)^q = a^q+b^q$ верно по тем же причинам, что и в эндоморфизме Фробениуса. Вообще, возведение в $q$-ю степень равносильно применению $n$ раз эндоморфизма Фробениуса, из этого тоже следует справедливость равенства).
        \item Замкнутость по взятию обратного: пусть $a^q \in M$, т.е. $a^q=a$. Тогда $(a^{-1})^q = (a^q)^{-1}=a^{-1} \Rightarrow a^{-1} \in M$.
        \item Замнкутость по взятию противоположного: пусть $a^q=a$, тогда $(-a)^q = (-1)^q a^q = -1 \cdot a = -a \Rightarrow (-a) \in M$  (Здесь стоит проверить, что $(-1)^q=-1$. Действительно, если $p\neq 2$, то $p$ нечетно $\Rightarrow$ $q$ нечетно $\Rightarrow$ $(-1)^q = -1$; если же $p = 2$, то $q$ четно и $(-1)^q=1$, но в поле с характеристикой $2$ выполняется $-1=1$, поэтому тоже все хорошо).
    \end{enumerate}\medskip

    Но тогда поле $M$ и является искомым!
    (Более того, поскольку мы выбирали $L$ как поле разложения $X^q-X$, а $M$ уже является полем и содержит все корни этого многочлена, то $L=M$).
\end{proof}

\subsection{Единственность поля из $p^n$ элементов}
\begin{lemma}
    Пусть $L/K$~--- расширение конечных полей.
    Тогда $L/K$ простое (т.е. $L=K(a)$ для некоторого $a\in L$).
\end{lemma}
\begin{proof}
    По теореме (примерно последняя доколлочная теорема) конечная подгруппа мультипликативной группы поля обязана быть циклической.
    Т.е. $L^* = \langle a \rangle$ для некоторого $a\in L$.
    Но тогда, очевидно, $L=K(a)$.
\end{proof}
\follow $\forall n\in \N$ существует неприводимый многочлен $f$, т.ч. $\deg f = n$.
\begin{proof}
    Рассмотрим поле из $p^n$ элементов.
    Оно является расширением степени $n$ над $\mathbb{F}_p$ и по лемме оно равно $\mathbb{F}_p(a)$.
    Тогда минимальный многочлен элемента $a$ и будет искомым.
\end{proof}

\begin{theorem}(Псевдотеорема Эйлера)
    Пусть $K$~--- поле, $|K|=q<\infty$, $a$~--- произвольный элемент $K$.
    Тогда $a^q=a$.
\end{theorem}
\begin{proof}
    Как мы поняли, $K^* = \langle d\rangle$ для некоторого $d$.
    Тогда если $a\in K^*$, то $a^{|K^*|}=1$, т.е. $a^{q-1}=1$ (это следствие из теоремы Лагранжа о том, что размер подгруппы делит размер группы~--- см. строение циклических групп), откуда $a^q=a$.
    В противном случае $a=0$ и тогда тоже, очевидно, $a^q=a$.
\end{proof}
\notice Название выдуманное, но будет встречаться пару раз в конспекте.
Если ссылаться на теорему, то лучше всего быстро передоказать / сказать ``очевидно''.

\begin{theorem}
    Пусть $p\in \mathbb{P}, n\in \N$, $F_1, F_2$~--- произвольные поля, такие что $|F_1|=|F_2|=p^n$.
    Тогда $F_1\cong F_2$.
\end{theorem}
\begin{proof}
    По лемме, $F_1=\mathbb{F}_p(a)$ для некоторого $a$.
    Пусть $f$~--- минимальный многочлен $a$.
    Рассмотрим многочлен $X^q-X$ (опять же, для удобства обозначем $p^n$ за $q$)
    По псевдотеореме Эйлера все элементы $F_1, F_2$ являются его корнями $\Rightarrow$ и там, и там он раскладывается на линейные множители (ибо мы нашли $q$ различных корней, а степень многочлена равна $q$).
    Более того, $a$ является корнем $X^q-X$, откуда, в силу минимальности $f$, $f \mid (X^q-X)$.
    Значит, $f$ раскладывается на линейные множители в $F_2$.

    Значит, можем рассмотреть $a'$~--- произвольный корень $f$ в $F_2$ (их там даже $n$, подойдет любой).
    Заметим, что $f$ будет минимальным многочленом для $a'$ (действительно, он является зануляющим $\Rightarrow$ должен делиться на минимальный. Но он неприводим $\Rightarrow$ делится (из тех, у кого старший коэффициент равен $1$) только на себя и на $1$, единица очевидно не подходит).
    Рассмотрим теперь $\mathbb{F}_p(a')$.
    Это поле изморфно $\mathbb{F}_p[X] / (f)$, а потому содержит $p^n$ элементов и при этом является подполем в $F_2$, которое тоже состоит из $p^n$ элементов.
    Значит, $F_2 = \mathbb{F}_p(a')$.
    Отсюда окончательно
    \begin{gather*}
        F_1 = \mathbb{F}_p(a) \cong \mathbb{F}_p[X] / (f) \cong \mathbb{F}_p(a') = F_2
    \end{gather*}
\end{proof}

\notice Теперь мы знаем, что для всякого $p^n$ существует ровно одно поле такого размера, поэтому можно ввести для него обозначение $\mathbb{F}_{p^n}$

\notice В доказательстве мы, вообщем-то, сразу поняли, что $F_1$ и $F_2$ являются полями разложения многочлена $X^q-X$.
Отсюда сразу следует, что $F_1\cong F_2$ т.к. поле разложения единственно, но последний факт мы не доказывали, поэтому им все же не стоит пользоваться.

\subsection{Группа автоморфизмов поля из $p^n$ элементов}
\begin{conj}
    $Aut(K)$~--- множество всех автоморфизмов поля $K$ (очевидно, это группа относительно композиции).

    Аналогично $Aut(L/K)$~--- группа всех автоморфизмов расширения (т.е. таких автоморфизмов $L$, которые каждый элемент из $K$ переводят в себя же).
\end{conj}
\begin{lemma}
    \begin{gather*}
        \left| Aut(K(a) / K) \right| \le [K(a) : K]
    \end{gather*}
\end{lemma}
\begin{proof}
    Пусть $f$~--- минимальный многочлен $a$ над $K$.
    Рассмотрим отображение $\lambda$, действующее в множество всех корней $f$.
    \begin{gather*}
        Aut(K(a)/K) \longrightarrow \{ x \in K(a) \mid f(x) = 0 \} \\
        \varphi \mapsto \varphi(a)
    \end{gather*}

    Надо проверить, что $\varphi(a)$ обязано являться корнем $f$.
    Действительно, пусть $f=\alpha_0 + \alpha_1 X + \cdots + \alpha_d X^d$.
    Тогда $\varphi(f(a)) = \varphi(\alpha_0) + \varphi(\alpha_1) \varphi(a) + \cdots + \varphi(\alpha_d) \varphi(a)^d = \alpha_0 + \alpha_1 \varphi(a) + \cdots + \alpha_d (\varphi(a))^d = f(\varphi(a))$ (поскольку коэффициенты многочлена были из $K$, то автоморфизм (по определению изоморфизма расширений) должен коэффициенты переводить в себя же).
    Полученное равенство дает $f(\varphi(a))=\varphi(f(a))=\varphi(0)=0$~--- проверили что образ $\lambda$ действительно содержится в множестве корней $f$.

    С другой стороны, автоморфизм такого расширения однозначно задается своим значением на $a$.
    Действительно, мы знаем структуру $K(a)$:
    \begin{gather*}
        K(a) = \{ \alpha_0 + \alpha_1 a + \alpha_{d-1}a^{d-1} \mid \alpha_0, \alpha_1, \ldots, \alpha_{d-1} \in K \}
    \end{gather*}
    Тогда зная лишь $\varphi(a)$ и держа в уме, что $\varphi$ все элементы $K$ по определению обязан переводить самих в себя, можем получить значение $\varphi$ для произвольного элемента из $K(a)$:
    \begin{gather*}
        \varphi(\alpha_0 + \alpha_1 a + \alpha_{d-1}a^{d-1}) = \alpha_0 + \alpha_1 \varphi(a) + \alpha_{d-1}\varphi(a)^{d-1}
    \end{gather*}
    Таким образом $\lambda$ инъективно.
    Значит, $\left|Aut(K(a)/K) \right|$ не превосходит числа различных корней $f$, т.е. $[K(a):K]$

    (Если коротко: мы убедились, что $\varphi(a)$ должен быть корнем $f$ и что $\varphi$ единственным образом восстанавливается по $\varphi(a)$, поэтому различных автоморфизмов не больше чем различных значений $\varphi(a)$, которых не больше чем корней $f$).
\end{proof}

\begin{theorem-non}
    $Aut(\mathbb{F}_{p^n})$~--- циклическая группа порядка $n$, порожденная автоморфизмом Фробениуса.
\end{theorem-non}
\begin{proof}
    Заметим, что любой автоморфизм должен переводить единицу в единицу, поэтому сумму $k$ единиц он должен перевести в сумму $k$ единиц, т.е. любой автоморфизм $\mathbb{F}_{p^n}$ должен переводить элементы $\mathbb{F}_p$ в себя же $\Rightarrow$ $Aut(\mathbb{F}_{p^n}) = Aut(\mathbb{F}_{p^n} / \mathbb{F}_p)$.
    Очевидно, группа автоморфизмов, порожденная $Fr$~--- подгруппа в в $Aut(\mathbb{F}_{p^n} / \mathbb{F}_p)$.
    Поэтому по лемме достаточно лишь проверить, что $|\langle Fr\rangle| = n$.
    Поскольку группа циклическая, то ее порядок равен минимальной степени $d$, такой, что $Fr^d=id$.
    При $d=n$ имеем $Fr^n$~--- отображение, переводящее $x \mapsto x^{p^n}$, однако $x^{p^n}=x$ по псевдотеореме Эйлера, поэтому $Fr^n = id$.
    При $d<n$ имеем $Fr^d$, которое действует $x \mapsto x^{p^d}$. 
    Заметим, что элементы, которые оно переводит в себя же, являются корнями многочлена $X^{p^d}-X$, которых не больше $p^d$, поэтому при $d<n$ всегда найдется элемент, который переводится не в себя $\Rightarrow$ при $d<n$, $Fr^d \neq id$.
    Таким образом, $|\langle Fr\rangle | = n$, что мы и хотели доказать.
\end{proof}

\subsection{Подполя из $p^n$ элементов}
\begin{theorem-non}
    Пусть $p\in\mathbb{P}, n\in \N$.
    Тогда
    \begin{enumerate}
        \item $\forall m: m\mid n$ в $\mathbb{F}_{p^n}$ есть подполе из $p^m$ элементов
        \item других подполей нет
    \end{enumerate}
\end{theorem-non}
\begin{proof}
    Сначала докажем второй пункт.
    Пусть $K$~--- подполе в $\mathbb{F}_{p^n}$.
    Тогда можем рассмотреть расширение $\mathbb{F}_{p^n} / K$.
    Если его степень равна $d$, то получается, что $\mathbb{F}_{p^n}$~--- $d$-мерное пространство над полем из $|K|$ элементов, откуда $|\mathbb{F}_{p^n}| = |K|^d$.
    Характеристика $K$~--- такая же, как у надполя, поэтому $|K|=p^m$ для некоторого $m$.
    Окончательно, $p^n = (p^m)^d = p^{md}$, откуда $|K| = p^m$ и $m \mid n$.\medskip

    Найдем теперь подполе из $p^m$ элементов.
    Пусть $n=mr$.
    Заметим, что в этом случае $(p^m-1) \mid (p^n-1)$.
    Действительно, $p^n = (p^m)^r$, $1=1^r$, поэтому можно расписать $p^n-1$ по формуле разностей $r$-х степеней и получить $(p^m)^r - 1^r = (p^m-1)((p^m)^{r-1} + (p^m)^{r-2}+\cdots + p^m + 1)$, отсюда сразу видно, что $p^n-1$ делится на $p^m-1$.
    После этого абсолютно аналогично доказывается, что $(X^{p^m-1}-1) \mid (X^{p^n-1}-1)$ (достаточно в предыдущем предложении заменить везде $p$ на $X$, $m$ на $p^m-1$, $n$~--- на $p^n-1$).
    Отсюда домножением на $X$ получаем, что $(X^{p^m}-X) \mid (X^{p^n}-X)$.
    Но мы-то помним, что $X^{p^n}-X$ раскладывается на попарно различные линейные множители (см. псевдотеорему Эйлера).
    Значит, $X^{p^m}-X$ тоже раскладывается на попарно различные линейные множители.
    Теперь, ровно как и в теореме о существовании поля из $p^n$ элементов, можем рассмотреть множество $M$ корней многочлена $X^{p^m}-X$ и убедиться, что оно будет подполем, да еще и нужного размера.
    Единственность такого подполя опять же следует из псевдотеоремы Эйлера: в любом поле из $p^m$ элементов для любого элемента должно выполняться $a^p=a$, но таких элементов в $\mathbb{F}_{p^n}$ всего $p^m$ и все они лежат в $M$, поэтому подполе из $p^m$ элементов обязано равняться $M$.
\end{proof}
\follow Эта теорема позволяет установить биекцию между подгруппами группы $Aut(\mathbb{F}_{p^n})$ и подполями $\mathbb{F}_{p^n}$.
Действительно, как мы убеждались, группа $Aut(\mathbb{F}_{p^n})$~--- циклическая порядка $n$ и порождена автоморфизмом Фробениуса $Fr$.
Тогда (согласно какому-то утверждению из доколлочной части второго семестра) все ее подгруппы имеют вид $\langle Fr^d\rangle $, где $d \mid n$.
Несложно проверить, что $\langle Fr^d\rangle $ будет группой всевозможных автоморфизмов расширения $\mathbb{F}_{p^n} / \mathbb{F}_{p^d}$ (т.е. таких автоморфизмов $\mathbb{F}_{p^n}$, которые переводят элементы $\mathbb{F}_{p^d}$ в себя же).
Таким образом подгруппе $\langle Fr^d \rangle$ естественным образом сопоставляется подполе $\mathbb{F}_{p^d}$ и такое сопоставление будет биективным.