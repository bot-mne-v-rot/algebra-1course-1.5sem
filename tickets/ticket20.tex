\section{Нормальные формы произвольного оператора}
Помимо жордановой формы существуют еще и другие, которые применимы уже к произвольным операторам.

В предыдущем параграфе мы поняли, что матрица любого оператора в нужном базисе примет вид:
\[
  [ \A ]_{E} = \left(\begin{array}{ccc}
  C(\mu_{\A, v_1}) &  & 0 \\ 
   & \ddots &  \\ 
  0 &  & C(\mu_{\A, v_t})
  \end{array}\right)  
\]
Пусть мы хотим сократить количество клеток в этой матрице, т.е разложить пространство в минимальную прямую сумму инвариантных подпространств.
Тогда, если выполняется условие этой леммы, мы можем это сделать:
\begin{lemma}
    Если $(\mu_{\A, v_1}, \mu_{\A, v_2}) = 1$ (т.е. они взаимно просты), то $L_{v_1} \otimes L_{v_2} = L_{v_1 + v_2}$.
\end{lemma}
\begin{proof}
    Упражнение. Надо бы доказать.
\end{proof}

Применяя эту лемму много раз, мы придем к так называемой Фробениусовой нормальной форме.
Именно она характеризуется минимальностью разложения в прямую сумму (то есть $t$ тут минимально):
    \[
  [ \A ]_{E} = \left(\begin{array}{ccc}
  C(f_1) &  & 0 \\ 
   & \ddots &  \\ 
  0 &  & C(f_t)
  \end{array}\right)  
\]
Тут $f_i$ уже не будут неприводимыми, но будут обладать таким свойством: $f_1 | f_2, f_2 | f_3, \dots, f_{t - 1} | f_t$.
Заметим, что такое разложение каноническое (однозначное), так как свойство делимости $(i + 1)$-го многочлена на $i$-тый запрещает нам переставлять клетки.

Приведем еще один вид матрицы произвольного оператора.
Пусть мы разложили наше пространство в прямую сумму инвариантных, т.е. матрица оператора является блочно-диагональной: \begin{gather*}
    \begin{pmatrix}
        D_1 &  & 0 \\ 
        & \ddots &  \\ 
        0 &  & D_m
    \end{pmatrix}
\end{gather*}
Тогда опять же, если правильно подобрать базис, каждый блок $D_i$ будет иметь вид: 
$$
\left[
    \begin{array}{c;{2pt/2pt}c;{2pt/2pt}c;{2pt/2pt}c;{2pt/2pt}cc}
        C(p) & 0 & \dots & \dots & 0 \\ \hdashline[2pt/2pt]
        1 & C(p) & \dots & \dots & 0 \\ \hdashline[2pt/2pt]
        0 & 1 & \ddots & 0 & 0 \\ \hdashline[2pt/2pt]
        0 & 0 & \dots & C(p) & 0 \\ \hdashline[2pt/2pt]
        0 & 0 & \dots & 1 & C(p)
    \end{array}
\right]
$$
Тут $C(p)$ это блок, являющийся сопровождающей матрицей многочлена $p$.
Важно, что единицы стоят не на всей диагонали, а только в местах состыковки блоков.

В некотором роде это обобщение жордановой нормальной формы для произвольного оператора.
Действительно, если $p = (x - \lambda)$, то каждое $C(p)$ это просто $\lambda$, и весь блок $D(i)$ превращается в жорданову клетку с собственным значением $\lambda$.
