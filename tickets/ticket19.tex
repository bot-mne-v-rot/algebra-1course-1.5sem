\section{Разрешимые группы. Критерий разрешимости}
\begin{conj}
    Группа называется разрешимой, если существует такая цепочка подгрупп \\ $\{e\} = G_n < ... < G_1 < G_0 = G$, что $G_{k + 1} \lhd G_k$ и $G_k / G_{k + 1}$ -- абелева группа для всех $k \in [0, n - 1]$.
\end{conj}
Введем обозначение $k$-того коммутанта: 
$G^{(k)} = \begin{cases}
    G, & k = 0 \\
    [G^{(k - 1)}, G^{(k - 1)}], & k > 0
 \end{cases}$.
 
\begin{theorem}
    Группа разрешима, если и только если ее ряд коммутантов заканчивается на тривиальной группе (иными словами, $\exists n : G^{(n)} = \{e\}$).
\end{theorem}
\begin{proof} \quad \\
     $"\Leftarrow":$ Ряд коммутантов подходит в определение разрешимой группы:
     \[ \{e\} = G^{(n)} < \dots < G^{(1)} < G^{(0)} = G \]
     Необходимые свойства ($G^{(k + 1)} \lhd G^{(k)}, G^{(k)} / G^{(k + 1)}$ -- абелева) выполняются по предыдущим двум предложениям.
    
     $"\Rightarrow":$ У нас есть ряд, обладающий необходимыми свойствами.
     Докажем по индукции, что $G^{(k)} \subset G_k$.

     \quad \underline{База:} при $k = 0 \;\;\; G^{(0)} = G = G_0$

     \quad \underline{Переход:}
     \begin{gather*}
        \begin{cases}
            G^{(k + 1)} = [G^{(k)}, G^{(k)}] \underbrace{\subset}_{\text{инд. переход}} [G_k, G_k] \\
            G_k / G_{k + 1} \; \text{абелева} \underbrace{\Rightarrow}_{\text{предложение}} G_{k + 1} \supset [G_k, G_k]
           \end{cases} \Rightarrow G^{(k + 1)} \subset [G_k, G_k] \subset G_{k + 1}
     \end{gather*}
    Значит, $G^{(n)} \subset G_n = \{e\} \Rightarrow G^{(n)} = \{e\}$.
\end{proof}
Таким образом, для конечных групп верно следующее:
\begin{itemize}
    \item либо $\exists n : G^{(n)} = \{e\}$
    \item либо $\exists n : G^{(n)} = G^{(n - 1)} \neq \{e\}$
\end{itemize}

\underline{Примеры разрешимых и неразрешимых групп:}
\begin{enumerate}
    \item Все абелевы группы разрешимы (уже первый коммутант будет равен $\{e\}$).
    \item $S_3$ является разрешимой.
    \item $A_5$ являеется неразрешимой группой минимального порядка (уже первый коммутант равен ей самой).
\end{enumerate}
