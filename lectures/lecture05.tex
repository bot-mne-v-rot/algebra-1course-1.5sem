\section{Лекция номер 5}
\subsection{Центр и коммутант группы}
\begin{conj}
    Центр группы -- множесто элементов данной группы, которые коммутируют со всеми ее элементами:
    \[ Z(G) = \{ g \in G \, | \, \forall h \in G : gh = hg \} \]
\end{conj}

\begin{theorem-non}
    Центр группы является ее нормальной подгруппой.
\end{theorem-non}
\begin{proof} \quad \\
    \begin{itemize}
        \item Замкнутость остносительно операции: 
        \begin{gather*}
            g_1, g_2 \in Z(G), h \in G \\
            h(g_1g_2) = g_1hg_2 = (g_1g_2)h \Rightarrow g_1g_2 \in Z(G)
        \end{gather*}
        \item Очевидно, что нейтральный элемент лежит в $Z(G)$
        \item Существование обратного:
        \begin{gather*}
            g \in Z(G), h \in G \\
            g^{-1}h = (h^{-1}g)^{-1} = (gh^{-1})^{-1} = hg^{-1} \Rightarrow g^{-1} \in Z(G)
        \end{gather*}
        \item Нормальность:
        \begin{gather*}
            g \in Z(G), h \in G \\
            hgh^{-1} = hh^{-1}g = g \Rightarrow hgh^{-1} \in G
        \end{gather*}
    \end{itemize}
\end{proof}

\begin{conj}
    Коммутатором для элементов $g, h \in G$ называется эллемент $[g, h] = ghg^{-1}h^{-1}$.
\end{conj}
\begin{notice}
    \begin{enumerate}
        \item Коммутатором двух элементов является нейтральный элемент, если и только если они коммутируют.
        \begin{proof} \quad \\
            $"\Rightarrow": ghg^{-1}h^{-1} = e \Rightarrow ghg^{-1} = h \Rightarrow gh = hg$ \\
            $"\Leftarrow": gh = hg \Rightarrow ghg^{-1} = h \Rightarrow ghg^{-1}h^{-1} = e$
        \end{proof}
        \item Обратный к коммутатору тоже является коммутатором, причем $[g, h]^{-1} = [h, g]$.
        \begin{proof}
            $[g, h]^{-1} = (ghg^{-1}h^{-1})^{-1} = hgh^{-1}g^{-1} = [h, g]$
        \end{proof}
    \end{enumerate}
    
\end{notice}


\begin{conj}
    Коммутант -- подгруппа, порожденная всеми коммутаторами группы. Обозначается как $[G, G]$.
\end{conj}

\begin{theorem-non}
    Коммутант является нормальной подгруппой.
\end{theorem-non}
\begin{proof}
    Это подгруппа по определнию, значит, осталось проверить нормальность.
    Сперва проверим, что сопряжение коммутатора дает коммутатор. \\
    Введем следующее обозначение для сопряжения: $aga^{-1} = g^a$.
    Легко видеть, что отображение $g \mapsto g^a$ является автоморфизмом.
    Тогда
    \begin{gather*}
        [g, h]^a = g^ah^a(g^{-1})^a(h^{-1})^a = g^ah^a(g^a)^{-1}(h^a)^{-1} = [g^a, h^a]
    \end{gather*}
    Отсюда уже легко доказать нормальность, ведь любой элемент $[G, G]$ раскладывается в произведение коммутаторов:
    \[ ([g_1, h_1][g_2, h_2]\dots[g_m, h_m])^a = [g_1^a, h_1^a][g_2^a, h_2^a]\dots[g_m^a, h_m^a] \in [G, G] \]
\end{proof}

\begin{theorem-non}
    Пусть $H \lhd G$. 
    Тогда следующие 2 утверждения эквивалентны: 
    \begin{enumerate}
        \item $G / H$ абелева группа
        \item $H \supset [G, G]$
    \end{enumerate}
\end{theorem-non}
\begin{proof} \quad \\
    \begin{gather*}
        \begin{split}
            \text{$G / H$ абелева} 
            &\Leftrightarrow \forall g_1, g_1 \in G \quad g_1Hg_2H = g_2Hg_1H \\
            &\Leftrightarrow (g_1H)(g_2H)(g_1H)^{-1}(g_2H)^{-1} = eH  \\
            &\Leftrightarrow (g_1g_2g_1^{-1}g_2^{-1})H = eH (\text{произведение классов -- класс произведения})\\
            &\Leftrightarrow [g_1, g_2] \in H \\
            &\Leftrightarrow [G, G] \subset H
        \end{split}
    \end{gather*}
\end{proof}

\begin{conj}
    Группа называется разрешимой, если существует такая цепочка подгрупп \\ $\{e\} = G_n < ... < G_1 < G_0 = G$, что $G_{k + 1} \lhd G_k$ и $G_k / G_{k + 1}$ -- абелева группа для всех $k \in [0, n - 1]$.
\end{conj}
Введем обозначение $k$-того коммутанта: 
$G^{(k)} = \begin{cases}
    G, & k = 0 \\
    [G^{(k - 1)}, G^{(k - 1)}], & k > 0
 \end{cases}$.
 
\begin{theorem}
    Группа разрешима, если и только если ее ряд коммутантов заканчивается на тривиальной группе (иными словами, $\exists n : G^{(n)} = \{e\}$).
\end{theorem}
\begin{proof} \quad \\
     $"\Leftarrow":$ Ряд коммутантов подходит в определение разрешимой группы:
     \[ \{e\} = G^{(n)} < \dots < G^{(1)} < G^{(0)} = G \]
     Необходимые свойства ($G^{(k + 1)} \lhd G^{(k)}, G^{(k)} / G^{(k + 1)}$ -- абелева) выполняются по предыдущим двум предложениям.
    
     $"\Rightarrow":$ У нас есть ряд, обладающий необходимыми свойствами.
     Докажем по индукции, что $G^{(k)} \subset G_k$.

     \quad \underline{База:} при $k = 0 \;\;\; G^{(0)} = G = G_0$

     \quad \underline{Переход:}
     \begin{gather*}
        \begin{cases}
            G^{(k + 1)} = [G^{(k)}, G^{(k)}] \underbrace{\subset}_{\text{инд. переход}} [G_k, G_k] \\
            G_k / G_{k + 1} \; \text{абелева} \underbrace{\Rightarrow}_{\text{предложение}} G_{k + 1} \supset [G_k, G_k]
           \end{cases} \Rightarrow G^{(k + 1)} \subset [G_k, G_k] \subset G_{k + 1}
     \end{gather*}
    Значит, $G^{(n)} \subset G_n = \{e\} \Rightarrow G^{(n)} = \{e\}$.
\end{proof}
Таким образом, для конечных групп верно следующее:
\begin{itemize}
    \item либо $\exists n : G^{(n)} = \{e\}$
    \item либо $\exists n : G^{(n)} = G^{(n - 1)} \neq \{e\}$
\end{itemize}

\underline{Примеры разрешимых и неразрешимых групп:}
\begin{enumerate}
    \item Все абелевы группы разрешимы (уже первый коммутант будет равен $\{e\}$).
    \item $S_3$ является разрешимой.
    \item $A_5$ являеется неразрешимой группой минимального порядка (уже первый коммутант равен ей самой).
\end{enumerate}

\subsection{Действие группы на множестве}
\begin{conj}
    Говорят, что задано действие группы $G$ на множестве $M$, если
    \begin{enumerate}
        \item задано отображение $(g, m) \mapsto gm$ из  $G \times M$ в $M$, обладающее двумя свойствами
        \begin{enumerate}
            \item $\forall g_1, g_2 \in G \;\; \forall m \in M : (g_1g_2)m = g_1(g_2m)$
            \item $\forall m \in M \;\; em = m$
        \end{enumerate}
        \item задан гомоморфизм $\varphi: G \to S(M)$, где $S(M)$ - группа биекций $M$ на себя
    \end{enumerate}
\end{conj}
Получим из одного определения другое.

\underline{Определение 1 $\to$ Определение 2:} Определим наш гомоморфизм $\varphi$ так:
\begin{gather*}
    \varphi: G \to S(M) \\
    g \mapsto (m \mapsto gm)
\end{gather*} 
Отображение $m \mapsto gm$ являестя биекцией, так как у него есть обратное отображение $m \mapsto g^{-1}m$.

Изучим отображение $\varphi(g_1g_2)$. Это элемент $S(M)$, давайте применим его к какому-то $m$.
\begin{gather*}
    \varphi(g_1g_2)(m) \underbrace{=}_{\text{по опр. гом-ма}} (g_1g_2)m \underbrace{=}_{\text{св-во опр. 1}} g_1(g_2m) = \varphi(g_1)(g_2m) = \varphi(g_1)(\varphi(g_2)(m)) \\
    \Rightarrow \varphi(g_1g_2) = \varphi(g_1) \circ \varphi(g_2)
\end{gather*}
Значит, гомоморфизм построен корректно.

\underline{Определение 2 $\to$ Определение 1:} Определим наше отображение из $G \times M$ в $M$ так:
\begin{gather*}
    G \times M \to M \\
    (g, m) \mapsto \varphi(g)(m)
\end{gather*}
Проверим, что сохранились свойства:
\begin{enumerate}
    \item \begin{gather*}
        (g_1g_2)m \underbrace{=}_{\text{по опр. отображения}} \varphi(g_1g_2)(m) \underbrace{=}_{\text{св-во гом-ма}} (\varphi(g_1) \circ \varphi(g_2))(m) = g_1(g_2m)
    \end{gather*}
    \item \[ em = \varphi(e)(m) = id_m(m) = m \]
\end{enumerate}

\underline{Примеры:}
\begin{enumerate}
    \item $G = \R, M = \mathbb{C}$ \\ $gm := m(\cos g + i\sin g)$ -- поворот на комплексной плоскости на угол $g$
    \item $G = GL_n(K)$ -- обратимые матрицы порядка $n$, $M = K^n$ -- столбцы \\ $gm := g * m$ - умножение матрицы на столбец
    \item $G$ -- любая группа, $M = G$ \\ $gm := g * m$, где $*$ -- групповая операция \\ Это действие $G$ на себе левыми сдвигами.
    \item $G$ -- любая группа, $M = G$ \\ $gm := m^g = gmg^{-1}$ \\ Это действие $G$ на себе сопряжениями.
\end{enumerate}

\begin{conj}
    Пусть задано действие группы $G$ на множестве $M$.
    Тогда орбита элемента $m$ -- это множество $Gm = \{gm\; | \; g \in G\}$.
\end{conj}

\begin{theorem-non}
    Орбиты двух элементов $M$ либо не пересекаются, либо совпадают.
\end{theorem-non}
\begin{proof}
    \begin{gather*}
        h \in Gm_1 \cap Gm_2 \\
        h = g_1m_1 = g_2m_2 \\
        m_2 = g_2^{-1}g_1m_1  \\
        \Rightarrow \forall g \in G: gm_2 = gg_2^{-1}g_1m_1 \in Gm_1 \\
        \Rightarrow Gm_2 \subset Gm_1 \\
        \text{Аналогично } Gm_1 \subset Gm_2 \\
        \Rightarrow Gm_1 = Gm_2
    \end{gather*}
\end{proof}
Посмотрим, какие орбиты у нас получились в примерах:
\begin{enumerate}
    \item Взяли точку на комплексной плоскости и начали поворачивать на всевозможные углы \\ 
    $\Rightarrow$ получили окружность с нужным радиусом $\Rightarrow Gm = \{ z \in \mathbb{C} \; | \; |z| = |m| \}$
    \item Нетрудно доказать, что из любого ненулевого столбца мы можем получить любой другой ненулевой. 
    Тогда $Gm = \begin{cases}
        \{ 0 \}, & m = 0 \\
        K^n \setminus \{ 0 \}, & m \neq 0
    \end{cases} $
    \item Очевидно, что будет лишь одна орбита: $Gm = M = G$.
    \item Орбитами будут классы сопряженности.
\end{enumerate}

Если $\exists m \in M : Gm = M$, то действие $G$ на $M$ называется транзитивным. 
Иными словами, мы можем перевести каждый элемент в любой другой. 
Легко видеть, что действие на каждой отдельной орбите транзитивно.

\begin{conj}
    Пусть $G$ действует на $M$. 
    Стабилизатором элемента $m$ называется множество $St_m = \{ g \in G \; | \; gm = m \}$.
\end{conj}

\begin{theorem-non} \quad \\
    \begin{enumerate}
        \item $St_m < G$ (поэтому стабилизатор иногда называют стационарной подгруппой) 
        \item Существует биекция: 
        \begin{gather*}
            \varphi: G / St_m \to Gm \\
            gSt_m \mapsto gm
        \end{gather*}
    \end{enumerate}
\end{theorem-non}
\begin{proof} \quad \\
    \begin{enumerate}
        \item Легко проверить замкнутость относительно умножения, существование нейтрального и обратного.
        \item Проверим корректность отображения:
        \begin{gather*}
            g_1St_m = g_2St_m \\
            g_2 = g_1s, s \in St_m \\
            g_2m = g_1sm \underbrace{=}_{\text{так как } s \in St_m} g_1m
        \end{gather*}
        Проверим инъективность:
        \begin{gather*}
            \varphi(gSt_m) = \varphi(g'St_m) \\
            gm = g'm \\
            m = g^{-1}g'm \Rightarrow g^{-1}g' \in St_m \\
            g'\in gSt_m \Rightarrow g'St_m = gSt_m
        \end{gather*}
        Сюръективность тривиальна, так как мы можем брать любые $g$ (значит, мы и получим всю орбиту $m$).
    \end{enumerate}
\end{proof}

\begin{follow}
    Пусть $G$ -- конечная группа, $m \in M$.
    Тогда длина орбиты делит порядок группы: $|Gm| \mid |G|$.
\end{follow}
\begin{proof}
    Из предыдущего предложения мы поняли, что $Gm$ биективно отображается на $G / St_m$, следовательно, $|Gm| = |G / St_m|$.
    Но $|G / St_m|$ является индексом подгруппы $St_m$ по определению, значит, делит порядок группы.
\end{proof}

Посмотрим, какие стабилизаторы у нас получились в примерах:
\begin{enumerate}
    \item $St_0 = \R$, так как 0 всегда будет оставатсья 0 при повороте \\ $St_{m \neq 0} = \langle 2\pi \rangle$
    \item Неочевидно (в первый раз это слово употребляется с приставкой не)
    \item $St_g = \{ e \}$
    \item $St_g = \{ g \in G \, | \; hg = gh \} = Z_g$ -- централизатор элемента $g$ 
\end{enumerate}

\begin{theorem-non}
    Пусть $G$ -- конечная $p$-группа (т.е. $|G| = p^n, p$ -- простое).
    Тогда у нее нетривиальный центр: $Z(G) \neq \{ e \}$.
\end{theorem-non}
\begin{proof}
    Пусть $G$ действует на себе сопряжением. 
    Из следствия мы знаем, что длина орбиты является делителем порядка группы. 
    Тогда орбиты могут быть следующих длин: $1, p, p^2, \dots, p^n$. 

    Пусть $m_i$ -- число орбит длины $p^i$. 
    Заметим, что $|G| = p^n = \sum\limits_{i = 0}^n p^im_i$, так как каждый элемент $G$ лежит в орбите, и причем только одной.
    Перепишем полученную сумму: \[ \underbrace{p^n}_{\mod p = 0} = m_0 + \underbrace{pm_1 + p^2m_2 + \dots + p^nm_n}_{\mod p = 0} \Rightarrow p \mid m_0 \]
    Мы поняли, что $p$ делит $m_0$. 
    Осталось заметить, что $m_0 = |Z(G)|$, так как это ровно те элементы, которые при любом сопряжении всегда дают себя (у них орбита длины 1), то есть они коммутируют со всеми.

    Следовательно, $p \mid |Z(G)|$, а в таком случае $Z(G)$ никак не может равняться $\{ e \}$.
\end{proof}

\begin{follow}
    Любая конечная $p$-группа разрешима (ггвп, а не следствие).
\end{follow}

\begin{proof}
    Индукция по $n$.
    
    Пусть $|G| = p^n$.
    Обозначим: $G' = G / Z(G)$.
    По предыдущему предложению $Z(G) \neq \{ e \} \Rightarrow |G'| = \frac{|G|}{(G : Z(G))} < |G|$ и $G'$ тоже являеется $p$-группой.
    Если $G' = \{ e \}$, то $G = Z(G) \Rightarrow G$ абелева, а значит разрешима.

    Теперь разберем содержательный случай, когда $G' \neq \{ e \}$. 
    Применим к ней индукционное предложение: \[ \{e \} = G_n' < \dots < G_1' < G_0' = G', \] где $G_{k+1}' \lhd G_k'$ и $G_k'/G_{k+1}'$ абелева.
    Посмотрим на гомоморфизм проекций на факторгруппу (каждому элементу сопоставляется его класс):
    \begin{gather*}
        \pi_{Z(G)}: G \to G / Z(G) = G' \\
        g \mapsto gZ(G) 
    \end{gather*}
    Введем $G_k = \pi_{Z(G)}^{-1}(G_k')$ -- все элементы, которые перешли в данные классы ($G_k'$ подгруппа в $G'$, то есть это просто подгруппа классов).
    Тогда мы можем выписать ряд c соотвествующими $G_k$: \[ \underbrace{\Ker \pi_{Z(G)}}_{\text{прообраз $e$}} = G_n < \dots < G_1 < G_0 = G \] 
    По определнию $\Ker \pi_{Z(G)} = Z(G)$. 
    Возьмем внутри $Z(G)$ тривиальную подгруппу.
    Тогда наш ряд примет вид: \[ \{ e \} = G_{n+1} < Z(G) = G_n < \dots < G_1 < G_0 = G \]  
    Покажем, что данный ряд является рядом в определении разрешимой группы:
    \begin{itemize}
        \item $G_{k + 1} \lhd G_k$.
        
        Отдельно рассмотрим случай $G_{n + 1} \lhd G_n$. Это очевидно, потому что $G_{n+1} = \{ e \}$.
        
        Теперь случай $k \in [0, n - 1]$. Наше отображение $\pi_{Z(G)}$ переводило $G_k$ в $G_k'$, тогда по теореме о соответствии из $G_{k+1}' \lhd G_k'$ следует, что $G_{k+1} \lhd G_k$. 
        \item $G_k / G_{k+1}$ абелева.
        
        Отдельно рассмотрим случай $G_n / G_{n+1}$. 
        Это $Z(G) / \{ e \} \cong Z(G)$, а $Z(G)$ абелева.

        Теперь случай $k \in [0, n - 1]$. 
        Можно понять, что $G_k' \cong G_k / Z(G)$ по определению этих самых $G_k'$ и $G_k$.
        Тогда  \[ \underbrace{G_k' / G_{k+1}'}_{\text{абелева}} \cong (G_k / Z(G)) / (G_{k+1} / Z(g)) \cong G_k / G_{k+1} \] 
        Второй переход это теорема о факторизации факторгруппы.
    \end{itemize}
    Следовательно, $G$ разрешима. 
\end{proof}