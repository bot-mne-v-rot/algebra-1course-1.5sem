\section{Нормальные подгруппы}
\begin{conj}
    Пусть $G$ -- группа, $H < G$. \\Тогда $H$ --
    \textbf{нормальная подгруппа} $G$, если
    $\forall g \in G \; \forall h \in H \quad
    ghg^{-1} \in H$. \\
    Обозначается $H \lhd G$.
\end{conj}

\notice Если $G$ абелева, то любая подгруппа нормальна.

\begin{theorem-non}
    Пусть $G, G'$ -- группы, $\varphi \colon G \to G'$ -- гомоморфизм.
    Тогда $\Ker \varphi \lhd G$.
\end{theorem-non}
\begin{proof} $ $

    Возьмём $h \in \Ker \varphi$, $g \in G$.\\
    Тогда $\varphi(ghg^{-1}) = \varphi(g) \cdot 
    \underbrace{\varphi(h)}_{= e}
    \cdot \varphi(g^{-1}) =
    \varphi(g) \cdot \varphi(g^{-1}) =
    \varphi(g) \cdot \varphi(g)^{-1} = e$. \\
    Из этого следует, что $ghg^{-1} \in \Ker \varphi$.
\end{proof}

\begin{theorem-non}
    Пусть $G$ -- группа, $H < G$. Тогда эквивалентны:
    \begin{enumerate}
        \item $H \lhd G$
        \item $\forall g \in G \quad gHg^{-1} \subset H$
        \item $\forall g \in G \quad gHg^{-1} = H$
        \item $\forall g \in G \quad gH = Hg$
    \end{enumerate}
\end{theorem-non}
\notice Про 4-е свойство ещё говорят, что левое и правое
разложение Лагранжа совпадают, где левое (правое) разложение Лангранжа
--- разбиение группы в объединение её левых (правых) смежных классов.\\
\textit{Замечание это от Жукова, термин этот не гуглится :(}

\begin{proof} $ $

    \begin{itemize}
        \item ``$1 \Leftrightarrow 2$'': тривиально.
        
        $gHg^{-1} = \{ghg^{-1} \mid h \in H\}$, \\
        поэтому
        $\forall g \in G \;\; gHg^{-1} \subset H$
        $\Longleftrightarrow $
        $\forall g \in G \; \forall h \in H \;\;
        ghg^{-1} \in H$

        \item ``$3 \Rightarrow 2$'': очевидно.
        
        \item ``$2 \Rightarrow 3$'':
        
        $\forall g \in G \;\; gHg^{-1} \subset H$.
        Значит, для $g^{-1}$ верно, что $g^{-1}Hg \subset H$. \\
        Посмотрим на $g^{-1}Hg \subset H$, выполним домножения: \\
        $g^{-1}Hg \subset H \Rightarrow g^{-1}H \subset Hg^{-1}
        \Rightarrow H \subset gHg^{-1}$. \\
        Получаем, что $gHg^{-1} = H$.

        \item ``$3 \Leftrightarrow 4$'':
        
        $\equalto{gHg^{-1}g}{gH} = Hg \Leftrightarrow gHg^{-1} = H$
    \end{itemize}
\end{proof}

\begin{example}
    $\cycle {(1 \; 2)}$ не нормальна в $S_3$, но $A_3 \lhd S_3$.
\end{example}

\notice $(G : H) = 2 \Rightarrow H \lhd G$.
\begin{proof} $ $\\
    $H \in G/H \Rightarrow G/H = \{H, G - H\}$.\\
    $H \in H\backslash G \Rightarrow 
    H\backslash G = \{H, G - H\}$.\\
    $G/H = H\backslash G \Rightarrow H \lhd G$.
\end{proof}