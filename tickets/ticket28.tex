\section{Ортогональные матрицы. Матрица перехода между ортогональными базисами}
Когда у нас уже есть некий ортонормированный базис, задать новый базис -- это то же самое, что задать матрциу перехода. Вопрос следующий: Какой должна быть 
матрица перехода, чтобы получающийся базис был ортонормированный? Ответ на этот вопрос мы собственно уже знаем.  
Новый базис будет ортонормированным, если матрица Грама формы $\B$ в новом базисе будет единичной. Мы знаем, что: 
\begin{gather*}
    [\B]_{E'} =C^T [\B]_E C
\end{gather*}
Раз исходный базис ортонормированный, то $[\B]_E$ -- единичная. Легко видеть, что $[\B]_{E'}$ будет единичной тогда и только тогда, когда $C^T C = E_n$ 
\begin{conj}
    Матрица $C$ называется ортогональной, если $C^T C = E_n$ 
\end{conj}
Итого, отвечая на наш вопрос, можно сказать, что новый базис будет тоже ортонормированным, если и только если матрциа перехода была ортогональной матрицей.

Квадратная матрица $A \in M(n, K)$ называется ортогональной, если она обратна к своей транспонированной: $AA^T = E_n$. 
Давайте поймем, какие критерии данное определение налагает на $A$. Пусть $A = (a_{ij})$, тогда: 
\begin{gather*}
    AA^T [i, j] = \sum\limits_{k}^n a_{i k} \cdot a_{jk}
\end{gather*}
Отсюда видим, что строки нашей матрицы $A$ ортогональны, а также нормированны.
\begin{lemma}
    $A$ ортогональна $\Longrightarrow A^T$ ортогональна
    \begin{gather*}
        A^T A^{TT} = \underbrace{A^T}_{= A^{-1}} A = E_n
    \end{gather*}
\end{lemma}
\begin{conj}
    Введем обозначение для множества ортогональных матриц: 
    \begin{gather*}
        O(n, K) = \{ A \in M(n, K) \mid A \text{ ортогональна } \}
    \end{gather*}
\end{conj}
\begin{theorem-non}
    Множество ортогональных -- подгруппа в множестве обратимых матриц.
\end{theorem-non}
\begin{proof} \quad 

    \begin{itemize}
        \item Проверим замкнутость относительно умножения: 
            \begin{gather*}
                A, B \in O(n, K) \\
                (AB)(AB)^T = A\underbrace{B B^T}_{E_n} A^T = A A^T = E_n 
            \end{gather*}
        \item Проверим замкнутость относительно взятия обратного: 
            \begin{gather*}
                A \in O(n, K) \\
                A^{-1} (A^{-1})^T = A^T A^{TT} = E_n
            \end{gather*}
    \end{itemize}
    --- ортогональная группа степени $n$ над полем $K$
\end{proof}

\begin{theorem-non}
    $V$ евклидово протранство, $E$ -- ортонормированный базис, $E'$ -- какой-либо базис $V$,
    $C$ -- матрица перехода от $E$ к $E'$. Тогда $E'$ ортонормированный $\Longleftrightarrow C \in O(n, K)$ 
\end{theorem-non}
\begin{proof}
    Пусть $\Gamma_E$ -- матрица Грама скалярного произведения в базисе $E$. Тогда
    \begin{gather*}
        \Gamma_{E'} = C^T \underbrace{\Gamma_E}_{E_n} C \\ 
        \Gamma_{E'} = E_n \Longleftrightarrow C^T C = E_n \Longleftrightarrow C \in O(n, K)
    \end{gather*} 
\end{proof}