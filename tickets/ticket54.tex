\section{Единственность поля из $p^n$ элементов}
\begin{lemma}
    Пусть $L/K$~--- расширение конечных полей.
    Тогда $L/K$ простое (т.е. $L=K(a)$ для некоторого $a\in L$).
\end{lemma}
\begin{proof}
    По теореме (примерно последняя доколлочная теорема) конечная подгруппа мультипликативной группы поля обязана быть циклической.
    Т.е. $L^* = \langle a \rangle$ для некоторого $a\in L$.
    Но тогда, очевидно, $L=K(a)$.
\end{proof}
\follow $\forall n\in \N$ существует неприводимый многочлен $f$, т.ч. $\deg f = n$.
\begin{proof}
    Рассмотрим поле из $p^n$ элементов.
    Оно является расширением степени $n$ над $\mathbb{F}_p$ и по лемме оно равно $\mathbb{F}_p(a)$.
    Тогда минимальный многочлен элемента $a$ и будет искомым.
\end{proof}

\begin{theorem}(Псевдотеорема Эйлера)
    Пусть $K$~--- поле, $|K|=q<\infty$, $a$~--- произвольный элемент $K$.
    Тогда $a^q=a$.
\end{theorem}
\begin{proof}
    Как мы поняли, $K^* = \langle d\rangle$ для некоторого $d$.
    Тогда если $a\in K^*$, то $a^{|K^*|}=1$, т.е. $a^{q-1}=1$ (это следствие из теоремы Лагранжа о том, что размер подгруппы делит размер группы~--- см. строение циклических групп), откуда $a^q=a$.
    В противном случае $a=0$ и тогда тоже, очевидно, $a^q=a$.
\end{proof}
\notice Название выдуманное, но будет встречаться пару раз в конспекте.
Если ссылаться на теорему, то лучше всего быстро передоказать / сказать ``очевидно''.

\begin{theorem}
    Пусть $p\in \mathbb{P}, n\in \N$, $F_1, F_2$~--- произвольные поля, такие что $|F_1|=|F_2|=p^n$.
    Тогда $F_1\cong F_2$.
\end{theorem}
\begin{proof}
    По лемме, $F_1=\mathbb{F}_p(a)$ для некоторого $a$.
    Пусть $f$~--- минимальный многочлен $a$.
    Рассмотрим многочлен $X^q-X$ (опять же, для удобства обозначем $p^n$ за $q$)
    По псевдотеореме Эйлера все элементы $F_1, F_2$ являются его корнями $\Rightarrow$ и там, и там он раскладывается на линейные множители (ибо мы нашли $q$ различных корней, а степень многочлена равна $q$).
    Более того, $a$ является корнем $X^q-X$, откуда, в силу минимальности $f$, $f \mid (X^q-X)$.
    Значит, $f$ раскладывается на линейные множители в $F_2$.

    Значит, можем рассмотреть $a'$~--- произвольный корень $f$ в $F_2$ (их там даже $n$, подойдет любой).
    Заметим, что $f$ будет минимальным многочленом для $a'$ (действительно, он является зануляющим $\Rightarrow$ должен делиться на минимальный. Но он неприводим $\Rightarrow$ делится (из тех, у кого старший коэффициент равен $1$) только на себя и на $1$, единица очевидно не подходит).
    Рассмотрим теперь $\mathbb{F}_p(a')$.
    Это поле изморфно $\mathbb{F}_p[X] / (f)$, а потому содержит $p^n$ элементов и при этом является подполем в $F_2$, которое тоже состоит из $p^n$ элементов.
    Значит, $F_2 = \mathbb{F}_p(a')$.
    Отсюда окончательно
    \begin{gather*}
        F_1 = \mathbb{F}_p(a) \cong \mathbb{F}_p[X] / (f) \cong \mathbb{F}_p(a') = F_2
    \end{gather*}
\end{proof}

\notice Теперь мы знаем, что для всякого $p^n$ существует ровно одно поле такого размера, поэтому можно ввести для него обозначение $\mathbb{F}_{p^n}$

\notice В доказательстве мы, вообщем-то, сразу поняли, что $F_1$ и $F_2$ являются полями разложения многочлена $X^q-X$.
Отсюда сразу следует, что $F_1\cong F_2$ т.к. поле разложения единственно, но последний факт мы не доказывали, поэтому им все же не стоит пользоваться.
