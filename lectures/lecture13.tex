\section{Лекция номер 13}

\subsection{Нормальные операторы в евклидовом пространстве}

Вспомним, что по любому пространству $V / \R$ мы можем построить комплисификацию $V_{\C} / \C$, той же размерности.
Можно представлять, что $V_{\C} / \C$ состоит из векторов вида $\{ v + iw \, | \, v,w \in V \}$, где $i$ -- мнимая единица. 
Также мы знаем, что если $n$ векторов $v_1, \dots, v_n \in V$ образуют базис $V$ $\Longleftrightarrow v_1, \dots, v_n$ образуют базис $V_{\C}$. 
Заметим, что стрелочка влево есть только при условии, что наши $n$ вещественных векторов принадлежали $V$. 
Действительно, если $\alpha_1 v_1 + \dots + \alpha_n v_b = 0 \Longrightarrow \alpha_1 = \dots = \alpha_n = 0$.
Давайте условимся называть вектор с нулевой мнимой частью вещественным вектором: $v + i \cdot 0 = v$. Тогда, строго говоря, 
наш базис $v_1, \dots, v_n$ -- базис из вещественных векторов. 

Далее, если на $V$ была евклидова структура, то мы естественным образом продолжаем ее до унитарной структуры на комплексификации. 
Из евклидова пространства в комплексификации получается унитарное. 

Примерно на этом мы и остановились. Давайте теперь научимся продолжать операторы на комплексификациях.

\subsection*{Операторы на комплексификациях}

\begin{conj}
Пусть у нас был оператор $\A \in \End{V}$.
Тогда $\A_{\C} \in \End{V_{\C}}$ -- это оператор комплексификации. Задается он следующим образом:
\begin{gather*}
    \A_{\C}(v + iw) = Av + iAw
\end{gather*}
Проверка, что мы правда получили оператор, тривиальна. 
\end{conj}

\begin{conj}
Пусть $E$ -- базис $V$
Тогда в таком базисе матрица оператора комплексификации будет такой же, как и матрица самого оператора. 
\begin{gather*}
    [\A_{\C}]_{E} = [\A]_{E}
\end{gather*}
Так как у нас базисные вектора вещественные, то есть оператор комплексификации 
для них будет давать то же самое, что и просто оператор $\A$. 

Отсюда следует равенство характеристических многочленов, так как для его вычисления мы можем брать любой базис: 
\begin{gather*}
    \chi_{\A_{\C}} = \chi_{\A}
\end{gather*}
\end{conj}

\begin{lemma}
    \begin{gather*}
        (\A_{\C})^* = (\A^*)_{\C}
    \end{gather*}
\end{lemma}
\begin{proof}
    \emptyln
    Достаточно проверить, что выполняется характеристическое свойство:
    \begin{gather*}
        (\A(v + iw), v' + iw') \stackrel{?}{=} (v + iw, (\A^*)_{\C}(v' + iw'))
    \end{gather*}
    Давайте проверять:
    \begin{align*}
        (\A(v + iw), v' + iw') &= (A v, v') + (A w, w') + i ((A w, v') - (A v, w')) \\ 
        (v + iw, (\A^*)_{\C}(v' + iw')) &= (v, A^* v') + (w, A^* w') + i ( (w, A^* v') - (v, A^* w'))
    \end{align*}
    По характеристическому свойству для $\A$ эти штуки равны. 
\end{proof}

\follow $\A$ -- нормальный оператор $\Longrightarrow A_{\C}$ нормальный опреатор (используется факт, что комплексификация композиции -- это композиция их комплексификаций)

Пусть $\A$ -- нормальный оператор в евклидовом пространстве $V$. 
Рассмотрим его характеристический многочлен $\chi_{\A} \in \R[x]$. 
Если разложить его на линейные множители над полем комплексных чисел, то 
мнимые корни будут встречаться парами: 
\begin{gather*}
    \chi_{\A}(z) = 0 \Longrightarrow \chi_{\A}(\bar{z}) = 0
\end{gather*}
И кратность этих корней будет одинаковой, поэтому его можно записать в следующем виде: 
\begin{gather*}
    \chi_{\A} = \pm \prod_{p = 1}^{s} (x - \mu_p)^{a_p} \prod_{q = 1}^{t} (x - \lambda_q)^{b_q} (x - \overline{\lambda_q})^{b_q}, \text { где } \mu_p \in \R, \lambda_q \not \in \R
\end{gather*}

Отсюда мы получаем, что не вещественные собственные числа встречаются парами. Также у комплексносопряженных собственных значений будет одна и та же алгебраическая кратность. 
А так как опреатор диагонализируем, то совпадать будут и геометрические кратности. Если покопаться, то, на самом деле, можно сказать еще больше:

\begin{lemma}
    Пусть вектора $v_1 + iw_1, \dots, v_l + iw_l$ -- это базис $V_{\lambda}$, где $\lambda$ не вещественное, а $(V_{\lambda} = \Ker{\A_{\C} - \lambda \mathcal{E}})$.
    Тогда сопряженные к ним векторы $v_1 - iw_1, \dots, v_l - iw_l$~--- базис $V_{\bar{\lambda}}$

    Более того, если первый базис ортонормированный, то и второй ортонормированный.
\end{lemma}


\begin{proof} \quad 
    
Нетрудно проверить, что:
\begin{gather*}
    v + iw \in V_{\lambda} \Longrightarrow v - iw \in V_{\bar{\lambda}}
\end{gather*}
Также:
\begin{gather*}
    V_{\lambda} = \Lin(v_1 + iw_1, \dots, v_l + iw_l) \Longrightarrow V_{\bar{\lambda}} = \Lin(v_1 - iw_1, \dots, v_l - iw_l)
\end{gather*}
Давайте это проверим. Запишем произвольный вектор как линейную комбинацию базисных векторов: 
\begin{gather*}
    v + iw = \sum_{j = 1}^{l} (\alpha_j + i \beta_j) (v_j + i w_j) \Longrightarrow v - iw = \sum_{}^{} (\alpha_j - i \beta_j) (v_j - i w_j)
\end{gather*}
Аналогично преверяем линейную независимость и первое утверждение доказано. 

Теперь докажем утверждение про ортонормированность:
\begin{gather*}
    (v_j + i w_j, v_k + iw_k) = \delta_{jk} \Longrightarrow  (v_j - i w_j, v_k - iw_k) = \delta_{jk}
\end{gather*}
Просто применили комплексное сопряжение.
\end{proof}

\begin{theorem}()

    Пусть $\A$~--- нормальный оператор в конечномерном евклидовом пространстве $V$ 
       
    Тогда в некотором ортонормированном базисе матрица $\A$ блочно-диагональная и все блоки имеют вид $(\mu)$~--- блок 1х1 или $\left(\begin{array}{cc}
        \alpha & \beta \\ 
        -\beta & \alpha
        \end{array}\right)$
\begin{proof}
    $ \A $ ~--- нормальный оператор

    $V_{\C} = V_{\mu_1} \otimes \dots \otimes V_{\mu_s} \otimes V_{\lambda_1} \otimes V_{\bar{\lambda_1}} \otimes \dots \otimes V_{\lambda_t} \otimes V_{\bar{\lambda_t}}$

    В каждом из вещественных слагаемых $V_{\mu_i}$ и паре слагаемых $V_{\lambda_i}$ и $V_{\bar{\lambda_i}}$ выберем новый базис 

    \textbf{Вещественные слагаемые} 

    Заметим, что в каждом таком кусочке можно выбрать вещественный базис.

    В $V_{\mu_p}$ выберем произвольный базис: \\
    $v_1 + iw_1, \dots, v_l + i w_l$ \\
    $\Longrightarrow V_{\mu_p} = \Lin (v_1, w_1, \dots, v_l, w_l)$ \\
    Из этого набора можно выбрать базис $V_{\mu_p}: u_1, \dots, u_l$ \\
    Ортог-ия (Грама-Шмидта) + нормирование $\Longrightarrow \tilde{u_1}, \dots, \tilde{u_p}$~--- ортонормированный, вещественный базис $V_{\mu_p}$

    $ [ \A \, |_{V_{\mu_p} }]_{\dots} = diag (\underbrace{\mu_p, \dots, \mu_p}_{\text{$ l $ раз}} ) $

    \textbf{Теперь о парах}
    
    $ \A |_{V_{\lambda_q} \otimes V_{\lambda_{\bar{q}}}} $

    $v_1 + iw_1, \dots, v_l + iw_l$~--- ортонормированный базис $V_{\lambda_q}$\\
    По одной из предыдущих лемм мы знаем, что $v_1 - iw_1, \dots, v_l - iw_l$~--- ортонормированный базис $V_{\bar{\lambda_q}}$ 

    Выберем новый базис в сумме этих подпространств следующим образом: \\
    $ \Lin (v_1, w_1, \dots, v_l, w_l) = V_{\lambda_q} \otimes V_{\bar{\lambda_q}} $ \\
    
    \[ \dim(V_{\lambda_q} \otimes V_{\lambda_{\bar{q}}}) = 2l \Longrightarrow \text{это базис}  \]
    
    Для краткости обозначим $u_j := v_j + iw_j$

    Знаем, что \begin{itemize}
        \item $(u_j, u_j) = 1$
        \item $(u_j, u_r) = 0, \quad j \neq r$
        \item $(u_j, \bar{u_j}) = 0$
        \item $(u_j, \bar{u_r}) = 0, \quad j \neq r$
    \end{itemize}
    Последние два равенства верны, потому что это собственные вектора нормиального оператора

    Если теперь перейти от введённого обозначения $u_j$ и расписать по линейности, то получим
    
    \begin{itemize}
        \item $ (v_j, v_j) + (w_j, w_j) = 1  $
        \item $ (u_j, \bar{u_j}) = 0 = (v_j + iw_j, v_j - iw_j) = (v_j, v_j) - (w_j, w_j) + i ( (w_j, v_j) + (v_j, w_j) )$
        \item $(u_j, u_r) = 0 = (v_j, v_r) + (w_j, w_r) + i ( (w_j, v_r ) - (v_j, w_r) )$
        \item $(u_j, \bar{u_r}) = 0 = (v_j, v_r) - (w_j, w_r) + i ( (w_j, v_r ) + (v_j, w_r) )$
    \end{itemize}
    
    Раз последние два равенства равны нулю, то что вещественные, что мнимые части равны нулю $\Longrightarrow (v_j, v_r) = (w_j, w_r) = (w_j, v_r) = (v_j, w_r) = 0$ \\
    Из второго следует, что $(v_j, w_j) = 0$ \\
    Из первого равенства мы понимаем, что $(v_j, v_j) = (w_j, w_j) = \frac{1}{2}$

    Мы получили ортогональный базис в наборе из $2l$ векторов, где каждый вектор ортогонален каждому, кроме себя, но он не нормированн (длина будет равна $ \frac{1}{\sqrt{2}} $) \\
    Ну и ортонормируем: $\tilde{v_j} = \sqrt{2} \cdot v_j, \tilde{w_j} = \sqrt{2} \cdot w_j$ 


    $ \tilde{v_1}, \tilde{w_1}, \dots, \tilde{v_l}, \tilde{w_l} $~--- ортонормированный базис $V_{\lambda_q} \otimes V_{\lambda_{\bar{q}}}$

    Как будет выглядеть матрица для нашего оператора в этом базисе? \\
    Пусть $\lambda_q = \alpha_q + i \beta_q$
    Посмотрим на $\A_{\C}(v_j + iw_j) = (\alpha_q + i \beta_q) (v_j + iw_j)$ \\
    $Av_j = \alpha_q v_j -  \beta_q w_j$ \\
    $Aw_j = \beta_q v_j + \alpha_q w_j$ \\
    $\Longrightarrow $ аналогично для $A\tilde{v_j}, A\tilde{w_j}$


    Итог: 

    Пусть $U_q = \Lin (\underbrace{\tilde{v_1}, \tilde{w_1}, \dots, \tilde{v_l}, \tilde{w_l}}_{\text{$ E_q $}} )$
    
    $[\A |_{U_q}]_{E_q} = \left(\begin{array}{ccccc}
    \alpha_q & \beta_q & 0 & \dots & 0 \\ 
    -\beta_q & \alpha_q & 0 & \dots & 0 \\ 
    0 & 0 & \dots & 0 & 0 \\ 
    \vdots & \vdots & \vdots & \vdots & \vdots \\
    0 & 0 & 0 & \alpha_q & \beta_q \\ 
    0 & 0 & 0 & -\beta_q & \alpha_q
    \end{array}\right)$
\end{proof} 
\end{theorem}


\notice

Первый частный случай:

    $\A = \A^* \Longrightarrow [\A]_{E} = [\A]^T_{E}$ \\
    $\left(\begin{array}{cc}
    \alpha & \beta \\ 
    -\beta & \alpha
    \end{array}\right) = \left(\begin{array}{cc}
    \alpha & -\beta \\ 
    \beta & \alpha
    \end{array}\right) \Longrightarrow \beta = 0 \Longrightarrow [\A]_{E}$

    Таким образом: $\A$ самосопр. $\Longleftrightarrow$ матрица $\A$ в некотором ортонормированном базисе диагональна

Второй частный случай:

    $\A \A^* = \epsilon$ % id 
    $\Longleftrightarrow [\A]_{E} \cdot [\A]^T_{E} \Longleftrightarrow $ для всех блоков $B: BB^T = E_{\dots}$

    $B = (\mu) \Longrightarrow \mu^2 = 1 \Longrightarrow \mu = \pm 1$

    $B = \left(\begin{array}{cc}
    \alpha & \beta \\ 
    -\beta & \alpha
    \end{array}\right) \Longrightarrow BB^T = E_2 = \left(\begin{array}{cc}
    \alpha^2 + \beta^2 & 0 \\ 
    0 & \alpha^2 + \beta^2
    \end{array}\right) \Longleftrightarrow \alpha^2 + \beta^2 = 1$

$ \Longrightarrow \exists \varphi : \begin{cases}
    \alpha = \cos{\varphi} \
    \beta = \sin{\varphi}
\end{cases} $

Т.о.: $\A$ ортогональный оператор $\Longleftrightarrow$ матрица $\A$ в некотором ортонормированном базисе имеет блочно-диагональный вид с блоками вида $(\pm 1)$ и $( \left(\begin{array}{cc}
\cos{\varphi} & \sin{\varphi} \\ 
-\sin{\varphi} & \cos{\varphi}
\end{array}\right) )$

Можно воспользоваться тем, что $\left(\begin{array}{cc}
1 & 0 \\ 
0 & 1
\end{array}\right) = \left(\begin{array}{cc}
\cos{0} & \sin{0} \\ 
-\sin{0} & \cos{0}
\end{array}\right)$

$ \left(\begin{array}{cc}
-1 & 0 \\
0 & -1
\end{array}\right)$
$= \left(\begin{array}{cc}
    \cos{\pi} & \sin{\pi} \\
    -\sin{\pi} & \cos{\pi}
    \end{array}\right)$

$[\A]_{E} = \left(\begin{array}{cccc}
R_{\varphi_1} &  &  & 0 \\ 
 & \ddots &  &  \\ 
 &  & R_{\varphi_m} &  \\ 
0 &  &  & D
\end{array}\right)$

$D = ()$ или $(\pm 1)$ или $\left(\begin{array}{cc}
    1 & 0 \\ 
    0 & -1
    \end{array}\right)$

В частност, если $\dim V = 3$, $| \A | = 1 \Longrightarrow$ в некотором $E : [\A]_{E} = \left(\begin{array}{cc}
R_{\varphi} & 0 \\ 
0 & 1 
\end{array}\right)$ ($| \A |$~--- определитель оператора) %красивая А, странно, но он так пишет 

Вспомним, что $AA^T = E_n \Longrightarrow |A| = \pm 1$

\begin{theorem}(Характеристика ортогональных операторов)
    Пусть $V$ евклидово пространство, $\A \in \End{V}$ 

    Тогда эквивалентны следующие утверждения:
    \begin{enumerate}
        \item $\A$ ортогонален
        \item $\forall v, w \in V: (\A v, \A w) = (v, w)$
        \item $\forall v \in V: ||\A v|| = || v || $
    \end{enumerate}
    \begin{proof}
        $1 \to 2$

        $\A^* \A = \epsilon$ \\ % id
        $(\A v, \A w) = (v, \A^*\A w) = (v, w)$

        $2 \to 3 $

        $||Av|| = \sqrt{(Av, Av)} = \sqrt{(v, v)} = ||v||$

        $3 \to 2$
        
        $2(v, w) = ||v + w||^2 - ||v||^2 - ||w||^2$

        $2(Av, Aw) = ||Av + Aw||^2 - ||Av||^2 - ||Aw||^2$
        
        $\Longrightarrow (Av, Aw) = (v, w)$

        $2 \to 1$

        Нужно доказать, что $\forall v \in V: \, \A^* \A v = v$

        $w \in V: \, (w, \A^* \A v) = (\A w, \A v) = (w, v)$ \\
        $(\A^* \A v - v) \perp w$ \\
        $\A^* \A v - v \in V^{\perp} = 0$ \\
        $\A^* \A v = v$
    \end{proof}    
\end{theorem}


\section*{Операторы проектирования}

Пусть у нас есть такое векторное пространство $V = W_1 \otimes W_2$

\begin{conj}(Оператор проектирования)
    Оператор $\A \in \End{V}:$ \\
    $\A(w_1 + w_2) = w_1$~--- оператор проектирования на $W_1$ вдоль $W_2$
\end{conj}

\begin{theorem}()
    $\A \in \End{V}$ \\
    Тогда эквивалентны два утверждения:
    \begin{enumerate}
        \item $\A$~--- оператор проектирования (т.е. $\exists W_1, W_2$, т.ч. $W_1 \otimes W_2$ и $\A \dots $ на $W_1$ вдоль $W_2$)
        \item $\A^2 = \A$ (идемпотентность)
    \end{enumerate}
    \begin{proof}
        $1 \to 2$ \\
        $\A^2 (w_1 + w_2) = \A(w_1) = \A(w_1 + 0)= w_1 = \A(w_1 + w_2)$

        $2 \to 1$

        $W_1 = \Imm{\A}$
        $W_2 = \Ker{\A}$

        Проверим, что $V = W_1 \otimes W_2$ \\
        $v = \underbrace{\A v }_{\text{$ \in W_1 $}} + \underbrace{(v - \A v)}_{\text{$ \in W_2 $}} $ \\
        т.к. $A(v - Av) = Av - A v = 0$

        $w \in W_1 \cap W_2$ \\
        Раз $w \in W_1$, то $Aw = A^2 w = Av = w$ \\
        Но при этом $w \in W_2$, а значит $Aw = 0$

        Осталось доказать, что $A(\underbrace{w_1}_{\text{$ w_1 $}}  + \underbrace{w_2}_{\text{$ w_2 $}} ) = \underbrace{A w_1}_{\text{$ w_1 $}}  + \underbrace{A w_2}_{\text{$ 0 $}}  $ т.е. это оператор проектирования.
        
    \end{proof}
\end{theorem}

\begin{conj}
    Пусть $V$ --- евклидово или унитарное, $W \subset V$~--- подпространство \\
    Оператор проектирования на $W$ вдоль $W^{\perp}$ называется оператором ортогонального проектирования
\end{conj}

\begin{theorem}(Характеристика оператора ортогонального проектирования)

    $\A \in \End{V}$

    Тогда эквивалентны следующие условия:
    \begin{enumerate}
        \item $\A$ оператор ортогонального проектирования
        \item $\A$ самоспопряжённый идемпотентный
    \end{enumerate}

    \begin{proof}
    \emptyln
    $ 1 \to 2 $

    $ \A = \A^2 $ знаем \\
    Пусть $e_1, \dots, e_m$ --- ортонормированный базис $W$, а $e_{m + 1}, \dots, e_n$ ортонормированный базис $W^{\perp}$ \\
    $E = (e_1, \dots, e_n)$ \\
    $[\A]_E = \left(\begin{array}{cc}
    E_m & 0 \\ 
    0 & 0
    \end{array}\right)$ \\
    $[\A]^*_E = \left(\begin{array}{cc}
    E_m & 0 \\ 
    0 & 0
    \end{array}\right) = [\A]_E \Longrightarrow \A^* = \A$

    $2 \to 1$

    В следующий раз
    \end{proof}
\end{theorem}
