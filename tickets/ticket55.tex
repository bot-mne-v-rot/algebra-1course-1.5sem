\section{Группа автоморфизмов поля из $p^n$ элементов}
\begin{conj}
    $Aut(K)$~--- множество всех автоморфизмов поля $K$ (очевидно, это группа относительно композиции).

    Аналогично $Aut(L/K)$~--- группа всех автоморфизмов расширения (т.е. таких автоморфизмов $L$, которые каждый элемент из $K$ переводят в себя же).
\end{conj}
\begin{lemma}
    \begin{gather*}
        \left| Aut(K(a) / K) \right| \le [K(a) : K]
    \end{gather*}
\end{lemma}
\begin{proof}
    Пусть $f$~--- минимальный многочлен $a$ над $K$.
    Рассмотрим отображение $\lambda$, действующее в множество всех корней $f$.
    \begin{gather*}
        Aut(K(a)/K) \longrightarrow \{ x \in K(a) \mid f(x) = 0 \} \\
        \varphi \mapsto \varphi(a)
    \end{gather*}

    Надо проверить, что $\varphi(a)$ обязано являться корнем $f$.
    Действительно, пусть $f=\alpha_0 + \alpha_1 X + \cdots + \alpha_d X^d$.
    Тогда $\varphi(f(a)) = \varphi(\alpha_0) + \varphi(\alpha_1) \varphi(a) + \cdots + \varphi(\alpha_d) \varphi(a)^d = \alpha_0 + \alpha_1 \varphi(a) + \cdots + \alpha_d (\varphi(a))^d = f(\varphi(a))$ (поскольку коэффициенты многочлена были из $K$, то автоморфизм (по определению изоморфизма расширений) должен коэффициенты переводить в себя же).
    Полученное равенство дает $f(\varphi(a))=\varphi(f(a))=\varphi(0)=0$~--- проверили что образ $\lambda$ действительно содержится в множестве корней $f$.

    С другой стороны, автоморфизм такого расширения однозначно задается своим значением на $a$.
    Действительно, мы знаем структуру $K(a)$:
    \begin{gather*}
        K(a) = \{ \alpha_0 + \alpha_1 a + \alpha_{d-1}a^{d-1} \mid \alpha_0, \alpha_1, \ldots, \alpha_{d-1} \in K \}
    \end{gather*}
    Тогда зная лишь $\varphi(a)$ и держа в уме, что $\varphi$ все элементы $K$ по определению обязан переводить самих в себя, можем получить значение $\varphi$ для произвольного элемента из $K(a)$:
    \begin{gather*}
        \varphi(\alpha_0 + \alpha_1 a + \alpha_{d-1}a^{d-1}) = \alpha_0 + \alpha_1 \varphi(a) + \alpha_{d-1}\varphi(a)^{d-1}
    \end{gather*}
    Таким образом $\lambda$ инъективно.
    Значит, $\left|Aut(K(a)/K) \right|$ не превосходит числа различных корней $f$, т.е. $[K(a):K]$

    (Если коротко: мы убедились, что $\varphi(a)$ должен быть корнем $f$ и что $\varphi$ единственным образом восстанавливается по $\varphi(a)$, поэтому различных автоморфизмов не больше чем различных значений $\varphi(a)$, которых не больше чем корней $f$).
\end{proof}

\begin{theorem-non}
    $Aut(\mathbb{F}_{p^n})$~--- циклическая группа порядка $n$, порожденная автоморфизмом Фробениуса.
\end{theorem-non}
\begin{proof}
    Заметим, что любой автоморфизм должен переводить единицу в единицу, поэтому сумму $k$ единиц он должен перевести в сумму $k$ единиц, т.е. любой автоморфизм $\mathbb{F}_{p^n}$ должен переводить элементы $\mathbb{F}_p$ в себя же $\Rightarrow$ $Aut(\mathbb{F}_{p^n}) = Aut(\mathbb{F}_{p^n} / \mathbb{F}_p)$.
    Очевидно, группа автоморфизмов, порожденная $Fr$~--- подгруппа в в $Aut(\mathbb{F}_{p^n} / \mathbb{F}_p)$.
    Поэтому по лемме достаточно лишь проверить, что $|\langle Fr\rangle| = n$.
    Поскольку группа циклическая, то ее порядок равен минимальной степени $d$, такой, что $Fr^d=id$.
    При $d=n$ имеем $Fr^n$~--- отображение, переводящее $x \mapsto x^{p^n}$, однако $x^{p^n}=x$ по псевдотеореме Эйлера, поэтому $Fr^n = id$.
    При $d<n$ имеем $Fr^d$, которое действует $x \mapsto x^{p^d}$. 
    Заметим, что элементы, которые оно переводит в себя же, являются корнями многочлена $X^{p^d}-X$, которых не больше $p^d$, поэтому при $d<n$ всегда найдется элемент, который переводится не в себя $\Rightarrow$ при $d<n$, $Fr^d \neq id$.
    Таким образом, $|\langle Fr\rangle | = n$, что мы и хотели доказать.
\end{proof}