\section{Возможные порядки конечного поля. Существование поля из $p^n$ элементов}
Мы поняли, что размер поля имеет вид $p^n$, где $p\in\mathbb{P}$, $n\in \N$ (см. следствие из пункта про простые поля).
На самом деле, верно и обратное
\begin{theorem-non}
    Пусть $p$ простое, $n$~--- натуральное.
    Тогда существует поле из $p^n$ элементов.
\end{theorem-non}
\begin{proof}
    Для удобства введем $q := p^n$
    Пусть $L$~--- поле разложения многочлена $X^q-X$ над $\mathbb{F}_p$.
    Заметим, что у $X^q-X$ нет кратных корней.
    Действительно, в первом семестре мы проверяли что кратный корень должен быть также корнем формальной производной многочлена.
    Но производная равна $qX^{q-1}-1$.
    Поскольку характеристика поля равна $p$ а $p\mid q$, то первое слагаемое зануляется и остается константная $-1$ $\Rightarrow$ у производной корней нет.

    С другой стороны $X^q-X$ раскладывается на линейные множители в $L$ (ведь это поле разложения).
    Значит, можем рассмотреть множество $M$ корней многочлена $X^q-X$ в $L$ и для него будет верно $|M|=q$.
    Однако $M$ будет являться подполем!
    Проверим это.
    \begin{enumerate}
        \item $0,1\in M$~--- очевидно
        \item Замкнутость по умножению: пусть $a, b\in M$, т.е. $a,b$~--- корни $X^q-X$.
        Соответственно, $a^q=a, b^q=b$.
        Тогда $(ab)^q=a^q b^q = ab \Rightarrow$ $ab$ тоже корень $X^q-X$ $\Rightarrow$ $ab\in M$
        \item Замкнутость по сложению: пусть $a, b\in M$, опять же, $a^q=a, b^q=a$.
        Тогда $(a+b)^q = a^q + b^q = a + b$ $\Rightarrow$ $a+b \in M$ (равенство $(a+b)^q = a^q+b^q$ верно по тем же причинам, что и в эндоморфизме Фробениуса. Вообще, возведение в $q$-ю степень равносильно применению $n$ раз эндоморфизма Фробениуса, из этого тоже следует справедливость равенства).
        \item Замкнутость по взятию обратного: пусть $a^q \in M$, т.е. $a^q=a$. Тогда $(a^{-1})^q = (a^q)^{-1}=a^{-1} \Rightarrow a^{-1} \in M$.
        \item Замнкутость по взятию противоположного: пусть $a^q=a$, тогда $(-a)^q = (-1)^q a^q = -1 \cdot a = -a \Rightarrow (-a) \in M$  (Здесь стоит проверить, что $(-1)^q=-1$. Действительно, если $p\neq 2$, то $p$ нечетно $\Rightarrow$ $q$ нечетно $\Rightarrow$ $(-1)^q = -1$; если же $p = 2$, то $q$ четно и $(-1)^q=1$, но в поле с характеристикой $2$ выполняется $-1=1$, поэтому тоже все хорошо).
    \end{enumerate}\medskip

    Но тогда поле $M$ и является искомым!
    (Более того, поскольку мы выбирали $L$ как поле разложения $X^q-X$, а $M$ уже является полем и содержит все корни этого многочлена, то $L=M$).
\end{proof}