\section{Характеристический многочлен сужения оператора на циклическое подпространство}
Поймем, как устроена матрица индуцированного на $L_v$ оператора $\A$.
Зафиксируем базис $E = \{ v, \A v, \A^2 v, \dots, \A^{d-1} v \}$.
Тогда: \[ [\A |_{L_{v}}]_E = 
\left(\begin{array}{ccccc}
    0 & 0 & \dots & 0 & -\alpha_0 \\ 
    1 & 0 & \dots & 0 & -\alpha_1 \\ 
    0 & 1 & \dots & 0 & -\alpha_2 \\ 
    \vdots & \vdots & \vdots & \vdots & \vdots \\ 
    0 & 0 & 0 & 1 & -\alpha_{d - 1}
\end{array}\right) - \text{ сопровождающая матрица многочлена $\mu_{\A, v}$.} \]
Действительно, первый вектор переходит во второй, второй в третий и т.д. При применении оператора к последнему появляются те самые коэффициенты из доказательства. 

Матрица называется сопровождающей матрицей многочлена $\mu_{\A, v}$, так как он полностью определяет ее.

\vspace*{3mm}

\begin{lemma}
    Характеристический многочлен индуцированного оператора с точностью до знака равен соответствующему минимальному многочлену:
    \[ \chi_{\A |_{L_v}} = \pm \mu_{\A, v} \]
\end{lemma}
\begin{proof}
    \[ \chi_{\A |_{L_v}} = \begin{vmatrix*}
        -x & 0 & \dots & 0 & -\alpha_0 \\ 
        1 & -x & \dots & 0 & -\alpha_1 \\ 
        0 & 1 & \dots & 0 & -\alpha_2 \\ 
        \vdots & \vdots & \vdots & \vdots & \vdots \\ 
        0 & 0 & 0 & 1 & -\alpha_{d - 1} - x
    \end{vmatrix*} \]

    Воспользуемся разложением по последнему столбцу.
    Для этого надо просуммировать следующие слагаемые: \begin{itemize}
        \item $(-\alpha_0) \cdot (-1)^{d+1} \cdot \begin{vmatrix}
            1 & 0 & \dots & 0 \\
            0 & 1 & \dots & 0 \\
            0 & 0 & \ddots & 0 \\
            0 & 0 & \dots & 1 
        \end{vmatrix} = (-1)^d \alpha_0$  
        \item $(-\alpha_1) \cdot (-1)^{d+2} \cdot \begin{vmatrix}
            -x & 0 & \dots & 0 \\
            0 & 1 & \dots & 0 \\
            0 & 0 & \ddots & 0 \\
            0 & 0 & \dots & 1 
        \end{vmatrix} = (-1)^d \alpha_1 x$
        \item \dots
        \item $(-\alpha_{d-1} - x) \cdot (-1)^{d+d} \cdot \begin{vmatrix}
            -x & 0 & \dots & 0 \\
            0 & -x & \dots & 0 \\
            0 & 0 & \ddots & 0 \\
            0 & 0 & \dots & -x 
        \end{vmatrix} = (-1)^d (\alpha_{d-1}x^{d-1} + x^d)$
    \end{itemize}
    Таким образом: \[ \chi_{\A |_{L_v}} = (-1)^d(\alpha_0 + \alpha_1 x + \dots + \alpha_{d - 1}x^{d-1} + x^d) = \pm \mu_{\A, v}  \]
\end{proof}
