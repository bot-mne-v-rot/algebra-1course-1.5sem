\section{Собственные значения. Лин. независимость собственных векторов, принадлежащих разным собств. значениям. Следствия}
\begin{conj}
    Собственным вектором линейного оператора $\A$ называется такой ненулевой вектор $v \in V$, что для некоторого $\lambda \in K$ выполнено $\A v = \lambda v$.

    Собственным значением линейного оператора $\A$ называется такой скаляр $\lambda \in K$, что для некоторого ненулевого вектора $v \in V$ выполнено $\A v = \lambda v$
    (говорят, что в таком случае вектор $v$ принадлежит скаляру $\lambda$).
\end{conj}

Заметим, что данное условие можно переписать так:
\[ \A v = \lambda v \Leftrightarrow (\A - \lambda\mathcal{E}_V)v = 0 \Leftrightarrow v \in \Ker(\A - \lambda\mathcal{E}_V) \]
То есть, если $\lambda$ является собственным значением, то все собственные векторы этого $\lambda$ лежат в $\Ker(\A - \lambda\mathcal{E}_V)$. 
Поэтому это подпространство называют собственным.

Введем важную характеристику этого подпространства.

\begin{conj}
    Геометричекая кратность собственного значения -- размерность, принадлажащего ему собственного подпространства:
    \[ g_\lambda = \dim \Ker(\A - \lambda\mathcal{E}) \]
\end{conj}

\begin{theorem-non}
    Собственные векторы, принадлежащие разным собственным значениям, линейно независимы.
\end{theorem-non}

\begin{proof}
    Пусть $v_1, v_2, \dots, v_m$ -- собственные векторы, принадлежащие собственным значениям $\lambda_1, \lambda_2, \dots, \lambda_m$.
    Будем доказывать по индукции.
    
    База $m = 1$: по определению собственного вектора $v_1 \neq 0 \Rightarrow v_1$ -- ЛНС.

    Переход $m - 1 \to m$: предположим, что существует нетривиальная линейная комбинация, которая дает 0:
    \begin{gather}
        \alpha_1v_1 + \alpha_2v_2 \dots + \alpha_mv_m = 0
    \end{gather}
    \quad Сразу заметим, что $\alpha_m \neq 0$, так как в ином случае нарушилось бы ИП. 

    \quad Воспользуемся тем, что образ 0 это 0:
    \begin{gather*}
        \A(\alpha_1v_1 + \alpha_2v_2 \dots + \alpha_mv_m) = 0 \\
        \alpha_1\lambda_1v_1 + \alpha_2\lambda_2v_2 \dots + \alpha_m\lambda_mv_m = 0
    \end{gather*}
    \quad Сделам так, чтобы ушло последнее слагаемое.
    Для этого вычтем выражение $(1)$, домноженное на $\lambda_m$:
    \begin{gather*}
        \alpha_1(\lambda_1 - \lambda_m)v_1 + \alpha_2(\lambda_2 - \lambda_m)v_2 \dots + \alpha_{m - 1}(\lambda_{m - 1} - \lambda_m)v_{m - 1} = 0
    \end{gather*}
    \quad Осталось заметить, что $v_1, v_2, \dots, v_{m - 1}$ -- ЛНС по ИП, и множители $(\lambda_i - \lambda_m) \neq 0$, так как собственные значения различны по условию.
    Следовательно $\alpha_1, \alpha_2, \dots, \alpha_{m-1} = 0$. 
    Но $a_m \neq 0$ и $v_m \neq 0$, следовательно эта комбинация не дает 0, и мы пришли к противоречию.
\end{proof}

\begin{follow}
    Пусть $\lambda_1, \lambda_2, \dots, \lambda_m$ -- собственные значения $\A$, $V_1, V_2, \dots, V_m$ -- соответствующие им собственные подпространства.
    Тогда \[ V_1 + V_2 + \dots + V_m = V_1 \oplus V_2 \oplus \dots \oplus V_m. \] 
    Это означает, что любой вектор из $V_1 + V_2 + \dots + V_m$ единственным образом раскладывается в сумма конкретных векторов этих подпространств.
\end{follow}

\begin{proof}
    От обратного. Пусть \[ v_1 + v_2 + \dots + v_m = v_1' + v_2' + \dots + v_m', \]
    \quad где $v_i, v_i' \in V_i$.
    Тогда \[ \underbrace{(v_1 - v_1')}_{\in V_1} + \underbrace{(v_2 - v_2')}_{\in V_2} + \dots + \underbrace{(v_m - v_m')}_{\in V_m} = 0. \]
    \quad Если есть какие-то $v_i - v_i' \neq 0$, то наше равенство противоречит линейной независимости (предыдущее предложение), следовательно $v_i - v_i' = 0$ для всех $i$.
\end{proof}

\begin{follow}
    Пусть $\lambda_1, \lambda_2, \dots, \lambda_m$ -- все собственные значения $\A$. 
    Тогда сумма их геометрических кратностей не превосходит размерности $V$:
    \[ g_{\lambda_1} + g_{\lambda_2} + \dots + g_{\lambda_m} \leqslant \dim V \]
\end{follow}
\begin{proof}
    Это очевидно, так как сумма соответствующих собственных пространств будет прямой суммой:
    \[ \dim(V_1 + V_2 + \dots + V_m) 
    = \dim(V_1 \oplus V_2 \oplus \dots \oplus V_m)
    =  g_{\lambda_1} + g_{\lambda_2} + \dots + g_{\lambda_m} \]
    \quad А это то, что нам надо, так как размерность подпространства не превосходит размерности пространства.
\end{proof}
