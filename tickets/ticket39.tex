\section{Каноническая форма нормального оператора в унитарном пространстве}
\begin{theorem}
    Пусть $V$ -- унитарное пространство, $\A \in \End V$ -- нормальный. 
    Тогда в $V$ есть ортонормированный базис $E$ такой, что $[\A]_E$ -- диагональная.
\end{theorem}
\begin{proof}
    Индукция по $n = \dim V$.
    \begin{itemize}
        \item База $n = 1$ тривиальна. Возьмем любой вектор и отнормируем его, матрица будет диагональной.
        \item Переход $n - 1 \to n$. 
        Так как $V$ -- унитарное, характеристический многочлен $\chi_{\A}$ раскладывается на линейный множители. 
        Возьмем любой из его корней, это будет какое-то собственное значение, и возьмем любой нормированный вектор, принадлежащий этому собственному значению. 
        Обозначим этот вектор за $e_1$. 

        Воспользуемся первым предложением: так как $\Lin(e_1)$ была $\A$-инвариантна (ведь $e_1$ собственный), $W := \Lin(e_1)^\perp$ будет $\A^*$-инвариантна.
        Согласно третьему предложению, $e_1$ будет также собственным вектором $\A^*$, значит, он будет $\A^*$-инвариантен, значит, $W$ будет $\A$-инвариантно.

        Так как $W$ будет и $\A$-инвариантно, и $\A^*$-инвариантно, операторы $\A$ и $\A^*$ останутся сопряженными при сужении на $W$: $(\A\big|_W)^* = \A^*\big|_W$. 
        Значит, они все еще будут нормальными, и мы можем применить к $W$ индукционное предположение: $\dim W = n - 1 \Rightarrow$ найдется ортонормированный базис $e_2, \dots, e_n$ с диагональной матрицей.
        Тогда базис $e_1, e_2, \dots, e_n$ -- искомый, потому что матрица очевидно будет диагональной ($e_1$ собственный), $e_1$ нормированный и ортогональный $e_2, \dots, e_n$.
    \end{itemize}
\end{proof}
