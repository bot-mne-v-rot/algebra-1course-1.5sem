\section{Лекция номер 3}

\subsection{Канонический гомоморфизм}

\begin{conj}
Канонический гомоморфизм проекции группы на фактор-группу.

$\pi_{H}: G \to G / H$

$\quad \quad g \mapsto gH$

\begin{proof}
    $\pi_{H}(g) \cdot \pi_{H}(g') = gH \cdot g'H = gg'H = \pi_{H}(gg')$
\end{proof}
$\Imm \pi_{H} = G / H$ (т.е. это сюръективный гомоморфизм)

$\Ker \pi_{H} = H$ (т.к. $gH = eH \Leftrightarrow g \in H$) 
\end{conj}

\begin{conj}
Канонический гомоморфизм вложения подгруппы $K$ в группу $G$

$i_{K}: K \to G$

$\quad \quad g \mapsto g$

$\Imm i_{K} = K$

$\Ker i_{K} = \{ e \}$ (т.е. это инъективный гомоморфизм)
\end{conj}

\notice 

$\varphi: G \to G'$~--- гомоморфизм

$\varphi$~--- инъекция $\Longleftrightarrow \Ker \varphi = \{ e \}$

\begin{proof}
    "$\Longrightarrow$"
    
    $\varphi$~--- инъекция $\Longrightarrow \abs{\varphi(e_{G'})^{-1}} \leqslant 1$
    
    (не в точности $1$, так как $\varphi$ может и не быть сюръекцией, но точно не больше $1$ из-за инъекции)

    $\varphi(e_{G'})^{-1} = \Ker \varphi \ni e_{G}$
    (всегда, т.к. гомоморфизм переводит нейтральный в нейтральный)

    "$\Longleftarrow$"

    $\Ker \varphi = \{ e \}$

    Пусть $\varphi(g_1) = \varphi(g_2)$ для некоторых $g_1, g_2 \in G$

    Рассмотрим $\varphi(g_2^{-1}g_1) = \varphi(g_2^{-1})\varphi(g_1) = \varphi(g_2)^{-1}\varphi(g_1) = \varphi(g_2)^{-1}\varphi(g_2) = e_{G'}$

    $\Longrightarrow g_2^{-1}g_1 \in \Ker \varphi \Longrightarrow g_2^{-1}g_1 = e \Longrightarrow g_1 = g_2 \Longrightarrow \varphi$~--- инъекция. 
\end{proof}

\subsection{Разложение гомоморфизма через канонический гомоморфизм}

Теперь, определив два таких гомоморфизма, мы можем доказать два полезных разложения любого гомоморфизма групп. 

\begin{theorem-non}
    Разложение гомоморфизма через каноническую проекцию на фактор-группу.

    $\varphi: G \to G'$~--- гомоморфизм групп.

    $H \vartriangleleft G, H \subset \Ker \varphi$

    Тогда $\exists !$ гомоморфизм $\varphi': G/H \to G':$

    $\varphi = \varphi' \circ \pi_{H}$

    Иными словами, гомоморфизм двух групп единственным образом раскладывается в 2 гомоморфизма:

    $G \stackrel{\varphi}{\to} G'$ то же, что и $G \stackrel{\pi_{H}}{\to} G / H \stackrel{\varphi'}{\to} G'$

    \begin{proof}
        $ $ \\
        Существование:

        $\varphi'(gH) := \varphi(g)$

        Проверим корректность определения:

        $g_1H = g_2H \stackrel{?}{\Longrightarrow} \varphi(g_1) = \varphi(g_2)$
        
        $g_1H = g_2H \Longrightarrow g_2 = g_1h,\, h \in H$

        $\varphi(g_2) = \varphi(g_1)\underbrace{\varphi(h)}_{\text{$H \subset \Ker \varphi$}} = \varphi(g_1)e = \varphi(g_1)$

        Единственность:

        $\varphi(g) = (\varphi' \circ \pi_{H})(g) = \varphi'(gH)$

        $\Longrightarrow \varphi'(gH)$ определён однозначно.

        Проверим гомоморфизм:
        \begin{gather*}
            \varphi'(g_1H \cdot g_2H) = \varphi'(g_1g_2H) = \varphi(g_1g_2) = \varphi(g_1)\varphi(g_2) = \varphi'(g_1H)\varphi'(g_2H)    
        \end{gather*}
        
        $\forall g \in G: \varphi(g) = (\varphi' \circ \pi_{H})(g)$~--- равенство выполняется.
    \end{proof}
\end{theorem-non}

\begin{theorem-non}
    Разложение гомоморфизма через каноническое вложение подгруппы в группу. 

    $\varphi: G \to G'$~--- гомоморфизм групп.

    $K < G', \Imm \varphi \subset K$

    Тогда $\exists !$ гомоморфизм $\tilde{\varphi}: G \to K:$

    $\varphi = \tilde{\varphi} \circ i_K$

    Иными словами, гомоморфизм двух групп единственным образом раскладывается в 2 гомоморфизма:

    $G \stackrel{\varphi}{\to} G'$ то же, что и $G \stackrel{\tilde{\varphi}}{\to} K \stackrel{i_{K}}{\to} G'$

    \begin{proof}
        $ $ \\
        $\tilde{\varphi}(g) := \varphi(g)$
        
        То же самое отображение, только образ не $G'$, а $K$.
        
    \end{proof}
\end{theorem-non}

\subsection{Индуцированный гомоморфизм}

\begin{theorem-non}
    Разложение гомоморфизма в композицию трёх гомоморфизмов.

    $H \vartriangleleft G, H \subset \Ker \varphi$

    $K \subset G', \Imm \varphi \subset K$

    $\exists !$ гомоморфизм $\varphi'': G / H \to K:$

    $\varphi =  i_K \circ \varphi'' \circ \pi_H$

    Иными словами, гомоморфизм двух групп единственным образом раскладывается в 3 гомоморфизма:

    $G \stackrel{\varphi}{\to} G'$ то же, что и $G \stackrel{\pi_{H}}{\to} G/H \stackrel{\varphi''}{\to} K \stackrel{i_{K}}{\to} G'$

    \begin{proof}
        Воспользуемся двумя предложениями выше.

        Сперва построим гомоморфизм $\varphi': G / H \to G'$ \, такой, что $\varphi = \varphi' \circ \pi_H$

        Заметив, что $\Imm \varphi' = \Imm \varphi \subset K$, мы можем разложить $\varphi'$ через:

        $\varphi'': G / H \to K$ \, такой, что $\varphi' = i_K \circ \varphi''$

        Тогда получим, что: 
        \begin{gather*}
            \varphi = \varphi' \circ \pi_H = i_K \circ \varphi'' \circ \pi_H
        \end{gather*}
    \end{proof}

    \notice Гомоморфизм $\varphi''$ из $G / H$ в $K$ называется индуцированным гомоморфизмом $\varphi$
\end{theorem-non}

\subsection{Теорема о гомоморфизме}

\begin{theorem}
    О гомоморфизме.

    $\varphi: G \to G'$~--- гомоморфизм групп.
    
    Тогда индуцированный гомоморфизм
    $\varphi'': G / \Ker \varphi \to \Imm \varphi$
    является изоморфизмом.
    
    Здесь мы берём за $H$ всё $\Ker \varphi$, а за $K$ весь $\Imm \varphi$,
    то есть это частный случай индуцированного гомоморфизма.

    \begin{proof}
        $\varphi''(g\Ker \varphi) = \varphi(g)$

        Сюръективность:

        $g' \in \Imm \varphi \Longrightarrow \exists g \in G$

        $g' = \varphi(g) = \varphi''(g \Ker \varphi) \Longrightarrow g' \in \Imm \varphi''$

        т.е. $\Imm \varphi'' = \Im \varphi $

        Инъективность:

        Проверим тривиальность ядра

        Пусть $g \Ker \varphi \in \Ker \varphi''$

        $\varphi''(g  \Ker \varphi ) = e_{G'} \Longrightarrow \varphi(g) = e_{G'} \Longrightarrow g \Ker \varphi = e_{G} \Ker \varphi$

        Таким образом $\Ker \varphi'' = \{ e_{G} \Ker \varphi \}$, т.е. $\varphi''$~--- инъективен.
        
    \end{proof}
\end{theorem}

\underline{Примеры нахождения изоморфизма:}
\begin{enumerate}
    \item $G := \C^* \; H := \mathbb{T} = \{ z \, | \, |z| = 1 \}$
    
    $\C^* / \mathbb{T} = ?$ 

    $\varphi: \C^* \to \R^{*}$

    $\quad \quad z \mapsto |z|$

    $\Imm \varphi = \R^{*}_{+}$

    $\Ker \varphi = \mathbb{T}$

    По Т. о гомоморфизме, индуцированный гомоморфизм $\C^{*} / \mathbb{T} \to \R^{*}_{+}$~--- изоморфизм.

    Т.о. $\C^* / \mathbb{T} \cong \R^{*}_+$

    \notice $\R^{*}_+ \cong \R (a \mapsto \ln{a})$

    \item $G / H := \C^* / \cycle{i}$
    
    $\ord i = 4, \, \cycle{i} = \{ i, -1, -i, 1 \}$
    
    $\varphi: \C^* \to \C^*$
    
    $\quad \quad z \mapsto z^4$

    $\Ker \varphi = \cycle{i}$

    $\Imm \varphi = \C^*$

    $\Longrightarrow \C^* / \cycle{i} \cong \C^*$

    \item $G / H := S_n / A_n \quad (A_n \vartriangleleft S_n)$
    
    $\varphi: S_n \to \Z^* = \{ 1, -1 \}$

    $\quad \quad \sigma \mapsto \sign(\sigma)$

    $\Ker \varphi = A_n$

    $\Imm \varphi = \Z^*$

    $\Longrightarrow S_n / A_n \cong \Z^{*}$
\end{enumerate}

\subsection{Классификация циклических групп}

\begin{theorem-non}
    $G$~--- циклическая группа 

    \begin{enumerate}
        \item $G$~--- бесконечная $\Longrightarrow G \cong \Z$
        \item $|G| = n < +\infty \Longrightarrow G \cong \Z / n \Z$
    \end{enumerate}

    \begin{proof}
        
        $G = \cycle{g}$

        $\varphi: \Z \to G$ (очевидно, гомоморфизм)

        $\quad \quad a \mapsto g^a$

        \begin{enumerate}
            \item
            $G$~--- бесконечная $\Longrightarrow g^m \neq e \, \forall m \in \N$
            
            $\Longrightarrow g^{-m} \neq e \, \forall m \in \N$

            Т.о. $\Ker \varphi = \{ e \} \Longrightarrow \varphi$~--- инъективен

            $\Imm \varphi = \{ g^a \, | \, a \in \Z \} = \cycle{g} = G$

            $\Longrightarrow \Z / \Ker \varphi \cong \Imm \varphi \Longrightarrow \Z \cong G$

            \item
            $|G| = n < +\infty \Longrightarrow \ord{g} = n$
            
            Посмотрим на ядро.

            Знаем, что:
            \begin{itemize}
                \item $n \in \Ker \varphi$
                \item $0 < m < n \Longrightarrow m \not \in \Ker \varphi$
                \item $t \in \Z \; t = nq + r \, 0 \leqslant r \leqslant n - 1$
                $r = 0 \Longrightarrow t \in \Ker \varphi$
                $r \neq 0 \Longrightarrow r \not \in \Ker \varphi \Longrightarrow t \not \in \Ker \varphi$
            \end{itemize}

            Т.о. $\Ker \varphi = \{ nq \, | \, q \in \Z \} = n \Z$

            $\Longrightarrow \Z / \Ker \varphi \cong \Imm \varphi \Longrightarrow \Z / n \Z \cong G$
             
        \end{enumerate}
    \end{proof}

    Второе доказательство:
    \begin{proof}
        
        $G \cong \Z / \Ker \varphi$

        Поищем, каким может быть ядро.

        Знаем, что в $\Z$ идеалы выглядят так: $(n) \, n \in \N_{0}$

        Любая подгруппа $H$ в $\Z$~--- идеал в $\Z:$

        $la = \underbrace{a + \dots + a}_{\text{l}} \in H \quad l \in \N, a \in H$

        $(-l)a = -(la) \in H$

        $0 \cdot a = 0 \in H$

        Любая подгруппа~--- идеал, а все идеалы главные.

        $\Longrightarrow G \cong \Z / (n):$
        \begin{itemize}
            \item $n < +\infty \Longrightarrow G \cong \Z / n \Z$
            \item Иначе $G \cong \Z$
        \end{itemize}
    \end{proof}
\end{theorem-non}

\subsection{Две теоремы об изоморфизме}

\begin{theorem-non}
    Образ и прообраз подгруппы.

    $\varphi: G \to G'$~--- гомоморфизм групп.

    \begin{enumerate}
        \item $H < G \Longrightarrow \varphi(H) < G'$
        \item $K < G' \Longrightarrow \varphi^{-1}(K) < G$
    \end{enumerate}

    \begin{proof}
        $ $ \\
        \begin{enumerate}
            \item 
            $\varphi(H) \neq \varnothing \; (e_G \in H \Rightarrow e_{G'} \in \varphi(H))$
        
            $\varphi(h_1) \varphi(h_2) = \varphi(h_1h_2) \in \varphi(H) \quad h_1, h_2 \in H$

            $\varphi(h)^{-1} = \varphi(h^{-1}) \in \varphi(H) \quad h \in H$
            \item 
            $\varphi^{-1}(K) \neq \varnothing \; (e_{G'} \in K \Rightarrow e_{G} \in \varphi^{-1}(K))$

            $\varphi(g_1g_2) = \underbrace{\varphi(g_1)}_{\text{$\in K$}} \underbrace{\varphi(g_2)}_{\text{$\in K$}} \in K \quad g_1, g_2 \in \varphi^{-1}(K)$
        \end{enumerate}
        \end{proof}

\end{theorem-non}

\begin{theorem}
    О соответствии.

    $H \vartriangleleft G$

    Тогда следующие два отображения являются взаимнообратными биекциями:
    \begin{itemize}
        \item
        $\{ K < G \, | \, H \subset K \} \rightarrow \{ K' < G / H \}$

        $K \mapsto \pi_H(K)$
        \item
        $\{ K' < G / H \} \rightarrow \{ K < G \, | \, H \subset K \} $

        $K' \mapsto \pi^{-1}_{H}(K')$
    \end{itemize}

    При этом $K \vartriangleleft G \Longleftrightarrow \pi_{H}(K) \vartriangleleft G / H$
    
    \begin{proof}
        Знаем, что $\pi_H(K) < G / H$ и $H \subset \pi^{-1}_{H}(K') < G$

        Проверим, что они взаимнообратные:

        $\pi^{-1}_{H}(\pi_{H}(K)) = \bigcup_{k \in K} kH \stackrel{H \subset K}{=} K$

        
        $\pi_H(\pi^{-1}_{H}(K')) = K' \quad (\pi^{-1}_{H}(K') < G / H)$

        Проверим заявленную нормальность:

        $K \vartriangleleft G \Longleftrightarrow \forall g \in G, k \in K: gkg^{-1} \in K 
        \Longleftrightarrow \forall g \in G, k \in K: \pi_{H}(gkg^{-1}) \in \pi_{H}(K)$

        $\Longleftrightarrow \underbrace{\forall g \in G, k \in K \; (gH)(kH)(gH)^{-1} \in \pi_H(K)}_{\text{*}}$

        Но так как:
        \begin{itemize}
            \item $\{ kH \, | \, k \in K \} = \pi_H(K)$
            \item $\{ gH \, | \, g \in G \} = G / H$
        \end{itemize}
        То $ * \Longleftrightarrow \pi_H(K) \vartriangleleft G / H$
    \end{proof}

\end{theorem}