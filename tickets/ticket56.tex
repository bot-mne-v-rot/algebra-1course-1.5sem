\section{Подполя поля из $p^n$ элементов}
\begin{theorem-non}
    Пусть $p\in\mathbb{P}, n\in \N$.
    Тогда
    \begin{enumerate}
        \item $\forall m: m\mid n$ в $\mathbb{F}_{p^n}$ есть подполе из $p^m$ элементов
        \item других подполей нет
    \end{enumerate}
\end{theorem-non}
\begin{proof}
    Сначала докажем второй пункт.
    Пусть $K$~--- подполе в $\mathbb{F}_{p^n}$.
    Тогда можем рассмотреть расширение $\mathbb{F}_{p^n} / K$.
    Если его степень равна $d$, то получается, что $\mathbb{F}_{p^n}$~--- $d$-мерное пространство над полем из $|K|$ элементов, откуда $|\mathbb{F}_{p^n}| = |K|^d$.
    Характеристика $K$~--- такая же, как у надполя, поэтому $|K|=p^m$ для некоторого $m$.
    Окончательно, $p^n = (p^m)^d = p^{md}$, откуда $|K| = p^m$ и $m \mid n$.\medskip

    Найдем теперь подполе из $p^m$ элементов.
    Пусть $n=mr$.
    Заметим, что в этом случае $(p^m-1) \mid (p^n-1)$.
    Действительно, $p^n = (p^m)^r$, $1=1^r$, поэтому можно расписать $p^n-1$ по формуле разностей $r$-х степеней и получить $(p^m)^r - 1^r = (p^m-1)((p^m)^{r-1} + (p^m)^{r-2}+\cdots + p^m + 1)$, отсюда сразу видно, что $p^n-1$ делится на $p^m-1$.
    После этого абсолютно аналогично доказывается, что $(X^{p^m-1}-1) \mid (X^{p^n-1}-1)$ (достаточно в предыдущем предложении заменить везде $p$ на $X$, $m$ на $p^m-1$, $n$~--- на $p^n-1$).
    Отсюда домножением на $X$ получаем, что $(X^{p^m}-X) \mid (X^{p^n}-X)$.
    Но мы-то помним, что $X^{p^n}-X$ раскладывается на попарно различные линейные множители (см. псевдотеорему Эйлера).
    Значит, $X^{p^m}-X$ тоже раскладывается на попарно различные линейные множители.
    Теперь, ровно как и в теореме о существовании поля из $p^n$ элементов, можем рассмотреть множество $M$ корней многочлена $X^{p^m}-X$ и убедиться, что оно будет подполем, да еще и нужного размера.
    Единственность такого подполя опять же следует из псевдотеоремы Эйлера: в любом поле из $p^m$ элементов для любого элемента должно выполняться $a^p=a$, но таких элементов в $\mathbb{F}_{p^n}$ всего $p^m$ и все они лежат в $M$, поэтому подполе из $p^m$ элементов обязано равняться $M$.
\end{proof}
\follow Эта теорема позволяет установить биекцию между подгруппами группы $Aut(\mathbb{F}_{p^n})$ и подполями $\mathbb{F}_{p^n}$.
Действительно, как мы убеждались, группа $Aut(\mathbb{F}_{p^n})$~--- циклическая порядка $n$ и порождена автоморфизмом Фробениуса $Fr$.
Тогда (согласно какому-то утверждению из доколлочной части второго семестра) все ее подгруппы имеют вид $\langle Fr^d\rangle $, где $d \mid n$.
Несложно проверить, что $\langle Fr^d\rangle $ будет группой всевозможных автоморфизмов расширения $\mathbb{F}_{p^n} / \mathbb{F}_{p^d}$ (т.е. таких автоморфизмов $\mathbb{F}_{p^n}$, которые переводят элементы $\mathbb{F}_{p^d}$ в себя же).
Таким образом подгруппе $\langle Fr^d \rangle$ естественным образом сопоставляется подполе $\mathbb{F}_{p^d}$ и такое сопоставление будет биективным.