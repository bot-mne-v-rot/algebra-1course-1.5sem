\section{Критерий диагонализируемости в терминах геометрических и алгебраических кратностей. Примеры недиагонализируемых операторов}
\begin{theorem}(О диоганолизации оператора и характеристическом многочлене)

    Пусть $\A \in \End{V}$.

    Тогда эквивалентны 2 утверждения:
    \begin{enumerate}
        \item $\A$ диагонализируемый
        \item $\chi_{\A}$ раскладывается на линейные множители
        и для любого собственного значения $\lambda:$
        \[ g_{\lambda} = a_{\lambda} \]
    \end{enumerate}

    \begin{proof} \quad

    \quad$1 \Longrightarrow 2:$
    Знаем, что матрица оператора $\A$ в некотором базисе $E$ выглядит так: 
    \[
        \left(\begin{array}{ccc}
            \lambda_1 &  & 0 \\ 
            & \ddots &  \\ 
            0 &  & \lambda_n
        \end{array}\right)
    \]

    Очевидно, что характеристический многочлен этого оператора выглядит так:
        \[ \chi_{\A} = \prod_{i = 1}^{n} (\lambda_i - x) \]
    По определению $g_{\lambda}$ равно максимальному числу линейно независимых векторов, принадлежащих собственному значению $\lambda$.
    Базисные векторы у нас линейно независимы, поэтому $g_\lambda$ равно числу вхождений $\lambda$ в $\{ \lambda_1, \dots, \lambda_n \}$ (вообше это надо бы построже доказать, но в целом это очевидно).
    Алгебраическая кратность $\lambda$ это кратность $\lambda$ как корня $\chi_\A$, что по сути то же самое, поэтому $g_\lambda = a_\lambda$.
    
    \quad$2 \Longrightarrow 1:$
    Пусть $\lambda_1, \dots, \lambda_m$ -- все собственные значения без повторений.
    Тогда \[ a_{\lambda_1} + \dots + \alpha_{\lambda_m} = \deg{\chi_{\A}} = n \Longrightarrow g_{\lambda_1} + \dots + g_{\lambda_m} = n \Longrightarrow \A \text{ диагонализируем} \]
    \end{proof}
\end{theorem}

С помощью этого предложения мы теперь видим две причины, по которым оператор может быть не диагонализируем:
\begin{enumerate}
    \item Характеристический многочлен может не раскладываться на линейные множители (поле $K$ не является алгебраически замкнутым).
    \item Алгебраическая и геометрическая кратности какого-то собственного значения просто напросто отличаются.
\end{enumerate}

\vspace*{4mm}

\underline{Рассмотрим примеры:}
\begin{enumerate}
    \item Пусть $ K = \R$, $V$ -- произвольное двумерное, $E$ -- его базис, и матрица оператора имеет вид \[ [\A]_{E} = \left(\begin{array}{cc}
    0 & 1 \\ 
    -1 & 0
    \end{array}\right) \] 
    Тогда $\chi_{\A} = x^2 + 1$. 
    У этого многочлена нет корней в $R$, следовательно, $\A$ не диагонализируемый.

    \item Пусть $K = \C$, $V$ -- произвольное двумерное, $E$ -- его базис, и матрица оператора имеет вид
    \[ [\A]_{E} = \left(\begin{array}{cc}
        0 & 0 \\ 
        1 & 0
    \end{array}\right)\] 
    Тогда $\chi_\A = x^2$, а это означает, что $a_0 = 2$.
    Заметим, что наш оператор действует так: $e_1 \mapsto e_2 \mapsto 0$.
    Значит, только вектор $e_2$ и все кратные ему переходят в 0 $\Rightarrow g_0 = \Lin(e_2) = 1$.
    Получили, что $g_0 < a_0 \Longrightarrow \A$ не диагонализируемый.
\end{enumerate}