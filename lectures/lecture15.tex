\section{Лекция номер 15}
\mybox[orange!15]{Законспектированная лекция -- это то, что Жуков назвал лекцией на похожую тематику. 
Тут, откровенно говоря, происходит какая-то жесть. Авторы ни за что не ручаются.}

\begin{conj}
    Если $L$ -- некоторое поле и $K$ -- его подполе, то $L$ называется расширением поля $K$.
\end{conj}

\begin{conj}
Расширение $L \supset K$ называется конечным, если поле $L$ есть конечномерное линейное пространство над $F$. 
Размерность этого пространства над $K$ называется степенью расширения и обозначается как $[L : K]$. 
\end{conj}

\textbf{Fun fact}

Пусть $x$ -- алгебраический элемент над полем $K$. $f_0$ -- минимальный многочлен $x$, 
то есть многочлен минимальной степени, 
для которого $x$ является корнем. Логично, что такой многочлен всегда неприводим. И тогда мы можем сказать, что:
\begin{gather*}
    K(x) \cong K[x] / (f_0) 
\end{gather*}
В общем случае, если многочлен неприводим, то факторкольцо по порожденному им идеалом всегда является полем. 

Рассмотрим пример. Как мы знаем, поле комплексных получается из $\R$ присоединением мнимой единицы, значит: 
\begin{gather*}
    \C = \R(i) \cong \R[x] / (x^2 + 1)
\end{gather*}
Еще мы уже рассматривали $\Q(\sqrt{2})$, которое будет изоморфно $\Q[x](x^2 - 2)$. 

Отсюда сразу видно, чему будет равна степень такого расширения.

\follow 
\begin{gather*}
    [K(x) : K] = \deg{f_0}
\end{gather*}
\begin{proof}
    Пусть $d := \deg{f_0}$. Посмотрим на классы $1, \bar{X}, \bar{X}^2, \dots, \bar{X}^{d-1}$. 
    Утверждается, что они линейно независимы. Почему? У нас есть данный изоморфизм:
    \begin{gather*}
        K(x) \cong K[x] / (f_0) \\
        x \longmapsto \bar{X}
    \end{gather*}
    Значит из того, что $f(\bar{X}) = 0$ следует, что $f(x) = 0 \Longrightarrow f \; \vdots \; f_0 \Longrightarrow f = 0$. 
    Значит элементы линейно независимы, а значит порождают всё поле $K[x]/(f_0)$, а значит это базис. Значит размерность этого векторого пространства 
    равна как раз $d$. 
\end{proof}

\mybox[orange!15]{Мы знаем, что структура простого расширения опеределяется минимальным 
многочленом присоединяемого элемента. Но! На самом деле мы можем для любого неприводимого многочлена построить 
простое расширение так, чтобы этот самый многочлен и был минимальным ногочленом присоединяемого элемента.}

\begin{theorem}
    Пусть $K$ -- поле, $f \in K[x]$ -- неприводимый многочлен над этим полем. Тогда существует простое расширение $L/K$, 
    такое, что $L = K(x)$ и $f$ -- минимальный многочлен $x$. 
\end{theorem}
\begin{proof}
    Просто возьмем $L = K[x]/(f)$. $f$ -- неприводим по условию, значит $L$ -- поле. Нам теперь нужно, чтобы $K$ было его подполем. 
    Построим отображение:
    \begin{gather*}
        K \longhookrightarrow L \\
        \alpha \longmapsto \bar{\alpha}
    \end{gather*} 
    Понтовая стрелочка означает, что отображение инъективно. Просто отождествим константу $\alpha$ с классом многочлена $\alpha$. Инъективность 
    следует из того, что константа не может делить многочлен. Из инъективности следует, что $K$ будет подполем $L$. Элемент $x \; (\operatorname{mod} \ f) \in L$  
    есть корень многочлена $f$. 
\end{proof}
\begin{conj}
    Пусть $K$ -- поле, $f \in K[x]$ -- некоторый многочлен, $L/K$ -- расширение. Тогда $L$ называется \textbf{полем разложения} $f$, если:
    \begin{enumerate}
        \item $f$ там раскладывается на линейные множетели. $f = a_0 \cdot (X - x_1) \dots (X - x_n)$ в $L[x]$. 
        \item $L$ получается из $K$ присоединением всех этих корней. $L = K(x_1, \dots, x_n)$. То есть $L$ является наименьшим полем с таким свойством. 
    \end{enumerate}
\end{conj}
\begin{theorem}
    Пусть $K$ -- поле, $f$ -- его многочлен. Тогда у $f$ существует поле разложения над $K$. 
\end{theorem}
\begin{proof}
    Индукция по степени $f$. 

    \quad \underline{База $f = 1$:} Тогда у него есть единственный корень и он лежит в $K$, то есть $K$ -- само поле разложения. 

    \quad \underline{Переход:} Пусть $f_0$ -- любой неприводимый делитель $f$. Возьмем и присоединим к $K$ любой его корень. $L := K(x)$, где $x$ -- корень $f_0$.  
\end{proof}
