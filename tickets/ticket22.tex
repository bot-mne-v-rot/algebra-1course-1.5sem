\section{Центр конечной p-группы}
\begin{theorem-non}
    Пусть $G$ -- конечная $p$-группа (т.е. $|G| = p^n, p$ -- простое).
    Тогда у нее нетривиальный центр: $Z(G) \neq \{ e \}$.
\end{theorem-non}
\begin{proof}
    Пусть $G$ действует на себе сопряжением. 
    Из следствия мы знаем, что длина орбиты является делителем порядка группы. 
    Тогда орбиты могут быть следующих длин: $1, p, p^2, \dots, p^n$. 

    Пусть $m_i$ -- число орбит длины $p^i$. 
    Заметим, что $|G| = p^n = \sum\limits_{i = 0}^n p^im_i$, так как каждый элемент $G$ лежит в орбите, и причем только одной.
    Перепишем полученную сумму: \[ \underbrace{p^n}_{\mod p = 0} = m_0 + \underbrace{pm_1 + p^2m_2 + \dots + p^nm_n}_{\mod p = 0} \Rightarrow p \mid m_0 \]
    Мы поняли, что $p$ делит $m_0$. 
    Осталось заметить, что $m_0 = |Z(G)|$, так как это ровно те элементы, которые при любом сопряжении всегда дают себя (у них орбита длины 1), то есть они коммутируют со всеми.

    Следовательно, $p \mid |Z(G)|$, а в таком случае $Z(G)$ никак не может равняться $\{ e \}$.
\end{proof}
