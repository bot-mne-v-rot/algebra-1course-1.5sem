\section{Орбиты и стабилизаторы}
\begin{conj}
    Пусть задано действие группы $G$ на множестве $M$.
    Тогда орбита элемента $m$ -- это множество $Gm = \{gm\; | \; g \in G\}$.
\end{conj}

\begin{theorem-non}
    Орбиты двух элементов $M$ либо не пересекаются, либо совпадают.
\end{theorem-non}
\begin{proof}
    \begin{gather*}
        h \in Gm_1 \cap Gm_2 \\
        h = g_1m_1 = g_2m_2 \\
        m_2 = g_2^{-1}g_1m_1  \\
        \Rightarrow \forall g \in G: gm_2 = gg_2^{-1}g_1m_1 \in Gm_1 \\
        \Rightarrow Gm_2 \subset Gm_1 \\
        \text{Аналогично } Gm_1 \subset Gm_2 \\
        \Rightarrow Gm_1 = Gm_2
    \end{gather*}
\end{proof}
Посмотрим, какие орбиты у нас получились в примерах:
\begin{enumerate}
    \item Взяли точку на комплексной плоскости и начали поворачивать на всевозможные углы \\ 
    $\Rightarrow$ получили окружность с нужным радиусом $\Rightarrow Gm = \{ z \in \mathbb{C} \; | \; |z| = |m| \}$
    \item Нетрудно доказать, что из любого ненулевого столбца мы можем получить любой другой ненулевой. 
    Тогда $Gm = \begin{cases}
        \{ 0 \}, & m = 0 \\
        K^n \setminus \{ 0 \}, & m \neq 0
    \end{cases} $
    \item Очевидно, что будет лишь одна орбита: $Gm = M = G$.
    \item Орбитами будут классы сопряженности.
\end{enumerate}

Если $\forall m \in M : Gm = M$, то действие $G$ на $M$ называется транзитивным. 
Иными словами, мы можем перевести каждый элемент в любой другой. 
Легко видеть, что действие на каждой отдельной орбите транзитивно.

\begin{conj}
    Пусть $G$ действует на $M$. 
    Стабилизатором элемента $m$ называется множество $St_m = \{ g \in G \; | \; gm = m \}$.
\end{conj}

\begin{theorem-non} \quad 

    \begin{enumerate}
        \item $St_m < G$ (поэтому стабилизатор иногда называют стационарной подгруппой) 
        \item Существует биекция: 
        \begin{gather*}
            \varphi: G / St_m \to Gm \\
            gSt_m \mapsto gm
        \end{gather*}
    \end{enumerate}
\end{theorem-non}
\begin{proof} \quad \\
    \begin{enumerate}
        \item Легко проверить замкнутость относительно умножения, существование нейтрального и обратного.
        \item Проверим корректность отображения:
        \begin{gather*}
            g_1St_m = g_2St_m \\
            g_2 = g_1s, s \in St_m \\
            g_2m = g_1sm \underbrace{=}_{\text{так как } s \in St_m} g_1m
        \end{gather*}
        Проверим инъективность:
        \begin{gather*}
            \varphi(gSt_m) = \varphi(g'St_m) \\
            gm = g'm \\
            m = g^{-1}g'm \Rightarrow g^{-1}g' \in St_m \\
            g'\in gSt_m \Rightarrow g'St_m = gSt_m
        \end{gather*}
        Сюръективность тривиальна, так как мы можем брать любые $g$ (значит, мы и получим всю орбиту $m$).
    \end{enumerate}
\end{proof}

\begin{follow}
    Пусть $G$ -- конечная группа, $m \in M$.
    Тогда длина орбиты делит порядок группы: $|Gm| \mid |G|$.
\end{follow}
\begin{proof}
    Из предыдущего предложения мы поняли, что $Gm$ биективно отображается на $G / St_m$, следовательно, $|Gm| = |G / St_m|$.
    Но $|G / St_m|$ является индексом подгруппы $St_m$ по определению, значит, делит порядок группы.
\end{proof}

Посмотрим, какие стабилизаторы у нас получились в примерах:
\begin{enumerate}
    \item $St_0 = \R$, так как 0 всегда будет оставатсья 0 при повороте \\ $St_{m \neq 0} = \langle 2\pi \rangle$
    \item Неочевидно (в первый раз это слово употребляется с приставкой не)
    \item $St_g = \{ e \}$
    \item $St_g = \{ g \in G \, | \; hg = gh \} = Z_g$ -- централизатор элемента $g$ 
\end{enumerate}