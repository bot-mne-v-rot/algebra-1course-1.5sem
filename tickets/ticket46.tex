\section{Степень расширения. Мультипликативность степени}
\begin{conj}
    Пусть $L / K$ --- расширение. Над $L$ есть структура линейного пространства над $K$.
    
    Просто забудем, что мы умеем умножать элементы из $L$ друг на друга. Очевидно, все аксиомы ЛП выполняются.
\end{conj}

\begin{conj}
    Пусть $L / K$ --- расширение. Тогда можно рассматривать размерность поля $L$, как размерность линейного пространства над $K$, т.е. $\dim_K L$. Её называют \textbf{степенью расширения} и обозначают: $(L : K)$, $[L : K]$, $\abs{L : K}$. Мы будем использовать второй вариант.

    Если $[L : K] < \infty$, то $L$ --- \textbf{конечное расширение}  $K$. В противном случае, $L$ --- \textbf{бесконечное расширение} $K$.
\end{conj}

\textbf{Примеры:} 
\begin{itemize}
    \item $\C / \R$ --- конечное расширение, т.к. $[\C : \R] = 2$.
    \item $\R / \Q$ --- бесконечное расширение из-за мощностных соображений.
    \item $K(x) / K$ --- бесконечное расширение, т.к. $K(x)$ содержит многочлены, которые могут быть любой длины.
\end{itemize}

\begin{theorem}[Мультипликативность степени]
    Пусть $M / L$, $L / K$ --- конечные расширения. Тогда $M / K$ тоже конечно и $[M : K] = [M : L] \cdot [L : K]$.
\end{theorem}
\begin{proof} $ $

    Пусть $e_1, \dots, e_m$ --- базис $L / K$, $f_1, \dots, f_m$ --- базис $M / L$. Докажем, что $(e_i f_j)$ --- базиc $M / K$.

    Проверим, что это порождающая система. Возьмём $c \in M$. Тогда:
    \begin{align*}
        c &= \sum_{j=1}^n \beta_j f_j \quad \text{ (где $\beta_j \in L $)} \\
        &= \sum_{j=1}^n \left( \sum_{i=1}^m \alpha_{ij} e_i \right) f_j \quad \text{ (где $\alpha_{ij} \in K$) } \\
        &= \sum_{i=1}^{m} \sum_{j=1}^{n} \alpha_{ij} e_i f_j \\
        &\in \Lin_K(e_i f_j \mid i=1..m, j=1..n)
    \end{align*}

    Проверим линейную независимость. Пусть:
    \begin{align*}
        0 &= \sum_{i=1}^{m} \sum_{j=1}^{n} \alpha_{ij} e_i f_j \quad \text{ (где $\alpha_{ij} \in K$) } \\
        &= \sum_{j=1}^n \underbrace{\left( \sum_{i=1}^m \alpha_{ij} e_i \right)}_{\in L} f_j \\
        &\Rightarrow \sum_{i=1}^m \alpha_{ij} e_i = 0 \; \;\forall j=1..n \quad \text{ (т.к. $(f_j)$ --- базис $M / L$) } \\
        &\Rightarrow \alpha_{ij} = 0 \; \; \forall i=1..m, j=1..n \quad \text{(т.к. $(e_i)$ --- базис $L / K$)}
    \end{align*}
\end{proof}