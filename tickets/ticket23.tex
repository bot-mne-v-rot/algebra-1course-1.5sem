\section{Разрешимость конечной p-группы}
\follow (Из теоремы в предыдущем билете)
    Любая конечная $p$-группа разрешима (ггвп, а не следствие).

\begin{proof}
    Индукция по $n$.
    
    Пусть $|G| = p^n$.
    Обозначим: $G' = G / Z(G)$.
    По предыдущему предложению $Z(G) \neq \{ e \} \Rightarrow |G'| = \frac{|G|}{(G : Z(G))} < |G|$ и $G'$ тоже являеется $p$-группой.
    Если $G' = \{ e \}$, то $G = Z(G) \Rightarrow G$ абелева, а значит разрешима.

    Теперь разберем содержательный случай, когда $G' \neq \{ e \}$. 
    Применим к ней индукционное предложение: \[ \{e \} = G_n' < \dots < G_1' < G_0' = G', \] где $G_{k+1}' \lhd G_k'$ и $G_k'/G_{k+1}'$ абелева.
    Посмотрим на гомоморфизм проекций на факторгруппу (каждому элементу сопоставляется его класс):
    \begin{gather*}
        \pi_{Z(G)}: G \to G / Z(G) = G' \\
        g \mapsto gZ(G) 
    \end{gather*}
    Введем $G_k = \pi_{Z(G)}^{-1}(G_k')$ -- все элементы, которые перешли в данные классы ($G_k'$ подгруппа в $G'$, то есть это просто подгруппа классов).
    Тогда мы можем выписать ряд c соотвествующими $G_k$: \[ \underbrace{\Ker \pi_{Z(G)}}_{\text{прообраз $e$}} = G_n < \dots < G_1 < G_0 = G \] 
    По определнию $\Ker \pi_{Z(G)} = Z(G)$. 
    Возьмем внутри $Z(G)$ тривиальную подгруппу.
    Тогда наш ряд примет вид: \[ \{ e \} = G_{n+1} < Z(G) = G_n < \dots < G_1 < G_0 = G \]  
    Покажем, что данный ряд является рядом в определении разрешимой группы:
    \begin{itemize}
        \item $G_{k + 1} \lhd G_k$.
        
        Отдельно рассмотрим случай $G_{n + 1} \lhd G_n$. Это очевидно, потому что $G_{n+1} = \{ e \}$.
        
        Теперь случай $k \in [0, n - 1]$. Наше отображение $\pi_{Z(G)}$ переводило $G_k$ в $G_k'$, тогда по теореме о соответствии из $G_{k+1}' \lhd G_k'$ следует, что $G_{k+1} \lhd G_k$. 
        \item $G_k / G_{k+1}$ абелева.
        
        Отдельно рассмотрим случай $G_n / G_{n+1}$. 
        Это $Z(G) / \{ e \} \cong Z(G)$, а $Z(G)$ абелева.

        Теперь случай $k \in [0, n - 1]$. 
        Можно понять, что $G_k' \cong G_k / Z(G)$ по определению этих самых $G_k'$ и $G_k$.
        Тогда  \[ \underbrace{G_k' / G_{k+1}'}_{\text{абелева}} \cong (G_k / Z(G)) / (G_{k+1} / Z(g)) \cong G_k / G_{k+1} \] 
        Второй переход это теорема о факторизации факторгруппы.
    \end{itemize}
    Следовательно, $G$ разрешима. 
\end{proof}