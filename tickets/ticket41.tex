\section{Лемма о базисах в сопряжённых собственных подпространствах}
\begin{lemma}
    Пусть вектора $v_1 + iw_1, \dots, v_l + iw_l$ -- это базис $V_{\lambda}$, где $\lambda$ не вещественное, а $(V_{\lambda} = \Ker{\A_{\C} - \lambda \mathcal{E}})$.
    Тогда сопряженные к ним векторы $v_1 - iw_1, \dots, v_l - iw_l$~--- базис $V_{\bar{\lambda}}$

    Более того, если первый базис ортонормированный, то и второй ортонормированный.
\end{lemma}


\begin{proof} \quad 
    
Нетрудно проверить, что:
\begin{gather*}
    v + iw \in V_{\lambda} \Longrightarrow v - iw \in V_{\bar{\lambda}}
\end{gather*}
Также:
\begin{gather*}
    V_{\lambda} = \Lin(v_1 + iw_1, \dots, v_l + iw_l) \Longrightarrow V_{\bar{\lambda}} = \Lin(v_1 - iw_1, \dots, v_l - iw_l)
\end{gather*}
Давайте это проверим. Запишем произвольный вектор как линейную комбинацию базисных векторов: 
\begin{gather*}
    v + iw = \sum_{j = 1}^{l} (\alpha_j + i \beta_j) (v_j + i w_j) \Longrightarrow v - iw = \sum_{}^{} (\alpha_j - i \beta_j) (v_j - i w_j)
\end{gather*}
Аналогично преверяем линейную независимость и первое утверждение доказано. 

Теперь докажем утверждение про ортонормированность:
\begin{gather*}
    (v_j + i w_j, v_k + iw_k) = \delta_{jk} \Longrightarrow  (v_j - i w_j, v_k - iw_k) = \delta_{jk}
\end{gather*}
Просто применили комплексное сопряжение.
\end{proof}