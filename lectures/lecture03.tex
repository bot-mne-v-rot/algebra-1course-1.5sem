\section{Лекция номер 3}

\begin{conj}
Канонический гомоморфизм проекции группы на фактор-группу.

$\pi_{H}: G \to G / H$

$\quad \quad g \mapsto gH$

\begin{proof}
    $\pi_{H}(g) \cdot \pi_{H}(g') = gH \cdot g'H = gg'H = \pi_{H}(gg')$
\end{proof}
$\Imm \pi_{H} = G / H$ (т.е. это сюръективный гомоморфизм)

$\Ker \pi_{H} = H$ (т.к. $gH = eH \Leftrightarrow g \in H$) 
\end{conj}

\begin{conj}
Канонический гомоморфизм вложения подгруппы $K$ в группу $G$

$i_{K}: K \to G$

$\quad \quad g \mapsto g$

$\Imm i_{K} = K$

$\Ker i_{K} = \{ e \}$ (т.е. это инъективный гомоморфизм)
\end{conj}

\notice 

$\varphi: G \to G'$~--- гомоморфизм

$\varphi$~--- инъекция $\Longleftrightarrow \Ker \varphi = \{ e \}$

\begin{proof}
    "$\Longrightarrow$"
    
    $\varphi$~--- инъекция $\Longrightarrow \abs{\varphi(e_{G'})^{-1}} \leqslant 1$
    
    (не в точности $1$, так как $\varphi$ может и не быть сюръекцией, но точно не больше $1$ из-за инъекции)

    $\varphi(e_{G'})^{-1} = \Ker \varphi \ni e_{G}$
    (всегда, т.к. гомоморфизм переводит нейтральный в нейтральный)

    "$\Longleftarrow$"

    $\Ker \varphi = \{ e \}$

    Пусть $\varphi(g_1) = \varphi(g_2)$ для некоторых $g_1, g_2 \in G$

    Рассмотрим $\varphi(g_2^{-1}g_1) = \varphi(g_2^{-1})\varphi(g_1) = \varphi(g_2)^{-1}\varphi(g_1) = \varphi(g_2)^{-1}\varphi(g_2) = e_{G'}$

    $\Longrightarrow g_2^{-1}g_1 \in \Ker \varphi \Longrightarrow g_2^{-1}g_1 = e \Longrightarrow g_1 = g_2 \Longrightarrow \varphi$~--- инъекция. 
\end{proof}

Теперь, определив два таких гомоморфизма, мы можем доказать два полезных разложения гомоморфизмов. 

\begin{theorem-non}
    $\varphi: G \to G'$~--- гомоморфизм групп.

    $H \vartriangleleft G, H \subset \Ker \varphi$

    Тогда $\exists !$ гомоморфизм $\varphi': G/H \to G':$

    $\varphi = \pi_{H} \circ \varphi'$

    Иными словами, гомоморфизм двух групп единственным образом раскладывается в 2 гомоморфизма:

    $G \stackrel{\varphi}{\to} G'$ то же, что и $G \stackrel{\pi_{H}}{\to} G / H \stackrel{\varphi'}{\to} G'$

    \begin{proof}
        $ $ \\
        Существование:

        $\varphi'(gH) := \varphi(g)$

        Проверим корректность определения:

        $g_1H = g_2H \stackrel{?}{\Longrightarrow} \varphi(g_1) = \varphi(g_2)$
        
        $g_1H = g_2H \Longrightarrow g_2 = g_1h,\, h \in H$

        $\varphi(g_2) = \varphi(g_1)\underbrace{\varphi(h)}_{\text{$H \subset \Ker \varphi$}} = \varphi(g_1)e = \varphi(g_1)$

        Единственность:

        $\varphi(g) = (\varphi'(g) \circ \pi_{H})(g) = \varphi'(gH)$

        $\Longrightarrow \varphi'(gH)$ определён однозначно.

        Проверим гомоморфизм:
        \begin{gather*}
            \varphi'(g_1H \cdot g_2H) = \varphi'(g_1g_2H) = \varphi(g_1g_2) = \varphi(g_1)\varphi(g_2) = \varphi'(g_1H)\varphi'(g_2H)    
        \end{gather*}
        
        $\forall g \in G: \varphi(g) = (\varphi' \circ \pi_{H})(g)$~--- равенство выполняется.
    \end{proof}
\end{theorem-non}