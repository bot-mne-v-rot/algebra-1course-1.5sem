\section{Матрица композиции линейных отображений}
\begin{theorem}
    Пусть $U, V, W$ --- конечномерные ЛП/$K$; $E, F, G$ --- их базисы;
    $\varphi \in \Hom(U, V)$, $\psi \in \Hom(V, W)$. Тогда
    $[\psi \circ \varphi]_{E,G} = [\psi]_{F,G} \cdot [\varphi]_{E,F}$.
\end{theorem}
\begin{proof}
    По определению:
    \begin{align*}
        \varphi(E) &= F \cdot [\varphi]_{E, F} \\
        \psi(F) &= G \cdot [\psi]_{F, G} \\
        \psi(\varphi(E)) &= G \cdot [\psi \circ \varphi]_{E, G} 
    \end{align*}
    Применим $\psi$ к первому равенству.
    Важно, что $[\varphi]_{E, F}$ это матрица-скаляр, поэтому при применении к ней $\psi$ ничего не происходит:
    \[ \psi(\varphi(E)) = \psi(F) \cdot [\varphi]_{E, F} = G \cdot [\psi]_{F, G} \cdot [\varphi]_{E, F}  \]
    Получилось то же самое, что и в третьем равенстве:
    \[ \psi(\varphi(E)) = G \cdot [\psi \circ \varphi]_{E, G} = G \cdot [\psi]_{F, G} \cdot [\varphi]_{E, F}  \]
    $G$ --- базис $\Longrightarrow$ 
    $[\psi \circ \varphi]_{E,G} = [\psi]_{F,G} \cdot [\varphi]_{E,F}$
\end{proof}
