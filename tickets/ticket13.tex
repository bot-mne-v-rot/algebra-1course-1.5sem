\section{Минимальный аннулятор вектора. Циклическое подпространство}

По аналогии с аннулятором оператора можно ввести понятие минимального аннулятора вектора.
Действительно, то, что множество аннуляторов вектора это главный идеал в $K[x]$, доказывается аналогично.
Выделив образующий, получаем минимальный аннулятор вектора.

\begin{conj}
    Пусть $I_v = \{ f \, | \, f(\A) (v) = 0 \} = (\mu_{\A, v})$.
    Тогда $\mu_{\A, v}$ -- минимальный аннулятор $v$ (по отношению к $\A$).
\end{conj}

Очевидно, что любой аннулятор оператора также является аннулятором любого вектора.
Из этого в частности следует, что $I_v \neq 0$ и $\mu_{\A, v} \neq 0$. 

\vspace*{5mm}

\begin{conj}
    Пусть $v \in V$.
    Тогда циклическое подпространство -- это минимальное подпространство $V$, содержащее $v$ и инвариантное относительно $\A$.
    Обозначается как $L_v$ и по определению равно $\Lin (v, \A v, \A^2 v, \dots)$.
\end{conj}

\vspace*{3mm}

Несмотря на то, что это линейная оболочка бесконечного количества векторов, это подпространство конечномерного $V$, поэтому можно выделить базис.

\vspace*{2mm}

\begin{theorem}
    Пусть $d = \deg \mu_{\A, v}$.
    Тогда $\{ v, \A v, \A^2 v, \dots, \A^{d - 1} v \}$ -- базис $L_v$.
\end{theorem}

\begin{proof}
    Докажем линейную независимость.
    Предположим, что $v, \A v, \A^2 v, \dots, \A^{d - 1} v $ -- ЛЗС:
    \[ \exists \beta_0, \dots, \beta_{d-1} : \;\; \beta_0 v + \dots + \beta_{d - 1} \A^{d - 1} v = 0 \]
    \quad Составим многочлен $f = \beta_0 + \beta_1 x + \dots + \beta_{d - 1} x^{d - 1}$. 
    Тогда: \[ f(\A)(v) = 0 \Longrightarrow \mu_{\A, v} | f \Longrightarrow f = 0 \Longrightarrow \beta_0 = \dots = \beta_{d - 1} = 0 \]
    \quad Второй переход верен, так как $\deg \mu_{\A, v} = d$, а $\deg f = d - 1$.
    
    \quad Теперь докажем, что последующие степени нам не интересны, так как они выражаются через первые $d$ штук.
    Пусть $W = \Lin (v, \A v, \dots, \A^{d - 1} v)$. 
    Докажем, что $\forall m \geqslant d$ выполняется $\A^m v \in W$.
    Будем доказывать по индукции: \begin{itemize}
        \item База $m = d$. Пусть минимальный аннулятор $v$ имеет вид: \[ \mu_{\A, v} = x^d + \alpha_{d - 1} x^{d - 1} + \dots + \alpha_0 \]
        Тогда: \begin{gather*}
            \mu_{\A, v}(\A)(v) = \A^dv + \alpha_{d - 1} \A^{d - 1}v + \dots + \alpha_1 \A v + \alpha_1 v = 0 \\
            \Longrightarrow \A^d v = - \alpha_{d - 1} \A^{d - 1}v - \dots - \alpha_0 v \in W
        \end{gather*}
        \item Переход $k \to k + 1$.  
        По индукционному предположению: \[ \A^k v = \alpha_0 v + \dots + \alpha_{d - 1} \A^{d - 1}v \]
        Применим к левой и правой части оператор $\A$: \[ \A^{k+1}v = \underbrace{\alpha_0\A v + \dots}_{\in W} + \underbrace{\alpha_{d-1}\A^d v}_{\in W} \Longrightarrow \A^{k+1}v \in W \]
    \end{itemize}
    \quad Таким образом, $L_v = W$.
\end{proof}

