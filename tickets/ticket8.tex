\section{Характеристический многочлен линейного оператора. Алг. кратность собственного значения}
Опять воспользуемся тем фактом, что значение $\lambda$ собственное тогда и только тогда, когда $\Ker(\A - \lambda\mathcal{E}) \neq 0$.
Мы знаем, что это равносильно тому, что соответствующая матрица в любом базисе будет вырожденной, то есть иметь нулевой определитель: \[ |A - \lambda E_n| = 0 \]
Здесь $A = [\A]_E$ для какого-то базиса $E$.

Посмотрим на эту матрицу:
\begin{gather*}
    \begin{pmatrix}
        a_{11} - \lambda & a_{12} & \dots & a_{1n} \\
        a_{21} & a_{22} - \lambda  & \dots & a_{2n} \\
        \dots & \dots & \dots & \dots \\
        a_{n1} & a_{n2} & \dots & a_{nn} - \lambda 
    \end{pmatrix}
\end{gather*}
Здесь $a_{ij}$ -- соответствующие элементы матрицы $A$.

Давайте честно распишем ее определитель и воспользуемся тем фактом, что он равен 0.
Получится какой-то многочлен от $\lambda$.
Он и называется характеристическим многочленом.

Например, для какой-то матрицы $A$ порядка 2 характеристический многочлен будет следующим:
\begin{gather*}
    \begin{vmatrix}
        a_{11} - \lambda & a_{12} \\
        a_{21} & a_{22} - \lambda
    \end{vmatrix} = 0 \\
    (a_{11} - \lambda)(a_{22} - \lambda) - a_{12}a_{21} = 0 \\
    \lambda^2 - (a_{11} + a_{22})\lambda + a_{11}a_{22} - a_{12}a_{21} = 0
\end{gather*}

Введем формальное определение.
\begin{conj}
    Характеристический многочлен матрицы $A \in M_n(K)$ -- это многочлен от $x$ вида $\chi_A = |A - xE_n|$.
\end{conj}

Зафиксируем очевидное, но важное свойство характеристического многочлена.
\begin{lemma}
    Пусть $A \in M_n(K)$ и $\lambda \in K$. Тогда \[ (A - \lambda E_n) \in GL_n(K) \Leftrightarrow \chi_A(\lambda) \neq 0 \]
\end{lemma}
\begin{proof}
    \[ (A - \lambda E_n) \in GL_n(K) \Leftrightarrow |A - \lambda E_n| \neq 0 \Leftrightarrow \chi_A(\lambda) \neq 0 \]
\end{proof}

\begin{conj}
    Характеристический многочлен оператора -- характеристический многочлен матрицы оператора в каком-либо базисе.
\end{conj}
То есть утверждается, что выбор базиса особой роли не играет.
Проверим это.

\quad Зафиксируем 2 базиса: $E$ и $E'$. 
Мы знаем, что если $C$ -- матрица перехода между $E$ и $E'$, то имеет место равенство \[ [\A]_{E'} = C^{-1}[\A]_EC \]
\quad Распишем характеристические многочлены:
\begin{gather*}
    \begin{split}
        \chi_{[A]_{E'}} &= \chi_{C^{-1}[\A]_EC} \\
        &= |C^{-1}[\A]_EC - xE_n| \\
        &= |C^{-1}[\A]_EC - xCC^{-1}| \\
        &= |C^{-1}[\A]_EC - CxC^{-1}| \; \text{(нам все равно где писать $x$)} \\
        &= |C^{-1}([\A]_E - xE_n)C| \\
        &= |C^{-1}| * |[\A]_E - xE_n| * |C| \; \text{(формула определителя произведения)}\\
        &= |C^{-1}| * \chi_{[A]_E} * |C| = \chi_{[A]_E}
    \end{split}
\end{gather*}

В общем случае коэффициенты характеристического многочлена принимают произвольные значения, но все-таки про некоторые их них мы можем кое-что сказать: 
\begin{itemize}
    \item При $x^n$ будет очевидно коэффициент $(-1)^n$
    \item При $x^{n-1}$ будет коэффициент $Tr A \cdot (-1)^{n-1}x^{n-1}$. 
    Тут надо понять, что $x^{n-1}$ может получиться только из произведения элементов на главной диагонали. 
    Действительно, если мы возьмем один элемент не с диагонали, то нам придется взять еще один не с диагонали, и нужную степень мы уже не получим.
    В каждом из $n$ множителей вида $(a_{ii} - x)$ в произведение мы можем взять либо $a_{ii}$, либо $-x$.
    Мы возьмем $(n - 1)$ раз $(-x)$ и произвольную $a_{ii}$.
    Легко видеть, что получится необходимая формула. ($Tr A = a_{11} + \dots + a_{nn}$--- след матрицы) 
    \item Свободный член получается при подстановке $x = 0$, следовательно, он равен $|A|$.
\end{itemize}

\vspace{5mm}

Мы уже поняли, что \[ \lambda - \text{собственное значение} \Leftrightarrow \chi_{\A}(\lambda) = 0 \]
У корня многочлена всегда есть кратность, поэтому логичным образом вытекает следующее определение.
\begin{conj}
    Алгебраическая кратность собственного значения -- кратность этого значения как корня характеристического многочлена.

    Обозначается как $a_\lambda$.
\end{conj}
