\section{Диагонализируемые операторы. Критерий диагонализируемости в терминах геометрических кратностей}
Особо интресен случай, когда сумма геометрических кратностей в точности равна размерности пространства.
Для этого введем понятие диагонализируемого оператора.
\begin{conj}
    Оператор называется диагонализируемым, если в некотором базисе его матрица диагональная 
    (все элементы, стоящие вне главной диагонали равны 0).
\end{conj}

\begin{theorem}
    Для $\A \in \End V$ следующие 3 условия эквивалентны:
    \begin{enumerate}
        \item $\A$ диагонализируем.
        \item В $V$ существует базис из собственных векторов.
        \item Сумма геометрических кратностей всех собственных значений равна размерности пространства.
    \end{enumerate} 
\end{theorem}

\begin{proof} \quad

    \quad$1 \Rightarrow 2:$ Если $\A$ диагонализируем, то каждый базисный вектор, при попытке применить к нему оператор $\A$, просто на что-то умножится, значит он является собственным
    
    \quad$2 \Rightarrow 1:$ Если базис составлен из собственных векторов и мы считаем матрицу оператора, то каждый такой вектор умножается на соответсвующее собственное значение и при разложении по базису у нас появится это собственное значение, а остальные координаты будут нули.

    \quad$1 \Rightarrow 3:$ Матрица оператора имеет следующий вид:
    \begin{gather*}
        [\A]_E = \begin{pmatrix}
            \beta_1 & 0 & 0 & \dots & 0 & \\
            0 & \beta_2 & 0 & \dots & 0 & \\
            \dots & \dots & \dots & \dots & \dots \\
            0 & 0 & 0 & \dots & \beta_n
        \end{pmatrix}
    \end{gather*}
    \quad Рассмотрим скаляр $\lambda$. 
    Чтобы он был собственным значением, оператор $(A - \lambda\mathcal{E}_V)$ должен иметь нетривиально ядро. 
    Тогда согласно самому первому предложению этой лекции, он должен быть необратим, то есть иметь вырожденную матрицу.
    Посмотрим на его матрицу:
    \begin{gather*}
        [\A - \lambda\mathcal{E}_V]_E = \begin{pmatrix}
            \beta_1 - \lambda & 0 & 0 & \dots & 0 & \\
            0 & \beta_2 - \lambda & 0 & \dots & 0 & \\
            \dots & \dots & \dots & \dots & \dots \\
            0 & 0 & 0 & \dots & \beta_n - \lambda
        \end{pmatrix}
    \end{gather*}
    \quad Чтобы она была вырожденной, $\lambda$ должна совпасть с одной из $\beta_i$. 
    Таким образом, \[ \lambda - \text{собственное значение} \Leftrightarrow \lambda \in \{\beta_1, \beta_2, \dots, \beta_n \}  \]
    \quad Теперь поймем, какая кратность будет у такого собственного значения $\lambda$.
    Заметим, что если значение в $i$-том столбце совпало с $\lambda$, то есть что $\beta_i = \lambda$, то этот столбец в матрице $[\A - \lambda\mathcal{E}_V]_E$ будет нулевым.
    То есть, $(\A - \lambda\mathcal{E}_V) e_i = 0$, а значит $e_i \in \Ker { (\A - \lambda\mathcal{E}_V) } \Longrightarrow \Ker { (\A - \lambda\mathcal{E}_V) } = \Lin (e_j \, | \, \beta_j = \lambda)$ 
    %То есть некоторые столбцы в матрице $[\A - \lambda\mathcal{E}_V]_E$ будут нулевыми, а некоторые нет.
    Получается, что $g_\lambda$ будет в точности равно количеству нулевых столбцов.
    %Это так потому что базисные векторы, которые не зануляются, очевидно не будут образующими ядра.

    \quad Итак, разные $\lambda$ будут занулять разные столбцы и в итоге сумма по $g_\lambda$ будет равна $n$ -- размерности пространства.

    $\quad 3 \Rightarrow 2:$ Пусть $V_i$ -- собственное подпространство $i$-того собственного значения.
    По предыдущим предложениям мы знаем, что \[ \dim(V_1 + V_2 + \dots + V_m) 
    = \dim(V_1 \oplus V_2 \oplus \dots \oplus V_m)
    =  \underbrace{g_{\lambda_1} + g_{\lambda_2} + \dots + g_{\lambda_m}}_{ = n} \]
    \quad Только размерность самого пространства равна $n$, следовательно $V_1 \oplus V_2 \oplus \dots \oplus V_m = V$. 
    Объединив базисы $V_1, V_2, \dots, V_m$, мы получим искомый базис $E$, который будет состоять из собственных векторов.
\end{proof}
