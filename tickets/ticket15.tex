\section{Теорема о произведении подгрупп}
\begin{theorem}
    (Теорема о произведении подгрупп)

    Пусть $H \lhd G, K < G$. Тогда $H \lhd KH, H \cap K \lhd K$ и $KH/H \cong K/(K \cap H)$
\end{theorem}

\begin{proof} \quad 

    \begin{itemize}
        \item $H \lhd KH$ -- тривиально 
        \item Возьмем $c \in K \cap H, k \in K$. Тогда: 
        \begin{align*}
            kck^{-1} &\in H, \text{ т. к. } c \in H \\
            kck^{-1} &\in K, \text{ т. к. все три этих элемента } \in K
        \end{align*} А значит, $kck^{-1} \in K \cap H$
        \item Применим теорему о гомоморфизме. Введем следующее отображение: 
        \begin{gather*}
            \varphi: K \longrightarrow KH/H \\
            k \longmapsto kH
        \end{gather*}
        Элемент $K$ одновременно является элементом $KH$, так как $k = ke$. И поскольку 
        класс произведения равен произведению классов, $\varphi$ -- гомоморфизм. 

        Посмотрим на образ $\varphi$: $\Imm{\varphi} = \{kH \mid k \in K\}$, при этом 
        $KH/H = \{\underbrace{khH}_{kH} \mid k \in K, h \in H\} = \{kH \mid k \in K\}$.
        Получили, что $\varphi$ сюръективен. $\Ker{\varphi} = \{k \in K \mid kH = eH\} =
        \{k \in K \mid k \in H\} = K \cap H$. Итого, по теореме о гомоморфизме: 
        \begin{gather*}
            K/(K \cap H) \cong KH/H
        \end{gather*}
    \end{itemize}
\end{proof}

\follow Пусть $G$ -- конечная группа. Тогда 
\begin{gather*}
    \frac{\abs{K}}{\abs{K \cap H}} = \left( K : (K \cap H)\right) = (KH : H) = 
    \frac{\abs{KH}}{\abs{H}} \Longrightarrow \\
    \abs{KH} = \frac{\abs{K} \cdot \abs{H}}{\abs{K \cap H}}
\end{gather*}