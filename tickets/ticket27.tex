\section{Ортонормированный базис в евклидовом пространстве. Процесс ортогонализации Грама – Шмидта}
\begin{conj}
    Базис $E$ евклидова пространства $V$ называется ортонормированным, если матрица скалярного произведения в нем равна единичной матрице. То есть если $\forall i, j: (e_i, e_j) = \delta_{ij}$, где 
    $e_1, \dots, e_n$ -- базисные вектора $E$.  
\end{conj}
Давайте теперь опишем алгоритм построения ортонормированного базиса. 
\begin{theorem-non} (Ортогонализация Грама-Шмидта)

    Пусть $f_1, \dots, f_n$ -- произвольный базис $V$. Тогда в $V$ существует ортонормированный базис $e_1, \dots, e_n$, причем $\forall j : e_j \in \Lin{(f_1, \dots, f_j)}$.  
\end{theorem-non}
\begin{proof}
    Наш ортонормированный базис будем строить рекурсивно. Первый вектор $e_1 = \frac{1}{\abs{f_1}} f_1$. Теперь пусть $k$ векторов $e_1, \dots, e_k$, где $k < n$ уже построены. Построим очередной вектор.  
    \begin{gather*}
        e_{k+1}^\circ = f_{k+1} + \alpha_1 e_1 + \dots + \alpha_k e_k
    \end{gather*} 
    То, что $e_{k+1}^\circ \in \Lin{(f_1, \dots, f_{k+1})}$ -- очевидно. Нужно выбрать такие $\alpha$, чтобы была ортогональность. Значит хотим, чтобы выполнялось равенство:
    \begin{gather*}
        (e_{k+1}^\circ, e_j) = 0, \; j=1, \dots, k \\
        (e_{k+1}^\circ, e_j) = (f_{k+1}, e_j) + \alpha_j \Longrightarrow \alpha_j = -(f_{k+1} , e_j)
    \end{gather*} 
    Получаем, что: 
    \begin{gather*}
        e_{k+1}^\circ = f_{k+1} - \sum\limits_{j=1}^k (f_{k+1}, e_j)e_j
    \end{gather*}
    Ортогональность есть, пренадлежность линейной комбинации $f$ тоже есть, осталось нам этот вектор отнормировать. Получаем: 
    \begin{gather*}
        e_{k+1} = \frac{1}{\abs{e_{k+1}^\circ}} \cdot e_{k+1}^\circ
    \end{gather*}
    Ортогональность никуда не делась, а ортонормированность появилась. 
\end{proof}