\section{Свойства сопряжённого оператора}
\begin{theorem-non}
    Свойства сопряженного оператора:
    \begin{enumerate}
        \item $(\A^*)^* = \A$
        \item $(\A + \B)^* = \A^* + \B^*$
        \item $(\alpha \A)^* = \bar{\alpha} \A^*$
        \item $(\A\B)^* = \B^*\A^*$
    \end{enumerate}
\end{theorem-non}
\begin{proof} \quad

    \begin{enumerate}
        \item Достаточно проверить, что $(\A^*v, w) = (v, \A w)$. Тут надо воспользоваться эрмитовостью, чтобы поменять порядок: $(\A^*v, w) = \overline{(w, \A^*v)}$ и $(v, \A w) = \overline{(\A w, v)}$. Легко видеть, что части под сопряжением равны, следовательно, равны и изначальные выражения.
        \item Достаточно проверить, что $((\A + \B)v, w) = (v, (\A^* + \B^*)w)$. Это очевидно из соображений линейности.
        \item Аналогично пункту 2.
        \item Достаточно проверить, что $((\A\B)v, w) = (v, (\B^*\A^*)w)$. Это действительно так, ведь $((\A\B)v, w) = (\A(\B v), w) = (\B v, \A^*w) = (v, \B^*\A^*w)$.
    \end{enumerate}
\end{proof}

Посчитаем матрицу сопряженного оператора. Это удобно делать в ортонормированном базисе.
\begin{theorem-non}
    Пусть $E$ -- ортонормированный базис $V$, $\A \in \End V$, $[\A]_E = A$.
    Тогда $[\A^*]_E = A^*$.
\end{theorem-non}
\begin{proof}
    Для удобства будем обозначать $A$ как $(a_{ij})$, а $[\A^*]_E$ как $(b_{ij})$.
    Посмотрим на $j$-тый столбец матрицы $A$: $\A e_j = a_{1j}e_1 + \dots + a_{nj}e_n$.
    Умножим его скалярно на $e_i$: вследствие ортонормированности получаем $(\A e_j, e_i) = a_{ij}$. 
    Воспользуемся сопряженным оператором: $(\A e_j, e_i) = (e_j, \A^* e_i) = a_{ij}$. 
    Заметим, что $\A^* e_i$ -- это $i$-тый столбец $A^*$, значит при скалярном умножении сократится почти все, кроме коэффициента $b_{ji}$, который вынесется с комплексным сопряжением (так как взят из правой части).
    Итого $a_{ij} = \bar{b}_{ji}$ для любых $i, j$, следовательно $[\A^*]_E = A^*$.
\end{proof}