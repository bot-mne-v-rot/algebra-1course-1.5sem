\section{Примарные подпространства. Разложение пространства в прямую сумму (максимальных) примарных подпространств}
\begin{conj}
    Вектор $v \in V$ называется $p$-примарным, если $\mu_{\A, v} = p^t$, где $p \in K[x]$ неприводимый и $t \geqslant 0$.
\end{conj}

\notice Случай, когда $t = 0$ тривиален, так как тогда $v = 0$.

\vspace*{3mm}

\begin{lemma}
    Все $p$-примарные векторы образуют инвариантное подпространство: \[ W_p = \{ v \in V \, | \, v - \text{ $p$-примарный}  \} - \text{инвариантное подпространство } V \]
\end{lemma}
\begin{proof} \quad

    \begin{itemize}
        \item Замкнутость по сложению: \begin{gather*}
            w_1, w_2 \in W_p \Rightarrow p^{t_1}(\A)(w_1) = 0 \, \text{ и } \,  p^{t_2}(\A)(w_2) = 0 \\
            \Longrightarrow p^t(\A)(w_1 + w_2) = p^t(\A)(w_1) + p^t(\A)(w_2) = 0, \quad \text{где } t = \max(t_1, t_2) \\
            \Longrightarrow \mu_{\A, w_1 + w_2} \, | \, p^t \Longrightarrow (w_1 + w_2) \in W_p
        \end{gather*}
        \item Умножение на скаляр: \[ \mu_{\A, w} = \mu_{\A, \lambda w} \, \text{ по определению } \]
        \item Инвариантность: \begin{gather*}
            W_p = \bigcup_{t \geqslant 0} \Ker p^t(\A) \\
            \forall t \geqslant 0 \;\; \Ker p^t(\A) - \text{ $\A$-инвариантно } \Longrightarrow W_p - \text{ $\A$-инвариантно }
        \end{gather*}
        Вообще говоря, мы рассматриваем объединение конечномерных, вложенных друг в друга подпространств, поэтому оно конечно и равно $\Ker p^N(\A)$ при достаточно большом $N$.
    \end{itemize}    
\end{proof}

\begin{conj}
    Векторное пространство $V$ называется $p$-примарным (относительно $\A$), если $V = W_p$.
\end{conj}

\vspace*{3mm}
Докажем техническую лемму, которая понадобится нам при доказательстве последующей теоремы.

\begin{lemma}
    Пусть $\mu_{\A} = fg$, причем $(f, g) = 1$.
    Тогда: \begin{gather*}
        V = V_1 \oplus V_2, \; \text{ где } V_1, V_2 - \text{ $\A$-инвариантны} \\
        \mu_{\A |_{V_1}} \, | \, f \, \text{ и } \,  \mu_{\A |_{V_2}} \, | \, g
    \end{gather*}
\end{lemma}
\begin{proof}
    Пусть $ V_1 := \Ker f(\A), V_2 := \Ker g(\A)$ -- инвариантные подпространства.
    Докажем, что они подходят.
    \begin{itemize}
        \item Поймём, что $V_1 + V_2 = V$. 
        Воспользуемся тем, что многочлены взаимно просты: \[ (f, g) = 1 \Longrightarrow fa + gb = 1, \quad a, b \in K[x] \]
        Теперь возьмем произвольный вектор из $V$ и разложим его в сумму: \[ \forall v \in V \;\; v = \mathcal{E}(v) = (f(\A)a(\A) + g(\A)b(\A))(v) = f(a(\A)(v)) + g(b(\A)(v)) \]
        Осталось заметить, что $f(\A)g(\A) = g(\A)f(\A) = \mu_\A(\A) = 0$ по условию. 
        Поэтому, если мы действуем оператором $g(\A)$ на $v$, то обязаны попасть в ядро $f(\A)$, и наоборот.
        Таким образом: \[ \begin{cases}
            f(a(\A)(v)) \in \Ker g(\A) = V_2 \\
            g(b(\A)(v)) \in \Ker f(\A) = V_1
        \end{cases} \]
        Получается, что разложили любой вектор $v$ в сумму двух из $V_1$ и $V_2$.

        \item Поймем, что $V_1 \cap V_2 = 0$.
        Возьмем $v \in V_1 \cap V_2$: \begin{gather*}
            \begin{cases}
                v \in V_1 = \Ker f(\A) \Rightarrow f(\A)(v) = 0 \Rightarrow \mu_{\A, v} | f \\
                v \in V_2 = \Ker g(\A) \Rightarrow g(\A)(v) = 0 \Rightarrow \mu_{\A, v} | g
            \end{cases}
        \end{gather*}
        Но $(f, g) = 1$ по условию $\Longrightarrow \mu_{\A, v} = 1 \Longrightarrow v = 0$.
        
        \item Поймем, что $ \mu_{\A |_{V_1}} \, | \, f$ и $\mu_{\A |_{V_2}} \, | \, g$.
        На самом деле мы это выяснили еще в предыдущем пункте.
        Для любого $v \in V_1$ выполняется $\mu_{\A, v} | f$, следовательно $f$ -- аннулятор всех $v$ из $V_1$, т.е. $ \mu_{\A |_{V_1}} \, | \, f$. 
        Для $V_2$ и $g$ аналогично.
    \end{itemize}
\end{proof} 

\begin{theorem-non}
    Пусть $\A \in \End V$ и $\mu_{\A} = p^{n_1}_1 \cdot \dots \cdot p^{n_s}_s$, где $p_i$ -- различные унитарные неприводимые многочлены.
    Тогда
    \[ V = \bigoplus_{i = 1}^{s} W_{p_i} \]
    То есть любое пространство можно разложить в прямую сумму $p$-примарных.
\end{theorem-non}
\begin{proof}
    Для доказательства применим несколько раз предыдущую лемму:
    \[ V = \bigoplus_{i = 1}^{s} V_i, \quad V_i \text{--- $\A$---инвариантно } \]  
    \quad Также по лемме: $\mu_{\A |_{V_i}} | p^{n_i}_i$.

    \quad Минимальный аннулирующий многочлен вектора делит минимальный многочлен оператора, поэтому каждый $v \in V_i$ является $p_i$-примарным.
    То есть мы доказали, что $V_i \subset W_{p_i}$.

    \quad Теперь проверим, что $W_{p_i} \subset V_i$. 
    Для простоты будем проверять только для $i = s$, очевидно, что для других индексов все будет аналогично.
    Возьмём $w \in W_{p_s}$ и разложим в сумму векторов из $V = \bigoplus\limits_{i = 1}^{s} V_i$:
    \begin{gather*}
        w = v_1 + v_2 + \dots + v_s, \quad v_i \in V_i \\
        \underbrace{w - v_s}_{\text{$ \in W_{p_s} $}}  = v_1 + v_2 + \dots + v_{s - 1} \\
    \end{gather*}
    \quad Знаем, что $\forall j \in [1, s - 1] \;\; \mu_{\A, v_j} \mid p^{n_j}_j$.
    Т.к. $p_s$ неприводим и $\mu_{\A, w - v_s} \mid \mu_{\A \mid_{V_s}} \mid p_s^{n_s}$,
    получаем, что $\mu_{\A, w - v_s} = p_s^k$, где $k \leqslant n_s$.
    Очевидно, что произведение аннуляторов будет обнулять каждый из $v_j$, где $j=1..(s-1)$: $(\mu_{\A, v_1} \dots \mu_{\A, v_{s - 1}})(\A)(v_j) = 0$.
    Тогда это произведение будет аннулятором суммы: 
    \begin{gather*}
        (\mu_{\A, v_1} \dots \mu_{\A, v_{s - 1}})(\A)(\underbrace{v_1 + \dots + v_{s - 1}}_{\text{$ w - v_s $}} ) = 0 \\
        \Longrightarrow (\mu_{\A, v_1} \dots \mu_{\A, v_{s - 1}}) \, \vdots \, \mu_{\A, w - v_s} \Longrightarrow (p_1^{n_1} \dots p_{s-1}^{n_{s-1}}) \, \vdots \, p_s^{k} \\
        \Longrightarrow k = 0 \Longrightarrow w - v_s = 0 \Longrightarrow w \in V_s
    \end{gather*}
\end{proof}

Эта теорема очень важна, потому что она сводит изучение произвольных операторов к изучению операторов, у которых минимальный многочлен это степень какого-либо неприводимого многочлена.