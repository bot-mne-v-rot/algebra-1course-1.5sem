\section{Алгебраичность конечных расширений. Расширение, порождённое данным конечным множеством}
\begin{theorem}
    Если $L / K$ --- конечное расширение, то $L / K$ --- алгебраическое.
\end{theorem}
\begin{proof} $ $

    Пусть $[L : K] = d$. Возьмём $a \in L$. Тогда $1, a, \dots, a^d$ --- ЛЗС над $K$, т.к. их $d + 1$ штука. Это означает, что существует их нетривиальная ЛК, равная 0:
    $$ \alpha_0 + \alpha_1 a + \dots + \alpha_d a^d = 0 \text{ для некоторых $\alpha_0, \dots, \alpha_d \in K$, причём не все 0} $$
    Пусть $f(x) = \alpha_0 + \alpha_1 x + \dots + \alpha_d x^d$. Тогда $f \neq 0$ и $f(a) = 0$. А значит, $a$ алгебраично.
\end{proof}

\notice Обратное не верно.

\textbf{Пример:} $\overline{\Q} / \Q$, где $\overline{\Q} = \{ a \in \C \mid \text{$a$ алгебраично над $Q$} \}$. Очевидно, это алгебраическое расширение. Поймём, почему оно бесконечное. Пусть $[\overline{\Q} : \Q] = d < \infty$. Возьмём корни из первых $d + 1$ простых чисел. Тогда это ЛЗС. Т.е. корень из какого-то простого числа выражается через остальные корни. Это очевидно невозможно над $\Q$.

Почему $\overline{\Q}$ --- вообще поле? Оказывается, сумма или произведение алгебраических чисел --- алгебраическое число. Пока это не очевидно. Поэтому мы пока не будем это доказывать. Пока просто поверим, что $\overline{\Q}$ --- поле.

\begin{conj}
    Пусть $L / K$ --- расширение; $a_1, \dots, a_n \in L$.
    Тогда:
    $$ K(a_1, \dots, a_n) := \bigcap_{\substack{K \subset F \subset L \\ \substack{F \text{ --- подполе}} \\ a_1, \dots, a_n \in F}} F $$
    $K(a_1, \dots, a_n)$ --- самое маленькое подполе $L$, содержащее $K$ и содержащее все элементы $a_1, \dots, a_n$. Читается ``$K$ от $a_1, \dots, a_n$''. Называется оно ``\textbf{поле, полученное присоединием к полю $K$ элементов $a_1, \dots, a_n$}''.
\end{conj}
\textbf{Примеры:} 
\begin{itemize}
    \item Расширение $\Q(i) / \Q$. Интересуемся $\Q(i)$, т.е. $K = \Q$, $n = 1$, $a_1 = i$. У нас получаются одинаковые обозначения, конечно. Поймём, что мы имеем в виду одно и то же. Если $F$ содержит $1$ и $i$, тогда оно содержит все гауссовые числа. Ну тогда само поле гауссовых чисел и будет самым маленьким подполем.

    \item $\Q(\sqrt{5}) = \{ a + b \sqrt{5} \mid a, b \in \Q \}$.
    \begin{proof} $ $
        \begin{itemize}
            \item[``$\supset$'':] По определению $1, \sqrt{5} \in \Q(\sqrt{5})$. Тогда если $a, b \in \Q \subset \Q(\sqrt{5})$, то $a + b \sqrt{5} \in \Q(\sqrt{5})$.
             
            \item[``$\subset$'':] Пусть $M = \{ a + b \sqrt{5} \mid a, b \in \Q \}$. Это поле! Действительно, замкнутость относительно сложения и умножения очевидна. А чтобы делить нужно делать примерно то же самое, что при делении комплексных чисел:
            $$ \frac{a + b\sqrt{5}}{c + d\sqrt{5}} = \frac{(a + b\sqrt{5})(c - d\sqrt{5})}{c^2 - 5d} \in M $$
            Раз это поле, то по определению $K(a_1, \dots, a_n)$ получаем, что $\Q(\sqrt{5}) \subset M$.
        \end{itemize}
    \end{proof}
\end{itemize}

\notice Аналогично можно определить присоединение бесконечного количества элементов, но мы этого делать не будем.

\notice Не трудно понять из определений, что $K(a_1, \dots, a_n) = K(a_1)(a_2)\dots(a_n)$. Т.е. присоединение нескольких элементов получается последовательным присоединением каждого элемента.

Это замечание показывает важность расширения поля через присоединение одного элемента.