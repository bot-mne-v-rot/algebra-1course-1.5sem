\section{Ортогональная проекция и ортогональная составляющая. Расстояние от вектора до линейного подпространства}
\begin{conj}
    Пусть у нас есть некоторое $U$ -- подпространство $V$. 
    Если мы раскладываем вектор $v$ в сумму двух векторов, первый из которых лежит в $U$, а второй -- в $U^T$, то они имеют спецаильные 
    названия: 
    \begin{gather*}
        v = u_1 + u_2, u_1 \in U, u_2 \in U^T
    \end{gather*}
    $u_1$ -- это ортогональная проекция $u$ на $u_1$ \\
    $u_2$ -- это ортогональное дополнение $u$ по отношению к $u_1$
\end{conj}

Отметим, что длина ортогонального дополнения, это и есть в точности расстояние от точки до подпространства. 
Напомним, что в евклидовом пространстве у нас есть метрика $\rho(u, v) = \norm{u - v}$. Введем расстояние от точки до множества:
    
Пусть $M$ --- метрическое пространство, $x \in M, N \subset M$, тогда: 
\begin{gather*}
    \rho(x, N) = \inf\limits_{\mu \in N} \rho(x, \mu)
\end{gather*}

\begin{theorem-non}
    Пусть $V$ -- конечномерное евклидово пространство. $U \subset V$ -- подпространство. 
    $v \in V$, сразу разложим его в сумму ортогональной проекции и ортогонального дополнения: $v = u_1 + u_2, u_1 \in U, u_2 \in U^{\perp}$. Тогда: 
    \begin{gather*}
         \rho(v, U) = \norm{u_2}, \quad \rho(v, U) = \rho(v, u_1)
    \end{gather*}
\end{theorem-non}
\begin{proof} \quad

        Зафиксируем произвольное $z \in U$ и посчитаем расстояние от $v$ до $u_1 + z$. 
        \begin{align*}
            \rho(v, u_1 + z)^2 &= \rho(u_1 + u_2, u_1 + z)^2 \\
            &= \norm{z - u_2}^2 \\
            &= (z - u_2, z - u_2) \\
            (\text{так как } z \perp u_2) &= {\underbrace{\norm{z}}_{\geqslant 0}}^2 + \norm{u_2}^2 \\
            &\geqslant \norm{u_2}^2
        \end{align*}
        То есть $\rho(v, u_1 + z)^2$ достигает минимума при $\norm{z} = 0$ и в таком случае будет равна $\norm{u_2}^2$

        Таким образом: 
        \begin{gather*}
            \inf\limits_z \rho(v, u_1 + z) = \norm{u_2}
        \end{gather*}
        А этот инфимум и является расстоянием $\rho(v, U)$, то есть $\rho(v, u_1) = \norm{u_2}$
    \end{proof}