\section*{Классификация билинейных форм}

\subsection*{Ортогональность}

\begin{conj}(Ортогональность векторов по отношению к билинейной форме)
    $\B$~--- билинейная форма на $V$ \\
    $u, v \in V$ \\
    $u \perp v$ (ортогональны), если $\B(u, v) = 0$
\end{conj}

\notice $\B$~--- симметрическая билинейная форма
$\Longrightarrow$ ортогональность~--- симмитрическое отношение.

\begin{conj}(Ортогональный базис)
    Базис $e_1, \dots, e_n$ пространства $V$ называется ортогональным,
    если $e_i \perp e_j \, \forall i \neq j$
\end{conj}

\begin{theorem}(Теорема Лагранжа)
    Пусть $\B$~--- симмитрическая билинейная форма на $V$, $\dim V = n < +\infty$

    Тогда в $V$ существует ортогональный базис.

    \begin{proof}
    \emptyln
        Индукция по $n:$ \\
        $n = 1 \Longrightarrow$ любой базис ортогонален \\
        Переход:
        $\B = 0 \Longrightarrow$ любой базис ортогонален \\
        $\B \neq 0 \Longrightarrow \exists e_1 \in V: \B(e_1, e_1) \neq 0$ \\
        Предположим, что это не так, т.е. $\forall v \in V q(v) = 0 \, (q(v) = \B(v, v))$ \\
        Но вспомним, что $\forall v, w: \, \B(v, w) = \frac{1}{2}( q(v + w) - q(v) - q(w) ) = \frac{1}{2}(0 - 0 - 0) = 0 ?!$ \\
        Пускай $W = \{ w \in V \, | \, e_1 \perp w \}$~--- линейное подпространство (по определению билинейной формы) \\
        Теперь мы хотим посчитать $\dim{W}$, для этого рассмотрим отображение: \\
        $\lambda: V \to K$ \\
        $\quad v \mapsto \B(v, e_1)$ \\
        $\Imm{\lambda} \neq 0 \Longrightarrow \Imm{\lambda} = K$ (т.к. $K$~--- одномерное пространство, то его подпространство либо $0$, либо $K$) \\
        $\Longrightarrow \dim{W} = \dim{\Ker{\lambda}} = \dim{V} - \dim{\Imm{\lambda}} = n - 1$ \\
        Сузим $\B$ на $W \times W$, т.е. $\B' := \B |_{W \times W}$~--- симмитрическая билинейная форма на $W$. \\
        По индукционному предположению у $W \exists$ ортогональный базис $e_2, \dots, e_n$ \\
        $e_1 \not \in \Lin(e_2, \dots, e_n) = W$ (помним, что он сам себе не ортогонален, т.к. $\B(e_1, e_1) \neq 0$) \\
        $\Longrightarrow e_1, e_2, \dots, e_n$~--- ЛНС $\Longrightarrow$ это искомый ортогональный базис $V$ ($e_1$ ортогонален всем остальным, а $e_2, \dots, e_n$ уже и так ортогональный базис)
    \end{proof}
\end{theorem}

\notice Пусть $E$ ортогональный базис $V$ (по отношению к $\B$) $\Longleftrightarrow [\B]_E$ диагональная. \\
Действительно, $\B(e_i, e_j) = 0 \Longrightarrow$ все элементы \textbf{не} на диагонали равны нулю.

\notice Пусть $E = (e_1, \dots, e_n)$ ортогональный базис $V$, $E' = (\alpha_1 e_1, \dots, \alpha_n e_n), \quad \alpha_i \neq 0$ \\
$[\B]_E =
\left(\begin{array}{ccc}
\lambda_1 & \dots & 0 \\ 
\dots & \dots & \dots \\ 
0 & \dots & \lambda_n
\end{array}\right)
\Longrightarrow [\B]_{E'} = 
\left(\begin{array}{ccc}
\alpha_1^2 \lambda_1 & \dots & 0 \\ 
\dots & \dots & \dots \\ 
0 & \dots & \alpha_n^2 \lambda_n
\end{array}\right)$ (т.к. $\B(\alpha_i e_i, \alpha_i e_i) = \alpha_i^2 \B(e_i, e_i) = \alpha_i^2 \lambda_i$) 

\begin{theorem}
    Пусть $V$~--- конечномерное ЛП / $\C$ \\
    $\B$~--- симмитрическая билинейная форма на $V$ \\
    Тогда $\exists$ базис $E$ пространства $V$, т.ч.
    $[\B]_E = \left(\begin{array}{cc}
    E_r & 0 \\ 
    0 & 0
    \end{array}\right)$ для некоторого $r$, $r$~--- инвариант $\B$.
    \begin{proof}
        $\exists E_0 : [\B]_{E_0} = diag(\lambda_1, \dots, \lambda_n)$ \\
        можно считать, что $\lambda_1, \dots, \lambda_r \neq 0, \lambda_{r + 1} = \dots = \lambda_n = 0$ \\
        $\exists \alpha_j \in \C: \alpha^2_j = \lambda_j$ \\ 
        $E_0 = (e_1, \dots, e_n)$ \\
        $E = (\alpha^{-1}_1 e_1, \dots, \alpha^{-1}_r e_r, e_{r + 1}, \dots, e_n)$ \\
        $[\B]_E = diag(\alpha^{-2}_1 \lambda_1, \dots, \alpha^{-2}_r \lambda_r, \lambda_{r + 1}, \dots, \lambda_n) = diag(1, \dots, 1, 0, \dots, 0)$ \\
        $r = \rk \left(\begin{array}{cc}
        E_r & 0 \\ 
        0 & 0
        \end{array}\right) = \rk{\B}$
    \end{proof}
\end{theorem}

\begin{theorem}(Закон инерции вещественной билинейной формы)
    Пусть $V$~--- конечномерное ЛП / $\R$, $\B$~--- симмитричная билинейная форма на $V$,

    тогда $\exists$ базис $E$ такой, что 

    $[\B]_E = diag(\underbrace{1, \dots, 1}_{\text{s}},\underbrace{-1, \dots, -1}_{\text{t}}, 0, \dots, 0)$, 
    при это $s$ и $t$ --- инварианты $\B$.

    \begin{proof}
    \emptyln
    Существование $E$ доказывается аналогично предыдущему предложению (нужно взять квадратный корень из модуля, каждое число, кроме нуля, поделится на свой модуль).
    
    $s + t = \rk [\B]_E = \rk \B$~--- инвариант $\B$.

    Предположим, что $s$~--- не инвариант $\B$ \\
    Тогда $\exists E_1, E_2:$ \\
    $[\B]_{E_1} = diag(\underbrace{1, \dots, 1}_{\text{s_1}},\underbrace{-1, \dots, -1}_{\text{t_1}}, 0, \dots, 0)$ \\
    $[\B]_{E_2} = diag(\underbrace{1, \dots, 1}_{\text{s_2}},\underbrace{-1, \dots, -1}_{\text{t_2}}, 0, \dots, 0)$ \\
    $s_1 \neq s_2$ (НУО $s_1 > s_2$) \\
    $E_1 = (e_1, \dots, e_n)$ \\
    $E_2 = (f_1, \dots, f_n)$ \\

    $U_1 = \Lin (e_1, \dots, e_{s_1})$ \\
    $U_2 = \Lin (f_{s_2 + 1}, \dots, f_n)$ \\

    $\forall v \in U_1, v \neq 0: \B(v, v) > 0$ \\
    Почему это так? $v = \alpha_1 e_1 + \dots + \alpha_{s_1} e_{s_1}$ \\
    $\B(v, v) = \sum_{i = 1}^{s_1} \alpha^2_i > 0$ (т.к. $\B(e_i, e_i) = 1, \B(e_i, e_j) = 0, \, i \neq j)$ \\

    $\forall v \in U_2: \B(v, v) \leqslant 0$ \\
    $v = \alpha_{s_2+ 1} f_{s_2 + 1} + \dots + \alpha_nf_n$ \\
    $\B(v, v) = \sum_{i = s_2 + 1}^{n} \alpha^2_i \B(f_i, f_i) = -\sum_{i = s_2 + 1}^{s_2 + t_2} \alpha^2_i \leqslant 0$ \\

    Следовательно, $U_1 \cap U_2 = 0$ \\
    Так как $\dim U_1 = s_1, \dim U_2 = n - s_2 \Longrightarrow \dim U_1 + \dim U_2 = s_1 + n - s_2 > n$ (т.к. НУО $s_1 > s_2$) \\
    и $\dim U_1 \cap U_2 = \dim U_1 + \dim U_2 - \dim (U_1 + U_2) > 0 ?!$
    \end{proof}
\end{theorem}

\begin{conj}
    $(s, t)$~--- сигнатура $\B$
\end{conj}

\notice Если вы встретили формулировку "... билинейная форма ранга $r$ и сигнатуры $t$", то у нас будет $(r - t, t), \quad r - t$~--- количество единиц, а $t$~--- количество минус единиц.

\underline{Пример:}
$V = \mathbb{F}_p$
$|\mathbb{F}^*_p / (\mathbb{F}^*_p)^2| = 2$ (квадратов вдвое меньше, чем элементов группы $\mathbb{F}_p$)
Билинейных форм будет $3$:
\begin{itemize}
    \item 0
    \item 1
    \item $\varepsilon \not \in (\mathbb{F}_p)^2$
\end{itemize}

\subsection*{Евклидовы пространства}

\begin{conj}(Положительно определенная билинейная форма)
    Симмитрическая билинейная форма $\B$ на $V$~--- ЛП / $\R$ называется положительно определенной, если:
    \begin{itemize}
        \item $\forall v \in V: \B(v, v) \geqslant 0$
        \item $\B(v, v) = 0$ при $v = 0$
    \end{itemize}
\end{conj}
\notice Аналогично определяются отрицательно / неположительно / неотрицательно определенная симмитрическая билинейная форма.

\begin{theorem}(Критерий определённости симмитрической билинейной формы)
    Для сигнатуры $(s, t)$ и размерности $n$
    \begin{itemize}
        \item Положительная определённость $\Longleftrightarrow s = n$
        \item Неотрицательная определённость $\Longleftrightarrow t = 0$
        \item Отрицательная определённость $\Longleftrightarrow t = n$
        \item Неположительная определённость $\Longleftrightarrow s = 0$
    \end{itemize}
    \begin{proof}
    \emptyln
    Оставлено в качестве упражнения.
    \end{proof}
\end{theorem}

\begin{conj}(Евклидово пространство)
    Евклидово пространство это ЛП $V / \R$ с фиксированной положительно определенной симмитрической билинейной формой $(\, , \,)$~--- скалярным произведением.
\end{conj}

\notice В определении нет конечномерности (т.е. евклидово пространство может быть бесконечномерным) \\
\begin{enumerate}
    \item $V = R^n$ \\
    $(\left(\begin{array}{c}
    \alpha_1 \\ 
    \vdots \\ 
    \alpha_n
    \end{array}\right)
    , \left(\begin{array}{c}
    \beta_1 \\ 
    \vdots \\ 
    \beta_n
    \end{array}\right)
    ) ) = \alpha_1 \beta_1 + \dots + \alpha_n \beta_n$
    \item $V = C[0, 1]$ \\
    $(f, g) = \int_{0}^{1} fg $
\end{enumerate}

\begin{conj}(Длина (или норма) вектора)
    $V$~--- евклидово пространство, $v \in V$ \\
    $|v| = \sqrt{(v, v)}$~--- длина (норма) вектора.
\end{conj}

\begin{theorem}(Неравенство Коши-Буняковского)
    Пусть $v, w \in V$, тогда \\
    $|(v, w)| \leqslant |v| \cdot |w|$ 
    \begin{proof}
    \emptyln
    Возьмём $t \in \R$, рассмотрим \\
    $(v + tw, v + tw) \geqslant 0$ \\
    $(v + tw, v + tw) = (v, v) + t^2 (w, w) + t((v,w) + (w, v)) = |w|^2t^2 + 2(v, w)t + |v|^2$ ~--- квадратный многочлен относительно $t$ \\
    $D \leqslant 0$ \\
    $(v, w)^2 - |w|^2 \cdot |v|^2 \leqslant 0$ \\
    $(v, w)^2 \leqslant |w|^2 \cdot |v|^2$ \\
    $\Longrightarrow |(v, w)| \leqslant |v| \cdot |w|$ 
    \end{proof}
\end{theorem}

\follow Пусть $v, w \in V$ \\
Тогда $|v + w| \leqslant |v| + |w|$
\begin{proof}
\emptyln
$|v + w|^2 = (v + w, v + w) = (v, v) + (w, w) + 2(v, w) \leqslant |v|^2 + |w|^2 + 2 \cdot |v| \cdot |w| = (|v| + |w|)^2$ \\
$\Longrightarrow |v + w| \leqslant |v| + |w|$
\end{proof}

\begin{conj}(Метрика на евклидовом пространстве)
    $\rho(v, w) = |v - w|$ \\
    $\rho$~--- метрика на $V$ \\
    т.е. $\rho: V \times V \to \R_{\geqslant 0}$ и
    \begin{enumerate}
        \item $\rho(v, w) = 0 \Longleftrightarrow v = w$
        \item $\rho(v, w) = \rho(w, v)$
        \item $\rho(v, w) \leqslant \rho(v, u) + \rho(u, w)$ \\
        \begin{proof}
            $u - w = (u - v) + (v - w) \Longrightarrow |u - w| \leqslant |u - v| + |v - w|$ (по неравенству КБ)
        \end{proof}
    \end{enumerate}
\end{conj}

\begin{conj}(Угол между векторами)
    Пусть $v, w \neq 0 \Longrightarrow$ угол между $v, w$ это $\alpha \in [0, \pi]$, т.ч. \\
    $(v, w) = |v| \cdot |w| \cdot \cos{\alpha}$

    \begin{proof}
    $-1 \leqslant \frac{(v, w)}{|v| \cdot |w|} \leqslant 1 \Longrightarrow \exists ! \alpha \in (0, \pi): \, \cos{\alpha} = \frac{(v, w)}{|v| \cdot |w|}$
    \end{proof}
\end{conj}

\begin{conj}(Ортонормированный базис)
    Базис $E$ евклидова пространства $V$ называется ортонормированным, если \\
    $E = (e_1, \dots, e_n)$ и $\forall i, j: (e_i, e_j) = \delta_{ij}$ (Символ кронекера, т.е. $1$, если они совпадают, иначе $0$).
\end{conj}

\begin{theorem}(Построение ортонормированного базиса)
    Пусть $f_1, \dots, f_n$~--- произвольный базис $V$

    Тогда существует ортонормированный базис $V: e_1, \dots, e_n$, т.ч. \\
    $e_j \in \Lin (f_1, \dots, f_j)$ (или по-другому $\Lin (e_1, \dots, e_j) = \Lin (f_1, \dots, f_j)$)

    \begin{proof}
    \emptyln
    $e_1 = \frac{1}{|f_1|} \cdot f_1$ \\

    Пусть $e_1, \dots, e_k$ уже построены, но $k < n$ \\
    
    $e^0_{k + 1} = f_{k + 1} + \alpha_1 e_1 + \dots + \alpha_{k} e_k \in \Lin (f_1, \dots, f_k)$ \\
    Хотим, чтобы $(e^0_{k + 1}, e_j) = 0, \, j = 1, \dots, k$ \\

    $(e^0_{k + 1}, e_j) = (f_{k + 1}, e_j) + \alpha_j \Longrightarrow \alpha_j = - (f_{k + 1}, e_j)$ \\
    Таким образом $e^0_{k + 1} = f_{k + 1} - \sum_{j = 1}^{k} (f_{k + 1}, e_j)$ \\
    Тогда $e_{k + 1} = \frac{1}{|e^0_{k + 1}|} e^0_{k + 1}$ (от умножения на скаляр свойство ортонормированности не исчезает) \\

    $(e_{k + 1}, e_1) = \dots = (e_{k + 1}, e_k) = 0$ \\
    $(e_{k + 1}, e_{k + 1}) = 1$
    \end{proof}
\end{theorem}

\notice Такое построение называется ортогонализацией Грама-Шмидта

\notice Пусть $[\B]_{E'} = C^T \cdot [\B]_E \cdot C, \quad [\B]_E = E_n$, т.е. $E$~--- ортонормированный базис
$E'$ будет тоже ортонормированным, если $C^TC = E_n$