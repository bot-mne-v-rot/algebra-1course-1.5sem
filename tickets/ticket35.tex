\section{Двойственное пространство к евклидову и унитарному пространству}
Из равенства размерностей обычного и двойственного пространств очевидным образом следует их изоморфизм, но никакого выделенного изоморфизма между ними нет, то есть изоморфизм будет зависеть от выбора базиса.
В то же время, если рассматривать дважды двойственное пространство $V^{**} = (V^*)^*$, то оно будет канонически изоморфно $V$. 
Чтобы показать это, заметим, что любой вектор $v \in V$ однозначным образом определяет отображение $\alpha_v$: \begin{gather*}
    \alpha_v: V^* \to K \\
    f \mapsto f(v)
\end{gather*}
Все такие $\alpha_v \in V^{**}$, поэтому можно ввести такое отображение и доказать, что его ядро тривиально (отсюда уже будет следовать, что это изоморфизм, так как размерности равны): \begin{gather*}
    \alpha: V \to V^{**} \\
    v \mapsto \alpha_v
\end{gather*}

\vspace*{7mm}

В общем и целом, дважды двойственные пространства нам не очень интересны, поэтому вернемся к просто двойственным пространствам.
Оказывается, что если рассматривать евклидовы или унитарные пространства, то канонический изоморфизм там все-таки найдется.

Пусть $V$ -- евклидово пространство, $w \in V$. 
Тогда отображение $l_w \in V^*$: \begin{gather*}
    l_w: V \to K \\
    v \mapsto (v, w)
\end{gather*} 

\begin{theorem-non}
    Данное отображение $l$ -- изоморфизм линейных пространств: \begin{gather*}
        l: V \to  V^* \\
        w \mapsto  l_w
    \end{gather*}
\end{theorem-non}
\begin{proof}
    Размерности у этих пространств одинаковы, поэтому достаточно проверить либо инъективность, либо сюръективность.
    Будем проверять инъективность, а именно тривиальность ядра. Пусть $w \in \Ker l$. Это означает, что $\forall v \in V \; (v, w) = 0$, в частности $(w, w) = 0$, а это равносильно тому, что $w = 0$.
\end{proof}

\vspace*{7mm}

Теперь посмотрим на ту же конструкцию в унитарном пространстве.
Пусть $V$ -- унитарное пространство, $w \in V$. 
Тогда отображение $l_w \in V^*$ (это действительно так, ведь скалярное произведение в унитарном пространстве от первого аргумента зависит линейно): \begin{gather*}
    l_w: V \to K \\
    v \mapsto (v, w)
\end{gather*}
\begin{theorem-non}
    Данное отображение $l$ -- полулинейная биекция: \begin{gather*}
        l: V \to V^* \\
        w \mapsto l_w
    \end{gather*}
    Свойства полулинейности: \begin{enumerate}
        \item $l(w_1 + w_2) = l(w_1) + l(w_2)$
        \item $l(\alpha w) = \bar{\alpha}\, l(w)$
    \end{enumerate}
\end{theorem-non}
\begin{proof}
    Полулинейность очевивдна. 
    Ядро тривиально аналогично предыдущему предложению. 
    Но опустить доказательство сюръективности, сославшись на равенство размерностей, мы уже не можем, так как соотнешение размерностей ядра и образа работает только для линейных отображений.
    Поэтому докажем сюръективность с помощью следующего трюка.

    Введем векторное пространство $\tilde{V}$, которое будет совпадать с $V$ за исключением операции умножения.
    Для $\tilde{V}$ она будет определена как $\alpha \cdot v := \bar{\alpha}v$. 
    Мы можем записать следующую цепочку отображений: \[ \tilde{V} \overset{id}{\to} V \overset{l}{\to} V^* \]
    Они оба полулинейны, следовательно $l \circ id$ линейно. 
    Поэтому к нему мы можем применить предыдущие соображения: \[ \Ker(l \circ id) = \Ker l = 0 \Rightarrow \Imm(l \circ id) = V^* \Rightarrow \Imm l = V^*  \]
    Таким образом, $l$ сюръективно.
\end{proof}

\begin{notice}
    \[ e_1, e_2, \dots, e_n - \text{ ортонормированный базис } \Rightarrow le_1, le_2, \dots, le_n - \text{двойственный базис} \]
    Действительно, из-за ортонормированности отображения (они же скалярные произведения) будут действовать как надо.
\end{notice}
