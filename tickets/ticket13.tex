\section{Теорема о факторгруппе факторгруппы}
\begin{conj}
    Пусть у нас есть $H \lhd G $ и факторгруппа $G/H$. Чтобы неким образом описать 
    подгруппы этой факторгруппы введем отображение:  
    \begin{gather*}
        \pi_H: G \longrightarrow G/H
    \end{gather*}
    Все подгруппы $G/H$ имеют вид $: \pi_H(K), $ где $K < G, H \subset K$

    Заметим также, что группу $\pi_H(k)$ можно, в общем то, представлять в более явном виде. 
    $\pi_H(k) = kH,$ то есть не что иное, как правый/левый смежный класс. И тогда: 
    \begin{gather*} 
        \pi_H(K) = \{kH \mid k \in K\} = K/H
    \end{gather*}
    Также можно отметить из того, что $H \lhd G$ тривиальным образом следует, что $H \lhd K$
\end{conj}

\begin{theorem} (О факторгруппе факторгруппы)
    \begin{gather*}
        \text{Пусть есть } H, K \lhd G, \text{причем } H \subset K \\
        \text{Тогда } K/H \lhd G/H \text{ и } 
        (G/H)/(K/H) \cong G/K
    \end{gather*}
\end{theorem}
\begin{proof} \quad 

    Для начала рассмотрим гомоморфизм проекций на факторгруппу: 
    \begin{gather*}
        \pi_K : G \longrightarrow G/K \\
        \text{Мы знаем, что }\Ker \pi_K = K \Longrightarrow H \subset \Ker \pi_K \\
        \text{Так как } H \lhd G, \; \exists \text{ индуцированный гомоморфизм } 
        \varphi: G/H \longrightarrow G/K  \\
    \end{gather*}
    Действовать он будет следующим образом: $gH \longmapsto \pi_K(g) = gK$. То есть на классе некоторого 
    элемента $G$ он действует как действовал бы исходный гомоморфизм на самом элементе. 

    \notice Стоит отметить на будущее, что всегда можно гомоморфно отобразить факторгруппу по 
    меньшей подгруппе в факторгруппу по большей подгруппе. 

    Вот теперь к нашему $\varphi$ будем применять теорему о гомоморфизме. Для начала заметим, что 
    $\Imm \varphi = \{gK \mid g \in G\} = G/K$. Вышла вся подгруппа, то есть $\varphi$ сюръективен. Ну а 
    ядро будет следующим: $\Ker \varphi = \{gH \mid gK = eK\} = \{gH \mid g \in K\} = K/H$. Осталось подставить все 
    в теорему о гомоморфизме. Получаем, что $(G/H)/(K/H) \cong G/K$. 
\end{proof}