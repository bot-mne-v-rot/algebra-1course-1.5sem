\section{Комплексификация евклидова пространства}
\begin{theorem-non}
    Пусть $V$ евклидово пространство, $V_{\C}$ --- его комплексификация
    
    $(v + iw, v' + iw') = (v, v') + (w, w') + i((w, v') - (v, w'))$ -- скалярное произведение на $V_{\C}$, продолжающее 
    скалярное произведение на $V$.
    \end{theorem-non}
\begin{proof} \quad

    \begin{itemize}
        \item Линейность по первому аргументу тривиальна
        \item $(u, u') = \overline{(u', u)}$ (мнимая часть поменяет знак)
        \item $(v + iw, v + iw) = (v, v) + (w, w) + 0 \geqslant 0$. И равняется $0$ только если мы брали скалярный квадрат нулевого вектора, то есть при $v = w = 0$
    \end{itemize}
\end{proof}
\notice Пусть $E$ -- базис $V$. Тогда 
$(e_i, e_j)_V = (e_i, e_j)_{V_{\C}}$

Таким образом, если взять какой то базис, то Матрица Грама будет той же самой, когда мы перейдем в унитарное пространство. А также
если базис был ортонормированным, то и получающийся из него, комплексный базис будет ортонормированным. 