\section{Гомоморфизм, образ и ядро}

\begin{conj}
    Отображение $G \stackrel{\varphi}{\longrightarrow} G'$ $G, G'$ --- группы
  
    называется гомоморфизмом, если 
  
    $\forall g_1, g_2 \in G$ $\varphi(g_1 \cdot g_2) = \varphi(g_1) \cdot \varphi(g_2)$
  \end{conj}
  
  Примеры:
  \begin{itemize}
    \item $\mathbb{R} \longrightarrow \mathbb{C}$
    
    $\alpha \mapsto \cos{\alpha} + i\sin{\alpha}$
  
    Согласитесь, что если перемножить два числа с аргументами $\alpha$ и $\beta$, то получится число с аргументом $\alpha + \beta$.
  
    $(\cos{a} + i\sin{a})(\cos{b} + i\sin{b}) = \cos{a}\cdot\cos{b} - \sin{a}\cdot\sin{b} + i(\sin{a}\cdot\cos{b} + \sin{b}\cdot\cos{a}) = \cos(a + b) + i\sin(a + b)$
  
    \item $G$ --- абелева группа, $m\in \mathbb{Z}$
    
    $G \longrightarrow G$
  
    $g \mapsto g^m$
  
    $(g_1 \cdot g_2)^m = g_1^m \cdot g_2^m$
    
    Абелевость нужна, чтобы переставить элементы местами в 
    
    $\underbrace{(g_1 \cdot g_2) \cdot (g_1 \cdot g_2) \cdot \cdots (g_1 \cdot g_2)}_{m \text{ скобок}}$
    % в планах красиво подчеркнуть, что тут m скобок 
    % -- Подчеркнул!
  
    \item  $S_n \longrightarrow \mathbb{Z}^*$
  
    $\sigma \mapsto \sign \; \sigma$
  
    Знак произведения перестановок равен произведению знаков.
  \end{itemize}
  
  \begin{conj} \quad 

    $\Imm {\varphi} = \{ \varphi(g)\ |\ g\in G \}$ -- Ядро гомоморфизма
  
    $\Ker{\varphi} = \{ g \in G\ |\ \varphi(g) = e \}$ -- Образ гомоморфизма
  \end{conj}
  
  \begin{theorem-non}
    Образ нейтрального элемента в гомоморфизме -- 
    это нейтральный элемент:
  
    \begin{proof}
      Пусть $\varphi \colon G \to G'$.
      \begin{eqnarray*}
        e_G &=& e_G \cdot e_G \\
        \varphi(e_G) &=& \varphi(e_G \cdot e_G) \\
        \varphi(e_G) &=& \varphi(e_G) \cdot \varphi(e_G)
        \quad \mid \text{выполним сокращение} \\
        e_{G'} &=& \varphi(e_G)
      \end{eqnarray*}
    \end{proof}
  \end{theorem-non}
  
  \begin{theorem-non}
    Пусть $\varphi \colon G \to G'$, $g \in G$.
    Тогда $\varphi(g^{-1}) = \varphi(g)^{-1}$
  
    \begin{proof}
      \begin{eqnarray*}
        \varphi(g^{-1}) \cdot \varphi(g) &=& \varphi(g^{-1}g) \\
        \varphi(g^{-1}) \cdot \varphi(g) &=& \varphi(e_G) \\
        \varphi(g^{-1}) \cdot \varphi(g) &=& e_{G'} \\
        \Rightarrow \varphi(g^{-1}) &=& \varphi(g)^{-1}
      \end{eqnarray*}
    \end{proof}
  \end{theorem-non}
  
  
\begin{theorem-non}
    $\mathrm{Im}{\varphi} < G'$, $\Ker{\varphi} < G$
    \begin{proof} $ $
  
      Докажем утверждение про образ гомоморфизма.
  
      \begin{itemize}
        \item
        Наличие обратного:
  
        $\varphi(\equalto{g \cdot g^{-1}}{\varphi(g)\cdot \varphi(g^{-1})}) = \varphi(e_G) = e_{G'}$
        
        $\Longrightarrow \varphi(g)^{-1} = \varphi(g^{-1}) \in \mathrm{Im}{G}$
  
        \item
        Наличие нейтрального:
  
        $e_{G'} = \varphi(e_G) \in \mathrm{Im}{\varphi}$
  
        \item
        Замкнутость очевидна из свойств гомоморфизма. 
      \end{itemize}
  
      Таким образом, $\mathrm{Im}{\varphi} < G'$
  
      Теперь докажем утвержение про ядро.
  
      \begin{itemize}
        \item Наличие нейтрального:
        
        $\varphi(e_G) = e_{G'}$.
        Таким образом, $e_G \in \Ker \varphi$.
  
        \item
        Пусть $g_1, g_2 \in \Ker \varphi$. Проверим замкнутость:
  
        $\varphi(g_1 g_2) = \varphi(g_1) \cdot \varphi(g_2) =
        e_{G'} \cdot e_{G'} = e_{G'}$
  
        Таким образом $g_1 g_2 \in \Ker \varphi$.
  
        \item
        Проверим наличие обратного. 
        Пусть $g \in \Ker \varphi$.
        
        $\varphi(g^{-1}) = \varphi(g)^{-1} = e_{G'}^{-1} = e_{G'}$.
  
      \end{itemize}
  
    \end{proof}
  
    \notice Очевидно, композиция двух гомоморфизмов -- гомоморфизм.
\end{theorem-non}
  