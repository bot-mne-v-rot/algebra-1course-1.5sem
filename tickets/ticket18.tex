\section{Жорданова нормальная форма. Существование для нильпотентного оператора}
\begin{conj}
    Жорданова клетка порядка $m$ с собственным значением $\lambda$: 
    \[ J_m(\lambda) = \left(\begin{array}{cccc}
    \lambda &  &  & 0 \\ 
    1 & \lambda &  &  \\ 
     & \ddots & \ddots &  \\ 
    0 &  & 1 & \lambda
    \end{array}\right) \in M(m, K) \]
    (на главной диагонали стоят $\lambda$, под ними единицы, остальные нули).
\end{conj}

\vspace*{3mm}

Посмотрим на характеристический многочлен.
Матрица является нижнетреугольной, поэтому ее определитель равен произведению элементов на главной диагонали:
    \[  \chi(x) = (\lambda - x)^m \]
Сразу видно, что: $a_{\lambda} = m$.

Посчитаем геометрическую кратность. 
Обозначим за $\A := J_m \cdot$ -- оператор умножения на эту матрицу (то есть это оператор, у которого в каком-то базисе $E$ ровно такая матрица).
Тогда $ g_{\lambda} = \dim \Ker(\A - \lambda\mathcal{E})$. 
Заметим, что $[\A - \lambda\mathcal{E}]_E = J_{m}(0)$.
Таким образом, матрица оператора $\A - \lambda\mathcal{E}$ имеет вид:
\[ \left(\begin{array}{cccc}
0 &  &  & 0 \\
1 & 0 &  &  \\
 & \ddots & \ddots & 0 \\ 
0 &  & 1 & 0
\end{array}\right) \]

А значит, оператор будет действовать так: $e_1 \mapsto e_2 \mapsto \dots \mapsto e_m \mapsto 0$.
То есть только векторы, кратные $e_m$, будут переходить в 0, а это означает, что $g_\lambda = 1$.

\vspace*{4mm}

\begin{conj}
    Жорданова матрица -- это блочно-диагональная матрица, у которой блоки -- жордановы клетки.
\end{conj}
У блоков могут быть одинаковые, а могут быть и разные собственные значения.

\vspace*{5mm}

Теперь перейдем к ключевой теореме, связанной с жордановой формой.

Пусть характеристический многочлен раскладывается на линейные множители: $\chi_{\A} = \pm (x - \lambda_1)^{m_1}\dots(x - \lambda_t)^{m_t}$.
Из последнего следствия мы поняли, что все такие $(x - \lambda_i)$ будут содержаться и в разложении минимального многочлена.
Тогда по теореме, доказанной в начале, наше пространство разложится в прямую сумму слагаемых вида $W_{x - \lambda_i}$.

Упростим запись, введя следующее обозначение: $R_{\lambda} := W_{x - \lambda_i}$ -- корневое подпространство, принадлежащее $\lambda$.
По определению оно будет равно $\{ v \in V \, | \, (\A - \lambda \mathcal{E})^h(v) = 0, \text{ где } h \in \N \}$.
Заменив $W_{x - \lambda_i}$ на $R_{\lambda_i}$ получаем \[ V = \bigoplus_{i = 1}^{t} R_{\lambda_i} \]

Рассмотрим случай, когда какая-то $\lambda = 0$. 
Тогда $R_0 = \{ v \in V \, | \, \A^h(v) = 0, \text{ где } h \in \N \}$. 
Заметим, что мы можем выбрать универсальную степень $h$ для всех векторов.
Это следует из конечномерности пространства (достаточно взять базис, для каждого вектора найти степень, а потом взять максимум).
Получаем, что $\exists h : (\A \big|_{R_0})^h = 0$.
Это так называемый нильпотентный оператор.

\begin{conj}
    Нильпотентный оператор -- оператор, некоторая степень которого обращается в 0. 
\end{conj}

Вспомним, что жорданова матрица -- это блочно-диагональная матрица, у которой блоки являются жордановыми клетками.

\begin{theorem-non}
    Пусть $\A \in \End V$. Тогда эквивалентны: \begin{enumerate}
        \item Характеристический многочлен оператора $\A$ раскладывается на линейные множители.
        \item Существует базис $E$ пространства $V$, т.ч. $[\A]_E$ жорданова. 
    \end{enumerate}
    Такой базис $E$ называется жордановым, и в общем случае он не единственнен. 
    Его поиск называется приведением матрицы к жордановой форме.
\end{theorem-non}
\begin{proof} \quad 

    \begin{enumerate}
        \item Пусть $\A$ нильпотентный.
        Заметим, что тогда $t = 1$ и единственная $\lambda = 0$.
        Почему? Ну пусть это не так, и существует собственное значение $\lambda \neq 0$ и собственный вектор $v \neq 0$, принадлежащий ему.
        Тогда $\A^N(v) = \lambda^N$, но по нильпотентности это должно быть равно 0. Противоречие.
    
        Таким образом, $\chi_{\A} = \pm x^n$, где $n$ -- размерность пространства.
        По недоказанной теореме прошлого параграфа $V = \bigoplus\limits_{i=1}^s L_{v_i}$.
        Заметим, что минимальные аннуляторы этих $v_i$ будут иметь вид $x^{l_i}$, так как обязаны делить минимальный аннулятор оператора, а он обязан делить характеристический многочлен.
        Для каждого $v_i$ в стандартном циклическом базисе у нас будет своя сопровождающая матрица, но ее особенностью будет то, что все коэффиценты в последнем столбце будут равны 0, так как они по определению являются коэффицентами минимального аннулятора:
        \[
            \left(\begin{array}{ccccc}
            0 & 0 & \dots & 0 & -\alpha_0 = 0 \\ 
            1 & 0 & \dots & 0 & -\alpha_1 = 0 \\ 
            0 & 1 & \dots & 0 & -\alpha_2 = 0 \\ 
            \vdots & \vdots & \vdots & \vdots & \vdots \\ 
            0 & 0 & 0 & 1 & -\alpha_{l_i - 1} = 0
        \end{array}\right)    
        \]
        Легко видеть, что получившаяся матрица -- жорданова клетка $J_{l_i}(0)$. 
        Объединив такие клетки по всем $v_i$, получим жорданову матрицу. 
        А тогда жорданов базис -- это сконкатенированные стандартные циклические базисы.

        \item Общий случай доказан в следующем билете.
    \end{enumerate}
\end{proof}
