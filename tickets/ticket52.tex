\section{Эндоморфизм возведения в степень $p$ поля характеристики $p$}
\begin{lemma}
    Пусть $K$~--- поле характеристики $p$.
    Тогда отображение $\varphi: a \mapsto a^p$ является эндоморфизмом (и называется \underline{эндоморфизмом Фробениуса}).
\end{lemma}
\begin{proof}
    Проверяем все необходимые свойства.
    \begin{enumerate}
        \item $\varphi(0)=0, \varphi(1)=1$~--- выполнены
        \item $\varphi(ab) = (ab)^p = a^p \cdot b^p = \varphi(a)\varphi(b)$~--- проверили, что умножение сохраняется
        \item $\varphi(a+b)=\varphi(a)+\varphi(b)$.
        Это неочевидное, но тоже верное свойство: $\varphi(a+b)=(a+b)^p = \sum_{k=0}^p \binom{p}{k} a^k b^{p-k}$.
        Если вспомнить, что $\binom{p}{k} = \frac{p!}{k!(p-k)!}$ и что $p$ простое, то можно заметить, что $p \mid \binom{p}{k}$ при $k\notin \{0, p\}$.
        А поскольку характеристика поля равна $p$, то получается, что $\binom{p}{k}=0$ при всех $k$ кроме $0, p$.
        Так что в сумме остаются только два слагаемых и получается $\varphi(a+b)=(a+b)^p = a^p + b^p = \varphi(a) + \varphi(b)$.
    \end{enumerate}
\end{proof}
\follow Если $K$ конечно, то такое $\varphi$ является автоморфизмом $K$.
\begin{proof}
    Очевидно (например т.к. поле является областью целостности), что $\Ker \varphi = 0$, поэтому $\varphi$ инъективен.
    А т.к. $K$ конечно, то это дает и сюръективность $\Rightarrow$ $\varphi$ биективен, что и означает, что это автоморфизм.
\end{proof}
\notice Когда из контекста ясно, о каком поле идет речь, автоморфизм Фробениуса на нем можно обозначать $Fr$.
