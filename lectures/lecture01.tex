\section{Лекция номер 1}

\subsection{Что-то про СЛУ}

$A: V \rightarrow W$

Было доказано: $\dim{\ker{A}} + \dim{\mathrm{Im}{V}} = \dim{V}$

Как это применить к изучению СЛУ?

Представим, что есть система из $m$ уравнений, $n$ неизвестных и она однородна: 

$AX = 0$, $A\in M(m, n, k)$

Пусть $U = \{ X | AX = 0 \} \subset K^n$ - пространство решений (подпространство пронстранства всех столбцов высоты $n$).

Требуется понять, чему равна размерность пространства решений. 

Для этого нужно интерпретировать $U$ как ядро некоторого линейного отображения. Это нетрудно, потому что $U$ состоит из тех столбцов, из которых получается 0 в результате некоторой операции. Остается операцию (умножение на матрицу $A$) интерпретировать как некоторое линейное отображение. 

Умножение на матрицу $A$ сопоставляет столбцу высоты $n$ столбец высоты $m$: 

$K^n \stackrel{\varphi}{\rightarrow} K^m$

$c \mapsto  Ac$

$\varphi \in \Hom(K^n, K^m)$

Остается применить равенство с размерностями.

Ядро умножения на матрицу $A$ – $U$.

$\dim{U} + \dim{\mathrm{Im}{\varphi}} = n$

Осталось понять, что представляет из себя размерность образа.

Рассмотрим:

$A \left(
  \begin{matrix}
  c_1 \\ 
  \vdots \\
  c_n
\end{matrix}
\right) = c_1A_1 + c_2A_2 + \cdots + c_nA_n$

$A = (A_1A_2\cdots A_n)$

Следовательно, образ $\varphi$ – множество всевозможных линейных комбинаций столбцов: 

$\Rightarrow \mathrm{{Im}{\varphi}} = \Lin(A_1, \cdots, A_n)$

$\Rightarrow \dim{\mathrm{{Im}{\varphi}}} =\rk A$

\follow 
$\dim{U} = n - \rk A$

\chapter{Теория групп}

\section{Группы и подгруппы, новые примеры}
\begin{itemize}

\item Аддитивные группы колец:

\begin{itemize}
\item $\mathbb{Z}$, $\mathbb{Q}$, $\mathbb{R}$, $\mathbb{C}$, $A[X]$, $K(X)$, $M(n, A)$
\item Кольца вычетов (группы по сложению) $\mathbb{Z}/m\mathbb{Z} = \mathbb{Z}/(m) = \mathbb{Z}/\stackrel{m}{\equiv}\ = \{ [0], [1], \cdots, [m-1] \}$
\end{itemize}

Введем сложение и умножение на классах вычетов: 
\begin{gather*}
  [a] + [b] = [a + b] \\
  [a]\cdot [b] = [a \cdot b]
\end{gather*}

Это определение нуждается в проверке корректности.
\begin{gather*}
\begin{cases}
    [a] = [a']  \\
    [b] = [b']
\end{cases}
\stackrel{?}{\Rightarrow}
\begin{cases}
  [a' + b'] = [a + b]  \\
  [a' \cdot b'] = [a \cdot b]
\end{cases}
\end{gather*}

\begin{gather*}
\begin{cases}
  a \stackrel{m}{\equiv} a' \\
  b \stackrel{m}{\equiv} b'
\end{cases}
\stackrel{?}{\Rightarrow}
\begin{cases}
  a + b \equiv a' + b' \\ 
  a \cdot b \equiv a' \cdot b'
\end{cases}
\end{gather*}

Из курса дискретной математики известно, что замена представителей классов на других представителей тех же классов не влияет на результат операций.

\begin{theorem-non}
  $(\mathbb{Z}/m \mathbb{Z}, +, \cdot)$ --- коммутативное ассоциативное кольцо с единицей.

  \begin{proof}
    Рутинная проверка, тривиально.
  \end{proof}
\end{theorem-non}

\notice
Можно обобщить построение кольца классов вычетов до любого коммутативного кольца. 

$R$ --- коммутативное кольцо, $f\in R$, $\stackrel{f}{\equiv}$ --- отношение сравнимости по модулю элемента $f$.

$R/(f) = R/\stackrel{f}{\equiv}$ --- факторкольцо. 

В частности можно самостоятельно обдумать такое: 

\begin{itemize}
  \item $\mathbb{R}[X]/(X^2 + 1)$ --- легко отождествляется с полем комплексных чисел. 
  \item $R$ --- коммутативное кольцо, $I\subset R$ --- идеал в нем. Тогда $a \stackrel{I}{\equiv} b \Leftrightarrow a - b \in I$
\end{itemize} 

\item Группы обратимых элементов кольца

Пусть $A$ --- ассоциативное кольцо с единицей. 

Тогда мы имеем дело с еще одной группой. Элементы $A$ не образуют группу по умножению, но если ограничиться обратимыми элементами ($A^*$), то они образуют группу по умножению. 

Будем занудами, проверим замкнутость: 
\begin{gather*}
  ab = 1 \\
  a'b' = 1 = b'a' \\
  \Rightarrow (aa')(b'b) = 1  
\end{gather*}

Теперь имеем такие примеры: 

$\mathbb(Z)^*$, $\mathbb{R}^*$, $\mathbb{Q}^*$, $\mathbb{C}^*$, $M(n, K)^* = GL(n, K)$ --- общая линейная группа степени $n$ над полем $K$.

В $GL(n, K)$ есть важная подгруппа (специальная линейная группа), состоит из матриц, у которых определитель равен единице. 
\begin{gather*}
  SL(n, k) = \{ A \in GL(n, K) | \det{A} = 1 \} \\
  SL(n, k) < GL(n, k)
\end{gather*}

Теперь в кольце классов вычетов можно рассмотреть группу обратимых элементов: 

\begin{gather*}
  (\mathbb{Z}/m \mathbb{Z})^* = \{ [a] | (a, m) = 1 \} \\
  |(\mathbb{Z}/m \mathbb{Z})^*| = \varphi(m)
\end{gather*}

$\varphi(m)$--- функция Эйлера, сопоставляет число количество натуральных, не превышающих его и взаимно простых с ним

\item 
$S_n = S(\{ 1, \cdots, n \})$

$A_n = \{ \sigma \in S_n | \sign{\sigma} = 1 \}$ --- группа четных перестановок

$A_n < S_n$, так как произведение четных перестановок четно, обратная к четной перестановке четна и тождественная перестановка четна. 

\item Группы симметрий геометрических фигур

$\Phi$ --- фигура

$\Sigma(\Phi)$ --- самосовмещения этой фигуры

$D_n$ --- группы симметрий правильного $n$-угольника

\begin{tikzpicture}
  \draw[gray, thick] (-2,-2) -- (-2,2);
  \draw[gray, thick] (2,-2) -- (2,2);
  \draw[gray, thick] (2)

  \filldraw[black] (0,0) circle (2pt) node[anchor=west] {Intersection point};
  
\end{tikzpicture}

\end{itemize}