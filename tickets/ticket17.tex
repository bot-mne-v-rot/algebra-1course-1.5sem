\section{Разложение в прямую сумму циклических подпространств. Следствия}
\begin{theorem-non}
    Пусть $V$ это $p$-примарное пространство, тогда $\exists v_1, \dots, v_t \in V$, т.ч.
    \[ V = \bigoplus_{i = 1}^{t} L_{v_i} \]
\end{theorem-non} 

Скомбинировав эту теорему с предыдущей, получаем, что любое пространство с действующем на нем линейным оператором можно разложить в прямую сумму циклических подпространств.

Что это всё будет означать на матричном языке?
Когда мы раскладываем пространство в прямую сумму инвариантных, то матрица оператора становится блочно-диагональной.
Вспомним, что для оператора, ограниченного на циклическое подпространство, матрица имеет особый вид.
Это так называемая сопровождающая матрица: 
\[
    \left(\begin{array}{ccccc}
        0 & 0 & \dots & 0 & -\alpha_0 \\ 
        1 & 0 & \dots & 0 & -\alpha_1 \\ 
        0 & 1 & \dots & 0 & -\alpha_2 \\ 
        \vdots & \vdots & \vdots & \vdots & \vdots \\ 
        0 & 0 & 0 & 1 & -\alpha_{d - 1}
    \end{array}\right)    
\]
Обозначив ее как $C(\mu_{\A, v_i})$, получаем, что матрица оператора примет вид:
\[
  [ \A ]_{E} = \left(\begin{array}{ccc}
  C(\mu_{\A, v_1}) &  & 0 \\ 
   & \ddots &  \\ 
  0 &  & C(\mu_{\A, v_s})
  \end{array}\right)  
\]

\begin{follow}
    Пусть $\chi_{\A} = \pm p^{m_1}_1 \cdot \dots \cdot p^{m_s}_s$, где $p_i$ --- различные неприводимые.
    Тогда 
        \[ \mu_{\A} = p^{n_1}_1 \cdot \dots \cdot p^{n_s}_s, \text{ где } 1 \leqslant n_i \leqslant m_i \]
    То есть никакой множитель из разложения не пропадет.
    \begin{proof}
        Осталось на упражнение. Надо доказать.
    \end{proof}
\end{follow}