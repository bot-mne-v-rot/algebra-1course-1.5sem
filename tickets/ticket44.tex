\section{Оператор ортогонального проектирования}
Понятие оператора проектирования можно ввести в любом пространстве. Пусть у нас есть 
векторное пространство, которое раскладывается в прямую сумму подпространств: $V = W_1 \oplus W_2$. 

\begin{conj}
    Если для оператора $\A \in \End{V}$ выполняется, что:  
    $\A(w_1 + w_2) = w_1$, то $\A$ называется \textbf{оператором проектирования} на $W_1$ вдоль $W_2$. 
\end{conj}

\begin{theorem}
    Пусть $\A \in \End{V}$. 
    Тогда эквивалентны следующие условия:
    \begin{enumerate}
        \item $\A$~--- оператор проектирования (т.е. $\exists W_1, W_2$, т.ч. $V = W_1 \oplus W_2$ и 
            $\A$ -- оператор проектирования на $W_1$ вдоль $W_2$)
        \item $\A^2 = \A$ (идемпотентность)
    \end{enumerate}
    \begin{proof} \quad

        \begin{itemize}
            \item[``$1 \Rightarrow 2$'':]
            \begin{align*}
                \A^2 (w_1 + w_2) &= \A(w_1) \\
                &= \A(w_1 + 0) \\
                &= w_1 \\
                &= \A(w_1 + w_2)
            \end{align*}
            \item[``$2 \Rightarrow 1$'':] Ручками введем необходимые нам подпространства. 
                Пусть $W_1 = \Imm{\A}$, а $W_2 = \Ker{\A}$. 
                Хотим проверить, что $V = W_1 \oplus W_2$:
                \begin{gather*}
                    v = \inbelow{\A v }{W_1} + \inbelow{\underbrace{(v - \A v)}}{W_2}
                \end{gather*}
                Так как $\A(v - \A v) = \A v - \A^2 v = 0$. \\
                При этом, когда у нас что-то попадает в пересечение, то есть когда $w \in W_1 \cap W_2$, 
                то $Aw = A^2 w = Av = w$. С другой стороны при этом $w \in W_2$, то есть мы попали в ядро, а значит $Aw = 0$.
                
                Мы доказали, что получилось прямая сумма, осталось доказать, что выполянется нужное свойство. Проверим его: 
                \begin{gather*}
                    A(\inbelow{w_1}{W_1}  + \inbelow{w_2}{W_2} ) = \stackbelow{A w_1}{w_1}  + \stackbelow{A w_2}{0}
                \end{gather*}
                То что нужно. То есть $\A$ -- оператор проектирования на $W_1$ вдоль $W_2$. 
        \end{itemize}
    \end{proof}
\end{theorem}

\begin{conj}
    Пусть $V$ --- евклидово или унитарное, $W \subset V$ -- его подпространство. 
    Оператор проектирования на $W$ вдоль $W^{\perp}$ называется \textbf{оператором ортогонального проектирования} на $W$.
\end{conj}

\begin{theorem} (Характеристика оператора ортогонального проектирования)

    Пусть $\A \in \End{V}$. 
    Тогда эквивалентны следующие условия:
    \begin{enumerate}
        \item $\A$ оператор ортогонального проектирования
        \item $\A$ самоспопряжённый идемпотентный
    \end{enumerate}

    \begin{proof} \quad 
    
    \begin{itemize}
        \item[``$1 \Rightarrow 2$'':]
            То, что он идемпотентный знаем, осталось проверить самоспопряжённость. 
            Пусть $e_1, \dots, e_m$ --- ортонормированный базис $W$, а $e_{m + 1}, \dots, e_n$ ортонормированный базис $W^{\perp}$.
            Также пусть $E := (e_1, \dots, e_n)$. Тогда:
            \begin{gather*}
                [\A]_E = \left(\begin{array}{cc}
                E_m & 0 \\ 
                0 & 0
                \end{array}\right)
            \end{gather*}
            Ну а тогда:    
            \begin{gather*}
                [\A]^*_E = \left(\begin{array}{cc}
                E_m & 0 \\ 
                0 & 0
                \end{array}\right) = [\A]_E \Longrightarrow \A^* = \A
            \end{gather*}
            Что и означает самоспопряжённость.
        \item[``$2 \Rightarrow 1$'':]  
            В следующий раз
    \end{itemize}
    \end{proof}
\end{theorem}