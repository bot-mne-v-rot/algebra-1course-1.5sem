\section{Теорема Гамильтона — Кэли}
Предыдущая лемма нужна нам для того, чтобы доказать важную теорему.

\textbf{Теорема Гамильтона-Кэли.} 

\quad\quad Характеристический многочлен оператора -- его аннулятор: \[ \chi_A(\A) = 0 \]

\begin{proof}
    Пусть $v \in V$.
    В самом начале разговора о характеристических многочленах было доказано, что если $W$ инвариантное подпространство, то $\chi_{\A|_W}$ делит $\chi_\A$.
    $L_v$ -- инвариантное подпространство, поэтому $\chi_{\A|_{L_v}} | \chi_\A$. 
    Воспользовавшись леммой, заключаем, что $\mu_{\A, v} | \chi_\A$, а значит $\chi_\A(\A)(v) = 0$.
    Это выполняется для всех $v$, следовательно, $\chi_A$ -- аннулятор $\A$.
\end{proof}

\vspace*{3mm}

\follow Минимальный многочлен оператора делит его характеристический: \[ \mu_{\A} | \chi_{\A} \]
\begin{proof}
    Аннуляторы оператора являются главным идеалом, поэтому любой аннулятор делится на минимальный.
\end{proof}

\vspace*{3mm}

Разберем простой пример.

\begin{example}
    Пусть $[\A]_E = \left(\begin{array}{cc}
        \lambda_1 & 0 \\ 
        0 & \lambda_2
        \end{array}\right)$.
        Тогда: \[ \chi_{\A} = (x - \lambda_1)(x - \lambda_2) \]
        Очевидно, что $\mu_{\A, e_1} = x - \lambda_1$, $\mu_{\A, e_2} = x - \lambda_2$.
        Рассмотрим 2 случая:
        \begin{itemize}
            \item $ \lambda_1 \neq \lambda_2 \Longrightarrow \mu_{\A} = (x - \lambda_1) (x - \lambda_2) = \chi_\A$.
            \item $ \lambda_1 = \lambda_2 \Longrightarrow \mu_{\A} = x - \lambda_1 \neq \chi_\A$.
        \end{itemize} 
\end{example}
