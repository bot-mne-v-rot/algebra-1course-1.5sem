\section{Канонический вид матрицы и ортогонального оператора в евклидовом пространстве}
\textbf{Частные случаи}:
\begin{enumerate}
    \item $\A$ -- самоспопряжённый тогда и только тогда, когда матрица $\A$ в некотором ортонормированном базисе диагональна. 
    \begin{gather*}
        \A = \A^* \Longrightarrow [\A]_{E} = [\A]^T_{E} \\
        \left(\begin{array}{cc}
        \alpha & \beta \\ 
        -\beta & \alpha
        \end{array}\right) = \left(\begin{array}{cc}
        \alpha & -\beta \\ 
        \beta & \alpha
        \end{array}\right) 
        \Longrightarrow \beta = 0 \Longrightarrow [\A]_{E} \text{ диагональна}
    \end{gather*}
    В обратную сторону тривиально. 
    \item $\A$ -- ортогональный:
    \begin{align*}
        \A \A^* = \mathcal{E} &\Longleftrightarrow [\A]_{E} \cdot [\A]^T_{E} = E_n \\
                              &\Longleftrightarrow \text{ для всех блоков } B \text{ выполняется, что } BB^T = E_{\dots}
    \end{align*}
    Таким образом, если блок $B$ был блоком $1 \times 1$ вида $(\mu)$, то $\mu^2 = 1 \Longrightarrow \mu = \pm 1$. Если же блок имел размер $2 \times 2$, то:
    \begin{gather*}
        B = \left(\begin{array}{cc}
            \alpha & \beta \\ 
            -\beta & \alpha
            \end{array}\right) \Longrightarrow BB^T = \left(\begin{array}{cc}
            \alpha^2 + \beta^2 & 0 \\ 
            0 & \alpha^2 + \beta^2
        \end{array}\right) = E_2
        \Longleftrightarrow \alpha^2 + \beta^2 = 1
    \end{gather*}

    Еще подметим, что, в общем то, можно ввести такое $\varphi$, что:
    \begin{gather*}
        \begin{cases}
            \alpha = \cos{\varphi} \\
            \beta = \sin{\varphi}
        \end{cases} 
    \end{gather*}

    Тогда $\A$ ортогональный оператор $\Longleftrightarrow$ матрица $\A$ в 
    некотором ортонормированном базисе имеет блочно-диагональный вид с блоками $1 \times 1$ вида $(\pm 1)$ и
    блоками $2 \times 2$ вида:
    \begin{gather*}
        \left(\begin{array}{cc}
            \cos{\varphi} & \sin{\varphi} \\ 
            -\sin{\varphi} & \cos{\varphi}
        \end{array}\right)
    \end{gather*}

    Логика повествования давно пошла по жопе. Давайте в очередной раз заметим, что можно еще упростить вид нашей матрицы. 

    Воспользуемся тем, что блоки мы можем переставлять как душе угодно. Давайте в начало сгребем все блоки $2 \times 2$, потом единичные блоки 
    с единицей, а в самый конец единичные блоки с минус единицей. После этого 
    заметим, что два подряд идущих блока единиц можно преобразовать в блок $2 \times 2$. Аналогично с блоками из $-1$: 
    \begin{gather*}
        \left(\begin{array}{cc}
            1 & 0 \\ 
            0 & 1
        \end{array}\right) = \left(\begin{array}{cc}
            \cos{0} & \sin{0} \\ 
            -\sin{0} & \cos{0}
        \end{array}\right) \qquad \qquad
        \left(\begin{array}{cc}
            -1 & 0 \\
            0 & -1
        \end{array}\right) = \left(\begin{array}{cc}
            \cos{\pi} & \sin{\pi} \\
            -\sin{\pi} & \cos{\pi}
        \end{array}\right)
    \end{gather*}
    Теперь обозначим блоки $2 \times 2$ буковками $R$ от слова rotation и посмотрим на финальный вид нашей матрицы: 
    \begin{gather*}
        [\A]_{E} = \left(\begin{array}{cccc}
            R_{\varphi_1} &  &  & 0 \\ 
            & \ddots &  &  \\ 
            &  & R_{\varphi_m} &  \\ 
            0 &  &  & D
        \end{array}\right)
    \end{gather*}
    Где $D$ будет либо пустым, либо равен $(\pm 1)$, либо же блоку:
    \begin{gather*}
        \left(\begin{array}{cc}
            1 & 0 \\ 
            0 & -1
        \end{array}\right)
    \end{gather*}
    В частности, если $\dim V = 3$, то любой оператор первого рода, то есть такой, что его определитель равен $1$, а еще точнее просто ортогональный 
    (так как ортогональность влечет то, что у оператора опеределитель равен $\pm 1$), 
    в некотором базисе $E$ будет иметь следующую матрицу:
    \begin{gather*}
        [\A]_{E} = \left(\begin{array}{cc}
            R_{\varphi} & 0 \\ 
            0 & 1 
        \end{array}\right)
    \end{gather*}
    То есть наш оператор будет представлять из себя просто поворот на угол $\varphi$ в некоторой плоскости. 
    На самом деле получили красивый геометрический факт. Пусть мы взяли композицию двух поворотов на некоторые 
    углы относительно двух разных осей, проходящих через начало координат. И оказывается, мы сейчас поняли, 
    что композиция этих поворотов тоже будет поворотом на некий угол, по некоторой оси, проходящей через начало координат. 
\end{enumerate}

\begin{theorem}(Характеристика ортогональных операторов)

    Пусть $V$ -- евклидово пространство, $\A \in \End{V}$.
    Тогда эквивалентны следующие утверждения:
    \begin{enumerate}
        \item $\A$ ортогонален
        \item $\forall v, w \in V: (\A v, \A w) = (v, w)$
        \item $\forall v \in V: ||\A v|| = || v || $
    \end{enumerate}
    \mybox[orange!15]{Грубо говоря ортогональный оператор -- оператор, сохраняющий длины векторов и сохраняющий начало координат. В общем то 
    из этого предложения становится интуитивно понятно, почему у ортогональого оператора опеределитель $\pm 1$. 
    Ведь в терминах операторов определитель -- это то, как сильно оператор изменяет длины векторов.}
    \begin{proof} \quad

        \begin{itemize}
            \item[``$1 \Rightarrow 2$'':]
            \begin{gather*}
                \A^* \A = \mathcal{E} \\
                (\A v, \A w) = (v, \A^*\A w) = (v, w)
            \end{gather*} 
            \item[``$2 \Rightarrow 3$'':]
            \begin{gather*}
                ||Av|| = \sqrt{(Av, Av)} = \sqrt{(v, v)} = ||v||
            \end{gather*} 
            \item[``$3 \Rightarrow 2$'':]
            \begin{align*}
                2(v, w) &= ||v + w||^2 - ||v||^2 - ||w||^2 \\
                2(Av, Aw) &= ||Av + Aw||^2 - ||Av||^2 - ||Aw||^2 \\
                &\Longrightarrow (Av, Aw) = (v, w)
            \end{align*}  
            \item[``$2 \Rightarrow 1$'':]  \quad
             
            Нужно доказать, что $\forall v \in V: \, \A^* \A v = v$. Возьмем произвольный $w \in V$:
            \begin{align*}
                (w, \A^* \A v) &= (\A w, \A v) = (w, v) \\
                &\Longrightarrow (\A^* \A v - v) \perp w \\
                &\Longrightarrow (\A^* \A v - v) \in V^{\perp} = 0 \\
                &\Longrightarrow \A^* \A v = v
            \end{align*} 
        \end{itemize}
    \end{proof}    
\end{theorem}