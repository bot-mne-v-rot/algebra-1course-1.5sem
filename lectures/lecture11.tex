%Лекция 11 22.04.2021
\begin{conj}
    $A \in M(n, K)$ называется ортогональной, если $AA^T = E_n$
    
    Пусть $A = a_{ij}$

    $(AA^T)_{ij} = \sum_{k = 1}^{n} a_{ik} \cdot a_{jk}$

    Строки нашей матрицы друг другу ортогональны и нормированы.
\end{conj}

\begin{lemma}
    $A$ ортогональная $\Longrightarrow A^T$ ортогональная

    $A^T \cdot A^{TT} = \underbrace{A^T}_{\text{$A^{-1}$}}A = E^n$

    $O(n, K) = \{ A \in M(n, K) \, | \, A$ ортогональная \}
\end{lemma}

\begin{theorem}()
    $O(n, K) < GL(n, K)$
    \begin{proof}
    \emptyln

    Наличие обратных очевидно из определения $O(n, K)$

    $A, B \in O(n, K)$

    $(AB)(AB)^T = A\underbrace{BB^T}_{\text{$E_n$}}A^T = AA^T = E_n$

    $A \in O(n, K) \\
    A^{-1}(A^{-1})^T = A^T \cdot A^{TT} = E_n$
    \end{proof}
    
    Получили ортогональную группу степени $n$ над $K$.
\end{theorem}

\begin{theorem}()
    $V$ --- евклидово пространство, $E$ --- ортонормированный базис, $E'$ --- какой-либо базис $V$ и $C$ --- матрица перехода от $E$ к $E'$

    Тогда $E'$ ортонормированный $\Longleftrightarrow C \in O(n, K)$

    \begin{proof}
    \emptyln
    $\gamma_E$ --- МГ скалярного произведения в базисе $E$

    $\gamma_{E'} = C^T \gamma_E C, \quad \gamma_E = E_n$

    $\gamma_{E'} = E_n \Longleftrightarrow C^TC = E_n \Longleftrightarrow C \in O(n, K)$
    \end{proof}
\end{theorem}
% Заменить все T далее на значок ортонормирования
\begin{conj}
$U \subset V$ --- лин. подпространство евклидова пространства 

Ортогональное дополнение к $U$ --- это $U^{\perp} = \{ v \in V \, | \, \forall u \in U: \, u \perp v \}$
\end{conj}

\begin{theorem}(Свойства ортогонального дополнения)
    $\dim{V} < \infty$
    \begin{enumerate}
        \item $U^{\perp}$ лин. подпространство
        \item $V = U \otimes U^{\perp}$
        \item $U_1 \subset U_2 \Longrightarrow U_1^{\perp} \supset U_2^{\perp}$
        \item $(U^{\perp})^{\perp} = U$
        \item $(U_1 + U_2)^{\perp} = U_1^{\perp} \cap U_2^{\perp}$
        \item $(U_1 \cap U_2)^{\perp} = U_1^{\perp} + U_2^{\perp}$
        \item $V^{\perp} = 0, 0^{\perp} = V$
    \end{enumerate}
    \emptyln
    \begin{proof}
        \begin{enumerate}
            \item Тривиально из определения \\
            $(v_1 + v_2, u) = (v_1, u) + (v_2, u)$ \\
            $(\alpha v, u) = \alpha(v, u)$
            \item Пусть $f_1, \dots, f_m$ --- какой-либо базис $U$ \\
            $f_{m + 1}, \dots, f_n$ --- его дополнение до базиса $V$

            Согласно соответсвующему предложению, $\exists e_1, \dots, e_n$ --- ортонормированный базис $V$, т.ч.

            $\forall l: \Lin(e_1, \dots, e_l) = \Lin(f_1, \dots, f_l)$

            В частности $\Lin(e_1, \dots, e_m) = U$

            $U^{\perp} = ?$

            $v = \alpha_1 e_1 + \dots \alpha_n e_n \in U^{\perp} \Longleftrightarrow v T e_i, \, i = 1, \dots, m$

            $(v, e_i) = \alpha_i \Longleftrightarrow \alpha_i = 0, i = 1, \dots, m$

            $\Longleftrightarrow v \in \Lin(e_{m + 1}, \dots, e_n)$

            $U^{\perp} = Lin(e_{m + 1}, \dots, e_n) \Longrightarrow U \otimes U^{\perp} = V$
            
            \item Очевидно из определения
            \item $U \subset (U^{\perp})^{\perp}$ --- очевидно \\
            $U \supset (U^{\perp})^{\perp}:$ \\
            По свойству $2: \dim U^{\perp} = \dim V - \dim U$ \\
            $\dim (U^{\perp})^{\perp} = \dim V - \dim U^{\perp} = \dim U$ \\
            $\Longrightarrow U = (U^{\perp})^{\perp}$
            \item $v \in (U_1 + U_2)^{\perp} \Longleftrightarrow v \in U_1^{\perp}$ и $v \in U_2^{\perp}$ (в одну сторону по определнию, в другую по линейности)
            \item $(U_1 \cap U_2)^{\perp} = ((U_1^{\perp})^{\perp} \cap (U_2^{\perp})^{\perp})^{\perp} = ((U_1^{\perp} + U_2^{\perp})^{\perp})^{\perp} = U_1^{\perp} + U_2^{\perp}$
            \item $\dim V^{\perp} = \dim V - \dim V = 0$ \\
            $0^{\perp} = (V^{\perp})^{\perp} = V$
        \end{enumerate}
    \end{proof}
\end{theorem}

\begin{conj}
    $U \subset V$ \\
    $v = u_1 + u_2, \quad u_1 \in U, u_2 \in U^{\perp}$ \\
    $u_1$ --- ортогональная проекция $U$ на $u_1$ \\
    $u_2$ --- ортогональное дополнение $U$ по отношению к $u_1$ \\
\end{conj}

\begin{theorem}()
    $\rho(u, v) = \norma{u - v}$

    $M$ --- метрическое пространство, $x \in M, N \subset M$

    $\rho(x, N) = inf_{\xi \in N} \rho(x, \xi)$ 
\end{theorem}

\begin{theorem}()
    Пусть $V$ --- конечномерное евклидово пространство \\
    $U \subset V$ --- подпространство \\
    $v \in V, v = u_1 + u_2, \quad u_1 \in U, u_2 \in U^{\perp}$

    Тогда $\rho(v, U) = \norma{u_2},$ \\
    $\rho(v, U) = \rho(v, u_1)$

    \begin{proof}
    \emptyln
        $z \in U$

        $\rho(v, u_1 + z)^2 = \rho(u_1 + u_2, u_1 + z)^2 = \norma{z - u_2}^2 = (z - u_2, z - u_2) = ||z||^2 + ||u_2||^2 \geqslant ||u_2||^2$

        Вспомним, что $z T u_2$

        $\rho(v, u_2)^2 = ||u_2||^2$

        Т.о. $inf_z \rho(v, u_1 + z) = ||u_2|| (\rho(v, u_1 + z) = \rho(v, U))$

        Инфимум достигается при $z = 0$, т.е. на векторе $u_1: \rho(v, u_1) = ||u_2||$
    \end{proof}
\end{theorem}

\subsection*{Унтарные пространства}

\begin{conj}(Унитарные пространства)
    $V$ --- ЛП / $\C$
\end{conj}
\begin{conj}
    Полутролинейная форма на $V$ это 
    $\B: V \times V \to \C$, т.ч.
    \begin{enumerate}
        \item $\B$ линейно по 1 аргументу
        \item $\B(v, \alpha_1 w_1 + \alpha_2 w_2) = \bar{\alpha_1} \B(v, w_1) + \bar{\alpha_2}\B(v, w_2)$
    \end{enumerate}
    \underline{Примеры:}
    \begin{enumerate}
        \item $V = \C^n$ \\
        $\B(\left(\begin{array}{c}
        \alpha_1 \\ 
        \vdots \\ 
        \alpha_n
        \end{array}\right)), 
        \left(\begin{array}{c}
        \beta_1 \\ 
        \vdots \\ 
        \beta_n
        \end{array}\right)$
        $) = \alpha_1 \bar{\beta_1} + \dots + \alpha_n \bar{\beta_n}$
        \item $C_{\C} [0, 1]$ \\
        $\B(f, g) = \int_{0}^{1} f \cdot \bar{g}$
    \end{enumerate}    
\end{conj}

\begin{conj}
    $E$ --- базис $V$

    $[\B]_E = \B(e_i, e_j)$

    $v = E \cdot b$
    $w = E \cdot c$

    $\B(v, w) = b^T \cdot [\B]_E \cdot \bar{c}$
\end{conj}

Fan fact again (as with bilin. forms):
    $E' = EC$
    $[\B]_{E'} = C^T \cdot [\B]_{E} \cdot \bar{C}$

\begin{conj}
    Эрмитовой формой на $V$ называется полутролинейная форма $\B$, т.ч.
    
    $\forall u, v: \B(v, u) = \bar{\B(u, v)} \Longleftrightarrow \B(e_i, e_j) = \bar{\B(e_j, e_i)} \forall i, j$
 
\end{conj}

Для таких форм справедлива теорема Лагранжа.
Пусть $\dim V < \infty$, $\B$ ---  эрмитова форма на $V$

Тогда $\exists$ базис $E$, т.ч. $[\B]_E$ диагональна.

\notice $\B$ эрмитова $\Longleftrightarrow [\B]^*_E  = [\B]_E$
    
$A^* := \bar{A^T}$ --- матрица, сопряженная к $A$

В частности, если $[\B]_E$ диагональна, то она вещественна.

Для эрмитовой формы также справедлив закон инерции:

Число положительных и число отрицательных чисел в диагональной матрице $[\B]_E$ --- инварианты $\B$.

\begin{conj}
    Скалярным произведением на $V$ называется положительно определённая эрмитова форма на V, т.е. такая, что
    \begin{enumerate}
        \item $\forall v 'in V: \B(v, v) \geqslant 0$
        \item $\B(v, v) = 0 \Longleftrightarrow v = 0$
    \end{enumerate}
\end{conj}

\begin{conj}
    Унитарным пространством называется линейное пространство над $\C$ с фиксированным скалярным произведением.
\end{conj}

\begin{theorem}
    Базис $e_1, \dots, e_n$ унитарного пространства V называется ортонормированным,
    если $\forall i, j: \, (e_i, e_j)$ = символ кронекера{ij} ($\Longleftrightarrow \gamma_E = E_n$)
\end{theorem}

again fan fact:
$E' = EC, \quad C \in GL(n, \C)$

$\gamma_{E'} = C^T \cdot \gamma_E \bar{C}$

$E'$ ортонормированный $\Longleftrightarrow \gamma_{E'} = E_n Longleftrightarrow C^T \cdot \bar{C} = E_n \Longleftrightarrow C^* \cdot C = E_n$

\begin{conj}(Унитарная матрица)
    $C \in GL(n, \C)$ называется унитарной, если $C^* \cdot C = E_n$
\end{conj}

Обозначение: $U(n, \C) := U(n) := \{ C \in GL(n, \C) \, | \, C$ --- унитарна \}

\begin{theorem}(Унитарная группа степени $n$)
    $U(n) < GL(n, \C)$ --- унитарная группа степени $n$
\end{theorem}

\begin{theorem}(Неравенство КБ)
    $\forall u,v \in V:$
    $$
        |(u, v)|^2 \leqslant ||u|| \cdot ||v|| (||v|| = \sqrt{(v, v)})
    $$
    \begin{proof}
    \emptyln
    Осталось в качестве упражнения.
    \end{proof}
\end{theorem}

\subsection*{Комплесификация}

Пусть $V$ --- ЛН / $\R$ \\
Оно определяет $V_{\C}$ --- ЛП / $\C$ \\
$V_{\C} = V \times V$ \\

\begin{itemize}
    \item Сложение: $(v, w) + (v', w') = (v + v', w + w')$
    \item Умножение на комплексное число: $(\alpha + i \beta)(v, w) = (\alpha v - \beta w, \alpha w + \beta v)$
    Несложно проверить, что $V_{\C}$ с этими операциями --- ЛП / $\C$ (осталось на упражнение)
\end{itemize}

\begin{theorem}(Базис $V_{\C}$)
    Пусть $e_1, \dots, e_n$ --- базис $V$ \\
    Тогда $(e_1, 0), \dots, (e_n, 0)$ --- базис $V_{\C}$

    \begin{proof}
    \emptyln
    $(\alpha + i \beta) \cdot (v, 0) = (\alpha v, \beta v)$
    $\sum_{j=0}^{n} (\alpha_j + i \beta_j)(e_j, 0) = (\sum_{}^{} \alpha_j e_j, \sum_{}^{} \beta_j e_j)$

    Т.о. $\Lin(e_j, 0), j = 1, \dots, n) = V_{\C}$

    Если $\sum_{j=0}^{n} (\alpha_j + i \beta_j)(e_j, 0) = (\sum_{}^{} \alpha_j e_j, \sum_{}^{} \beta_j e_j) = 0$ \\
    $\Longrightarrow \alpha_j = \beta_j = 0, \, j = 1, \dots, n$
    \end{proof}
\end{theorem}

\follow $\dim V_{\C} = \dim V$

$V_{\C}$ называется комплексификацией $V$

$(v, w) = (v, 0) + (0, w) = (v, 0) + i(w, 0)$

Можно отождествить $(v, 0) с v, тогда (v, w) = v + iw$

Пусть $V$ евклидово пространство, $V_{\C}$ --- его комплесификация

Положим $(v + iw, v' + iw') = (v, v') + (w, w') + i((w, v') - (v, w'))$

\begin{theorem}(Получившееся определение является скалярным произведением на $V_{\C}$)
    \begin{itemize}
        \item Линейность по первому аргументу тривиальна
        \item $(u, u') = \bar{(u', u)}$ (мнимая часть поменяет знак)
        \item $(v + iw, v + iw) = (v, v) + (w, w) + 0 \geqslant 0$ \\
        И равняется $0$ при $v = w = 0$
    \end{itemize}
\end{theorem}
\notice $E$ --- базис $V$
$(e_i, e_j)_V = (e_i, e_j)_{V_{\C}}$
\begin{proof}
\emptyln
Очевидно, так как скалярное проивзедение будет таким же как и в V.
\end{proof}
Матрица Грама будет той же самой
Если базис был ортонормированным, то и базис будет ортонормированным


\paragraph{Двойственное пространство}
\begin{conj}
    $V$ --- ЛП / $K$ \\
    Двойственным к $V$ пространством называют $V^* = \Hom(V, K)$ \\
    Элементы $V^*$ называют линейными функционалами на $V$
\end{conj}

Пример
$V = C[0, 1]$

$f \to f(0)$ --- элемент $V^*$

\begin{theorem}()
    Пусть $\dim V = n < \infty$

    Тогда $\dim V^* = n$

    \begin{proof}
    \emptyln
    Пусть $V, W$ --- конечномерные

    $\Hom(V, W) \cong M(m, n, K), \quad m = \dim V, n = \dim W$

    $\Longrightarrow \dim \Hom(V, W) = mn$
    \end{proof}
\end{theorem}


\begin{conj}
    Пусть $e_1, \dots, e_n$ --- базис $V$
    $e^i : V \to K$
        $e_j \mapsto$ символ кронекера{ij}
\end{conj}

\begin{theorem}(Базис $V^*$)
    $(e^1, \dots, e^n)$ --- базис $V^*$

    \begin{proof}
    \emptyln
    $\alpha_1 e^1 + \dots + \alpha_n e^n = 0$ \\
    $\Longrightarrow (\alpha_1 e^1 + \dots + \alpha_n e^n) (e_j) = 0 = \alpha_j e_j$
    $\Longrightarrow \alpha_1 = \dots = \alpha_n = 0$
    \end{proof}
\end{theorem}

\begin{conj}
    $(e^1, \dots, e^n)$ --- двойственный базис к $e_1, \dots, e_n$
\end{conj}

упр.

пусть$ E, E_1$--- базисы $V$

$E^*, E^*_1$ соотв. двойственные базисы

$C$ --- матрица перехода от $E$ к $E_1$, тогда матрица перехода от $E_1$ к $E$ равна $C^T$

