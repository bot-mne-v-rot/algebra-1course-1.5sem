\section{Симметрические билинейные формы и квадратичные формы}
\begin{conj}
    Если $\forall v, w \in V: \, \B(v, w) = \B(w, v)$, то 
    $\B$ -- симметрическая билинейная форма.
\end{conj}

\begin{conj}
    Если $\forall v, w \in V: \, \B(v, w) = -\B(w, v)$, то 
    $\B$ -- кососимметрическая билинейная форма.
\end{conj}

\begin{conj}
    Пусть $V$ -- линейное пространство над полем $K$ характеристики отличной от 2.
    Тогда отображение $q: V \to K$ называется квадратичной формой,
    если существует симмитрическая билинейная форма $\B$ на $V$, т.ч.
    \[ \forall v \in V: q(v) = \B(v, v) \]
\end{conj}

Можно заметить, что квадратичную форму задает однородный (все мономы имеют одинаковую степень) многочлен второй степени от координат вектора.
Действительно, пусть $E = (e_1, \dots, e_n)$ -- базис.
Тогда $q(x_1e_1 + \dots + x_ne_n) = X^TBX$.

Приведем пример для $n = 2$: \begin{gather*}
    B = \left(\begin{array}{cc}
        b_{11} & b_{12} \\ 
        b_{12} & b_{22}
        \end{array}\right) \Rightarrow X^TBX = b_{11}x^2_1 + 2b_{12}x_1x_2 + b_{22}x^2_2
\end{gather*}

Билинейную форму можно восстановить по квадратичной (помним, что $char K \neq 2$):
$$
    \B(v, w) = \frac{1}{2}(
        \underbrace{q(v + w)}_{\text{$\B(v + w, v + w)$}} - q(v) - q(w)
        )
$$
Говорят, что $\B$ является поляризацией $q$.

Таким образом, мы получили, что симметрическая билинейная и квадратичная формы тесно связаны и фактически являются эквивалентными понятиями.