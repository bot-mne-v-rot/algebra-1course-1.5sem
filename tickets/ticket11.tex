\section{Применение теоремы о гомоморфизме к классификации циклических групп}
\underline{Примеры нахождения изоморфизма:}
\begin{enumerate}
    \item $G := \C^* \; H := \mathbb{T} = \{ z \, | \, |z| = 1 \}$
    
    $\C^* / \mathbb{T} = ?$ 

    $\varphi: \C^* \to \R^{*}$

    $\quad \quad z \mapsto |z|$

    $\Imm \varphi = \R^{*}_{+}$

    $\Ker \varphi = \mathbb{T}$

    По Т. о гомоморфизме, индуцированный гомоморфизм $\C^{*} / \mathbb{T} \to \R^{*}_{+}$~--- изоморфизм.

    Т.о. $\C^* / \mathbb{T} \cong \R^{*}_+$

    \notice $\R^{*}_+ \cong \R (a \mapsto \ln{a})$

    \item $G / H := \C^* / \cycle{i}$
    
    $\ord i = 4, \, \cycle{i} = \{ i, -1, -i, 1 \}$
    
    $\varphi: \C^* \to \C^*$
    
    $\quad \quad z \mapsto z^4$

    $\Ker \varphi = \cycle{i}$

    $\Imm \varphi = \C^*$

    $\Longrightarrow \C^* / \cycle{i} \cong \C^*$

    \item $G / H := S_n / A_n \quad (A_n \vartriangleleft S_n)$
    
    $\varphi: S_n \to \Z^* = \{ 1, -1 \}$

    $\quad \quad \sigma \mapsto \sign(\sigma)$

    $\Ker \varphi = A_n$

    $\Imm \varphi = \Z^*$

    $\Longrightarrow S_n / A_n \cong \Z^{*}$
\end{enumerate}

\subsection*{Классификация циклических групп}

\begin{theorem-non}
    $G$~--- циклическая группа 

    \begin{enumerate}
        \item $G$~--- бесконечная $\Longrightarrow G \cong \Z$
        \item $|G| = n < +\infty \Longrightarrow G \cong \Z / n \Z$
    \end{enumerate}

    \begin{proof}
        
        $G = \cycle{g}$

        $\varphi: \Z \to G$ (очевидно, гомоморфизм)

        $\quad \quad a \mapsto g^a$

        \begin{enumerate}
            \item
            $G$~--- бесконечная $\Longrightarrow g^m \neq e \, \forall m \in \N$
            
            $\Longrightarrow g^{-m} \neq e \, \forall m \in \N$

            Т.о. $\Ker \varphi = \{ e \} \Longrightarrow \varphi$~--- инъективен

            $\Imm \varphi = \{ g^a \, | \, a \in \Z \} = \cycle{g} = G$

            $\Longrightarrow \Z / \Ker \varphi \cong \Imm \varphi \Longrightarrow \Z \cong G$

            \item
            $|G| = n < +\infty \Longrightarrow \ord{g} = n$
            
            Посмотрим на ядро.

            Знаем, что:
            \begin{itemize}
                \item $n \in \Ker \varphi$
                \item $0 < m < n \Longrightarrow m \not \in \Ker \varphi$
                \item $t \in \Z \; t = nq + r \, 0 \leqslant r \leqslant n - 1$
                $r = 0 \Longrightarrow t \in \Ker \varphi$
                $r \neq 0 \Longrightarrow r \not \in \Ker \varphi \Longrightarrow t \not \in \Ker \varphi$
            \end{itemize}

            Т.о. $\Ker \varphi = \{ nq \, | \, q \in \Z \} = n \Z$

            $\Longrightarrow \Z / \Ker \varphi \cong \Imm \varphi \Longrightarrow \Z / n \Z \cong G$
             
        \end{enumerate}
    \end{proof}

    Второе доказательство:
    \begin{proof}
        
        $G \cong \Z / \Ker \varphi$

        Поищем, каким может быть ядро.

        Знаем, что в $\Z$ идеалы выглядят так: $(n) \, n \in \N_{0}$

        Любая подгруппа $H$ в $\Z$~--- идеал в $\Z:$

        $la = \underbrace{a + \dots + a}_{\text{l}} \in H \quad l \in \N, a \in H$

        $(-l)a = -(la) \in H$

        $0 \cdot a = 0 \in H$

        Любая подгруппа~--- идеал, а все идеалы главные.

        $\Longrightarrow G \cong \Z / (n):$
        \begin{itemize}
            \item $n < +\infty \Longrightarrow G \cong \Z / n \Z$
            \item Иначе $G \cong \Z$
        \end{itemize}
    \end{proof}
\end{theorem-non}