\section{Жорданова нормальная форма. Существование в общем случае}
Про жорданову форму написано в предыдущем билете.

\begin{theorem-non}
    Пусть $\A \in \End V$. Тогда эквивалентны: \begin{enumerate}
        \item Характеристический многочлен оператора $\A$ раскладывается на линейные множители.
        \item Существует базис $E$ пространства $V$, т.ч. $[\A]_E$ жорданова. 
    \end{enumerate}
    Такой базис $E$ называется жордановым, и в общем случае он не единственнен. 
    Его поиск называется приведением матрицы к жордановой форме.
\end{theorem-non}
\begin{proof} \quad 

    \begin{enumerate}
        \item Доказано в предыдущем билете.
        \item Просто $t = 1$, т.е. $\chi_{\A} = \pm (x - \lambda)^n$. 
        По теореме Гамильтона-Кэли характеристический многочлен оператора это его аннулятор, поэтому $(\A - \lambda \mathcal{E})^n = 0$. 
        Получаем, что оператор $\B := \A - \lambda \mathcal{E}$ нильпотентный, а значит по первому пункту существует базис $E$, т.ч. в нем матрица $\B$ жорданова. 
        Матрица оператора $\A$ в том же базисе получается прибавление скаляра на главной диагонали, что тоже дает жорданову матрицу.
        
        \item Общий случай. Разложим в прямую сумму: $V = \bigoplus\limits_{i = 1}^{r} R_{\lambda_i}$.
        Заметим, что у $\A \big|_{R_{\lambda_i}}$ единственным собственным значением будет $\lambda_i$.
        Это верно, потому что собственные векторы $\lambda_i$ очевидно лежат в $R_{\lambda_i}$, а $R_{\lambda_i} \cap R_{\lambda_j} = \varnothing$ по определению прямой суммы, значит других собственных векторов там нет.
        Согласно второму пункту у $\A \big|_{R_{\lambda_i}}$ есть жорданов базис.
        Так как сумма у нас прямая, мы можем просто объединить все эти базисы и получить уже искомый.
    \end{enumerate}
\end{proof}

\notice Как уже упоминалось, жорданов базис не единственнен, но вот жорданова матрица определена однозначно с точностью до перестановки жордановых клеток. 
