\section{Строение конечных и конечно порождённых абелевых групп}
\begin{conj} $ $\\
    Пусть $A, B$ -- группы.\\
    Тогда $A \oplus B := A \times B$ -- \textbf{прямая сумма} 
    групп $A$ и $B$ в аддитивной записи.
\end{conj}

\begin{lemma}
    Пусть $m, n \in \N$, $\gcd(m, n) = 1$.
    Тогда $\Z/(mn)\Z \cong (\Z/m\Z) \oplus (\Z/n\Z)$.
\end{lemma}
\begin{proof}
    Два доказательства:
    \begin{itemize}
        \item[I.] Рассмотрим отображение:
        \begin{flalign*}
            \varphi \colon \Z &\to (\Z/m\Z) \oplus (\Z/n\Z) &&\\
            a &\mapsto ([a]_m, [a]_n) &&
        \end{flalign*} 
        Очевидно, $\varphi$ -- гомоморфизм.
        $$\Ker \varphi = \{ a \mid a \equiv 0 \mod{m},
        a \equiv 0 \mod{n} \} = \{ a \mid a \equiv 0 
        \mod{\operatorname{lcm}(m, n)}\} = mn \Z$$

        По теореме о гомоморфизме:
        \begin{align*}
            &\Z/\Ker \varphi \cong \Imm \varphi \Leftrightarrow
            \Z/mn\Z \cong \Imm \varphi \Rightarrow 
            \abs{\Imm \varphi} = mn; \\
            &(\Z/m\Z) \oplus (\Z/n\Z) \subset \Imm \varphi,
            \abs{(\Z/m\Z) \oplus (\Z/n\Z)} = mn \Rightarrow \\
            &\Rightarrow \Imm \varphi = (\Z/m\Z) \oplus (\Z/n\Z)
            \cong \Z / mn \Z
        \end{align*}

        \item[II.]
        \begin{align*}
            & \text{ord}([1]_m, [1]_n) = \min\{ l \in \N \mid 
            ([l]_m, [l]_n) = ([0]_m, [0]_n)\} = mn \Rightarrow \\
            & \Rightarrow (\Z/m\Z) \oplus (\Z/n\Z) = 
            \cycle{([1]_m, [1]_n)} \Rightarrow \\
            & \Rightarrow \text{циклическая группа порядка } mn 
            \cong \Z/mn\Z
        \end{align*} 
    \end{itemize}
\end{proof}

\begin{conj}
    \textbf{Конечнопорождённая группа} --- группа с 
    конечным числом образующих.
\end{conj}
\begin{conj} $ $\\
    Пусть $G$ -- абелева группа.\\
    $G_{tor} = \{ g \in G \mid \ord g < \infty \}$
    --- \textbf{подгруппа кручений}.\\
    \textit{tortion -- кручение}
\end{conj}
\notice Абелевость $G$ здесь важна. В неабелевой группе
определённое таким образом множество не обязательно является
подгруппой $G$.

Легко видеть: $(G/G_{tor})_{tor} = \{ 0 \}$. 
(\textit{вот мне не легко, но кому легко, тот молодец})

\underline{\textbf{Факты:}}
\begin{enumerate}
    \item $G_{tor} = 0 \Rightarrow G \cong \Z^r$, 
    где $r$ -- инвариант.
    \item В общем случае, $G \cong (G / G_{tor}) \oplus G_{tor}$
    \item $G_{tor}$ -- конечная абелева группа
\end{enumerate}
\notice В частности, конечно порождённая абелева группа изоморфна
конечной прямой сумме циклических.

\begin{conj}
    $\Z/p^t\Z$ -- это $p$-примарная циклическая группа,
    где $p \in \mathbb{P}, t \in \N$.
\end{conj}

\begin{theorem-nonna}
    Пусть $G$ -- конечная абелева группа. Тогда $G$ изоморфна
    прямой сумме примарных циклических групп. Порядки этих
    групп определены однозначно с точностью до перестановки.
\end{theorem-nonna}
\begin{proof}
    \textit{Доказательства не будет -- автор принял линал.}
\end{proof}
\begin{example}
    Опишем все абелевы группы порядка 200:
    \begin{eqnarray*}
        200 &=& 5^2 \cdot 2^3 = \\
            &=& 5 \cdot 5 \cdot 2^3 = \\
            &=& 5^2 \cdot 2^2 \cdot 2 = \\
            &=& 5 \cdot 5 \cdot 2^2 \cdot 2 = \\
            &=& 5^2 \cdot 2 \cdot 2 \cdot 2 = \\
            &=& 5 \cdot 5 \cdot 2 \cdot 2 \cdot 2
    \end{eqnarray*}
    Этим разложениям соответствуют группы:
    \begin{gather*}
        (\Z/5^2\Z) \oplus (\Z/2^3\Z) \\
        (\Z/5\Z) \oplus (\Z/5\Z) \oplus (\Z/2^3\Z) \\
        (\Z/5^2\Z) \oplus (\Z/2^2\Z) \oplus (\Z/2\Z) \\
        (\Z/5\Z) \oplus (\Z/5\Z) \oplus (\Z/2^2\Z) \oplus (\Z/2\Z) \\
        \vdots \\
        (\Z/5\Z)^2 \oplus (\Z/2\Z)^3 \cong (\Z/10\Z)^2 \oplus (\Z/2\Z)
    \end{gather*}
\end{example}
