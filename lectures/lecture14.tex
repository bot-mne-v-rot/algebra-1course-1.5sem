\section{Лекция номер 14}
\subsection*{Расширение полей}

\begin{conj}
    $K$ --- \textbf{подполе} поля $L$, если это подкольцо, являющееся полем. Другими словами:
    \begin{itemize}
        \item $K$ --- подгруппа $L$ по сложению, т.е $(K, +)$ --- подгруппа $(L, +)$;
        \item $1 \in K$;
        \item $K$ замкнуто относительно умножения, т.е. $K \cdot K \subset K$;
        \item $K$ замкнуто относительно взятия обратного, т.е. $K^{-1} \subset K$.
    \end{itemize}
\end{conj}

\begin{conj}
    Если задано поле $L$ и его подполе $K$, то говорят, что задано
    \textbf{расширение полей}  $L/K$ (читается ``$L$ над $K$'').
\end{conj}

\textbf{Примеры:} 
\begin{itemize}
    \item $\C / \R$, $\R / \Q$, $K / K$. 
    \item $\Q (i) / \Q$, $\Q (i) = \{\alpha + i \beta \mid \alpha, \beta \in \Q \}$ --- поле гауссовых чисел.
\end{itemize}

\begin{conj}
    Поле $K$ называется \textbf{простым}, если оно не содержит
    подполей, не равных $K$.
\end{conj}

\textbf{Историческая справка}: $\{0\}$ не подполе, т.к. не содержит
единицы отличной от нуля.

\textbf{Примеры:}
\begin{enumerate}
    \item $Q$ --- простое. Пусть $K \subset Q$ --- подполе $Q$.
    Тогда $0, 1 \in K$. Тогда $\N \subset K$, т.к. любое натуральное
    число представимо в виде суммы единиц. Тогда $\Z \subset K$.
    А тогда и $\Q \subset K$.

    \item $\Z / p\Z$ --- кольцо классов вычетов по модулю $p$, где
    $p$ --- простое. Это простое поле, т.к. $\Z / p\Z = \{ 
        \underbrace{1 + \dots + 1}_{\text{$t$ штук}} \mid
        0 \leqslant t < p\}$, а любое подполе содержит единицу.
\end{enumerate}

\begin{conj}
    Пусть $L / K$ --- расширение, $a \in L$. 
    \begin{itemize}
        \item $a$ называется \textbf{алгебраическим} над полем $K$, если $\exists f \in K[x] : f \neq 0, f(a) = 0$. 
        \item $a$ называется \textbf{трансцендентным} над $K$, если он не является алгебраическим над $K$.
    \end{itemize} 
\end{conj}

\textbf{Примеры:} 
\begin{itemize}
    \item $\sqrt{5}, \sqrt[3]{5}, i$ --- алгебраические над $\Q$. Не трудно предъявить соответствующие многочлены: $x^2 - 5$, $x^3 - 5$, $x^2 + 1$.
    \item $\pi, e$ --- трансцендентные над $Q$. Это не тривиально доказывается средствами мат. анализа. Например, трансцендентность $e$ доказывается через ряд Тейлора похожим образом на то, как мы доказывали, что $e$ иррационально (т.е. не является корнем многочлена степени 1 с рациональными коэффициентами). Но это за рамками лекций по алгебре. 
    
    То, что можно доказать легко --- это то, что трансцендентные числа над $Q$ существуют. Это вытекает из мощностных соображений. В $\R$ (или $\C$) множество алгебраических чисел над $\Q$ счётно. Потому что многочленов над $\Q$ счётное количество, а корней у многочлена конечное число, поэтому алгебраических чисел не более, чем счётно. При этом легко предъявить как минимум счётное число алгебраических чисел. Например, это корни многочленов $x - q$, где $q \in \Q$. Выкидывание счётного множества не меняет мощности множества, поэтому в $\R$ (или $\C$) остаётся континуум трансцендентных чисел. 

    \item $\pi + i \cdot e$ алгебраично над $\R$. Т.к. многочлен с вещественными коэффициентами, то если $x = \pi + i e$ является корнем, то $x = \pi - i e$ тоже является корнем. Получаем: 
    \begin{align*}
        x &= \pi \pm ie \\
        x - \pi &= \pm ie \\
        (x - \pi)^2 &= -e^2 \\
    \end{align*}
    Искомый многочлен: $x^2 - 2\pi x + \pi^2 + e^2$. 
\end{itemize}

\begin{theorem}
    Пусть $a$ --- алгебраическое над $K$. Тогда
    $$ F = \{ f \in K[x] \mid f(a) = 0 \} \text{ --- главный идеал в $K[x]$} $$
\end{theorem}
\begin{proof}
    Действительно, если $f, h \in F$, $g \in K[x]$, то
    $(f + h)(a) = f(a) + h(a) = 0$ и $(fg)(a) = f(a) g(a) = 0$. А этот идеал главный, т.к. $K[x]$ --- евклидово кольцо.
\end{proof}
\begin{conj}
    Тогда $F = \langle f_0 \rangle$. $f_0$ --- \textbf{минимальный многочлен} 
    $a$.
\end{conj}

\begin{lemma}
    $f_0$ неприводим.
\end{lemma}
\begin{proof}
    Предположим, что $f_0$ приводим. Тогда $f_0 = gh$, где $1 \leqslant \deg g, \deg h < \deg f_0$. $f_0(a) = 0 \Rightarrow g(a) h(a) = 0$. Тогда либо $g(a) = 0$, либо $h(a) = 0$, либо оба. НУО, $g(a) = 0$. Тогда $g \in F$ $\Rightarrow$ $f_0 \mid g$ $\Rightarrow$ $\deg g \geqslant \deg f_0$.
\end{proof}

\notice $f_0$ ещё называют минимальным неприводимым многочленом. 
Не совсем стандартное обозначение: $\Irr_K a := f_0$. 

\textbf{Пример:} $\Irr_\R i = x^2 + 1$.

\begin{conj}
    $L / K$ --- \textbf{алгебраическое} расширение, если все его элементы алгебраичны над $K$, и \textbf{трансцендентное} в противном случае.
\end{conj}
\textbf{Примеры:}
\begin{itemize}
    \item $\C / \R$ --- алгебраическое расширение.
    \item $\R / \Q$ --- трансцендентное расширение.
    \item Для любого поля $K$, $K(x) / K$ --- трансцендентное, где $K(x)$ --- поле дробно-рациональных функций. 
    
    \begin{proof}
        Пусть $f \in K[x], f \neq 0$, тогда $x \in K(x)$ будет трансцендентным, т.к. $f(x) = f \in K(x)$, $f(x) \neq 0$. Т.е. мы просто в ненулевой многочлен подставляем переменную и получаем тот же самый ненулевой многочлен.
    \end{proof}
\end{itemize}

\begin{conj}
    Пусть $L / K$ --- расширение. Над $L$ есть структура линейного пространства над $K$.
    
    Просто забудем, что мы умеем умножать элементы из $L$ друг на друга. Очевидно, все аксиомы ЛП выполняются.
\end{conj}

\begin{conj}
    Пусть $L / K$ --- расширение. Тогда можно рассматривать размерность поля $L$, как размерность линейного пространства над $K$, т.е. $\dim_K L$. Её называют \textbf{степенью расширения} и обозначают: $(L : K)$, $[L : K]$, $\abs{L : K}$. Мы будем использовать второй вариант.

    Если $[L : K] < \infty$, то $L$ --- \textbf{конечное расширение}  $K$. В противном случае, $L$ --- \textbf{бесконечное расширение} $K$.
\end{conj}

\textbf{Примеры:} 
\begin{itemize}
    \item $\C / \R$ --- конечное расширение, т.к. $[\C : \R] = 2$.
    \item $\R / \Q$ --- бесконечное расширение из-за мощностных соображений.
    \item $K(x) / K$ --- бесконечное расширение, т.к. $K(x)$ содержит многочлены, которые могут быть любой длины.
\end{itemize}

\begin{theorem}
    Если $L / K$ --- конечное расширение, то $L / K$ --- алгебраическое.
\end{theorem}
\begin{proof} $ $

    Пусть $[L : K] = d$. Возьмём $a \in L$. Тогда $1, a, \dots, a^d$ --- ЛЗС над $K$, т.к. их $d + 1$ штука. Это означает, что существует их нетривиальная ЛК, равная 0:
    $$ \alpha_0 + \alpha_1 a + \dots + \alpha_d a^d = 0 \text{ для некоторых $\alpha_0, \dots, \alpha_d \in K$, причём не все 0} $$
    Пусть $f(x) = \alpha_0 + \alpha_1 x + \dots + \alpha_d x^d$. Тогда $f \neq 0$ и $f(a) = 0$. А значит, $a$ алгебраично.
\end{proof}

\notice Обратное не верно.

\textbf{Пример:} $\overline{\Q} / \Q$, где $\overline{\Q} = \{ a \in \C \mid \text{$a$ алгебраично над $Q$} \}$. Очевидно, это алгебраическое расширение. Поймём, почему оно бесконечное. Пусть $[\overline{\Q} : \Q] = d < \infty$. Возьмём корни из первых $d + 1$ простых чисел. Тогда это ЛЗС. Т.е. корень из какого-то простого числа выражается через остальные корни. Это очевидно невозможно над $\Q$.

Почему $\overline{\Q}$ --- вообще поле? Оказывается, сумма или произведение алгебраических чисел --- алгебраическое число. Пока это не очевидно. Поэтому мы пока не будем это доказывать. Пока просто поверим, что $\overline{\Q}$ --- поле.

\begin{theorem}[Мультипликативность степени]
    Пусть $M / L$, $L / K$ --- конечные расширения. Тогда $M / K$ тоже конечно и $[M : K] = [M : L] \cdot [L : K]$.
\end{theorem}
\begin{proof} $ $

    Пусть $e_1, \dots, e_m$ --- базис $L / K$, $f_1, \dots, f_m$ --- базис $M / L$. Докажем, что $(e_i f_j)$ --- базиc $M / K$.

    Проверим, что это порождающая система. Возьмём $c \in M$. Тогда:
    \begin{align*}
        c &= \sum_{j=1}^n \beta_j f_j \quad \text{ (где $\beta_j \in L $)} \\
        &= \sum_{j=1}^n \left( \sum_{i=1}^m \alpha_{ij} e_i \right) f_j \quad \text{ (где $\alpha_{ij} \in K$) } \\
        &= \sum_{i=1}^{m} \sum_{j=1}^{n} \alpha_{ij} e_i f_j \\
        &\in \Lin_K(e_i f_j \mid i=1..m, j=1..n)
    \end{align*}

    Проверим линейную независимость. Пусть:
    \begin{align*}
        0 &= \sum_{i=1}^{m} \sum_{j=1}^{n} \alpha_{ij} e_i f_j \quad \text{ (где $\alpha_{ij} \in K$) } \\
        &= \sum_{j=1}^n \underbrace{\left( \sum_{i=1}^m \alpha_{ij} e_i \right)}_{\in L} f_j \\
        &\Rightarrow \sum_{i=1}^m \alpha_{ij} e_i = 0 \; \;\forall j=1..n \quad \text{ (т.к. $(f_j)$ --- базис $M / L$) } \\
        &\Rightarrow \alpha_{ij} = 0 \; \; \forall i=1..m, j=1..n \quad \text{(т.к. $(e_i)$ --- базис $L / K$)}
    \end{align*}
\end{proof}

\begin{conj}
    Пусть $L / K$ --- расширение; $a_1, \dots, a_n \in L$.
    Тогда:
    $$ K(a_1, \dots, a_n) := \bigcap_{\substack{K \subset F \subset L \\ \substack{F \text{ --- подполе}} \\ a_1, \dots, a_n \in F}} F $$
    $K(a_1, \dots, a_n)$ --- самое маленькое подполе $L$, содержащее $K$ и содержащее все элементы $a_1, \dots, a_n$. Читается ``$K$ от $a_1, \dots, a_n$''. Называется оно ``\textbf{поле, полученное присоединием к полю $K$ элементов $a_1, \dots, a_n$}''.
\end{conj}
\textbf{Примеры:} 
\begin{itemize}
    \item Расширение $\Q(i) / \Q$. Интересуемся $\Q(i)$, т.е. $K = \Q$, $n = 1$, $a_1 = i$. У нас получаются одинаковые обозначения, конечно. Поймём, что мы имеем в виду одно и то же. Если $F$ содержит $1$ и $i$, тогда оно содержит все гауссовые числа. Ну тогда само поле гауссовых чисел и будет самым маленьким подполем.

    \item $\Q(\sqrt{5}) = \{ a + b \sqrt{5} \mid a, b \in \Q \}$.
    \begin{proof} $ $
        \begin{itemize}
            \item[``$\supset$'':] По определению $1, \sqrt{5} \in \Q(\sqrt{5})$. Тогда если $a, b \in \Q \subset \Q(\sqrt{5})$, то $a + b \sqrt{5} \in \Q(\sqrt{5})$.
             
            \item[``$\subset$'':] Пусть $M = \{ a + b \sqrt{5} \mid a, b \in \Q \}$. Это поле! Действительно, замкнутость относительно сложения и умножения очевидна. А чтобы делить нужно делать примерно то же самое, что при делении комплексных чисел:
            $$ \frac{a + b\sqrt{5}}{c + d\sqrt{5}} = \frac{(a + b\sqrt{5})(c - d\sqrt{5})}{c^2 - 5d} \in M $$
            Раз это поле, то по определению $K(a_1, \dots, a_n)$ получаем, что $\Q(\sqrt{5}) \subset M$.
        \end{itemize}
    \end{proof}
\end{itemize}

\notice Аналогично можно определить присоединение бесконечного количества элементов, но мы этого делать не будем.

\notice Не трудно понять из определений, что $K(a_1, \dots, a_n) = K(a_1)(a_2)\dots(a_n)$. Т.е. присоединение нескольких элементов получается последовательным присоединением каждого элемента.

Это замечание показывает важность расширения поля через присоединение одного элемента.

\begin{conj}
    Расширение $L / K$ называется \textbf{простым}, если $L = K(a)$ для некоторого $a \in L$.
\end{conj}

Прежде, чем заняться изучением структуры простого расширения, нам нужно разобраться с некоторыми вопросами факторколец. А именно, когда $K[x] / \langle f \rangle$ является полем. Вопрос аналогичен тому, когда кольцо классов вычетов $\Z / \langle f \rangle$ является полем. Мы знаем, что тогда и только тогда, когда $p$ --- простое. Здесь ответ аналогичен.

\begin{lemma}
    $K[x] / \langle f \rangle$ --- поле $\Longleftrightarrow$ $f$ неприводим.
\end{lemma}
\begin{proof} $ $

    \begin{itemize}
        \item[``$\Longrightarrow$'':] Предположим, это не так. Пусть $f = gh$, где $1 \leqslant \deg g, \deg h < \deg f$. В $K[x]/\langle f \rangle$ верно, что класс $[f] = 0$. А тогда $[g] \cdot [h] = 0$. Но $[g] \neq 0$ и $[h] \neq 0$, т.к. $f \nmid g$ и $f \nmid h$. Получаем, что в $K[x]/(f)$ есть делители нуля. А тогда это не область целостности и уж тем более не поле.

        \item[``$\Longleftarrow$'':] Хотим доказать, что любой ненулевой элемент обратим. Пусть $G \in K[x] / \langle f \rangle$, $G \neq 0$. Тогда $G = [g] = g + \langle f \rangle$ для некоторого $g \in K[x]$, $\deg g < \deg f$. Т.к. $G \neq 0$, то $g \neq 0$. Т.к. $f$ неприводим, получаем, что $\gcd(g, f) = 1$. А значит, $ga + fb = 1$ для некоторых $a, b \in K[x]$. Возвращаясь к классам, $[1] = G \cdot [a] + \underbrace{[f]}_{=[0]} \cdot [b] = G \cdot [a]$. Получаем, что $G$ обратим в $K[x] / \langle f \rangle$.
    \end{itemize}
\end{proof}

\textbf{Пример:} cовсем взрослое определение комплексных чисел: $\C = \R[x] / (x^2 + 1)$. 

Теперь займёмся изучением структуры простого расширения. Для этого нам важно различать два случая: когда $a$ --- алгебраическое, и когда $a$ --- трансцендентное.

\begin{theorem}
    Пусть $a$ --- алгебраический элемент над $K$. Тогда $K(a) \cong K[x] / \langle f \rangle$, где \\ $f = \Irr_K a$.
\end{theorem}
\begin{proof}
    Определим отображение:
    \begin{align*}
        \varphi : K[x] &\to K(a) \\
        g &\mapsto g(a) 
    \end{align*}
    $\varphi$ --- гомоморфизм колец, т.к. это просто гомоморфизм подстановки в многочлен (мы это изучали в первом семестре).
    
    Поймём, что $\Ker \varphi = \langle f \rangle$. По определению
    $\Ker \varphi = \{ g \in K[x] \mid g(a) = 0 \}$. Но это был главный идеал в $K[x]$, порождённый $f = \Irr_K a$.

    По теореме о гомоморфизме (для групп) $K[x] / \langle f \rangle \overset{\widetilde{\varphi}}\cong \Imm \varphi$, где $\widetilde{\varphi}$ --- изоморфизм групп.

    Заметим, что $\widetilde{\varphi}$ сохраняет умножение. Действительно, пусть:
    $$[g], [h] \in K[x]/\langle f \rangle, [g] = g + \langle f \rangle, [h] = h + \langle f \rangle$$
    $\widetilde{\varphi}$ --- индуцированный гомоморфизм. 
    И он определяется следующим образом: $\widetilde{\varphi}([g]) = \varphi(g)$. По определению произведения классов: $[g] \cdot [h] = [gh]$. Поэтому:
    $$\widetilde{\varphi}([g] \cdot [h]) = \widetilde{\varphi}([gh]) = \underbrace{\varphi(gh) = \varphi(g) \cdot \varphi(h)}_{\text{$\varphi$ --- гом-м колец}} = \widetilde{\varphi}([g]) \cdot 
    \widetilde{\varphi}([h])$$

    Таким образом, $\widetilde{\varphi}$ --- изоморфизм колец. Т.к. $f = \Irr_K a$ неприводим, $K[x] / \langle f \rangle$ --- поле.
    Тогда $\Imm \varphi$ --- тоже поле. 

    Заметим, что $K \subset \Imm \varphi$, т.к. $\varphi(c) = c \;\; \forall c \in K \subset K[x]$. Также заметим, что $\varphi(x) = a$. Действительно, подставляем $a$ в многочлен $g(x) = x$ и получаем $a$. Тогда $\Imm \varphi \supset K(a)$ по определению $K(a)$, но $\Imm \varphi \subset K(a)$, а значит, они равны.
\end{proof}

\notice Если отождествить класс константы $c + \langle f \rangle$ с константой $c$ в $K[x] / \langle f \rangle$, то 
\begin{enumerate}
    \item $K[x] / \langle f \rangle$ можно рассматривать, как расширение $K$.
    \item $\widetilde{\varphi} : K[x]/ \langle f \rangle \to K(a)$ --- изоморфизм расширений $K$ (изоморфизм над $K$).
\end{enumerate}

\begin{conj}
    Пусть $L_1 / K$, $L_2 / K$. $\psi : L_1 \to L_2$ --- \textbf{изоморфизм над $K$}, если это изоморфизм полей и $\forall c \in K : \psi(c) = c$. 
\end{conj}

\begin{theorem}
    Пусть $K$ --- поле, $f \in K[x]$ неприводим. Тогда для $K' = K[x] / \langle f \rangle$ справедливо, что $K' = K([x])$, $\Irr_K [x] = f$ и $[K' : K] = \deg f$.
\end{theorem}
\begin{proof}
    Очевидно (нет), $K' / K$ --- расширение. Пусть $d = \deg f$. Тогда \\ $K' = \Lin_K(1, [x], \dots, [x]^{d-1})$. А значит, $K([x]) = K'$.

    Проверим, что $\Irr_K [x] = f$. $f([x]) = f + \langle f \rangle =
    0 + \langle f \rangle = [0]$. Возьмём $g \in K[x]$, т.ч. $g \neq 0$ и $\deg g < \deg f$. Тогда $g([x]) = g + \langle f \rangle \neq [0]$, т.к. $g \nmid f$. Значит, $\Irr_K [x] = f$.

    Понятно (нет), что $(1, [x], \dots, [x]^{d-1})$ --- базис $K'$.
    Тогда $[K' : K] = d$.
\end{proof}

Таким образом, мы классифицировали простые расширения $K$. Это такие $K[x] / \langle f \rangle$, где $f \in K[x]$ неприводим, со стандартным вложением в них поля $K$.

\textbf{Примеры:}
\begin{itemize}
    \item $\Q(\sqrt{5}) = \{ a + b \sqrt{5} \mid a, b \in \Q \}$
    \item $\Q(\sqrt[3]{5}) = \{ a + b \sqrt{5} + c(\sqrt{5})^2 \mid a, b, c \in \Q \}$
    \item и тому подобные...
\end{itemize} 

\follow Конечнопорождённое расширение является алгебраическим тогда и только тогда, когда все присоединяемые элементы алгебраические.
\begin{proof}
    В одну сторону это понятно. Если расширение алгебраическое, тогда все присоединенные элементы будут алгебраическими просто по определению.

    В другую сторону утверждение более содержательное. Сформулируем и докажем даже более сильное утверждение: пусть $a_1, \dots, a_n$ --- алг. над $K$, тогда $K(a_1, \dots, a_n)/K$ --- конечное расширение. Из конечности расширения следует его алгебраичность.

    Докажем индукцией по $n$.
    \begin{itemize}
        \item База $n = 1$. Только что видели.
        \item Переход $n - 1 \to n$.
        
        Запишем присоединение по одному из прошлых замечаний:
        $$ K(a_1, \dots, a_n) = K(a_1)(a_2)\dots(a_n) $$

        Если $a_n$ алгебраично над $K$, то оно уж тем более алгебраично над более широким полем $K(a_1, \dots, a_{n-1})$ --- мы всегда можем взять тот же самый многочлен. Тогда \\$K(a_1, \dots, a_{n-1})(a_n) / K(a_1, \dots, a_{n-1})$ --- конечное расширение. По ИП $K(a_1, \dots, a_{n-1}) / K$ конечно. Применяя мультипликативность степени, получаем, что $K(a_1, \dots, a_n)$ тоже конечно.
    \end{itemize}
\end{proof}

\follow Пусть $a, b$ алгебраичны над $K$. Тогда $a \pm b$, $ab$ тоже алгебраичны над $K$.
\begin{proof}
    Как мы только что поняли, $K(a, b) / K$ --- конечное расширение. А значит, алгебраичное. А значит, любой элемент в $K(a, b)$ алгебраичен над $K$.
\end{proof}


