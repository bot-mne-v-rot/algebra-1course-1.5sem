\usepackage{amsmath}\section{Лекция номер 15}
\subsection{поля разложения}

\begin{conj}
    Пусть $K$ -- поле, $f \in K[x]$ -- некоторый многочлен, $L/K$ -- расширение. Тогда $L$ называется \textbf{полем разложения} $f$, если:
    \begin{enumerate}
        \item $f$ там раскладывается на линейные множетели. $f = a_0 \cdot (X - x_1) \dots (X - x_n)$ в $L[x]$. 
        \item $L$ получается из $K$ присоединением всех этих корней. $L = K(x_1, \dots, x_n)$. То есть $L$ является наименьшим полем с таким свойством. 
    \end{enumerate}
\end{conj}
Примеры:
\begin{enumerate}
    \item $\Q\left(\sqrt 2\right)$~--- поле разложения $X^2-2$ над $\Q$
    \item $\Q\left(\sqrt[3]{2}\right)$~--- не поле разложения $X^3-2$ над $\Q$ (содержит только один корень)
    \item $\Q\left(\sqrt[3]{2}, \omega\right)$ (где $\omega$~--- какой-то из комплексных корней $X^3-2$)~--- поле разложения $X^3-2$.
    \item $\C$~--- не поле разложения $X^3-2$ над $\Q$ (все корни содержит, но не минимально).
\end{enumerate}

\begin{theorem}
    Пусть $K$ -- поле, $f$ -- его многочлен. Тогда у $f$ существует поле разложения над $K$. 
\end{theorem}
\begin{proof}
    Индукция по степени $f$. 

    \quad \underline{База $f = 1$:} Тогда у него есть единственный корень и он лежит в $K$, то есть $K$ -- само поле разложения. 

    \quad \underline{Переход:} Пусть $f_0$ -- любой неприводимый делитель $f$.
    Возьмем $L := K[X] / (f_0)$, тогда, как мы выясняли ранее, $L=K(x_1)$, где $x_1$~--- некоторый корень $f_0$ и, соответственно, $f$.
    Тогда по теореме Безу в $L$ многочлен $f$ раскладывается как $f=(X-x_1)g$, где $\deg g = \deg f - 1$.
    Тогда по предположению индукции существует поле разложения $g$ в $L$, равное $M=L(x_2, x_3, \ldots, x_n)$, где $x_2,x_3,\ldots, x_n$~--- оставшиеся корни $f$.
    Но тогда $M=L(x_2, x_3, \ldots, x_n) = (K(x_1))(x_2, x_3, \ldots, x_n) = K(x_1, x_2, \ldots, x_n) \Rightarrow$ $M$ является полем разложения $f$ над $K$.
\end{proof}

\begin{conj}
    Поле называется \textbf{простым}, если оно не содержит собственных (т.е. отличных от себя самого) подполей.
\end{conj}

\begin{theorem-non}
    \begin{enumerate}
        \item Любое простое поле изоморфно либо $\Q$, либо $\mathbb{F}_p$
        \item Любое поле содержит единственное простое подполе.
    \end{enumerate}
\end{theorem-non}
\begin{proof}
    Рассмотрим произвольное поле $K$ и гомоморфизм $\varphi$, действующий
    \begin{align*}
        & \qquad \Z \longrightarrow K \\
        & n \mapsto \underbrace{1+1+\cdots+1}_{n \text{ раз}}\quad (\forall n\in \N) \\
        & 0 \mapsto 0 \\
        & -n \mapsto -(\underbrace{1+1+\cdots+1}_{n \text{ раз}})\quad (\forall n\in \N)
    \end{align*}

    Если $\text{char } K = p$, то $\Ker \varphi = p\Z$, отсюда по теореме о гомоморфизме для групп
    \begin{gather*}
        \Z / p\Z \cong \Imm \varphi
    \end{gather*}

    (Теорема дает нам только изоморфизм этих множеств как групп по сложению, однако тривиально проверяется, что $\varphi$ сохраняет умножение, поэтому это еще и изоморфизм колец с $1$).
    При этом мы знаем, что $\Z / p\Z$ является полем, так что $\Imm \varphi$ тоже будет полем, т.е. $\Imm \varphi$~--- подполе в $K$.
    Поэтому если мы (как в пункте $1$) хотим чтобы $K$ было простым, то необходимо $K=\Imm \varphi$
    Если мы (как в пункте $2$) рассматриваем произвольное $K$, то мы нашли его простое подполе.\medskip

    Если теперь $\text{char} K = 0$, то это значит $\Ker \varphi = 0$, что позволяет нам рассмотреть отображение $\varphi'$:
    \begin{gather*}
        \Q \longrightarrow K \\
        \frac mn \mapsto \frac{\varphi'(m)}{\varphi'(n)}
    \end{gather*}

    У него тоже нулевое ядро, поэтому по теореме о гомоморфизме
    \begin{gather*}
        \Q \cong \Imm \varphi'
    \end{gather*}
    При этом $\Q$ является полем $\Rightarrow$ $\Imm \varphi'$ будет подполем в $K$.
    Поэтому если (см. пункт $1$) $K$ простое, то $K=\Imm \varphi'$.
    А если (см. пункт $2$) $K$ произвольное, то мы нашли искомое простое подполе.\medskip

    Осталось проверить, что простое подполе единственно.
    Ну действительно, пусть $F_1, F_2$~--- простые подполя $K$.
    Пересечение полей тоже будет полем, поэтому $F_1 \cap F_2$ будет подполем в $F_1$ и $F_2$.
    Но все их подполя, по определению, собственные, поэтому $F_1 = F_1 \cap F_2 = F_2$.
\end{proof}

Таким образом, у любого поля существует простое подполе, полностью определяемое характеристикой исходного.
Отсюда вытекает

\follow Пусть $K$~--- конечное поле, $\text{char} K = p$. Тогда $|K|=p^n$ для некоторого $n\in\N$.
\begin{proof}
    Из доказательства предыдущей теоремы видно, что $\mathbb{F}_p$ (или изоморфное ему)~--- подполе в $K$.
    Но тогда $K$ можно рассматривать как линейное пространство над $\mathbb{F}_p$.
    Но это и значит, что $|K|=|\mathbb{F}_p|^n=p^n$, где $n = [K:\mathbb{F}_p]$.
\end{proof}