\section{Расширение поля. Алгебраические и трансцендентные элементы. Минимальный многочлен алгебраического элемента}
\begin{conj}
    $K$ --- \textbf{подполе} поля $L$, если это подкольцо, являющееся полем. Другими словами:
    \begin{itemize}
        \item $K$ --- подгруппа $L$ по сложению, т.е $(K, +)$ --- подгруппа $(L, +)$;
        \item $1 \in K$;
        \item $K$ замкнуто относительно умножения, т.е. $K \cdot K \subset K$;
        \item $K$ замкнуто относительно взятия обратного, т.е. $K^{-1} \subset K$.
    \end{itemize}
\end{conj}

\begin{conj}
    Если задано поле $L$ и его подполе $K$, то говорят, что задано
    \textbf{расширение полей}  $L/K$ (читается ``$L$ над $K$'').
\end{conj}

\textbf{Примеры:} 
\begin{itemize}
    \item $\C / \R$, $\R / \Q$, $K / K$. 
    \item $\Q (i) / \Q$, $\Q (i) = \{\alpha + i \beta \mid \alpha, \beta \in \Q \}$ --- поле гауссовых чисел.
\end{itemize}

\begin{conj}
    Поле $K$ называется \textbf{простым}, если оно не содержит
    подполей, не равных $K$.
\end{conj}

\textbf{Историческая справка}: $\{0\}$ не подполе, т.к. не содержит
единицы отличной от нуля.

\textbf{Примеры:}
\begin{enumerate}
    \item $Q$ --- простое. Пусть $K \subset Q$ --- подполе $Q$.
    Тогда $0, 1 \in K$. Тогда $\N \subset K$, т.к. любое натуральное
    число представимо в виде суммы единиц. Тогда $\Z \subset K$.
    А тогда и $\Q \subset K$.

    \item $\Z / p\Z$ --- кольцо классов вычетов по модулю $p$, где
    $p$ --- простое. Это простое поле, т.к. $\Z / p\Z = \{ 
        \underbrace{1 + \dots + 1}_{\text{$t$ штук}} \mid
        0 \leqslant t < p\}$, а любое подполе содержит единицу.
\end{enumerate}

\begin{conj}
    Пусть $L / K$ --- расширение, $a \in L$. 
    \begin{itemize}
        \item $a$ называется \textbf{алгебраическим} над полем $K$, если $\exists f \in K[x] : f \neq 0, f(a) = 0$. 
        \item $a$ называется \textbf{трансцендентным} над $K$, если он не является алгебраическим над $K$.
    \end{itemize} 
\end{conj}

\textbf{Примеры:} 
\begin{itemize}
    \item $\sqrt{5}, \sqrt[3]{5}, i$ --- алгебраические над $\Q$. Не трудно предъявить соответствующие многочлены: $x^2 - 5$, $x^3 - 5$, $x^2 + 1$.
    \item $\pi, e$ --- трансцендентные над $Q$. Это не тривиально доказывается средствами мат. анализа. Например, трансцендентность $e$ доказывается через ряд Тейлора похожим образом на то, как мы доказывали, что $e$ иррационально (т.е. не является корнем многочлена степени 1 с рациональными коэффициентами). Но это за рамками лекций по алгебре. 
    
    То, что можно доказать легко --- это то, что трансцендентные числа над $Q$ существуют. Это вытекает из мощностных соображений. В $\R$ (или $\C$) множество алгебраических чисел над $\Q$ счётно. Потому что многочленов над $\Q$ счётное количество, а корней у многочлена конечное число, поэтому алгебраических чисел не более, чем счётно. При этом легко предъявить как минимум счётное число алгебраических чисел. Например, это корни многочленов $x - q$, где $q \in \Q$. Выкидывание счётного множества не меняет мощности множества, поэтому в $\R$ (или $\C$) остаётся континуум трансцендентных чисел. 

    \item $\pi + i \cdot e$ алгебраично над $\R$. Т.к. многочлен с вещественными коэффициентами, то если $x = \pi + i e$ является корнем, то $x = \pi - i e$ тоже является корнем. Получаем: 
    \begin{align*}
        x &= \pi \pm ie \\
        x - \pi &= \pm ie \\
        (x - \pi)^2 &= -e^2 \\
    \end{align*}
    Искомый многочлен: $x^2 - 2\pi x + \pi^2 + e^2$. 
\end{itemize}

\begin{theorem}
    Пусть $a$ --- алгебраическое над $K$. Тогда
    $$ F = \{ f \in K[x] \mid f(a) = 0 \} \text{ --- главный идеал в $K[x]$} $$
\end{theorem}
\begin{proof}
    Действительно, если $f, h \in F$, $g \in K[x]$, то
    $(f + h)(a) = f(a) + h(a) = 0$ и $(fg)(a) = f(a) g(a) = 0$. А этот идеал главный, т.к. $K[x]$ --- евклидово кольцо.
\end{proof}
\begin{conj}
    Тогда $F = \langle f_0 \rangle$. $f_0$ --- \textbf{минимальный многочлен} 
    $a$.
\end{conj}

\begin{lemma}
    $f_0$ неприводим.
\end{lemma}
\begin{proof}
    Предположим, что $f_0$ приводим. Тогда $f_0 = gh$, где $1 \leqslant \deg g, \deg h < \deg f_0$. $f_0(a) = 0 \Rightarrow g(a) h(a) = 0$. Тогда либо $g(a) = 0$, либо $h(a) = 0$, либо оба. НУО, $g(a) = 0$. Тогда $g \in F$ $\Rightarrow$ $f_0 \mid g$ $\Rightarrow$ $\deg g \geqslant \deg f_0$.
\end{proof}

\notice $f_0$ ещё называют минимальным неприводимым многочленом. 
Не совсем стандартное обозначение: $\Irr_K a := f_0$. 

\textbf{Пример:} $\Irr_\R i = x^2 + 1$.

\begin{conj}
    $L / K$ --- \textbf{алгебраическое} расширение, если все его элементы алгебраичны над $K$, и \textbf{трансцендентное} в противном случае.
\end{conj}
\textbf{Примеры:}
\begin{itemize}
    \item $\C / \R$ --- алгебраическое расширение.
    \item $\R / \Q$ --- трансцендентное расширение.
    \item Для любого поля $K$, $K(x) / K$ --- трансцендентное, где $K(x)$ --- поле дробно-рациональных функций. 
    
    \begin{proof}
        Пусть $f \in K[x], f \neq 0$, тогда $x \in K(x)$ будет трансцендентным, т.к. $f(x) = f \in K(x)$, $f(x) \neq 0$. Т.е. мы просто в ненулевой многочлен подставляем переменную и получаем тот же самый ненулевой многочлен.
    \end{proof}
\end{itemize}