\section{Теорема Лагранжа и следствия из неё}
\follow Пусть $G$ -- конечная группа, $H < G$. 
Тогда $\abs{G} = (G : H) \abs{H}$.
\begin{proof} $ $

    Пусть $K := \{ e \}$ (тривиальная подгруппа), \\
    тогда $(G : K) = \abs{G}$, $(H : K) = \abs{H}$, \\
    т.к. $gK = ge = g$, $hK = he = h$, где $g \in G, h \in H$. \\
    Получаем, что $\abs{G} = (G : K) = (G : H)(H : K) =
    (G : H)\abs{H}$.
\end{proof}

\begin{theorem-nonna}[Лагранжа]
    Пусть $G$ -- группа,  $\abs{G} < \infty$, $H < G$. 
    Тогда $\abs{H} \mid \abs{G}$.
\end{theorem-nonna}
\begin{proof}
    Из предыдущего следствия: $\abs{G} = (G : H) \cdot \abs{H}$.
\end{proof}

\follow Пусть $G$ -- группа,  $\abs{G} < \infty$, $g \in G$.
Тогда $\ord g \mid G$.
\begin{proof}
    $\ord g = \abs{\cycle g}$.
\end{proof}

\follow Пусть $G$ -- группа,  $\abs{G} < \infty$, $g \in G$.
Тогда $g^{\abs{G}} = e$.
\begin{proof}
    $g^{\ord g} = e$ по определению $\Rightarrow g^{\abs{G}} =
    (g^{\ord g})^{\abs{G}/\ord g} = e$.
\end{proof}

\follow (Теорема Эйлера)
Пусть $m \in \N$, $a \in \Z$, $\gcd(a, m) = 1$.
Тогда $a^{\varphi(m)} \equiv 1 \; (\operatorname{mod} m)$.
\begin{proof} $ $\\
    Пусть $G = (\Z/m\Z)^*$, $g = [a]_m$. 
    Так как группа обратимых элементов кольца вычетов состоит из всех чисел, взаимно простых с $m$.(Когда-то доказывали)
    
    Т.к. $\gcd(a, m) = 1$, $g \in G$. \\
    Из предыдущего следствия $[a]_m^{\abs{G}} = [1]_m$. \\
    По определению функции Эйлера $\abs{G} = \abs{(\Z/m\Z)^*}= 
    \varphi(m)$. \\
    По определению классов $G$, $[a]_m^{\varphi(m)} = [1]_m$
    $\Rightarrow$ $a^{\varphi(m)} \equiv 1 \; (\operatorname{mod} m)$.
\end{proof}