\section{Теорема Лагранжа для симметрических билинейных форм}
\begin{conj}
    Пусть у нас есть $\B$ -- билинейная форма на $V$. И есть два вектора $u, v \in V$. Говорят, что векторы 
    $u$ и $v$ ортогональны, если $\B(u, v) = 0$.
\end{conj}

Если $\B$ симметричная билинейная форма, то это симметричное отноешние.
Ортогональность векторов $u$ и $v$ обозначают как $u \perp v$.

\begin{conj}
    Базис $e_1, \dots, e_n$ пространства $V$ будем называть ортогональным, если $e_i \perp e_j \forall i \neq j$. 
\end{conj}

\begin{theorem}
    (Лагранжа) Пусть $\B$ -- симметрическая билинейная форма на $V$, $\dim V = n < + \infty$. Тогда в $V$ существует ортогональный
    базис.
\end{theorem}
\begin{proof}
    Докажем теорему индукцией по размерности $V$. 
    \begin{itemize}
        \item База $n = 1$. Тогда любой базис ортогонален. 
        \item Переход $n-1 \to n$. Начнем с того, что может быть случай, когда форма нулевая. Тогда любой базис ортогональный. 
        
        В случае, если форма ненулевая, то должен существовать вектор $e_1 \in V : \B(e_1, e_1) \neq 0$. Предположим, что это не так. 
        То есть $\forall v \in V : \B(v, v) = 0$. Введем обозначение $q(v) := \B(v, v)$. Теперь самое время нам вспомнить про замечательную формулу: 
        \begin{gather*}
            \forall u, v \in V : \B(u, v) = \frac{1}{2} (q(u+v) - q(u) - q(v))
        \end{gather*}
        Из такого равенства очевидно, что если $q$ -- нулевая квадратичная форма, то $\B$ -- нулевая билинейная форма. Получаем противоречие.

        Теперь давайте рассмотрим множество всех векторов, ортогональных к $e_1$: $W = \{ w \in V \mid e_1 \perp w \}$ -- линейное подпространство в $V$. 
        Рассмотрим следующее отображение: 
        \begin{gather*}
            \lambda: V \longrightarrow K \\
            v \longmapsto \B(v, e_1)
        \end{gather*}
        Заметим, что образ $\lambda$ не может быть равен 0, так как $\B(e_1, e_1) \neq 0$. А значит $\Imm{\lambda} = K$, так как $K$ -- одномерное линейное пространство, то есть 
        подпространство может быть либо нулевым, либо всем пространством. 

        Таким образом:
        \begin{align*}
            \dim{W} &= \dim{\Ker{\lambda}} \\
            &= \dim{V} - \dim{\Imm{\lambda}} \\
            &= n - 1
        \end{align*}
        Заметим, что $\B$ можно ограничить на декартов квадрат $W$. То есть $\B' = \B \mid_{W \times W}$ -- симметрическая билинейная форма на $W$. Размерность пространства мы получили 
        на 1 меньше, так что, по индукционному предположению у 
        $W$ существует ортогональный базис $e_2, \dots, e_n$. Заметим теперь, что у нас есть пространство размерности $n-1$ и есть вектор $e_1$, который не пренадлежит $\Lin{(e_2, \dots, e_n)}$. Тогда все 
        вектора $e_1, \dots, e_n$ -- ЛНС. А значит это наш искомый ортогональный базис $V$. 
    \end{itemize}
\end{proof}    
\notice $E$ -- ортогональный базис $V$ по отношению к форме $\B \Longleftrightarrow$ матрица $\B$ в этом базисе диагональная.

Мы пока что не до конца решили классификаицонную задачу, так как мы пока что не знаем, могут ли разные диагональные матрицы оказаться матрицами Грама одной и той же 
билинейной формы в разных базисах?

\notice $E = (e_1, \dots, e_n)$ -- ортогональный базис $V$, $E' = (\alpha_1 e_1, \dots, \alpha_n e_n)$. Пускай 
$[\B]_E = \diag{\lambda_1, \dots, \lambda_n}$. Тогда в базисе $E'$ форма $\B$ тоже будет иметь диагональную матрицу. 
Проверим, что будет стоять на диагонали: 
\begin{gather*}
    \B(\alpha_i e_i, \alpha_i e_i) = \alpha_i^2 \B(e_i, e_i) = \alpha_i^2 \lambda_i
\end{gather*}
То есть $[\B]_{E'} = \diag{\alpha_1^2 \lambda_1, \dots, \alpha_n^2 \lambda_n}$. Отсюда уже видно, что $\lambda_1, \dots, \lambda_n$ -- не инвариант матрицы, так что ответ на вопрос, которым
мы задавались выше -- нет. 